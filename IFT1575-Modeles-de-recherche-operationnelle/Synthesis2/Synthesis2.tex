\documentclass{report}
%\usepackage[utopia]{mathdesign}
%\usepackage{amsmath, amsthm}

\usepackage[top=2.5cm, bottom=2.5cm, left=2cm, right=2cm]{geometry}


\usepackage{amsmath,amsfonts,amsthm,amssymb,mathtools}
\usepackage{nicefrac, xfrac}
%\usepackage[varbb]{newpxmath}
%\usepackage[osf,largesc,theoremfont]{newpxtext}
%\usepackage{coelacanth}
%\usepackage{beraserif} % Bitstream Vera Serif font
%\usepackage{berasans} % Bitstream Vera Sans font
%\usepackage{beramono} % Bitstream Vera Sans Mono font
%\usepackage{berasans}
%\usepackage{libertine}
%\usepackage{mathpazo}
%\usepackage{palatino}
%\usepackage{crimson}

% NEW ------- For pointilles lines
\usepackage{multido}
\usepackage{lipsum}

%% Choose one of the following (if not choosing the  
%% default, viz., Computer Modern, font family):
%\usepackage{lmodern}
\usepackage{bold-extra}
%%
%\usepackage{mathpazo}
% \usepackage{newpxmath}
%\usepackage{kpfonts} % Very good
%%
%\usepackage{mathptmx} %Very good
%\usepackage{stix} 
%\usepackage{txfonts} %Very good
\usepackage{newtxtext,newtxmath} %Very good
%%
%\usepackage{libertine} \usepackage[libertine]{newtxmath}
%\usepackage{libertine,libertinust1math} % added 2019/11/28
%%
%\usepackage{newpxtext} \usepackage[euler-digits]{eulervm}
%\usepackage{textcomp}
%\usepackage{bm}
\usepackage{contour}
\usepackage{adjustbox}
\usepackage{nicematrix}
\usepackage{booktabs}
\usepackage{tabularx}

%For multiple line boxed equations
\usepackage{empheq}





%%%%%%%%%%%%%%%%%%%%%%%%%%%%%%%%%%%%%%%%%%%%%%%%%%%%%%%%%%%%%%%%%%%%%%%%%%%%%%%%%%%%%%%%%%%%%%%%%
%                                 Additional Packages (commented)
%%%%%%%%%%%%%%%%%%%%%%%%%%%%%%%%%%%%%%%%%%%%%%%%%%%%%%%%%%%%%%%%%%%%%%%%%%%%%%%%%%%%%%%%%%%%%%%%%
%%%%%%%%%%%%%%%%%%%%%%%%%%%%%%%%%%%%%%%%%%%%%%%%%%%%%%%%%%%%%%%%%%%%%%%%%%%%%%%%%%%%%%%%%%%%%%%%%
%                                 Language and Encoding
%%%%%%%%%%%%%%%%%%%%%%%%%%%%%%%%%%%%%%%%%%%%%%%%%%%%%%%%%%%%%%%%%%%%%%%%%%%%%%%%%%%%%%%%%%%%%%%%%
\usepackage[french]{babel} % Utilisation du français pour la typographie et les hyphenations
\usepackage[T1]{fontenc} % Utilisation de l'encodage de police T1 pour une meilleure gestion des caractères accentués
\usepackage{titlesec}
\usepackage[utf8]{inputenc} % Permet l'utilisation de l'encodage UTF-8 dans le fichier source
\usepackage{csquotes} % Gestion avancée des guillemets
\usepackage{microtype}
\usepackage{listings}

%%%%%%%%%%%%%%%%%%%%%%%%%%%%%%%%%
% PACKAGE IMPORTS
%%%%%%%%%%%%%%%%%%%%%%%%%%%%%%%%%


\usepackage[tmargin=2cm,rmargin=1in,lmargin=1in,margin=0.85in,bmargin=2cm,footskip=.2in]{geometry}
\usepackage{xfrac}
\usepackage[makeroom]{cancel}
\usepackage{bookmark}
\usepackage{enumitem}
\usepackage{hyperref,theoremref}
\hypersetup{
	pdftitle={Assignment},
	colorlinks=true, linkcolor=black,
	bookmarksnumbered=true,
	bookmarksopen=true
}
\hypersetup{
    colorlinks=true,       % false: boxed links; true: colored links
    linkcolor=black,       % color of internal links
    citecolor=black,       % color of citation links
    filecolor=black,       % color of file links
    urlcolor=black         % color of external links (like \url and \href)
}
\usepackage[most,many,breakable]{tcolorbox}
\usepackage{xcolor}
\usepackage{varwidth}
\usepackage{varwidth}
\usepackage{etoolbox}
%\usepackage{authblk}
\usepackage{nameref}
\usepackage{multicol,array}
\usepackage{colortbl}
\usepackage{tikz-cd}
\usepackage[ruled,vlined,linesnumbered]{algorithm2e}
\usepackage{comment} % enables the use of multi-line comments (\ifx \fi) 
\usepackage{import}
\usepackage{xifthen}
\usepackage{pdfpages}
\usepackage{transparent}

\newcommand\mycommfont[1]{\footnotesize\ttfamily\textcolor{blue}{#1}}
\SetCommentSty{mycommfont}
\newcommand{\incfig}[1]{%
    \def\svgwidth{\columnwidth}
    \import{./figures/}{#1.pdf_tex}
}

\usepackage{tikzsymbols}
%\renewcommand\qedsymbol{$\Laughey$}


%\usepackage{import}
%\usepackage{xifthen}
%\usepackage{pdfpages}
%\usepackage{transparent}


%%%%%%%%%%%%%%%%%%%%%%%%%%%%%%
% SELF MADE COLORS
%%%%%%%%%%%%%%%%%%%%%%%%%%%%%%



\definecolor{lightBlue}{rgb}{0.88,1,1}
\definecolor{myg}{RGB}{56, 140, 70}
\definecolor{myb}{RGB}{45, 111, 177}
\definecolor{myr}{RGB}{199, 68, 64}
\definecolor{mytheorembg}{HTML}{F2F2F9}
\definecolor{mytheoremfr}{HTML}{00007B}
\definecolor{mylenmabg}{HTML}{FFFAF8}
\definecolor{mylenmafr}{HTML}{983b0f}
\definecolor{mypropbg}{HTML}{f2fbfc}
\definecolor{mypropfr}{HTML}{191971}
\definecolor{myexamplebg}{HTML}{F2FBF8}
\definecolor{myexamplefr}{HTML}{88D6D1}
\definecolor{myexampleti}{HTML}{2A7F7F}
\definecolor{mydefinitbg}{HTML}{E5E5FF}
\definecolor{mydefinitfr}{HTML}{3F3FA3}
\definecolor{notesgreen}{RGB}{0,162,0}
\definecolor{myp}{RGB}{197, 92, 212}
\definecolor{mygr}{HTML}{2C3338}
\definecolor{myred}{RGB}{127,0,0}
\definecolor{myyellow}{RGB}{169,121,69}
\definecolor{myexercisebg}{HTML}{F2FBF8}
\definecolor{myexercisefg}{HTML}{88D6D1}
\definecolor{myblue}{RGB}{0,82,155}

\usepackage{forest}
\usepackage{adjustbox}



%%%%%%%%%%%%%%%%%%%%%%%%%%%%
% chapter format
%%%%%%%%%%%%%%%%%%%%%%%%%%%%

\titleformat{\chapter}[display]
  {\normalfont\bfseries\color{doc!60}}
  {\filleft%
    \begin{tikzpicture}
    \node[
      outer sep=-4pt,
      text width=0.75cm,
      minimum height=1cm,
      fill=doc!60,
      font=\color{white}\fontsize{20}{25}\selectfont\fontfamily{lmss},
      align=center
      ] (num) {\thechapter};
    \node[
      rotate=90,
      anchor=south,
      font=\color{black}\normalsize\fontfamily{lmss}
      ] at ([xshift=-5pt]num.west) {\textls[180]{\textsc{}}};  
    \end{tikzpicture}%
  }
  {10pt}
  {\titlerule[2.5pt]\vskip1.5pt\titlerule\vskip4pt\large\bfseries\sc}
\titlespacing*{\chapter}{0pt}{*0}{*0}

\makeatletter
\patchcmd{\chapter}{\if@openright\cleardoublepage\else\clearpage\fi}{}{}{}
\makeatother




%%%%%%%%%%%%%%%%%%%%%%%%%%%%
% TCOLORBOX SETUPS
%%%%%%%%%%%%%%%%%%%%%%%%%%%%

\setlength{\parindent}{1cm}
%================================
% THEOREM BOX
%================================

\tcbuselibrary{theorems,skins,hooks}
\newtcbtheorem[number within=section]{Theorem}{Theorem}
{%
	enhanced,
	breakable,
	colback = mytheorembg,
	frame hidden,
	boxrule = 0sp,
	borderline west = {2pt}{0pt}{mytheoremfr},
	sharp corners,
	detach title,
	before upper = \tcbtitle\par\smallskip,
	coltitle = mytheoremfr,
	fonttitle = \bfseries\sffamily,
	description font = \mdseries,
	separator sign none,
	segmentation style={solid, mytheoremfr},
}
{th}

\tcbuselibrary{theorems,skins,hooks}
\newtcbtheorem[number within=chapter]{theorem}{Theorem}
{%
	enhanced,
	breakable,
	colback = mytheorembg,
	frame hidden,
	boxrule = 0sp,
	borderline west = {2pt}{0pt}{mytheoremfr},
	sharp corners,
	detach title,
	before upper = \tcbtitle\par\smallskip,
	coltitle = mytheoremfr,
	fonttitle = \bfseries\sffamily,
	description font = \mdseries,
	separator sign none,
	segmentation style={solid, mytheoremfr},
}
{th}


\tcbuselibrary{theorems,skins,hooks}
\newtcolorbox{Theoremcon}
{%
	enhanced
	,breakable
	,colback = mytheorembg
	,frame hidden
	,boxrule = 0sp
	,borderline west = {2pt}{0pt}{mytheoremfr}
	,sharp corners
	,description font = \mdseries
	,separator sign none
}

%================================
% Preuve
%================================

% Crée un environnement "Preuve" numéroté en fonction du document
\tcbuselibrary{theorems,skins,hooks}
\newtcbtheorem[number within=chapter]{Preuve}{Preuve}
{
	enhanced,
	breakable,
	colback=white,
	frame hidden,
	boxrule = 0sp,
	borderline west = {2pt}{0pt}{mytheoremfr},
	sharp corners,
	detach title,
	before upper = \tcbtitle\par\smallskip,
	coltitle = mytheoremfr,
	description font=\fontfamily{lmss}\selectfont,
	fonttitle=\fontfamily{lmss}\selectfont\bfseries,
	separator sign none,
	segmentation style={solid, mytheoremfr},
}
{th}



%================================
% Corollery
%================================
\tcbuselibrary{theorems,skins,hooks}
\newtcbtheorem[number within=section]{Corollary}{Corollaire}
{%
	enhanced
	,breakable
	,colback = myp!10
	,frame hidden
	,boxrule = 0sp
	,borderline west = {2pt}{0pt}{myp!85!black}
	,sharp corners
	,detach title
	,before upper = \tcbtitle\par\smallskip
	,coltitle = myp!85!black
	,fonttitle = \bfseries\sffamily
	,description font = \mdseries
	,separator sign none
	,segmentation style={solid, myp!85!black}
}
{th}
\tcbuselibrary{theorems,skins,hooks}
\newtcbtheorem[number within=chapter]{corollary}{Corollaire}
{%
	enhanced
	,breakable
	,colback = myp!10
	,frame hidden
	,boxrule = 0sp
	,borderline west = {2pt}{0pt}{myp!85!black}
	,sharp corners
	,detach title
	,before upper = \tcbtitle\par\smallskip
	,coltitle = myp!85!black
	,fonttitle = \bfseries\sffamily
	,description font = \mdseries
	,separator sign none
	,segmentation style={solid, myp!85!black}
}
{th}


%================================
% LENMA
%================================

\tcbuselibrary{theorems,skins,hooks}
\newtcbtheorem[number within=section]{Lenma}{Lemme}
{%
	enhanced,
	breakable,
	colback = mylenmabg,
	frame hidden,
	boxrule = 0sp,
	borderline west = {2pt}{0pt}{mylenmafr},
	sharp corners,
	detach title,
	before upper = \tcbtitle\par\smallskip,
	coltitle = mylenmafr,
	fonttitle = \bfseries\sffamily,
	description font = \mdseries,
	separator sign none,
	segmentation style={solid, mylenmafr},
}
{th}

\tcbuselibrary{theorems,skins,hooks}
\newtcbtheorem[number within=chapter]{lenma}{Lemme}
{%
	enhanced,
	breakable,
	colback = mylenmabg,
	frame hidden,
	boxrule = 0sp,
	borderline west = {2pt}{0pt}{mylenmafr},
	sharp corners,
	detach title,
	before upper = \tcbtitle\par\smallskip,
	coltitle = mylenmafr,
	fonttitle = \bfseries\sffamily,
	description font = \mdseries,
	separator sign none,
	segmentation style={solid, mylenmafr},
}
{th}


%================================
% PROPOSITION
%================================

\tcbuselibrary{theorems,skins,hooks}
\newtcbtheorem[number within=section]{Prop}{Proposition}
{%
	enhanced,
	breakable,
	colback = mypropbg,
	frame hidden,
	boxrule = 0sp,
	borderline west = {2pt}{0pt}{mypropfr},
	sharp corners,
	detach title,
	before upper = \tcbtitle\par\smallskip,
	coltitle = mypropfr,
	fonttitle = \bfseries\sffamily,
	description font = \mdseries,
	separator sign none,
	segmentation style={solid, mypropfr},
}
{th}

\tcbuselibrary{theorems,skins,hooks}
\newtcbtheorem[number within=chapter]{prop}{Proposition}
{%
	enhanced,
	breakable,
	colback = mypropbg,
	frame hidden,
	boxrule = 0sp,
	borderline west = {2pt}{0pt}{mypropfr},
	sharp corners,
	detach title,
	before upper = \tcbtitle\par\smallskip,
	coltitle = mypropfr,
	fonttitle = \bfseries\sffamily,
	description font = \mdseries,
	separator sign none,
	segmentation style={solid, mypropfr},
}
{th}


%================================
% CLAIM
%================================

\tcbuselibrary{theorems,skins,hooks}
\newtcbtheorem[number within=section]{claim}{Affirmation}
{%
	enhanced
	,breakable
	,colback = myg!10
	,frame hidden
	,boxrule = 0sp
	,borderline west = {2pt}{0pt}{myg}
	,sharp corners
	,detach title
	,before upper = \tcbtitle\par\smallskip
	,coltitle = myg!85!black
	,fonttitle = \bfseries\sffamily
	,description font = \mdseries
	,separator sign none
	,segmentation style={solid, myg!85!black}
}
{th}



%================================
% Exercise
%================================

\tcbuselibrary{theorems,skins,hooks}
\newtcbtheorem[number within=section]{Exercise}{Exercice}
{%
	enhanced,
	breakable,
	colback = myexercisebg,
	frame hidden,
	boxrule = 0sp,
	borderline west = {2pt}{0pt}{myexercisefg},
	sharp corners,
	detach title,
	before upper = \tcbtitle\par\smallskip,
	coltitle = myexercisefg,
	fonttitle = \bfseries\sffamily,
	description font = \mdseries,
	separator sign none,
	segmentation style={solid, myexercisefg},
}
{th}

\tcbuselibrary{theorems,skins,hooks}
\newtcbtheorem[number within=chapter]{exercise}{Exercice}
{%
	enhanced,
	breakable,
	colback = myexercisebg,
	frame hidden,
	boxrule = 0sp,
	borderline west = {2pt}{0pt}{myexercisefg},
	sharp corners,
	detach title,
	before upper = \tcbtitle\par\smallskip,
	coltitle = myexercisefg,
	fonttitle = \bfseries\sffamily,
	description font = \mdseries,
	separator sign none,
	segmentation style={solid, myexercisefg},
}
{th}

%================================
% EXAMPLE BOX
%================================

\newtcbtheorem[number within=section]{Example}{Example}
{%
	colback = myexamplebg
	,breakable
	,colframe = myexamplefr
	,coltitle = myexampleti
	,boxrule = 1pt
	,sharp corners
	,detach title
	,before upper=\tcbtitle\par\smallskip
	,fonttitle = \bfseries
	,description font = \mdseries
	,separator sign none
	,description delimiters parenthesis
}
{ex}

\newtcbtheorem[number within=chapter]{example}{Example}
{%
	colback = myexamplebg
	,breakable
	,colframe = myexamplefr
	,coltitle = myexampleti
	,boxrule = 1pt
	,sharp corners
	,detach title
	,before upper=\tcbtitle\par\smallskip
	,fonttitle = \bfseries
	,description font = \mdseries
	,separator sign none
	,description delimiters parenthesis
}
{ex}

%================================
% DEFINITION BOX
%================================

\newtcbtheorem[number within=chapter]{Definition}{Définition}{enhanced,
	before skip=2mm,after skip=2mm, colback=red!5,colframe=red!80!black,boxrule=0.5mm,
	attach boxed title to top left={xshift=1cm,yshift*=1mm-\tcboxedtitleheight}, varwidth boxed title*=-3cm,
	boxed title style={frame code={
			\path[fill=tcbcolback!10!red]
			([yshift=-1mm,xshift=-1mm]frame.north west)
			arc[start angle=0,end angle=180,radius=1mm]
			([yshift=-1mm,xshift=1mm]frame.north east)
			arc[start angle=180,end angle=0,radius=1mm];
			\path[left color=tcbcolback!10!myred,right color=tcbcolback!10!myred,
			middle color=tcbcolback!60!myred]
			([xshift=-2mm]frame.north west) -- ([xshift=2mm]frame.north east)
			[rounded corners=1mm]-- ([xshift=1mm,yshift=-1mm]frame.north east)
			-- (frame.south east) -- (frame.south west)
			-- ([xshift=-1mm,yshift=-1mm]frame.north west)
			[sharp corners]-- cycle;
		},interior engine=empty,
	},
	fonttitle=\bfseries,
	title={#2},#1}{def}
\newtcbtheorem[number within=chapter]{definition}{Definition}{enhanced,
	before skip=2mm,after skip=2mm, colback=red!5,colframe=red!80!black,boxrule=0.5mm,
	attach boxed title to top left={xshift=1cm,yshift*=1mm-\tcboxedtitleheight}, varwidth boxed title*=-3cm,
	boxed title style={frame code={
					\path[fill=tcbcolback]
					([yshift=-1mm,xshift=-1mm]frame.north west)
					arc[start angle=0,end angle=180,radius=1mm]
					([yshift=-1mm,xshift=1mm]frame.north east)
					arc[start angle=180,end angle=0,radius=1mm];
					\path[left color=tcbcolback!60!black,right color=tcbcolback!60!black,
						middle color=tcbcolback!80!black]
					([xshift=-2mm]frame.north west) -- ([xshift=2mm]frame.north east)
					[rounded corners=1mm]-- ([xshift=1mm,yshift=-1mm]frame.north east)
					-- (frame.south east) -- (frame.south west)
					-- ([xshift=-1mm,yshift=-1mm]frame.north west)
					[sharp corners]-- cycle;
				},interior engine=empty,
		},
	fonttitle=\bfseries,
	title={#2},#1}{def}


    \newtcbtheorem{Definitionx}{Définition}
    {
    enhanced,
    breakable,
    colback=red!5,
      before upper=\tcbtitle\par\Hugeskip,
    frame hidden,
    boxrule = 0sp,
    borderline west = {2pt}{0pt}{red!80!black},
    sharp corners,
    detach title,
    before upper = \tcbtitle\par\smallskip,
    coltitle = red!80!black,
    description font=\mdseries\fontfamily{lmss}\selectfont,
    fonttitle=\fontfamily{lmss}\selectfont\bfseries,
    fontlower=\fontfamily{cmr}\selectfont,
      separator sign none,
    segmentation style={solid, mytheoremfr},
    }
    {th}



%================================
% Solution BOX
%================================

\makeatletter
\newtcbtheorem{question}{Question}{enhanced,
	breakable,
	colback=white,
	colframe=myb!80!black,
	attach boxed title to top left={yshift*=-\tcboxedtitleheight},
	fonttitle=\bfseries,
	title={#2},
	boxed title size=title,
	boxed title style={%
			sharp corners,
			rounded corners=northwest,
			colback=tcbcolframe,
			boxrule=0pt,
		},
	underlay boxed title={%
			\path[fill=tcbcolframe] (title.south west)--(title.south east)
			to[out=0, in=180] ([xshift=5mm]title.east)--
			(title.center-|frame.east)
			[rounded corners=\kvtcb@arc] |-
			(frame.north) -| cycle;
		},
	#1
}{def}
\makeatother

%================================
% SOLUTION BOX
%================================

\makeatletter
\newtcolorbox{solution}{enhanced,
	breakable,
	colback=white,
	colframe=myg!80!black,
	attach boxed title to top left={yshift*=-\tcboxedtitleheight},
	title=Solution,
	boxed title size=title,
	boxed title style={%
			sharp corners,
			rounded corners=northwest,
			colback=tcbcolframe,
			boxrule=0pt,
		},
	underlay boxed title={%
			\path[fill=tcbcolframe] (title.south west)--(title.south east)
			to[out=0, in=180] ([xshift=5mm]title.east)--
			(title.center-|frame.east)
			[rounded corners=\kvtcb@arc] |-
			(frame.north) -| cycle;
		},
}
\makeatother

%================================
% Question BOX
%================================

\makeatletter
\newtcbtheorem{qstion}{Question}{enhanced,
	breakable,
	colback=white,
	colframe=mygr,
	attach boxed title to top left={yshift*=-\tcboxedtitleheight},
	fonttitle=\bfseries,
	title={#2},
	boxed title size=title,
	boxed title style={%
			sharp corners,
			rounded corners=northwest,
			colback=tcbcolframe,
			boxrule=0pt,
		},
	underlay boxed title={%
			\path[fill=tcbcolframe] (title.south west)--(title.south east)
			to[out=0, in=180] ([xshift=5mm]title.east)--
			(title.center-|frame.east)
			[rounded corners=\kvtcb@arc] |-
			(frame.north) -| cycle;
		},
	#1
}{def}
\makeatother

\newtcbtheorem[number within=chapter]{wconc}{Wrong Concept}{
	breakable,
	enhanced,
	colback=white,
	colframe=myr,
	arc=0pt,
	outer arc=0pt,
	fonttitle=\bfseries\sffamily\large,
	colbacktitle=myr,
	attach boxed title to top left={},
	boxed title style={
			enhanced,
			skin=enhancedfirst jigsaw,
			arc=3pt,
			bottom=0pt,
			interior style={fill=myr}
		},
	#1
}{def}


%%%%%%%%%%%%%%%%%%%%%%%%%%%%%%%%%%%%%%%%%%%%%%%%%%%%%%%%%%%%%%%%%%%%%%%%%%%%%%%%%%%%%%%%%%%%%%%%%
%                                Environnement Explication
%%%%%%%%%%%%%%%%%%%%%%%%%%%%%%%%%%%%%%%%%%%%%%%%%%%%%%%%%%%%%%%%%%%%%%%%%%%%%%%%%%%%%%%%%%%%%%%%%
\newtcbtheorem{Explication}{Explication}
{
	enhanced,
	breakable,
	colback=white,
	frame hidden,
	boxrule = 0sp,
	borderline west = {2pt}{0pt}{mytheoremfr},
	sharp corners,
	detach title,
	before upper = \tcbtitle\par\smallskip,
	coltitle = mytheoremfr,
	description font=\fontfamily{lmss}\selectfont,
	fonttitle=\fontfamily{lmss}\selectfont\bfseries,
	separator sign none,
	segmentation style={solid, mytheoremfr},
}
{th}



%%%%%%%%%%%%%%%%%%%%%%%%%%%%%%%%%%%%%%%%%%%%%%%%%%%%%%%%%%%%%%%%%%%%%%%%%%%%%%%%%%%%%%%%%%%%%%%%%
%                                Environnement EExample
%%%%%%%%%%%%%%%%%%%%%%%%%%%%%%%%%%%%%%%%%%%%%%%%%%%%%%%%%%%%%%%%%%%%%%%%%%%%%%%%%%%%%%%%%%%%%%%%%
% Crée un environnement "EExample" numéroté en fonction du document
\newtcbtheorem{EExample}{Exemple}
{
	enhanced,
	breakable,
	colback=white,
	frame hidden,
	boxrule = 0sp,
	borderline west = {2pt}{0pt}{myb},
	sharp corners,
	detach title,
	before upper = \tcbtitle\par\smallskip,
	coltitle = myb,
	description font=\mdseries\fontfamily{lmss}\selectfont,
	fonttitle=\fontfamily{lmss}\selectfont\bfseries,
	separator sign none,
	segmentation style={solid, mytheoremfr},
}
{th}

%%%%%%%%%%%%%%%%%%%%%%%%%%%%%%%%%%%%%%%%%%%%%%%%%%%%%%%%%%%%%%%%%%%%%%%%%%%%%%%%%%%%%%%%%%%%%%%%%
%                                Environnement Liste
%%%%%%%%%%%%%%%%%%%%%%%%%%%%%%%%%%%%%%%%%%%%%%%%%%%%%%%%%%%%%%%%%%%%%%%%%%%%%%%%%%%%%%%%%%%%%%%%%

% Pour créer un environnement "Liste" 

\tcbuselibrary{theorems,skins,hooks}
\newtcbtheorem[number within=chapter]{Liste}{Liste}
{%
	enhanced
	,breakable
	,colback = myp!10
	,frame hidden
	,boxrule = 0sp
	,borderline west = {2pt}{0pt}{myp!85!black}
	,sharp corners
	,detach title
	,before upper = \tcbtitle\par\smallskip
	,coltitle = myp!85!black
	,fonttitle = \bfseries\sffamily
	,description font = \mdseries
	,separator sign none
	,segmentation style={solid, myp!85!black}

}
{th}


%%%%%%%%%%%%%%%%%%%%%%%%%%%%%%%%%%%%%%%%%%%%%%%%%%%%%%%%%%%%%%%%%%%%%%%%%%%%%%%%%%%%%%%%%%%%%%%%%
%                                Environnement Syntaxe
%%%%%%%%%%%%%%%%%%%%%%%%%%%%%%%%%%%%%%%%%%%%%%%%%%%%%%%%%%%%%%%%%%%%%%%%%%%%%%%%%%%%%%%%%%%%%%%%%
\tcbuselibrary{theorems,skins,hooks}
\newtcbtheorem{Syntaxe}{Syntaxe.}
{%
	enhanced
	,breakable
	,colback = myp!10
	,frame hidden
	,boxrule = 0sp
	,borderline west = {2pt}{0pt}{myp!85!black}
	,sharp corners
	,detach title
	,before upper = \tcbtitle\par\smallskip
	,coltitle = myp!85!black
	,fonttitle = \bfseries\fontfamily{lmss}\selectfont 
	,description font = \mdseries\fontfamily{lmss}\selectfont 
	,separator sign none
	,segmentation style={solid, myp!85!black}
}
{th}



%%%%%%%%%%%%%%%%%%%%%%%%%%%%%%%%%%%%%%%%%%%%%%%%%%%%%%%%%%%%%%%%%%%%%%%%%%%%%%%%%%%%%%%%%%%%%%%%%
%                                Environnement Concept
%%%%%%%%%%%%%%%%%%%%%%%%%%%%%%%%%%%%%%%%%%%%%%%%%%%%%%%%%%%%%%%%%%%%%%%%%%%%%%%%%%%%%%%%%%%%%%%%%
% Crée un environnement "Concept" numéroté en fonction du document
\tcbuselibrary{theorems,skins,hooks}
\newtcbtheorem{Concept}{Concept.}
{
	enhanced,
	breakable,
	colback=mylenmabg,
	frame hidden,
	boxrule = 0sp,
	borderline west = {2pt}{0pt}{mylenmafr},
	sharp corners,
	detach title,
	before upper = \tcbtitle\par\smallskip,
	coltitle = mylenmafr,
	description font=\mdseries\fontfamily{lmss}\selectfont,
	fonttitle=\fontfamily{lmss}\selectfont\bfseries,
	separator sign none,
	segmentation style={solid, mytheoremfr},
}
{th}



%%%%%%%%%%%%%%%%%%%%%%%%%%%%%%%%%%%%%%%%%%%%%%%%%%%%%%%%%%%%%%%%%%%%%%%%%%%%%%%%%%%%%%%%%%%%%%%%%
%                                Environnement codeEx
%%%%%%%%%%%%%%%%%%%%%%%%%%%%%%%%%%%%%%%%%%%%%%%%%%%%%%%%%%%%%%%%%%%%%%%%%%%%%%%%%%%%%%%%%%%%%%%%%
% Crée un environnement "codeEx" numéroté en fonction du document
\tcbuselibrary{theorems,skins,hooks}
\newtcbtheorem{codeEx}{Exemple}
{
	enhanced,
	breakable,
	colback=white,
	frame hidden,
	boxrule = 0sp,
	borderline west = {2pt}{0pt}{gruvbox-bg},
	sharp corners,
	detach title,
	before upper = \tcbtitle\par\smallskip,
	coltitle = gruvbox-bg,
	description font=\md:wqseries\fontfamily{lmss}\selectfont,
	fonttitle=\fontfamily{lmss}\selectfont\bfseries,
	separator sign none,
	segmentation style={solid, mytheoremfr},
}
{th}



%%%%%%%%%%%%%%%%%%%%%%%%%%%%%%%%%%%%%%%%%%%%%%%%%%%%%%%%%%%%%%%%%%%%%%%%%%%%%%%%%%%%%%%%%%%%%%%%%
%                                Environnement Remarque
%%%%%%%%%%%%%%%%%%%%%%%%%%%%%%%%%%%%%%%%%%%%%%%%%%%%%%%%%%%%%%%%%%%%%%%%%%%%%%%%%%%%%%%%%%%%%%%%%
% Crée un environnement "Remarque" numéroté en fonction du document
\tcbuselibrary{theorems,skins,hooks}
\newtcbtheorem{codeRem}{Remarque}
{
	enhanced,
	breakable,
	colback=white,
	frame hidden,
	boxrule = 0sp,
	borderline west = {2pt}{0pt}{gruvbox-bg},
	sharp corners,
	detach title,
	before upper = \tcbtitle\par\smallskip,
	coltitle = gruvbox-bg,
	description font=\mdseries\fontfamily{lmss}\selectfont,
	fonttitle=\fontfamily{lmss}\selectfont\bfseries,
	separator sign none,
	segmentation style={solid, mytheoremfr},
}
{th}


%%%%%%%%%%%%%%%%%%%%%%%%%%%%%%%%%%%%%%%%%%%%%%%%%%%%%%%%%%%%%%%%%%%%%%%%%%%%%%%%%%%%%%%%%%%%%%%%%
%                                Environnement Identité
%%%%%%%%%%%%%%%%%%%%%%%%%%%%%%%%%%%%%%%%%%%%%%%%%%%%%%%%%%%%%%%%%%%%%%%%%%%%%%%%%%%%%%%%%%%%%%%%%
\tcbuselibrary{theorems,skins,hooks}
\newtcbtheorem{Identite}{Identité}
{
	enhanced,
	breakable,
	colback=white,
  before upper=\tcbtitle\par\Hugeskip,
	frame hidden,
	boxrule = 0sp,
	borderline west = {2pt}{0pt}{gruvbox-bg},
	sharp corners,
	detach title,
	before upper = \tcbtitle\par\smallskip,
	coltitle = gruvbox-bg,
	description font=\mdseries\fontfamily{lmss}\selectfont,
	fonttitle=\fontfamily{lmss}\selectfont\bfseries,
	fontlower=\fontfamily{cmr}\selectfont,
  separator sign none,
	segmentation style={solid, mytheoremfr},
}
{th}



%%%%%%%%%%%%%%%%%%%%%%%%%%%%%%%%%%%%%%%%%%%%%%%%%%%%%%%%%%%%%%%%%%%%%%%%%%%%%%%%%%%%%%%%%%%%%%%%%
%                                Environnement Exercice 
%%%%%%%%%%%%%%%%%%%%%%%%%%%%%%%%%%%%%%%%%%%%%%%%%%%%%%%%%%%%%%%%%%%%%%%%%%%%%%%%%%%%%%%%%%%%%%%%%
\tcbuselibrary{theorems,skins,hooks}
\newtcbtheorem{Exercice}{Exercice}
{
	enhanced,
	breakable,
	colback=white,
  before upper=\tcbtitle\par\Hugeskip,
	frame hidden,
	boxrule = 0sp,
	borderline west = {2pt}{0pt}{gruvbox-green},
	sharp corners,
	detach title,
	before upper = \tcbtitle\par\smallskip,
	coltitle = gruvbox-green,
	description font=\mdseries\fontfamily{lmss}\selectfont,
	fonttitle=\fontfamily{lmss}\selectfont\bfseries,
	fontlower=\fontfamily{cmr}\selectfont,
  separator sign none,
	segmentation style={solid, mytheoremfr},
}
{th}


%%%%%%%%%%%%%%%%%%%%%%%%%%%%%%%%%%%%%%%%%%%%%%%%%%%%%%%%%%%%%%%%%%%%%%%%%%%%%%%%%%%%%%%%%%%%%%%%%
%                                Environnement Réponse
%%%%%%%%%%%%%%%%%%%%%%%%%%%%%%%%%%%%%%%%%%%%%%%%%%%%%%%%%%%%%%%%%%%%%%%%%%%%%%%%%%%%%%%%%%%%%%%%%
% Crée un environnement "Réponse" numéroté en fonction du document
\tcbuselibrary{theorems,skins,hooks}
\newtcbtheorem{Reponse}{Réponse}
{
	enhanced,
	breakable,
	colback=white,
	frame hidden,
	boxrule = 0sp,
	borderline west = {2pt}{0pt}{mytheoremfr},
	sharp corners,
	detach title,
	before upper = \tcbtitle\par\smallskip,
	coltitle = mytheoremfr,
	description font=\fontfamily{lmss}\selectfont,
	fonttitle=\fontfamily{lmss}\selectfont\bfseries,
	separator sign none,
	segmentation style={solid, mytheoremfr},
}
{th}

\newtcbtheorem{Remarque}{Remarque.}
{
	enhanced,
	breakable,
	colback=white,
	frame hidden,
	boxrule = 0sp,
	borderline west = {2pt}{0pt}{myb},
	sharp corners,
	detach title,
	before upper = \tcbtitle\par\smallskip,
	coltitle = myb,
	description font=\mdseries\fontfamily{lmss}\selectfont,
	fonttitle=\fontfamily{lmss}\selectfont\bfseries,
	separator sign none,
	segmentation style={solid, mytheoremfr},
}
{th}





%================================
% NOTE BOX
%================================

\usetikzlibrary{arrows,calc,shadows.blur}
\usetikzlibrary {arrows.meta,backgrounds,fit,positioning,petri}
\tcbuselibrary{skins}
\newtcolorbox{note}[1][]{%
	enhanced jigsaw,
	colback=gray!20!white,%
	colframe=gray!80!black,
	size=small,
	boxrule=1pt,
	title=\textbf{Note:-},
	halign title=flush center,
	coltitle=black,
	breakable,
	drop shadow=black!50!white,
	attach boxed title to top left={xshift=1cm,yshift=-\tcboxedtitleheight/2,yshifttext=-\tcboxedtitleheight/2},
	minipage boxed title=1.5cm,
	boxed title style={%
			colback=white,
			size=fbox,
			boxrule=1pt,
			boxsep=2pt,
			underlay={%
					\coordinate (dotA) at ($(interior.west) + (-0.5pt,0)$);
					\coordinate (dotB) at ($(interior.east) + (0.5pt,0)$);
					\begin{scope}
						\clip (interior.north west) rectangle ([xshift=3ex]interior.east);
						\filldraw [white, blur shadow={shadow opacity=60, shadow yshift=-.75ex}, rounded corners=2pt] (interior.north west) rectangle (interior.south east);
					\end{scope}
					\begin{scope}[gray!80!black]
						\fill (dotA) circle (2pt);
						\fill (dotB) circle (2pt);
					\end{scope}
				},
		},
	#1,
}

%%%%%%%%%%%%%%%%%%%%%%%%%%%%%%
% SELF MADE COMMANDS
%%%%%%%%%%%%%%%%%%%%%%%%%%%%%%


\newcommand{\thm}[2]{\begin{Theorem}{#1}{}#2\end{Theorem}}
\newcommand{\cor}[2]{\begin{Corollary}{#1}{}#2\end{Corollary}}
\newcommand{\mlenma}[2]{\begin{Lenma}{#1}{}#2\end{Lenma}}
\newcommand{\mprop}[2]{\begin{Prop}{#1}{}#2\end{Prop}}
\newcommand{\clm}[3]{\begin{claim}{#1}{#2}#3\end{claim}}
\newcommand{\wc}[2]{\begin{wconc}{#1}{}\setlength{\parindent}{1cm}#2\end{wconc}}
\newcommand{\thmcon}[1]{\begin{Theoremcon}{#1}\end{Theoremcon}}
\newcommand{\ex}[2]{\begin{Example}{#1}{}#2\end{Example}}
\newcommand{\dfn}[2]{\begin{Definition}[colbacktitle=red!75!black]{#1}{}#2\end{Definition}}
\newcommand{\dfnc}[2]{\begin{definition}[colbacktitle=red!75!black]{#1}{}#2\end{definition}}
\newcommand{\qs}[2]{\begin{question}{#1}{}#2\end{question}}
\newcommand{\pf}[2]{\begin{myproof}[#1]#2\end{myproof}}
\newcommand{\nt}[1]{\begin{note}#1\end{note}}

\newcommand*\circled[1]{\tikz[baseline=(char.base)]{
		\node[shape=circle,draw,inner sep=1pt] (char) {#1};}}
\newcommand\getcurrentref[1]{%
	\ifnumequal{\value{#1}}{0}
	{??}
	{\the\value{#1}}%
}
\newcommand{\getCurrentSectionNumber}{\getcurrentref{section}}
\newenvironment{myproof}[1][\proofname]{%
	\proof[\bfseries #1: ]%
}{\endproof}

\newcommand{\mclm}[2]{\begin{myclaim}[#1]#2\end{myclaim}}
\newenvironment{myclaim}[1][\claimname]{\proof[\bfseries #1: ]}{}

\newcounter{mylabelcounter}

\makeatletter
\newcommand{\setword}[2]{%
	\phantomsection
	#1\def\@currentlabel{\unexpanded{#1}}\label{#2}%
}
\makeatother




\tikzset{
	symbol/.style={
			draw=none,
			every to/.append style={
					edge node={node [sloped, allow upside down, auto=false]{$#1$}}}
		}
}


% deliminators
% Manually define paired delimiters
\newcommand{\abs}[1]{\lvert#1\rvert}
\newcommand{\norm}[1]{\lVert#1\rVert}
\newcommand{\ceil}[1]{\lceil#1\rceil}
\newcommand{\floor}[1]{\lfloor#1\rfloor}
\newcommand{\round}[1]{\lfloor#1\rceil}

%
\newsavebox\diffdbox
\newcommand{\slantedromand}{{\mathpalette\makesl{d}}}
\newcommand{\makesl}[2]{%
\begingroup
\sbox{\diffdbox}{$\mathsurround=0pt#1\mathrm{#2}$}%
\pdfsave
\pdfsetmatrix{1 0 0.2 1}%
\rlap{\usebox{\diffdbox}}%
\pdfrestore
\hskip\wd\diffdbox
\endgroup
}
\newcommand{\dd}[1][]{\ensuremath{\mathop{}\!\ifstrempty{#1}{%
\slantedromand\@ifnextchar^{\hspace{0.2ex}}{\hspace{0.1ex}}}%
{\slantedromand\hspace{0.2ex}^{#1}}}}
\ProvideDocumentCommand\dv{o m g}{%
  \ensuremath{%
    \IfValueTF{#3}{%
      \IfNoValueTF{#1}{%
        \frac{\dd #2}{\dd #3}%
      }{%
        \frac{\dd^{#1} #2}{\dd #3^{#1}}%
      }%
    }{%
      \IfNoValueTF{#1}{%
        \frac{\dd}{\dd #2}%
      }{%
        \frac{\dd^{#1}}{\dd #2^{#1}}%
      }%
    }%
  }%
}
\providecommand*{\pdv}[3][]{\frac{\partial^{#1}#2}{\partial#3^{#1}}}
%  - others
\DeclareMathOperator{\Lap}{\mathcal{L}}
\DeclareMathOperator{\Var}{Var} % varience
\DeclareMathOperator{\Cov}{Cov} % covarience
\DeclareMathOperator{\E}{E} % expected

% Since the amsthm package isn't loaded

% I prefer the slanted \leq
\let\oldleq\leq % save them in case they're every wanted
\let\oldgeq\geq
\renewcommand{\leq}{\leqslant}
\renewcommand{\geq}{\geqslant}

% % redefine matrix env to allow for alignment, use r as default
% \renewcommand*\env@matrix[1][r]{\hskip -\arraycolsep
%     \let\@ifnextchar\new@ifnextchar
%     \array{*\c@MaxMatrixCols #1}}


%\usepackage{framed}
%\usepackage{titletoc}
%\usepackage{etoolbox}
%\usepackage{lmodern}


%\patchcmd{\tableofcontents}{\contentsname}{\sffamily\contentsname}{}{}

%\renewenvironment{leftbar}
%{\def\FrameCommand{\hspace{6em}%
%		{\color{myyellow}\vrule width 2pt depth 6pt}\hspace{1em}}%
%	\MakeFramed{\parshape 1 0cm \dimexpr\textwidth-6em\relax\FrameRestore}\vskip2pt%
%}
%{\endMakeFramed}

%\titlecontents{chapter}
%[0em]{\vspace*{2\baselineskip}}
%{\parbox{4.5em}{%
%		\hfill\Huge\sffamily\bfseries\color{myred}\thecontentspage}%
%	\vspace*{-2.3\baselineskip}\leftbar\textsc{\small\chaptername~\thecontentslabel}\\\sffamily}
%{}{\endleftbar}
%\titlecontents{section}
%[8.4em]
%{\sffamily\contentslabel{3em}}{}{}
%{\hspace{0.5em}\nobreak\itshape\color{myred}\contentspage}
%\titlecontents{subsection}
%[8.4em]
%{\sffamily\contentslabel{3em}}{}{}  
%{\hspace{0.5em}\nobreak\itshape\color{myred}\contentspage}



%%%%%%%%%%%%%%%%%%%%%%%%%%%%%%%%%%%%%%%%%%%
% TABLE OF CONTENTS
%%%%%%%%%%%%%%%%%%%%%%%%%%%%%%%%%%%%%%%%%%%

\usepackage{tikz}
\usetikzlibrary{automata, positioning}
\definecolor{doc}{RGB}{0,60,110}
\usepackage{titletoc}
\contentsmargin{0cm}
\titlecontents{chapter}[4.9pc]
{\addvspace{40pt}%
	\begin{tikzpicture}[remember picture, overlay]%
		\draw[fill=doc!60,draw=doc!60] (-7,-.1) rectangle (-0.9,.5);%
		\pgftext[left,x=-3.5cm,y=0.2cm]{\color{white}\Large\sc\bfseries Section \ \thecontentslabel};%
	\end{tikzpicture}\color{doc!60}\large\sc\bfseries}%
{}
{}
{\;\titlerule\;\large\sc\bfseries Page \thecontentspage
	\begin{tikzpicture}[remember picture, overlay]
		\draw[fill=doc!60,draw=doc!60] (2pt,0) rectangle (4,0.1pt);
	\end{tikzpicture}}%
\titlecontents{section}[3.7pc]
{\addvspace{2pt}}
{\contentslabel[\thecontentslabel]{2pc}}
{}
{\hfill\small \thecontentspage}
[]
\titlecontents*{subsection}[3.7pc]
{\addvspace{-1pt}\small}
{}
{}
{\ --- \small\thecontentspage}
[ \textbullet\ ][]

\makeatletter
\renewcommand{\tableofcontents}{%
	\chapter*{%
	  \vspace*{-20\p@}%
	  \begin{tikzpicture}[remember picture, overlay]%
		  \pgftext[right,x=15cm,y=0.2cm]{\color{doc!60}\Huge\sc\bfseries \contentsname};%
		  \draw[fill=doc!60,draw=doc!60] (13,-.75) rectangle (20,2);%
		  \clip (13,-.75) rectangle (20,1);
		  \pgftext[right,x=15cm,y=0.2cm]{\color{white}\Huge\sc\bfseries \contentsname};%
	  \end{tikzpicture}}%
	\@starttoc{toc}}
\makeatother






%From M275 "Topology" at SJSU
\newcommand{\id}{\mathrm{id}} % Identité
\newcommand{\taking}[1]{\xrightarrow{#1}} % Flèche avec annotation
\newcommand{\inv}{^{-1}} % Inverse

%From M170 "Introduction to Graph Theory" at SJSU
\DeclareMathOperator{\diam}{diam} % Diamètre
\DeclareMathOperator{\ord}{ord} % Ordre
\newcommand{\defeq}{\overset{\mathrm{def}}{=}} % Défini comme égal

%From the USAMO .tex files
\newcommand{\ts}{\textsuperscript} % Exposant
\newcommand{\dg}{^\circ} % Degré
\newcommand{\ii}{\item} % Item

% % From Math 55 and Math 145 at Harvard
% \newenvironment{subproof}[1][Proof]{%
% \begin{proof}[#1] \renewcommand{\qedsymbol}{$\blacksquare$}}%
% {\end{proof}}

\newcommand{\liff}{\leftrightarrow} % Si et seulement si
\newcommand{\lthen}{\rightarrow} % Implique
\newcommand{\opname}{\operatorname} % Opérateur générique
\newcommand{\surjto}{\twoheadrightarrow} % Flèche surjective
\newcommand{\injto}{\hookrightarrow} % Flèche injective
\newcommand{\On}{\mathrm{On}} % Ordinaux
\DeclareMathOperator{\img}{im} % Image
\DeclareMathOperator{\Img}{Im} % Image
\DeclareMathOperator{\coker}{coker} % Cokernel
\DeclareMathOperator{\Coker}{Coker} % Cokernel
\DeclareMathOperator{\Ker}{Ker} % Noyau
\DeclareMathOperator{\rank}{rank} % Rang
\DeclareMathOperator{\Spec}{Spec} % Spectre
\DeclareMathOperator{\Tr}{Tr} % Trace
\DeclareMathOperator{\pr}{pr} % Projection
\DeclareMathOperator{\ext}{ext} % Extension
\DeclareMathOperator{\pred}{pred} % Prédécesseur
\DeclareMathOperator{\dom}{dom} % Domaine
\DeclareMathOperator{\ran}{ran} % Image (range)
\DeclareMathOperator{\Hom}{Hom} % Homomorphisme
\DeclareMathOperator{\Mor}{Mor} % Morphismes
\DeclareMathOperator{\End}{End} % Endomorphisme

\newcommand{\eps}{\epsilon} % Épsilon
\newcommand{\veps}{\varepsilon} % Variance d'épsilon
\newcommand{\ol}{\overline} % Ligne au-dessus
\newcommand{\ul}{\underline} % Ligne en-dessous
\newcommand{\wt}{\widetilde} % Tilde large
\newcommand{\wh}{\widehat} % Chapeau large
\newcommand{\vocab}[1]{\textbf{\color{blue} #1}} % Texte en gras et bleu
\providecommand{\half}{\frac{1}{2}} % Fraction 1/2
\newcommand{\dang}{\measuredangle} % Angle dirigé
\newcommand{\ray}[1]{\overrightarrow{#1}} % Ray
\newcommand{\seg}[1]{\overline{#1}} % Segment
\newcommand{\arc}[1]{\wideparen{#1}} % Arc
\DeclareMathOperator{\cis}{cis} % cis
\DeclareMathOperator*{\lcm}{lcm} % Plus petit commun multiple
\DeclareMathOperator*{\argmin}{arg min} % Argument du minimum
\DeclareMathOperator*{\argmax}{arg max} % Argument du maximum
\newcommand{\cycsum}{\sum_{\mathrm{cyc}}} % Somme cyclique
\newcommand{\symsum}{\sum_{\mathrm{sym}}} % Somme symétrique
\newcommand{\cycprod}{\prod_{\mathrm{cyc}}} % Produit cyclique
\newcommand{\symprod}{\prod_{\mathrm{sym}}} % Produit symétrique
\newcommand{\Qed}{\begin{flushright}\qed\end{flushright}} % QED aligné à droite
\newcommand{\parinn}{\setlength{\parindent}{1cm}} % Indentation de paragraphe à 1 cm
\newcommand{\parinf}{\setlength{\parindent}{0cm}} % Pas d'indentation de paragraphe
% \newcommand{\norm}{\|\cdot\|} % Norme
\newcommand{\inorm}{\norm_{\infty}} % Norme infinie
\newcommand{\opensets}{\{V_{\alpha}\}_{\alpha\in I}} % Ensemble ouvert
\newcommand{\oset}{V_{\alpha}} % Ensemble ouvert V
\newcommand{\opset}[1]{V_{\alpha_{#1}}} % Ensemble ouvert V avec indice
\newcommand{\lub}{\text{lub}} % Plus petite borne supérieure
\newcommand{\del}[2]{\frac{\partial #1}{\partial #2}} % Dérivée partielle
\newcommand{\Del}[3]{\frac{\partial^{#1} #2}{\partial^{#1} #3}} % Dérivée partielle d'ordre élevé
\newcommand{\deld}[2]{\dfrac{\partial #1}{\partial #2}} % Dérivée partielle avec dfrac
\newcommand{\Deld}[3]{\dfrac{\partial^{#1} #2}{\partial^{#1} #3}} % Dérivée partielle d'ordre élevé avec dfrac
\newcommand{\lm}{\lambda} % Lambda
\newcommand{\uin}{\mathbin{\rotatebox[origin=c]{90}{$\in$}}} % Appartient, tourné de 90 degrés
\newcommand{\usubset}{\mathbin{\rotatebox[origin=c]{90}{$\subset$}}} % Sous-ensemble, tourné de 90 degrés
\newcommand{\lt}{\left} % Gauche
\newcommand{\rt}{\right} % Droite
\newcommand{\bs}[1]{\boldsymbol{#1}} % Symbole en gras
\newcommand{\exs}{\exists} % Il existe
\newcommand{\st}{\strut} % Strut
\newcommand{\dps}[1]{\displaystyle{#1}} % Disposition en ligne

\newcommand{\sol}{\setlength{\parindent}{0cm}\textbf{\textit{Solution:}}\setlength{\parindent}{1cm} } % Solution sans indentation initiale puis rétablie
\newcommand{\solve}[1]{\setlength{\parindent}{0cm}\textbf{\textit{Solution: }}\setlength{\parindent}{1cm}#1 \Qed}

\newcommand{\entoure}[1]{\fcolorbox{black}{gray!30}{\texttt{#1}}}

\renewcommand{\ttdefault}{cmtt}
\newcommand{\textttbf}[1]{\contour{yellow!45}{\texttt{#1}}}
\newcommand{\varitem}[3][black]{%
    \item [%
        \colorbox{#2}{\textcolor{#1}{\makebox(5.5,7){#3}}}%
    ]
}
% Allow you to do the non implication (implication barred)
\newcommand{\notimplies}{%
  \mathrel{{\ooalign{\hidewidth$\not\phantom{=}$\hidewidth\cr$\implies$}}}}


\newcommand*{\authorimg}[1]%
    { \raisebox{-1\baselineskip}{\includegraphics[width=\imagesize]{#1}}}
\newlength\imagesize 

% Things Lie
\newcommand{\kb}{\mathfrak b}
\newcommand{\kg}{\mathfrak g}
\newcommand{\kh}{\mathfrak h}
\newcommand{\kn}{\mathfrak n}
\newcommand{\ku}{\mathfrak u}
\newcommand{\kz}{\mathfrak z}
\DeclareMathOperator{\Ext}{Ext} % Ext functor
\DeclareMathOperator{\Tor}{Tor} % Tor functor
\newcommand{\gl}{\opname{\mathfrak{gl}}} % frak gl group
\renewcommand{\sl}{\opname{\mathfrak{sl}}} % frak sl group chktex 6

% More script letters etc.
\newcommand{\SA}{\mathcal A}
\newcommand{\SB}{\mathcal B}
\newcommand{\SC}{\mathcal C}
\newcommand{\SF}{\mathcal F}
\newcommand{\SG}{\mathcal G}
\newcommand{\SH}{\mathcal H}
\newcommand{\OO}{\mathcal O}

\newcommand{\SCA}{\mathscr A}
\newcommand{\SCB}{\mathscr B}
\newcommand{\SCC}{\mathscr C}
\newcommand{\SCD}{\mathscr D}
\newcommand{\SCE}{\mathscr E}
\newcommand{\SCF}{\mathscr F}
\newcommand{\SCG}{\mathscr G}
\newcommand{\SCH}{\mathscr H}

% Mathfrak primes
\newcommand{\km}{\mathfrak m}
\newcommand{\kp}{\mathfrak p}
\newcommand{\kq}{\mathfrak q}

% number sets
\newcommand{\RR}[1][]{\ensuremath{\ifstrempty{#1}{\mathbb{R}}{\mathbb{R}^{#1}}}}
\newcommand{\NN}[1][]{\ensuremath{\ifstrempty{#1}{\mathbb{N}}{\mathbb{N}^{#1}}}}
\newcommand{\ZZ}[1][]{\ensuremath{\ifstrempty{#1}{\mathbb{Z}}{\mathbb{Z}^{#1}}}}
\newcommand{\QQ}[1][]{\ensuremath{\ifstrempty{#1}{\mathbb{Q}}{\mathbb{Q}^{#1}}}}
\newcommand{\CC}[1][]{\ensuremath{\ifstrempty{#1}{\mathbb{C}}{\mathbb{C}^{#1}}}}
\newcommand{\PP}[1][]{\ensuremath{\ifstrempty{#1}{\mathbb{P}}{\mathbb{P}^{#1}}}}
\newcommand{\HH}[1][]{\ensuremath{\ifstrempty{#1}{\mathbb{H}}{\mathbb{H}^{#1}}}}
\newcommand{\FF}[1][]{\ensuremath{\ifstrempty{#1}{\mathbb{F}}{\mathbb{F}^{#1}}}}
% expected value
\newcommand{\EE}{\ensuremath{\mathbb{E}}}
\newcommand{\charin}{\text{ char }}
\DeclareMathOperator{\sign}{sign}
\DeclareMathOperator{\Aut}{Aut}
\DeclareMathOperator{\Inn}{Inn}
\DeclareMathOperator{\Syl}{Syl}
\DeclareMathOperator{\Gal}{Gal}
\DeclareMathOperator{\GL}{GL} % General linear group
\DeclareMathOperator{\SL}{SL} % Special linear group

%---------------------------------------
% BlackBoard Math Fonts :-
%---------------------------------------

%Captital Letters
\newcommand{\bbA}{\mathbb{A}}	\newcommand{\bbB}{\mathbb{B}}
\newcommand{\bbC}{\mathbb{C}}	\newcommand{\bbD}{\mathbb{D}}
\newcommand{\bbE}{\mathbb{E}}	\newcommand{\bbF}{\mathbb{F}}
\newcommand{\bbG}{\mathbb{G}}	\newcommand{\bbH}{\mathbb{H}}
\newcommand{\bbI}{\mathbb{I}}	\newcommand{\bbJ}{\mathbb{J}}
\newcommand{\bbK}{\mathbb{K}}	\newcommand{\bbL}{\mathbb{L}}
\newcommand{\bbM}{\mathbb{M}}	\newcommand{\bbN}{\mathbb{N}}
\newcommand{\bbO}{\mathbb{O}}	\newcommand{\bbP}{\mathbb{P}}
\newcommand{\bbQ}{\mathbb{Q}}	\newcommand{\bbR}{\mathbb{R}}
\newcommand{\bbS}{\mathbb{S}}	\newcommand{\bbT}{\mathbb{T}}
\newcommand{\bbU}{\mathbb{U}}	\newcommand{\bbV}{\mathbb{V}}
\newcommand{\bbW}{\mathbb{W}}	\newcommand{\bbX}{\mathbb{X}}
\newcommand{\bbY}{\mathbb{Y}}	\newcommand{\bbZ}{\mathbb{Z}}

%---------------------------------------
% MathCal Fonts :-
%---------------------------------------

%Captital Letters
\newcommand{\mcA}{\mathcal{A}}	\newcommand{\mcB}{\mathcal{B}}
\newcommand{\mcC}{\mathcal{C}}	\newcommand{\mcD}{\mathcal{D}}
\newcommand{\mcE}{\mathcal{E}}	\newcommand{\mcF}{\mathcal{F}}
\newcommand{\mcG}{\mathcal{G}}	\newcommand{\mcH}{\mathcal{H}}
\newcommand{\mcI}{\mathcal{I}}	\newcommand{\mcJ}{\mathcal{J}}
\newcommand{\mcK}{\mathcal{K}}	\newcommand{\mcL}{\mathcal{L}}
\newcommand{\mcM}{\mathcal{M}}	\newcommand{\mcN}{\mathcal{N}}
\newcommand{\mcO}{\mathcal{O}}	\newcommand{\mcP}{\mathcal{P}}
\newcommand{\mcQ}{\mathcal{Q}}	\newcommand{\mcR}{\mathcal{R}}
\newcommand{\mcS}{\mathcal{S}}	\newcommand{\mcT}{\mathcal{T}}
\newcommand{\mcU}{\mathcal{U}}	\newcommand{\mcV}{\mathcal{V}}
\newcommand{\mcW}{\mathcal{W}}	\newcommand{\mcX}{\mathcal{X}}
\newcommand{\mcY}{\mathcal{Y}}	\newcommand{\mcZ}{\mathcal{Z}}


%---------------------------------------
% Bold Math Fonts :-
%---------------------------------------

%Captital Letters
\newcommand{\bmA}{\boldsymbol{A}}	\newcommand{\bmB}{\boldsymbol{B}}
\newcommand{\bmC}{\boldsymbol{C}}	\newcommand{\bmD}{\boldsymbol{D}}
\newcommand{\bmE}{\boldsymbol{E}}	\newcommand{\bmF}{\boldsymbol{F}}
\newcommand{\bmG}{\boldsymbol{G}}	\newcommand{\bmH}{\boldsymbol{H}}
\newcommand{\bmI}{\boldsymbol{I}}	\newcommand{\bmJ}{\boldsymbol{J}}
\newcommand{\bmK}{\boldsymbol{K}}	\newcommand{\bmL}{\boldsymbol{L}}
\newcommand{\bmM}{\boldsymbol{M}}	\newcommand{\bmN}{\boldsymbol{N}}
\newcommand{\bmO}{\boldsymbol{O}}	\newcommand{\bmP}{\boldsymbol{P}}
\newcommand{\bmQ}{\boldsymbol{Q}}	\newcommand{\bmR}{\boldsymbol{R}}
\newcommand{\bmS}{\boldsymbol{S}}	\newcommand{\bmT}{\boldsymbol{T}}
\newcommand{\bmU}{\boldsymbol{U}}	\newcommand{\bmV}{\boldsymbol{V}}
\newcommand{\bmW}{\boldsymbol{W}}	\newcommand{\bmX}{\boldsymbol{X}}
\newcommand{\bmY}{\boldsymbol{Y}}	\newcommand{\bmZ}{\boldsymbol{Z}}
%Small Letters
\newcommand{\bma}{\boldsymbol{a}}	\newcommand{\bmb}{\boldsymbol{b}}
\newcommand{\bmc}{\boldsymbol{c}}	\newcommand{\bmd}{\boldsymbol{d}}
\newcommand{\bme}{\boldsymbol{e}}	\newcommand{\bmf}{\boldsymbol{f}}
\newcommand{\bmg}{\boldsymbol{g}}	\newcommand{\bmh}{\boldsymbol{h}}
\newcommand{\bmi}{\boldsymbol{i}}	\newcommand{\bmj}{\boldsymbol{j}}
\newcommand{\bmk}{\boldsymbol{k}}	\newcommand{\bml}{\boldsymbol{l}}
\newcommand{\bmm}{\boldsymbol{m}}	\newcommand{\bmn}{\boldsymbol{n}}
\newcommand{\bmo}{\boldsymbol{o}}	\newcommand{\bmp}{\boldsymbol{p}}
\newcommand{\bmq}{\boldsymbol{q}}	\newcommand{\bmr}{\boldsymbol{r}}
\newcommand{\bms}{\boldsymbol{s}}	\newcommand{\bmt}{\boldsymbol{t}}
\newcommand{\bmu}{\boldsymbol{u}}	\newcommand{\bmv}{\boldsymbol{v}}
\newcommand{\bmw}{\boldsymbol{w}}	\newcommand{\bmx}{\boldsymbol{x}}
\newcommand{\bmy}{\boldsymbol{y}}	\newcommand{\bmz}{\boldsymbol{z}}

%---------------------------------------
% Scr Math Fonts :-
%---------------------------------------

\newcommand{\sA}{{\mathscr{A}}}   \newcommand{\sB}{{\mathscr{B}}}
\newcommand{\sC}{{\mathscr{C}}}   \newcommand{\sD}{{\mathscr{D}}}
\newcommand{\sE}{{\mathscr{E}}}   \newcommand{\sF}{{\mathscr{F}}}
\newcommand{\sG}{{\mathscr{G}}}   \newcommand{\sH}{{\mathscr{H}}}
\newcommand{\sI}{{\mathscr{I}}}   \newcommand{\sJ}{{\mathscr{J}}}
\newcommand{\sK}{{\mathscr{K}}}   \newcommand{\sL}{{\mathscr{L}}}
\newcommand{\sM}{{\mathscr{M}}}   \newcommand{\sN}{{\mathscr{N}}}
\newcommand{\sO}{{\mathscr{O}}}   \newcommand{\sP}{{\mathscr{P}}}
\newcommand{\sQ}{{\mathscr{Q}}}   \newcommand{\sR}{{\mathscr{R}}}
\newcommand{\sS}{{\mathscr{S}}}   \newcommand{\sT}{{\mathscr{T}}}
\newcommand{\sU}{{\mathscr{U}}}   \newcommand{\sV}{{\mathscr{V}}}
\newcommand{\sW}{{\mathscr{W}}}   \newcommand{\sX}{{\mathscr{X}}}
\newcommand{\sY}{{\mathscr{Y}}}   \newcommand{\sZ}{{\mathscr{Z}}}


%---------------------------------------
% Math Fraktur Font
%---------------------------------------

%Captital Letters
\newcommand{\mfA}{\mathfrak{A}}	\newcommand{\mfB}{\mathfrak{B}}
\newcommand{\mfC}{\mathfrak{C}}	\newcommand{\mfD}{\mathfrak{D}}
\newcommand{\mfE}{\mathfrak{E}}	\newcommand{\mfF}{\mathfrak{F}}
\newcommand{\mfG}{\mathfrak{G}}	\newcommand{\mfH}{\mathfrak{H}}
\newcommand{\mfI}{\mathfrak{I}}	\newcommand{\mfJ}{\mathfrak{J}}
\newcommand{\mfK}{\mathfrak{K}}	\newcommand{\mfL}{\mathfrak{L}}
\newcommand{\mfM}{\mathfrak{M}}	\newcommand{\mfN}{\mathfrak{N}}
\newcommand{\mfO}{\mathfrak{O}}	\newcommand{\mfP}{\mathfrak{P}}
\newcommand{\mfQ}{\mathfrak{Q}}	\newcommand{\mfR}{\mathfrak{R}}
\newcommand{\mfS}{\mathfrak{S}}	\newcommand{\mfT}{\mathfrak{T}}
\newcommand{\mfU}{\mathfrak{U}}	\newcommand{\mfV}{\mathfrak{V}}
\newcommand{\mfW}{\mathfrak{W}}	\newcommand{\mfX}{\mathfrak{X}}
\newcommand{\mfY}{\mathfrak{Y}}	\newcommand{\mfZ}{\mathfrak{Z}}
%Small Letters
\newcommand{\mfa}{\mathfrak{a}}	\newcommand{\mfb}{\mathfrak{b}}
\newcommand{\mfc}{\mathfrak{c}}	\newcommand{\mfd}{\mathfrak{d}}
\newcommand{\mfe}{\mathfrak{e}}	\newcommand{\mff}{\mathfrak{f}}
\newcommand{\mfg}{\mathfrak{g}}	\newcommand{\mfh}{\mathfrak{h}}
\newcommand{\mfi}{\mathfrak{i}}	\newcommand{\mfj}{\mathfrak{j}}
\newcommand{\mfk}{\mathfrak{k}}	\newcommand{\mfl}{\mathfrak{l}}
\newcommand{\mfm}{\mathfrak{m}}	\newcommand{\mfn}{\mathfrak{n}}
\newcommand{\mfo}{\mathfrak{o}}	\newcommand{\mfp}{\mathfrak{p}}
\newcommand{\mfq}{\mathfrak{q}}	\newcommand{\mfr}{\mathfrak{r}}
\newcommand{\mfs}{\mathfrak{s}}	\newcommand{\mft}{\mathfrak{t}}
\newcommand{\mfu}{\mathfrak{u}}	\newcommand{\mfv}{\mathfrak{v}}
\newcommand{\mfw}{\mathfrak{w}}	\newcommand{\mfx}{\mathfrak{x}}
\newcommand{\mfy}{\mathfrak{y}}	\newcommand{\mfz}{\mathfrak{z}}

% lstlistingsEnvs.tex

\usepackage{minted}


\lstset{
  basicstyle=\ttfamily, % Set
  columns=fullflexible,
  keepspaces=true,
  language=Python % You can specify the language if you want syntax highlighting
}

%%%%%%%%%%%%%%%%%%%%%%%%%%%%%%%%%%%%%%%%%%%%%%%%%%%%%%%%%%%%%%%%%%%%%%%%%%%%%%%%%%%%%%%%%%%%%%%%%
%                                 Custom lstlisting Environments
%%%%%%%%%%%%%%%%%%%%%%%%%%%%%%%%%%%%%%%%%%%%%%%%%%%%%%%%%%%%%%%%%%%%%%%%%%%%%%%%%%%%%%%%%%%%%%%%%
% Gruvbox style for Python
\definecolor{Pgruvbox-bg}{HTML}{282828}
\definecolor{Pgruvbox-fg}{HTML}{ebdbb2}
\definecolor{Pgruvbox-red}{HTML}{fb4934}
\definecolor{Pgruvbox-green}{HTML}{b8bb26}
\definecolor{Pgruvbox-yellow}{HTML}{fabd2f}
\definecolor{Pgruvbox-blue}{HTML}{83a598}
\definecolor{Pgruvbox-purple}{HTML}{d3869b}
\definecolor{Pgruvbox-aqua}{HTML}{8ec07c}
\definecolor{BBBlack}{rgb}{0.05, 0.06, 0.09}



% JAVA LSTLISTING STYLE IN Gruvbox Colorscheme
\definecolor{gruvbox-bg}{rgb}{0.282, 0.247, 0.204}
\definecolor{gruvbox-fg1}{rgb}{0.949, 0.898, 0.776}
\definecolor{gruvbox-fg2}{rgb}{0.871, 0.804, 0.671}
\definecolor{gruvbox-red}{rgb}{0.788, 0.255, 0.259}
\definecolor{gruvbox-green}{rgb}{0.518, 0.604, 0.239}
\definecolor{gruvbox-yellow}{rgb}{0.914, 0.808, 0.427}
\definecolor{gruvbox-blue}{rgb}{0.353, 0.510, 0.784}
\definecolor{gruvbox-purple}{rgb}{0.576, 0.412, 0.659}
\definecolor{gruvbox-aqua}{rgb}{0.459, 0.631, 0.737}
\definecolor{gruvbox-gray}{rgb}{0.518, 0.494, 0.471}

\definecolor{lst-bg}{RGB}{45, 45, 45}
\definecolor{lst-fg}{RGB}{220, 220, 204}
\definecolor{lst-keyword}{RGB}{215, 186, 125}
\definecolor{lst-comment}{RGB}{117, 113, 94}
\definecolor{lst-string}{RGB}{163, 190, 140}
\definecolor{lst-number}{RGB}{181, 206, 168}
\definecolor{lst-type}{RGB}{218, 142, 130}

\lstdefinestyle{PythonGruvbox}{
    language=Python,
    identifierstyle=\color{lst-fg},
    basicstyle=\ttfamily\color{Pgruvbox-fg},
    keywordstyle=\color{Pgruvbox-yellow},
    keywordstyle=[2]\color{Pgruvbox-blue},
    stringstyle=\color{Pgruvbox-green},
    commentstyle=\color{Pgruvbox-aqua},
    backgroundcolor=\color{BBBlack},
    rulecolor=\color{BBBlack},
    showstringspaces=false,
    keepspaces=true,
    captionpos=b,
    breaklines=true,
    tabsize=4,
    showspaces=false,
    numbers=left,
    numbersep=5pt,
    numberstyle=\tiny\color{gray},
    showtabs=false,
    columns=fullflexible,
    morekeywords={True,False,None},
    morekeywords=[2]{and,as,assert,break,class,continue,def,del,elif,else,except,exec,
    finally,for,from,global,if,import,in,is,lambda,nonlocal,not,or,pass,print,raise,
    return,try,while,with,yield},
    morecomment=[s]{"""}{"""},
    morecomment=[s]{'''}{'''},
    morecomment=[l]{\#},
    morestring=[b]",
    morestring=[b]',
    literate=
    {0}{{\textcolor{Pgruvbox-purple}{0}}}{1}
    {1}{{\textcolor{Pgruvbox-purple}{1}}}{1}
    {2}{{\textcolor{Pgruvbox-purple}{2}}}{1}
    {3}{{\textcolor{Pgruvbox-purple}{3}}}{1}
    {4}{{\textcolor{Pgruvbox-purple}{4}}}{1}
    {5}{{\textcolor{Pgruvbox-purple}{5}}}{1}
    {6}{{\textcolor{Pgruvbox-purple}{6}}}{1}
    {7}{{\textcolor{Pgruvbox-purple}{7}}}{1}
    {8}{{\textcolor{Pgruvbox-purple}{8}}}{1}
    {9}{{\textcolor{Pgruvbox-purple}{9}}}{1}
}

% Gruvbox style for Java
\definecolor{gruvbox-bg}{rgb}{0.282, 0.247, 0.204}
\definecolor{gruvbox-fg1}{rgb}{0.949, 0.898, 0.776}
\definecolor{gruvbox-fg2}{rgb}{0.871, 0.804, 0.671}
\definecolor{gruvbox-red}{rgb}{0.788, 0.255, 0.259}
\definecolor{gruvbox-green}{rgb}{0.518, 0.604, 0.239}
\definecolor{gruvbox-yellow}{rgb}{0.914, 0.808, 0.427}
\definecolor{gruvbox-blue}{rgb}{0.353, 0.510, 0.784}
\definecolor{gruvbox-purple}{rgb}{0.576, 0.412, 0.659}
\definecolor{gruvbox-aqua}{rgb}{0.459, 0.631, 0.737}
\definecolor{gruvbox-gray}{rgb}{0.518, 0.494, 0.471}

\lstdefinestyle{JavaGruvbox}{
    language=Java,
    basicstyle=\ttfamily\color{Pgruvbox-fg},
    keywordstyle=\color{Pgruvbox-yellow},
    keywordstyle=[2]\color{lst-type},
    commentstyle=\itshape\color{lst-comment},
    stringstyle=\color{lst-string},
    numberstyle=\color{lst-number},
    backgroundcolor=\color{BBBlack},
    rulecolor=\color{gruvbox-aqua},
    showstringspaces=false,
    keepspaces=true,
    captionpos=b,
    breaklines=true,
    tabsize=4,
    showspaces=false,
    showtabs=false,
    columns=fullflexible,
    morekeywords={var},
    morekeywords=[2]{boolean, byte, char, double, float, int, long, short, void},
    morecomment=[s]{/}{/},
    morecomment=[l]{//},
    morestring=[b]",
    morestring=[b]',
    numbers=left,
    numbersep=5pt,
    numberstyle=\tiny\color{gray},
}

% Dracula style for Java
\definecolor{draculawhite-background}{RGB}{237, 239, 252}
\definecolor{draculawhite-comment}{RGB}{98, 114, 164}
\definecolor{draculawhite-keyword}{RGB}{189, 147, 249}
\definecolor{draculawhite-string}{RGB}{152, 195, 121}
\definecolor{draculawhite-number}{RGB}{249, 189, 89}
\definecolor{draculawhite-operator}{RGB}{248, 248, 242}

\lstdefinestyle{JavaDraculaWhite}{
    language=Java,
    backgroundcolor=\color{draculawhite-background},
    commentstyle=\itshape\color{draculawhite-comment},
    keywordstyle=\color{draculawhite-keyword},
    stringstyle=\color{draculawhite-string},
    basicstyle=\ttfamily\footnotesize\color{black},
    identifierstyle=\color{black},
    keywordstyle=\color{draculawhite-keyword}\bfseries,
    morecomment=[s][\color{draculawhite-comment}]{/**}{*/},
    showstringspaces=false,
    showspaces=false,
    breaklines=true,
    %frame=single,
    rulecolor=\color{draculawhite-operator},
    tabsize=2,  
    numbers=left,
    numbersep=4pt,
    numberstyle=\ttfamily\tiny\color{gray}
}

% Dracula style for Python
\definecolor{draculawhite-bg}{HTML}{FAFAFA}
\definecolor{draculawhite-fg}{HTML}{282A36}
\definecolor{pdraculawhite-keyword}{HTML}{BD93F9}
\definecolor{pdraculawhite-comment}{HTML}{6272A4}
\definecolor{draculawhite-number}{HTML}{FF79C6}

\lstdefinestyle{PythonDraculaWhite}{
    language=Python,
    basicstyle=\ttfamily\small\color{draculawhite-fg},
    backgroundcolor=\color{draculawhite-background},
    keywordstyle=\color{orange}\bfseries,
    stringstyle=\color{draculawhite-string},
    commentstyle=\color{pdraculawhite-comment}\itshape,
    numberstyle=\color{draculawhite-number},
    showstringspaces=false,
    showspaces=false,
    breaklines=true,
    frame=single,
    rulecolor=\color{draculawhite-operator}, 
    tabsize=4,
    morekeywords={as,with,1,2,3,4, 5,6,7,8,9,True,False},
    numbers=left,
    numbersep=5pt,
    numberstyle=\small\bfseries\ttfamily\color{htmlcomment},
}

% Dracula Dark style for HTML
\definecolor{htmltag}{HTML}{ff79c6}
\definecolor{htmlattr}{HTML}{f1fa8c}
\definecolor{htmlvalue}{HTML}{bd93f9}
\definecolor{htmlcomment}{HTML}{6272a4}
\definecolor{htmltext}{HTML}{401E31}
\definecolor{htmlbackground}{HTML}{282a36}
\definecolor{comphtmlbackground}{HTML}{8093FF}

\lstdefinestyle{HTMLDraculaDark}{
    basicstyle=\normalsize\bfseries\ttfamily\color{htmltext},
    commentstyle=\itshape\color{htmlcomment},
    keywordstyle=\bfseries\color{htmltag},
    stringstyle=\color{htmlvalue},
    emph={DOCTYPE,html,head,body,div,span,a,script},
    emphstyle={\color{htmltag}\bfseries},
    sensitive=true,
    showstringspaces=false,
    backgroundcolor=\color{white},
    inputencoding=utf8,
    extendedchars=true,
    language=HTML,
    tabsize=4,
    breaklines=true,
    breakatwhitespace=true,
    numbers=left,
    numbersep=10pt,
    numberstyle=\small\bfseries\ttfamily\color{htmlcomment},
    escapeinside={<@}{@>},
    rulecolor=\color{htmlbackground},
    xleftmargin=10pt,
    frame=none, 
    breaklines=true,
    postbreak=\mbox{\textcolor{gray}{$\hookrightarrow$}\space},
    showlines=false,
    moredelim=[s][\itshape\color{htmlcomment}]{<!--}{-->},
    morekeywords={id,class,type,name,value,placeholder,checked,src,href,alt},
    literate={é}{{\'e}}1 {è}{{\`e}}1 {ê}{{\^e}}1 {ë}{{\"e}}1 {à}{{\`a}}1 {ù}{{\`u}}1 {û}{{\^u}}1 {ç}{{\c{c}}}1 {â}{{\^a}}1 {î}{{\^i}}1 {ï}{{\"i}}1
}


\lstdefinestyle{Haskell}{
  frame=none,
  xleftmargin=2pt,
  stepnumber=1,
  numbers=left,
  numbersep=5pt,
  numberstyle=\ttfamily\tiny\color[gray]{0.3},
  belowcaptionskip=\bigskipamount,
  captionpos=b,
  escapeinside={*'}{'*},
  language=haskell,
  tabsize=2,
  emphstyle={\bf},
  %commentstyle=\it,
  stringstyle=\mdseries\ttfamily,
  showspaces=false,
  keywordstyle=\bfseries\ttfamily,
  columns=flexible,
  basicstyle=\small\ttfamily,
  showstringspaces=false,
  morecomment=[l]\%,
}



\lstdefinestyle{CSSDraculaLight}{
    basicstyle=\bfseries\scriptsize\ttfamily\color{htmltext},
    commentstyle=\color{htmlcomment},
    keywordstyle=\bfseries\color{htmlvalue},
    stringstyle=\color{htmlvalue},
    emph={DOCTYPE,html,head,body,div,span,a,script},
    emphstyle={\color{htmltag}\bfseries},
    sensitive=true,
    showstringspaces=false,
    backgroundcolor=\color{white},
    inputencoding=utf8,
    extendedchars=true, % Support extended characters
    frame=none, 
    %frame=tb,
    tabsize=4,
    breaklines=true,
    breakatwhitespace=true,
    numbers=left,
    numbersep=10pt,
    numberstyle=\small\bfseries\ttfamily\color{htmlcomment},
    escapeinside={<@}{@>},
    rulecolor=\color{htmlbackground},
    xleftmargin=20pt,
    % Add a vertical line for opening and closing tags
    %frame=lines,
    framesep=2pt,
    framexleftmargin=4pt,
    % Add a vertical line for closing tags that go through multiple lines
    breaklines=true,
    postbreak=\mbox{\textcolor{gray}{$\hookrightarrow$}\space},
    showlines=true,
    % Add a rule to apply commentstyle to keywords inside comments
    moredelim=[s][\color{htmlcomment}]{/*}{*/},
    literate={é}{{\'e}}1
             {è}{{\`e}}1
             {ê}{{\^e}}1
             {ë}{{\"e}}1
             {à}{{\`a}}1
             {ù}{{\`u}}1
             {û}{{\^u}}1
             {ç}{{\c{c}}}1
             {â}{{\^a}}1
             {î}{{\^i}}1
             {ï}{{\"i}}1,
    morekeywords={color, background, background-color, font-size, font-weight, border, border-radius, padding, margin, display, position, top, right, bottom, left, flex, grid, width, height, min-width, max-width, min-height, max-height, transition, transform, animation, keyframes, content, z-index,id,class,type,name,value,placeholder,checked,src,href,alt},
    morestring=[s][\color{htmltag}]{:}{;},
}


\usetikzlibrary{decorations.pathreplacing}

%For multiple line boxed equations
\newcommand*\widefbox[1]{\fbox{\hspace{2em}#1\hspace{2em}}}
\newtcbox{\othermathbox}[1][]{nobeforeafter, math upper, tcbox raise base, 
          % enhanced, 
          % sharp corners, 
          % colback=black!10, 
          colframe=black!60, 
          drop fuzzy shadow, left=1em, top=1em, right=3em, bottom=1em}



\def\blankpage{%
      \clearpage%
      \thispagestyle{empty}%
      \addtocounter{page}{-1}%
      \null%
      \clearpage}


\begin{document}
\begin{multicols*}{2}
    \chapter{Programmation dynamique probabiliste}

    \section{Principe}
    \textcolor{myr}{\textbf{Dans le contexte de programmation dynamique déterministe}}, l'état $S_{n+1}$ à l'étape $n+1$ est \textbf{complètement déterminé}                                
    par l'état $S_n$ à l'étape $n$ et la décision prise 
    dans l'état $S_n$. 


    \begin{figure}[H]
    \centering
            \begin{tikzpicture}[
            node distance=2cm and 2cm,
            every node/.style={font=\sffamily},
            box/.style={draw, fill=gray!30, minimum size=1.5cm, 
            rectangle, align=center}, % Ajout de fill=gray!30
            arrow/.style={-{Stealth}, thick},
            textred/.style={text=red, font=\sffamily\small}
        ]

        % Noeuds principaux
        \node[box] (box) {};
        \node[above=1cm of box] (xn) {$x_n$};
        \node[left=2cm of box] (Sn) {$S_n$};
        \node[right=2cm of box] (Sn1) {$S_{n+1} = S_n - x_n$};
        \node[below=1cm of box] (Cn) {$C_n$};

        % Flèches
        \draw[arrow] (xn) -- (box);
        \draw[arrow] (Sn) -- (box);
        \draw[arrow] (box) -- (Sn1);
        \draw[arrow] (box) -- (Cn);
% Annotation rouge
\node[draw=red, below left=2.25cm and -2cm of box, align=left] (textred) {
    $\textcolor{red}{C_n : }$ Contribution directe à la\\
    valeur de l'objectif    
};
\draw[red, thick, -{Stealth}] (Cn) -- ++(-2,0) -- (textred.north west);
        \end{tikzpicture}
    \end{figure}

    \textcolor{myr}{\textbf{Dans le contexte de programmation dynamique probabiliste}}, l'état $S_{n+1}$ à l'étape $n+1$ \textbf{n'est pas} 
    complètement déterminé par l'état $S_n$ à l'étape $n$ 
    et la décision $x_n$ prise dans l'état $S_n$

    On a donc une \textbf{distribution de probabilités}. On peut 
    Il est possible d'obtenir différents états $S_{n+1}$ 
    à l'état $n + 1$, en fonction de \textbf{la décision} $x_n$ 
    et de l'\textbf{état} $S_n$ : 







\begin{tikzpicture}[
    node distance=2cm and 2cm,
    every node/.style={font=\sffamily},
    box/.style={draw, fill=gray!30, minimum size=3cm, rectangle, align=center},
    dashedarrow/.style={dashed, thick, -{Stealth}},
    arrow/.style={thick, -{Stealth}},
    textmyb/.style={text=myb, font=\sffamily\footnotesize\itshape}
]

% Boîte principale
\node[box] (box) {};

% Entrées
\node[above=1.2cm of box] (xn) {$x_n$};
\node[left=1.5cm of box] (sn) {$S_n$};

% Sorties
\node[right=2cm of box, yshift=1cm] (sn1i1) {$S_{n+1}^{i_1}$};
\node[right=2cm of box, yshift=0.5cm] (sn1i2) {$S_{n+1}^{i_2}$};
\node[right=2cm of box, yshift=-1cm] (sn1it) {$S_{n+1}^{i_t}$};

% Variables C_n^i en bas
\node[below=1.5cm of box, xshift=-1.0cm] (ci1) {$C_n^{i_1}$};
\node[below=1.5cm of box, xshift=-0.5cm] (ci2) {$C_n^{i_2}$};
\node[below=1.5cm of box, xshift=1cm] (cit) {$C_n^{i_t}$};

% Labels p_n^i
\node[right=0.35cm of box,  yshift=1.35cm, textmyb] (pn1) {$p_n^{i_1}$};
\node[right=0.35cm of box, yshift=0.8cm, textmyb] (pn2) {$p_n^{i_2}$};
\node[right=0.35cm of box, yshift=-0.6cm, textmyb] (pnt) {$p_n^{i_t}$};

% Flèches extérieures
\draw[arrow] (xn) -- (box.north);
\draw[arrow] (sn) -- (box.west);

% Flèches pour i_1
\draw[arrow] (box.south) ++(-1.0cm, 0) -- (ci1); % Extérieure vers C_n^{i_1}
\draw[dashed] (ci1.north) -- ++(0, 3)  -- ++(0, 1.0) -- (box.east|-sn1i1); % Pointillée intérieure
\draw[arrow] (box.east|-sn1i1) -- (sn1i1); % Extérieure vers s_{n+1}^{i_1}

% Flèches pour i_2
\draw[arrow] (box.south) ++(-0.5cm, 0) -- (ci2); % Extérieure vers C_n^{i_2}
\draw[dashed] (ci2.north) -- ++(0, 3) -- ++(0, 0.5) -- (box.east|-sn1i2); % Pointillée intérieure

\draw[arrow] (box.east|-sn1i2) -- (sn1i2); % Extérieure vers s_{n+1}^{i_2}

% Flèches pour i_t
\draw[arrow] (box.south) ++(1cm, 0) -- (cit); % Extérieure vers C_n^{i_t}
\draw[dashed] (cit.north) -- ++(0, 1.5) -- ++(0, 0.5) -- (box.east|-sn1it); % Pointillée intérieure
\draw[arrow] (box.east|-sn1it) -- (sn1it); % Extérieure vers s_{n+1}^{i_t}

% Points de suspension
\node[below=1.3cm of box, xshift=0.25cm] (dots1) {$\cdots$};
\node[right=.6cm of box] (dots2) {$\vdots$};

\end{tikzpicture}

\begin{itemize}
    \item [$\rhd$ ] \textbf{Ensemble des états suivants possibles}  
        \begin{itemize}
            \item [$\blacktriangleright$ ] 
                $A_{n+1} =  
                s_{n+1}^{\textcolor{myb}{1}}, 
                s_{n+1}^{\textcolor{myb}{2}}, 
                \dots, s_{n+1}^{\textcolor{myb}{|A_{n+1}|}}$
        \end{itemize}
    \item [$\rhd$ ]                             
        \textbf{Probabilité d'atteindre l'état spécifique} 
        $\boldsymbol{s_{n+1}^{i}}$
        \begin{itemize}
            \item [$\blacktriangleright$ ] 
                $p_n^{\textcolor{myb}{i} } 
                = 
                \text{Prob}\{s_{n+1} 
                = 
        s_{n+1}^{\textcolor{myb}{i}}\} \quad i = 1, \dots, |A_{n+1}|$ 
        \end{itemize}
    \item [$\rhd$ ] 
        \textbf{États atteignabless} 
        $\boldsymbol{s_n^{\textcolor{myb}{i_{n+1}}}} 
        \in A_{\text{atteingnable}} \subseteq A_{n+1}$   
        \begin{itemize}
            \item [$\blacktriangleright$ ] 
                $p_n^{\textcolor{myb}{i} } > 0$ \quad $1 = i_1, \dots, i_t$ 
            \item [$\rhd$ ] 
                $\{
                s_{n+1}^{\textcolor{myb}{i_1}}, 
                s_{n+1}^{\textcolor{myb}{i_2}}, 
                \dots,
                s_{n+1}^{\textcolor{myb}{i_t}}
                \}% 
                \subseteq A_{n+1}$
    \end{itemize}
    \item [$\rhd$ ] 
        \textbf{États non atteignabless} 
        \begin{itemize}
            \item [$\blacktriangleright$ ] 
                $p_n^{\textcolor{myb}{i} } 
                = 
                0$ 
                \quad 
                $i = 1, \dots, |A_{n+1}|, i \neq i_1, \dots, i_t$ 
    \end{itemize}
\end{itemize}       

\begin{note}{}{}
   Dans le diagramme et le texte, \( p^i_n \) représente \textbf{la probabilité de transition} vers un état spécifique \( s_{n+1}^{i} \) à l’étape \( n+1 \), \textbf{conditionnellement à être dans l’état \( s_n \) et après avoir effectué l’action \( x_n \)}. 
\end{note}                                

    Pour chaque état $s_n$, on a donc le le système suivant : 
    \begin{align*}
            &\phantom{p}p_{n}^{\textcolor{myb}{i}} \geq 0 \quad i = 1, \dots, |A_{n+1}|  
            \\
            &\sum_{\textcolor{myb}{i} =1}^{\textcolor{myb}{|A_{n+1}|}}p_n^{\textcolor{myb}{i}} 
            = 1
    \end{align*}            

    \section{Exemple du problème de diligence}
    On peut imaginer un scénario où \textbf{il n'est pas garantit} que lorsqu'on 
    prend la décision de se déplacer à la ville $x_n \in A_{n+1}$ \textbf{en partant}  
    de $x$, il n'est pas garantit qu'on atteigne cette ville. 


\subsection{Tenter de se rendre à la ville $\boldsymbol{2}$}
    Supponons que nous voulons nous rendre à la ville $2$.
    Posons qu'il y a une \textbf{probabilité}  de $0.8$ qu'on arrive à la ville $2$ 
    et une \textbf{probabilité} de $0.2$ qu'on arrive à la ville $3$, lorsqu'on 
    prendre la \textbf{décision  de se rendre à la ville} $2$. Ponsons que la 
    distance de $1$ à $2$ et de $1$ à $3$ est de $2$ et $4$ unités, respectivement. 

    Nous avons alors la situation suivante :
    \begin{center}
        \begin{tikzpicture}[
        scale=0.85, 
            ->,>=stealth',shorten >=1pt, auto, 
            node distance=2cm and 2cm, thick,
            every node/.style={font=\sffamily},
            state/.style={circle, draw, minimum size=6mm, font=\footnotesize},
            weight/.style={font=\small},
            dottedarrow/.style={->, dotted, thick},
            arrow/.style={->, thick},
            costred/.style={text=red, font=\small},
            costmyb/.style={text=myb, font=\small},
            squarebox/.style={draw=green, thick, minimum size=6mm, font=\bfseries, align=center}
        ]

        % Noeuds
        \node[squarebox] (box1) at (0, 3) {1};
        \node[squarebox] (box2) at (2, 3) {2};
        \node[squarebox] (box5) at (5, 3) {5};

        \node[state] (node1) at (0, 0) {1};
        \node[state] (node2) at (2, 2) {2};
        \node[state] (node3) at (2, 0) {3};
        \node[state] (node4) at (2, -2) {4};
        \node[state] (node10) at (5, 0) {10};

        % Arcs
        \draw[arrow] (node1) -- (node2) node[midway, weight, above left] {\textcolor{red}{2}  };
        \draw[arrow] (node1) -- (node3) node[midway, weight, above right] {\textcolor{red}{4}  };
        \draw[arrow] (node1) -- (node4) node[midway, weight, below ] {\textcolor{red}{3}  };

        \draw[dottedarrow] (node2) -- (node10) node[midway, above right] {$f_2^*(2)$};
        \draw[dottedarrow] (node3) -- (node10)  node[midway, below ] {$f_2^*(3)$};


        % Coûts (textes colorés)


        \node[costmyb, below=0cm of node2 ]  {1};
        \node[costmyb, below=0cm of node3 ]  {2};
        \node[costmyb, below=0cm of node4 ]  {3};


        \node[right=0.2cm of node2, yshift=-0.2cm]  {0.8};
        \node[right=0.2cm of node3, yshift=0.3cm]  {0.2};


        \end{tikzpicture}
        
    \end{center}

    Ainsi, le chemin optimal pour se rendre de $1$ à $2$ dépend de 
    la probabilité de d'arriver à $2$ ou $3$ : 
    \begin{align*}
        f_1(1,2) = \texttt{0.8} (2 + f_2^{*}(2)) + \texttt{0.2} (4 + f_2^{*}(3))    
    \end{align*}


    On peut  également représenter la situation avec le diagramme d'état suivant :


\begin{center}
\begin{tikzpicture}[
    node distance=2cm and 2cm,
    every node/.style={font=\sffamily},
    box/.style={draw, fill=gray!30, minimum size=3cm, rectangle, align=center},
    dashedarrow/.style={dashed, thick, -{Stealth}},
    arrow/.style={thick, -{Stealth}},
    textmyb/.style={text=myb, font=\sffamily\footnotesize\itshape}
]

% Boîte principale
\node[box] (box) {};

% Entrées
\node[above=1.2cm of box] (xn) {$x_1 = 2$};
\node[left=1.5cm of box] (sn) {$s_1 = 1$};

% Sorties
\node[right=2cm of box, yshift=0.75cm] (s1) {$s_2^1 = 2$};
\node[right=2cm of box, yshift=-0.75cm] (s2) {$s_2^2 = 3$};

% Variables lambda
\node[below=1.5cm of box, xshift=-1cm] (lambda1) {$\lambda_{1,2} = 2$};
\node[below=1.5cm of box, xshift=1cm] (lambda2) {$\lambda_{1,3} = 4$};

% Probabilités
\node[right=0.3cm of box, yshift=1.05cm] (p1) {$0.8$};
\node[right=0.3cm of box, yshift=-0.45cm] (p2) {$0.2$};

% Flèches extérieures
\draw[arrow] (xn) -- (box.north);
\draw[arrow] (sn) -- (box.west);

% Flèche pour s_2^1
\draw[arrow] (box.south) ++(-1cm, 0) -- (lambda1); % Extérieure vers lambda_1,2
\draw[dashed] (lambda1.north) -- ++(0, 3.75)  -- (box.east|-s1); % Pointillée intérieure
\draw[arrow] (box.east|-s1) -- (s1); % Extérieure vers s_2^1

% Flèche pour s_2^2
\draw[arrow] (box.south) ++(1cm, 0) -- (lambda2); % Extérieure vers lambda_1,3
\draw[dashed] (lambda2.north) -- ++(0, 2.25) -- (box.east|-s2); % Pointillée intérieure
\draw[arrow] (box.east|-s2) -- (s2); % Extérieure vers s_2^2

% Probabilités

\end{tikzpicture}
    
\end{center}

\subsection{Tenter de se rendre à la ville $\boldsymbol{3}$}

\[%
\begin{aligned}
    &n = 1, \quad s_1 = \textcolor{red}{1}, \quad x_1 = \textcolor{red}{3}, \\
    &s_2^1 = \textcolor{myb}{2}, \quad s_2^2 = \textcolor{myb}{3}, \quad s_2^3 = \textcolor{myb}{4}, \\
    &p_1^1(1, 3) = \textcolor{red}{0.1}, \quad 
    p_1^2(1, 3) = \textcolor{red}{0.6}, \quad 
    p_1^3(1, 3) = \textcolor{red}{0.3}.
\end{aligned}
\]%


\begin{tikzpicture}[
        scale=0.85, 
            ->,>=stealth',shorten >=1pt, auto, 
            node distance=2cm and 2cm, thick,
            every node/.style={font=\sffamily},
            state/.style={circle, draw, minimum size=6mm, font=\footnotesize},
            weight/.style={font=\small},
            dottedarrow/.style={->, dotted, thick},
            arrow/.style={->, thick},
            costred/.style={text=red, font=\small},
            costmyb/.style={text=myb, font=\small},
            squarebox/.style={draw=green, thick, minimum size=6mm, font=\bfseries, align=center}
        ]

        % Noeuds
        \node[squarebox] (box1) at (0, 3) {1};
        \node[squarebox] (box2) at (2, 3) {2};
        \node[squarebox] (box5) at (5, 3) {5};

        \node[state] (node1) at (0, 0) {1};
        \node[state] (node2) at (2, 2) {2};
        \node[state] (node3) at (2, 0) {3};
        \node[state] (node4) at (2, -2) {4};
        \node[state] (node10) at (5, 0) {10};

        % Arcs
        \draw[arrow] (node1) -- (node2) node[midway, weight, above left] {\textcolor{red}{2}};
        \draw[arrow] (node1) -- (node3) node[midway, weight, above right] {\textcolor{red}{4}};
        \draw[arrow] (node1) -- (node4) node[midway, weight, below ] {\textcolor{red}{3}};

        \draw[dottedarrow] (node2) -- (node10) node[midway, above right] {$f_2^*(2)$};
        \draw[dottedarrow] (node3) -- (node10)  node[midway, below ] {$f_2^*(3)$};
        \draw[dottedarrow] (node4) -- (node10)  node[midway, below, yshift=-0.2cm ] {$f_2^*(4)$};

        % Coûts (textes colorés)
        \node[costmyb, below=0cm of node2 ]  {1};
        \node[costmyb, below=0cm of node3 ]  {2};
        \node[costmyb, below=0cm of node4 ]  {3};

        % Probabilités
        \node[right=0.2cm of node2, yshift=-0.2cm]  {0.1};
        \node[right=0.2cm of node3, yshift=0.3cm]  {0.6};
        \node[right=0.2cm of node4, yshift=0.9cm]  {0.3};

\end{tikzpicture}



\vspace{4em}
\begin{center}
\begin{tikzpicture}[
    node distance=2cm and 2cm,
    every node/.style={font=\sffamily},
    box/.style={draw, fill=gray!30, minimum size=3cm, rectangle, align=center},
    dashedarrow/.style={dashed, thick, -{Stealth}},
    arrow/.style={thick, -{Stealth}},
    textmyb/.style={text=myb, font=\sffamily\footnotesize\itshape}
]

% Boîte principale
\node[box] (box) {};

% Entrées
\node[above=1.2cm of box] (xn) {$x_1 = 3$};
\node[left=1.5cm of box] (sn) {$s_1 = 1$};

% Sorties
\node[right=2cm of box, yshift=1.2cm] (s1) {$s_2^1 = 2$};
\node[right=2cm of box, yshift=0cm] (s2) {$s_2^2 = 3$};
\node[right=2cm of box, yshift=-1.2cm] (s3) {$s_2^3 = 4$};

% Variables lambda
\node[below=1.5cm of box, xshift=-1.5cm] (lambda1) {$\lambda_{1,2} = 2$};
\node[below=1.5cm of box] (lambda2) {$\lambda_{1,3} = 4$};
\node[below=1.5cm of box, xshift=1.5cm] (lambda3) {$\lambda_{1,4} = 3$};

% Probabilités
\node[right=0.3cm of box, yshift=1.5cm] (p1) {$0.1$};
\node[right=0.3cm of box, yshift=0.3cm] (p2) {$0.6$};
\node[right=0.3cm of box, yshift=-0.9cm] (p3) {$0.3$};

% Flèches extérieures
\draw[arrow] (xn) -- (box.north);
\draw[arrow] (sn) -- (box.west);

% Flèche pour s_2^1
\draw[arrow] (box.south) ++(-1.5cm, 0) -- (lambda1); % Extérieure vers lambda_{1,2}
\draw[dashed, thick] (box.south) ++(-1.5, 0) -- ++(0, 2.7)  -- (box.east|-s1); % Pointillée intérieure
\draw[arrow] (box.east|-s1) -- (s1); % Extérieure vers s_2^1

% Flèche pour s_2^2
\draw[arrow] (box.south) -- (lambda2); % Extérieure vers lambda_{1,3}
\draw[dashed, thick] (box.south)  -- ++(0, 1.5)  -- (box.east|-s2); % Pointillée intérieure
\draw[arrow] (box.east|-s2) -- (s2); % Extérieure vers s_2^2

% Flèche pour s_2^3
\draw[arrow] (box.south) ++(1.5cm, 0) -- (lambda3); % Extérieure vers lambda_{1,4}

\draw[dashed, thick] (box.south)  ++(1.5,0)  -- (box.east|-s3); % Pointillée intérieure
\draw[arrow] (box.east|-s3) -- (s3); % Extérieure vers s_2^3

\end{tikzpicture}
\end{center}

\subsection{Tenter de se rendre à la ville $\boldsymbol{4}$}

\[%
\begin{aligned}
    &n = 1, \quad s_1 = \textcolor{red}{1}, \quad x_1 = \textcolor{red}{4}, \\
    &s_2^1 = \textcolor{myb}{2}, \quad s_2^2 = \textcolor{myb}{3}, \quad s_2^3 = \textcolor{myb}{4}, \\
    &p_1^1(1, 4) = 0, \quad 
    p_1^2(1, 4) = \textcolor{red}{0.3}, \quad 
    p_1^3(1, 4) = \textcolor{red}{0.7}.
\end{aligned}
\]%


\begin{tikzpicture}[
        scale=0.85, 
        ->,>=stealth',shorten >=1pt, auto, 
        node distance=2cm and 2cm, thick,
        every node/.style={font=\sffamily},
        state/.style={circle, draw, minimum size=6mm, font=\footnotesize},
        weight/.style={font=\small},
        dottedarrow/.style={->, dotted, thick},
        arrow/.style={->, thick},
        costred/.style={text=red, font=\small},
        costmyb/.style={text=myb, font=\small},
        squarebox/.style={draw=green, thick, minimum size=6mm, font=\bfseries, align=center}
    ]

    % Noeuds
    \node[squarebox] (box1) at (0, 3) {1};
    \node[squarebox] (box2) at (2, 3) {2};
    \node[squarebox] (box5) at (5, 3) {5};

    \node[state] (node1) at (0, 0) {1};
    \node[state] (node2) at (2, 2) {2};
    \node[state] (node3) at (2, 0) {3};
    \node[state] (node4) at (2, -2) {4};
    \node[state] (node10) at (5, 0) {10};

    % Arcs
    \draw[arrow] (node1) -- (node2) node[midway, weight, above left] {\textcolor{red}{2}};
    \draw[arrow] (node1) -- (node3) node[midway, weight, above right] {\textcolor{red}{4}};
    \draw[arrow] (node1) -- (node4) node[midway, weight, below ] {\textcolor{red}{3}};

    \draw[dottedarrow] (node3) -- (node10) node[midway, above right] {$0.3$};
    \draw[dottedarrow] (node4) -- (node10) node[midway, below, yshift=-0.2cm] {$0.7$};

    % Coûts (textes colorés)
    \node[costmyb, below=0cm of node2] {1};
    \node[costmyb, below=0cm of node3] {2};
    \node[costmyb, below=0cm of node4] {3};
\end{tikzpicture}


\begin{center}
\begin{tikzpicture}[
    node distance=2cm and 2cm,
    every node/.style={font=\sffamily},
    box/.style={draw, fill=gray!30, minimum size=3cm, rectangle, align=center},
    dashedarrow/.style={dashed, thick, -{Stealth}},
    arrow/.style={thick, -{Stealth}},
    textmyb/.style={text=myb, font=\sffamily\footnotesize\itshape}
]

% Boîte principale
\node[box] (box) {};

% Entrées
\node[above=1.2cm of box] (xn) {$x_1 = 4$};
\node[left=1.5cm of box] (sn) {$s_1 = 1$};

% Sorties
\node[right=2cm of box, yshift=0.6cm] (s2) {$s_2^2 = 3$};
\node[right=2cm of box, yshift=-0.6cm] (s3) {$s_2^3 = 4$};

% Variables lambda
\node[below=1.5cm of box, xshift=-0.8cm] (lambda2) {$\lambda_{1,3} = 4$};
\node[below=1.5cm of box, xshift=0.8cm] (lambda3) {$\lambda_{1,4} = 3$};

% Probabilités
\node[right=0.3cm of box, yshift=0.9cm] (p2) {$0.3$};
\node[right=0.3cm of box, yshift=-0.3cm] (p3) {$0.7$};

% Flèches extérieures
\draw[arrow] (xn) -- (box.north);
\draw[arrow] (sn) -- (box.west);

% Flèche pour s_2^2
\draw[arrow] (box.south) ++(-0.8cm, 0) -- (lambda2); % Extérieure vers lambda_{1,3}
\draw[dashed, thick] (box.south) ++(-0.8, 0) -- ++(0, 2.1) -- (box.east|-s2); % Pointillée intérieure
\draw[arrow] (box.east|-s2) -- (s2); % Extérieure vers s_2^2

% Flèche pour s_2^3
\draw[arrow] (box.south) ++(0.8cm, 0) -- (lambda3); % Extérieure vers lambda_{1,4}
\draw[dashed, thick] (lambda3.north) -- ++(0, 2.4) -- (box.east|-s3); % Pointillée intérieure
\draw[arrow] (box.east|-s3) -- (s3); % Extérieure vers s_2^3
\end{tikzpicture} 
\end{center}    



\subsection{Formule de récurrence}

    Considérons les probabilité suivantes :

\[
\begin{aligned}
    &p_1^1(1, 2) = 0.8, \quad p_1^2(1, 2) = 0.2, \quad p_1^3(1, 2) = 0.0, \\
    &p_1^1(1, 3) = 0.1, \quad p_1^2(1, 3) = 0.6, \quad p_1^3(1, 3) = 0.3, \\
    &p_1^1(1, 4) = 0.0, \quad p_1^2(1, 4) = 0.3, \quad p_1^3(1, 4) = 0.7.
\end{aligned}
\]

    Nous avons déterminé poids du \textbf{chemin probable}   pour chacune 
    des villes $2, 3, 4$ qu'on souhaite vistiter lorsqu'on se trouve
    à la \textbf{ville} $\boldsymbol{1}$, soient $f_1(1,2)$, $f_1(1,3)$, $f_1(1,4)$  :                  

    \[%
    \begin{aligned}
        f_1(1, 2) &= 0.8 \, (2 + f_2^*(2)) + 0.2 \, (4 + f_2^*(3)) + 0.0\, (3 + f_2^*(4)), \\
        f_1(1, 3) &= 0.1 \, (2 + f_2^*(2)) + 0.6 \, (4 + f_2^*(3)) + 0.3 \, (3 + f_2^*(4)), \\
        f_1(1, 4) &= 0.0 \, (2 + f_2^*(2)) + 0.3 \, (4 + f_2^*(3)) + 0.7 \, (3 + f_2^*(4)).
    \end{aligned}
    \]%

    En considérant chaque poids de chemin probables, 
    le poids du \textbf{chemin minimal probable} est 
    le poids minimal parmi ceux-ci :

    \[%
    \begin{aligned}
        f^*(1) &= \min \left\{ f_1(1, 2), f_1(1, 3), f_1(1, 4) \right\}, \\
               &= \min_{x_1 \in \{2, 3, 4\}} f_1(1, x_1), \\
               &= \min_{x_1 \in A_2} \left\{ \sum_{i=1}^{|A_2|} p_1^i(1, x_1) \, \big( \lambda_{1,s_2^i} + f_2^*(s_2^i) \big) \right\}.
    \end{aligned}
    \]%

    De façon générale, pour une étape $n$ et un état $s_n$, 
    nous avons la formule de récurence suivante : 


    \[%
    \boxed{
        f_n^*(s_n) = \min_{x_n \in A_{n+1}} \left\{ 
        \sum_{i=1}^{|A_{n+1}|} p_n^i(s_n, x_n) \left( \lambda_{s_n,s_{n+1}^i} + f_{n+1}^*(s_{n+1}^i) \right)
        \right\}
    }
    \]%

    \section{Généralisation}

    \subsection{Programmation probabiliste}
    \begin{align*}
        &f_n(s_n, x_n) = C_n + f_{n+1}^*(s_{n+1}) \\
        &f_n^*(s_n) = \min \text{ ou } \max_{x_n \in X(s_n)} \{ f_n(s_n, x_n) \}
    \end{align*}

    \subsection{Programmation probabiliste}
    \begin{align*}
        &f_n(s_n, x_n) 
        = 
        \sum_{i=1}^{|A_{n+1}|} 
        p_n^i(s_n, x_n) 
        \left( C_n^i + f_{n+1}^*(s_{n+1}^i) \right) 
        \\
        &f_n^*(s_n) 
        = 
        \min \text{ ou } 
        \max_{x_n \in X(s_n)} \{ f_n(s_n, x_n) \}
    \end{align*}
    où \( X(s_n) \) est ensemble des décisions possibles 
    dans l’état \( s_n \) à l’étape \(n \)
    \newpage

    % \chapter{Modèles sotachastiques}
    %
    % \section{Espace échantillon} 
    % Soit une expérience dont le résultat n'\textbf{est pas prévisible}  
    % et supposons que nous connaisssons à l'avance tous les résultatsa possibles 
    % pour constituer un ensemble dénoté \textbf{espace échantillon} $S$.   
    %
    % \noindent 
    % \textbf{Exemples}  
    % \begin{itemize}
    %     \item [$\rhd$ ] \textbf{Tirages d'une pièce de monnaire} : 
    %     \begin{align*}
    %         S = \{P, F\}
    %     \end{align*} 
    %     \item [$\rhd$ ] Lancement d'un dé : 
    %     \begin{align*}
    %         S = \{ 1, 2, 3, 4, 5, 6 \} 
    %     \end{align*}        
    %     \item [$\rhd$ ] Lancement d'un dé : 
    %     \begin{align*}
    %         S = \big\{ (i, j): i,j \in \{ 1, 2, 3, 4, 5, 6 \} \big\}
    %     \end{align*}                    
    % \end{itemize}   
    %
    % \subsection{Loi des grands nombres}
    % Soit les différents sous-ensembles 
    % $E \subseteq S$ :
    % \begin{align*}
    %   S =  \big\{ \\
    %         &\emptyset, 
    %         \{ 1 \}, \{ 2 \}, \dots, \{ 6 \}, 
    %         \\
    %         &\{ 1, 2 \}, \{ 1, 3 \}, 
    %         \\
    %         &\dots, 
    %         \\
    %         &\{ 1,2,3 \}, 
    %         \\
    %         &\dots, 
    %         \\
    %         &\{ 1, 2, 3, 4, 5, 6 \} 
    %         \\
    %     \big\}
    % \end{align*}
    % $\blacktriangleright$ Par exemple, $\{ 2, 4, 6\} \subseteq S$ 
    % est l'évènement qu'on obtienne un nombre pair. 
    %
    %
    % En  répétant une expérience de manière indépendante et 
    % dans des conditions identiques, 
    % \textbf{la fréquence empirique}  d'un événement donné $E$ 
    % tendra vers sa probabilité théorique $P(E)$.

    \chapter{File d'attente}


    \section{Principe}
    Un file d'attente est introduite lorsque le \textbf{demande} 
    excède la capacité d'\textbf{offre de service}. 
    Or, bien qu'\textbf{augmenter la capacité}  entraîne une 
    amélioration de la \textbf{qualité du service}, 
    cela entraîne aussi une augmentation de \textbf{coût}                   .



    \begin{center}
    \begin{tikzpicture}[
        block/.style={rectangle, draw, minimum height=2em, minimum width=4em}, 
        arrow/.style={-{Triangle[angle=60:1.5mm]}, thick},
        dashedbox/.style={dashed, draw, inner sep=0.5cm, rounded corners},
        node distance=3cm, auto]

        % Nodes
        \node[block] (population) at (0, 0) {Population};

        \node[block]  (queue) at (4, 0) {File d’attente};
        \node[block] (service) at (5.75, -4){Service};
        
        % Dashed box
        \node[dashedbox, fit={(queue) (service)}, label=above:] {};
        
        % Arrows
        \draw[arrow] (population) -- node[above, midway, xshift=-0.35cm] {clients} (queue);
        \draw[arrow] (queue) -- ++(1.75,0) --  (service);
        \draw[arrow] (service) -- ++(2cm,0);

    \end{tikzpicture}
    \end{center}

    La théorie des files d'attente permet de 
    \textbf{modéliser }      
    un tel problème pour aider le gestionnaire à 
    \textbf{prendre une décision éclairée}.  
    \begin{center}
    \begin{tikzpicture}[
        block/.style={rectangle, draw, minimum height=2em, minimum width=4em}, 
        arrow/.style={-{Triangle[angle=60:1.5mm]}, thick},
        dashedbox/.style={dashed, draw, inner sep=0.5cm, rounded corners},
        dot/.style={rectangle, draw, fill=black, minimum size=0.5em},
        redannotation/.style={rectangle, draw=red, thick, text=red, align=center},
        regularnnotation/.style={rectangle, draw=black, thick,align=center},
        node distance=3cm, auto
    ]

    % Nodes
    \node[block] (population) at (0, 0) {Population};
    \node[block] (queue) at (4, 0) {File d’attente};
    \node[block, minimum height=6em, minimum width=3em] (service) at (5.75, -4) {};

    % Dashed box
    \node[dashedbox, fit={(queue) (service)}, label=above:] {};

    % Servers inside the service node
    \node[dot] (server1) at (5.75, -3.25) {};
    \node[dot] (server2) at (5.75, -3.6) {};
    \node (dots) at (5.75, -3.95) {$\vdots$};
    \node[dot] (server3) at (5.75, -4.5) {};

    % Annotation for servers
    \node[redannotation, below=1.5cm of service] (annotation) {Un ou plusieurs serveurs};
    \node[regularnnotation] (annotation2) at (1,-2) {Serveur};
    \draw[arrow, red, dashed] (annotation.north) -- (service.south);

    % Arrows
    \draw[arrow] (population) -- node[above, midway, xshift=-0.35cm] {clients} (queue);
    \draw[arrow] (queue) -- ++(1.75, 0) -- (service);
    \draw[arrow] (service.east) -- ++(1.25cm,0);
    \draw[arrow,dashed] (annotation2.south) -- ++(0,-1) -- (server1.west);

    \end{tikzpicture}
    \end{center}

    \section{Notation}
    \begin{itemize}
        \item [$\rhd$ ] \( s \) : nombre de \textbf{serveurs}  
        \item [$\blacktriangleright$ ] Soient \( n \) client dans le système :
            \begin{itemize}
                \item [$\rhd$ ] \( \lambda_n \) : \textbf{Taux d'arrivée} des clients.
                    On suppose que : 
                \begin{align*}
                    \lambda_n = \lambda 
                \end{align*}
                Autrement dit, le nombre de clients $n$ dans le système n'affecte 
                pas le taux t'arrivée $\lambda$ et donc le 
                taux d'arrivé ne varie par selon $n$ : $\lambda_n = \lambda$. 
                \item [$\rhd$ ] \( \mu_n \) : \textbf{Taux moyen} de service 
                    \textit{pour l'ensemble des serveurs}.  
                \item [$\rhd$ ] \( \mu \) : Taux moyen de service \textbf{par serveur} : 
                    \begin{align*}
                        \mu_n  = s\mu, \quad \forall n \geq s
                    \end{align*}
                \item [$\rhd$ ] \( \sfrac{1}{\lambda_n} \) : Taux moyen entre 
                    \textbf{deux arrivées consécutives} 
            \end{itemize}
                \item [$\rhd$ ] \( \sfrac{1}{\mu} \) : \textbf{Temps}   moyen 
                    de service \textit{par serveur}.   
            \item [$\rhd$ ] \( \rho \) : \textbf{Facteur d'utilisation}   :
                \begin{align*}
                    \rho = \frac{\lambda}{s\mu}
                \end{align*}
    \end{itemize}

    \textcolor{myr}{\textbf{Au début}}, l'état du système 
    correspond à $n$, soit le nombre de client dans la 
    file et le nombre de client \textbf{entrain d'être servis}  

    Le système atteint éventuellement un \textbf{équilibre} où 
    il y a une distribution de probabilités 
    \( \{ P(n) = P_n : n \in \mathbb{N}  \}\)
    \begin{note}{}{}
        L'équilibre est atteint tant que le \textbf{facteur d'utilisation} 
        n'est pas excessif :
        \begin{align*}
            \rho < 1 
        \end{align*}
        Autre, la file \textcolor{myr}{\textbf{croît à l'infini}}.  
    \end{note}

    \begin{itemize}
        \item [$\rhd$ ] \( P_n \) : \textbf{Probabilité}  qu'il y ait \( n \) clients 
        dans le système
    \item [$\rhd$ ] \( L_{\phantom{q}} \) : \textbf{Nombre moyen} de clients dans le système. 
        \item [$\rhd$ ] \( L_q \) : \textbf{Nombre moyen} de client dans la file. 
            Soit $s$ représentant le service :
            \begin{align*}
                L = L_q + n_s
            \end{align*}
        \item [$\rhd$ ] \( W \) : \textbf{Temps moyen} passé par un client 
            \textbf{dans le système}
        \item [$\rhd$ ] \( W_q \) : \textbf{Temps moyen} passé par un client 
            \textbf{dans la file}

    \end{itemize}

    \section{Formule de Little}
    \begin{center}
        \begin{empheq}[box=\othermathbox]{align*}
            &L_{\phantom{q}} = \lambda W
            \\ 
            &L_q = \lambda W_q 
            \\ 
            &W_{\phantom{i}} = W_q + \sfrac{1}{\mu}
        \end{empheq}
    \end{center}

    \section{Loi de Poisson Rappel}
    La \textbf{Loi de Poisson} permet de d'écrire le 
    \textbf{nombre d'évènements} se produisant dans un intervalle 
    de temps donnée. 

    Une distribution d'évènements obéit à la loi de Poisson 
    si elle respecte \textbf{trois conditions} :
    \begin{itemize}
        \item [$\rhd$ ] Les évènements sont \textbf{indépendants}  
        \item [$\rhd$ ] La probabilité que deux évènements se 
            produisent sur un intervalle très court est \textbf{néglibable}.
    \end{itemize}
   \begin{center}
    \begin{tikzpicture}[
        block/.style={rectangle, draw, minimum height=2em, minimum width=4em}, 
        arrow/.style={-{Triangle[angle=60:1.5mm]}, thick},
        dashedbox/.style={dashed, draw, inner sep=0.5cm, rounded corners},
        dot/.style={rectangle, draw, fill=black, minimum size=0.5em},
        client/.style={circle, draw, fill=black, minimum size=0.5em},
        redannotation/.style={rectangle, draw=red, thick, text=red, align=center},
        regularnnotation/.style={rectangle, draw=black, thick,align=center},
        node distance=3cm, auto
    ]

    % Nodes
    \node[block] (population) at (0, 0) {Population};
    \node[block] (queue) at (4, 0) {File d’attente};
    \node[block, minimum height=6em, minimum width=3em] (service) at (5.75, -4) {};

    % Dashed box
    \node[dashedbox, fit={(queue) (service)}, label=above:] {};
    \node[client] (client1) at (2.55, 0.2) {}; 
    \node[client] (client2) at (2.55, -0.2) {}; 

    % Servers inside the service node
    \node[dot] (server1) at (5.75, -3.25) {};
    \node[dot] (server2) at (5.75, -3.6) {};
    \node (dots) at (5.75, -3.95) {$\vdots$};
    \node[dot] (server3) at (5.75, -4.5) {};

    % Annotation for servers
    \node[regularnnotation] (annotation2) at (1,-2) {Serveur};
    \node[] (indication) at (3.25, -1) {\( t = 0.5s\)};
    % Arrows
    \draw[arrow] (population) -- node[above, midway, xshift=-0.35cm] {clients} (queue);
    \draw[arrow] (queue) -- ++(1.75, 0) -- (service);
    \draw[arrow] (service.east) -- ++(1.25cm,0);
    \node[redannotation, below=5.5cm of client2] (annotation) 
        {Deux client ne peuvent pas \\ arriver \textit{exactement en même temps}};

    \draw[arrow,dashed] (annotation2.south) -- ++(0,-1) -- (server1.west);
    \draw[arrow, red, dashed] (annotation.north) -- (client2.south);

    \end{tikzpicture}
    \end{center}
    \begin{itemize}
        \item [$\rhd$ ] La probabilité d'occurence d'un évènement 
            sur un intervalle donné
            est proportionnelle à la taille de l'intervalle 
    \end{itemize}


    \begin{tikzpicture}[
        block/.style={rectangle, draw, minimum height=2em, minimum width=4em}, 
        arrow/.style={-{Triangle[angle=60:1.5mm]}, thick},
        dashedbox/.style={dashed, draw, inner sep=0.5cm, rounded corners},
        dot/.style={rectangle, draw, fill=black, minimum size=0.5em},
        client/.style={circle, draw, fill=black, minimum size=0.5em},
        redannotation/.style={rectangle, draw=red, thick, text=red, align=center},
        regularnnotation/.style={rectangle, draw=black, thick,align=center},
        node distance=3cm, auto
    ]

    % Nodes
    \node[block] (population) at (0, 0) {Population};
    \node[block] (queue) at (4, 0) {File d’attente};

    \node[] (indication2) at (5.25, -1) {\( t = 2 \text{ min}\)};

    \node[] (indication3) at (3.25, -2) {\( \lambda_n = 2\)};

    % Dashed box
    \node[dashedbox, fit={(queue) (indication3) (indication2)}, label=above:] {};
    \node[client] (client1) at (2.55, 0.2) {}; 
    \node[client] (client2) at (2.55, -0.2) {}; 


    \node[client] (client3) at (5.55, 0.6) {}; 
    \node[client] (client4) at (5.55, 0.2) {}; 
    \node[client] (client5) at (5.55, -0.2) {}; 
    \node[client] (client6) at (5.55, -0.6) {}; 

    % Servers inside the service node

    % Annotation for servers
    \node[] (indication) at (3.25, -1) {\( t = 1 \text{ min}\)};

    % Arrows
    \draw[arrow] (population) -- node[above, midway, xshift=-0.35cm] {clients} (queue);
    \node[redannotation, below=2.75cm of client2] (annotation) 
        {Si deux clients arrivent à $t_1$ \\ On s'attend à ce que 
        qu'il y ait  au total $4$ clients à $t_2$ };

    \draw[arrow, red, dashed] (annotation.north) -- (client2.south);

    \end{tikzpicture}
    \vspace{-2em}
    \subsection{Formule mathématique de la loi de poisson}
    Soit $i$, le nombre \textbf{d'évenèment} observé dans un intervalle 
    $\lambda$ le \textbf{taux moyens} d'évènement observables, 
    la probabilité d'observer un évènement $i$ dans cet intervalle 
    est donné par :

        \begin{empheq}[box=\othermathbox]{align*}
        p(i) = \dfrac{\lambda^{i}e^{-\lambda}}{i!}
        \quad 
        i = 0, 1, 2, \dots
        \end{empheq}

    \section{Arrivée des clients}
    Le \textbf{taux d'arrivé des clients} \( \lambda \) est une 
    \textbf{variable aléatoire} qui obéit à la Loi de Poisson. 
    \begin{note}{}{}
        Par concidence, la variable représentant le taux d'arrivé $\lambda$ 
        a le même symbole qu'une variable alétoire quelconque 
        obéissant à la Loi de Poisson et ayant un taux moyen de 
        $\lambda$. 
    \end{note}

    $\blacktriangleright$ Si le taux moyen d'arrivé est de $\lambda = 6$ client 
    pas minutes, la probabilité qu'il y a $2$ clients qui arrivent 
    dans la prochaine minute est de :
    \begin{align*}
         p(2) = \dfrac{6^{2}e^{-6}}{2!}
         \approx 0.0446 = 4.46 \%
    \end{align*}

    \subsection{Espérance du taux d'arrivé}
    \textcolor{myr}{\textbf{L'espérance d'une variable aléatoire 
    \( \boldsymbol{X}  \)}}  correspond à sa 
    \textbf{valeur moyenne sur le long terme}.
    Pour une loi de Poisson de paramètre \( \lambda \), 
    l'espérance est simplement \( \lambda \). 
    Cela signifie que si l' on observe le système sur une longue période, 
    le nombre moyen de clients arrivant par unité de temps sera 
    $\lambda$. Plus formellement, on a :


    \begin{empheq}[box=\othermathbox]{align*}
        E[X] 
        = 
        \sum_{i=0}^{\infty }ip(i) 
        =
        \sum_{i=0}^{\infty }\dfrac{\lambda^{i}e^{-\lambda}}{i!} 
        = 
        \lambda e^{-\lambda}e^{\lambda} = \entoure{$\lambda$}
    \end{empheq}


    \section{Variable aléatoire continue}
    Une \textbf{variable aléatoire continue} peut prendre une infinité 
    de valeur sur un intervalle donnée. La probabilité qu'elle 
    prenne une valeur spécifique est donc nulle, puisque l'ensemble 
    des valeurs possibles est \textbf{infini}. 
    pour étudier 
    une variable aléatoire continue, on utilise plutôt 
    une \textcolor{myr}{\textbf{fonction de densité de probabilité $\textbf{f(x)}$} }  qui satisfait les conditions :
            \begin{align*}
               \rhd \quad \forall x, f(x) \geq 0 \text{ et }  \quad\quad \int_{-\infty}^{\infty }f(x)dx = 1
            \end{align*}
    La \textcolor{myr}{\textbf{probabilité qu'une variable aléatoire continue 
    $\boldsymbol{X}$ prenne une valeur dans l'intervale \( \boldsymbol{[a, b]}   \)}} 
    est donné par :
    \begin{align*}
        P(a \leq X \leq b) = \int_{a}^{b}f(x)dx  
    \end{align*}

    \section{Variable aléatoire exponentielle}
    Il s'agit d'un \textbf{cas particulier} d'une variable aléatoire continue
    qui modélise le \textcolor{myr}{\textbf{temps entre deux évènements qui suivent 
    un processus Poisson}}---des évènements qui arrivent de manière 
    alétoire mais à taux constant. 
    

    Une variable aléatoire exponentielle $X$ a une 
    \textcolor{myr}{\textbf{fonction de densité de probabilité}} :
    \begin{align*}
        f(x) = 
        \begin{cases}
            \lambda e^{-\lambda x} \quad \text{ si } x \geq 0 
            \\ 
            0 \quad\quad\quad\text{ si } x < 0
        \end{cases}
    \end{align*}    

    En intégrant par partie on obtien l'espérence $E[X]$
        \begin{empheq}[box=\othermathbox]{align*}
            E[X] 
            = 
            \int_{-\infty}^{+\infty}\textcolor{myr}{x}f(x)dx 
            =
            \int_{-\infty}^{+\infty}\lambda e^{-\lambda x} =  \dfrac{1}{\lambda}
        \end{empheq}

    \section{Arrivées consécutives des clients }
    Le temps $t$ qui s'écoule entre deux arrivés consécutives 
    est un \textcolor{myr}{\textbf{variable aléatoire exponentielle de paramètre 
    \( \lambda \)}}. La \textbf{fonction de densité de probabilité} est donc    

    \begin{align*}
        f(t) = 
        \begin{cases}
            \lambda e^{-\lambda t} \quad \text{ si } t \geq 0 
            \\ 
            0 \quad\quad\quad \text{ si } t < 0
        \end{cases}
    \end{align*} 

    Soit $T$ une variable aléatoire exponentielle, la probabilité 
    qu'il s'écoule ($T > t$) plus d'une unité de temps entre 
    deux arrivées consicutives de clients est équivalent à 
    la probabilité qu'il n'y ai aucun client aui arrive pendant 
    une unité de temps : 
    \begin{align*}
        P(T > t) 
        = 
        1 - P(T \leq t) 
        &= 
        1 - \int_{0}^{t}f(x)dx 
        \\
        &= 
        1 - \int_{0}^{t}\lambda e^{-\lambda t} 
        = 
        e^{-\lambda t}
    \end{align*}
    On a alors :
    \begin{align*}
        P(T > 1) = e^{\lambda} = P(X = 0) = p(0)
    \end{align*}

    \section{Temps de service}
    Le temps  de service  est 
    \textcolor{myr}{\textbf{variable aléatoire exponentielle de paramètre 
    \( \mu \)}}. La \textbf{fonction de densité de probabilité} est donc : 

    \begin{align*}
        f(t) = 
        \begin{cases}
            \mu e^{-\mu t} \quad \text{ si } t \geq 0 
            \\ 
            0 \quad\quad\quad \text{ si } t < 0
        \end{cases}
    \end{align*} 

    L'espérance, c'est-à-dire le temps moyen de service, 
    est donné par $\sfrac{1}{\mu}$

    
    \newpage
    \chapter{File d'attente avec un serveur}

    De façon générale, \( \boldsymbol{P_n }  \) est la 
    \textcolor{myr}{\textbf{probabilité qu'il y ait \( \boldsymbol{n}  \) 
    client dans le système}}---qu'il s'agisse de clients 
    dans la file ou au service. 

    Il est possible dériver des équations d'états pour 
    montrer que :
    \begin{align*}
        P_n 
        = 
        \dfrac{\lambda_n}{\mu_n}P_0 
        = 
        \left( \dfrac{\lambda}{\mu} \right)^n P_0 = \rho^nP_0, 
        \quad \text{ puisque }  \rho = \dfrac{\lambda}{\mu}
    \end{align*}    


    De façon générale, on a :
    \begin{empheq}[box=\othermathbox]{align*}
        P_0 = \rho^nP_0, \quad n = 1, 2, \dots    
    \end{empheq}

    Puisque $n$ est une variable aléatoire l'espérence de $n$, 
    soit $L$, le nombre moyen de client dans le système est donné par :
    \begin{align*}
        E[n] = L = \sum_{n=0}^{\infty }nP_n 
    \end{align*}


    En dérivant partiellement, on obtient :

    \begin{empheq}[box=\othermathbox]{align*}
        L =  \dfrac{\rho}{1 - \rho}
    \end{empheq}

    Ou, puisqu'on sait que $p = \sfrac{\lambda}{\mu}$, on a :

    \begin{empheq}[box=\othermathbox]{align*}
        L =  \dfrac{\lambda}{\mu - \lambda}
    \end{empheq}

    Les équations de Little demeurent valides. Généralement, 
    il suffit d'observer les variables fournies par une problème 
    pour déduire les inconnues 
    \textcolor{myr}{\textbf{en appliquant les formules de Little}}.
    Lorsqu'on parvient à identifier soit $L$, $L_q$, $W$,  
    ou $W_q$, il est possible de déduire toutes les autres 
    inconnues. 
        \section{Formule de Little}
    \begin{center}
        \begin{empheq}[box=\othermathbox]{align*}
            &L_{\phantom{q}} = \lambda W
            \\ 
            &L_q = \lambda W_q 
            \\ 
            &W_{\phantom{i}} = W_q + \sfrac{1}{\mu}
        \end{empheq}
    \end{center}

    \chapter{File d'attent avec plusieurs serveurs}
    On considère \textbf{deux scénarios} :
    \begin{itemize}
        \item [$\rhd$ ]  \( n < s \) où le service augmente selon le nombre de clients
        \item [$\rhd$ ]  \( n \geq s \) où le le taux moyen de service $\mu$ reste stable à $s\mu$
    \end{itemize}
    \textcolor{myr}{\textbf{Lorsque \( \boldsymbol{n < s}  \)}} , \( \boldsymbol{P_n }  \) la 
    \textcolor{myr}{\textbf{probabilité qu'il y ait \( \boldsymbol{n}  \) 
    clients dans le système}} est donnée par :
    \begin{empheq}[box=\othermathbox]{align*}
        P_n 
        = 
        \dfrac{\left(\sfrac{\lambda}{\mu}\right)^n}{n!}P_0, \quad n = 1, 2, \dots, s
    \end{empheq}    

    \noindent Pour simplifier la notation, on considère le facteur 
    $C_n = \dfrac{\left(\sfrac{\lambda}{\mu}\right)^n}{n!}$ : 
    \begin{empheq}[box=\othermathbox]{align*}
        P_n 
        = 
        C_nP_0, \quad n = 1, 2, \dots, s
    \end{empheq}


    \textcolor{myr}{\textbf{Lorsque \( \boldsymbol{n \geq s}  \)}} , \( \boldsymbol{P_n }  \) la 
    \textcolor{myr}{\textbf{probabilité qu'il y ait \( \boldsymbol{n}  \) 
    clients dans le système}} est donnée par :
    \begin{empheq}[box=\othermathbox]{align*}
        P_n 
        = 
        \dfrac{\left(\sfrac{\lambda}{\mu}\right)^n}{s^{n-s}s!}P_0, \quad n = s + 1, s + 2, \dots
    \end{empheq}    
    \noindent Pour simplifier la notation, on considère le facteur 
    $C_n$ :
    \begin{empheq}[box=\othermathbox]{align*}
        \dfrac{\left(\sfrac{\lambda}{\mu}\right)^n}{s^{n-s}s!} 
    \end{empheq}                    

    Puisque \( P_n \) est une \textbf{distribution de probabilité}, nous avons 
    \( \sum_{n=0}^{\infty }P_n = 1\). De là, nous pouvons montrer que :
    \begin{align*}
        \sum_{n=0}^{\infty }P_n = P_0 + \sum_{n=1}^{\infty }P_n  = P_0 + \sum_{n=1}^{\infty }C_nP_0 = 1
    \end{align*}

    Nous pouvons réarranger les expressions pour montrer que :
    \begin{align*}
        P_0 = 
        \dfrac{1}{1+\sum_{n=1}^{\infty }C_n} 
        = 
        \left(
            1 + \sum_{n=1}^{\infty }C_n
        \right)^{-1}
    \end{align*}

    En simplifiant l'expression, on obtient la formule qui permet 
    de déterminer la \textcolor{myr}{\textbf{probabilité d'être à l'état 0 dans 
    une file d'attente de plusieurs serveur }} :
    \begin{empheq}[box=\othermathbox]{align*}
        \left( 
            \sum_{n=0}^{s-1}
            \dfrac{\left(\sfrac{\lambda}{\mu}\right)^{n}}{n!} 
            +   
            \left(
                \dfrac{\left(\sfrac{\lambda}{\mu}\right)^n}{s!} 
                \cdot
                \dfrac{1}{1 - \rho}
            \right)
        \right)^{-1}
    \end{empheq}        

    Soit 
    $L_q = 0\cdot P_0, + \dots + 0\cdot P_{s-1} + \sum_{n=s}^{\infty }(n-1)P_n$, 
    il est possible de dériver l'expression pour 
    obtenir une formule simplifiée de $L_q$ : 

    \begin{empheq}[box=\othermathbox]{align*}
        L_q
        =
        \dfrac{P_0\left( \sfrac{\lambda}{\mu} \right)^s}{s! \left(1 - \rho\right)^2}
    \end{empheq}

    \chapter{Problème du Camelo}
    Le problème illustre une situation où un camelot commande des journaux 
    \textbf{la veille} et veut les vendre \textbf{le lendemain}. Dans ce scénario, 
    la \textbf{demande} est une variable aléatoire discrète 
    puisqu'il est impossible 
    de savoir à l'avance la quantité de personne qui achetera 
    des journaux. 

    \section{Notation}
    \begin{itemize}
        \item [$\rhd$ ] \( x \) : Nombre de journaux commandés la veille 
        \item [$\rhd$ ] \( c \) : Coût de chaque journal
        \item [$\rhd$ ] \( p \) : Prix de vente de chaque journal
        \item [$\rhd$ ] \( D \) : Variable aléatoire représentant la demande
    \end{itemize}
    Soient les \textbf{bornes inférieure et supérieure} 
    \( \underline{d} \) et \( \overline{d} \) respectivement, 
    on a la fonction de probabilité suivante : 
    \begin{align*}
        p(d) = P(D = d), \quad \underline{d} \leq d \leq \overline{d}
    \end{align*}

    Si le demande est inférieure à la quantité de journaux commandé la veille, 
    le camelot vendras \textbf{autant de journaux que la demande}, soit \( d \).
    Sinon, il vendra la \textbf{totalité des journaux qu'il a commandé}, soit \( x \) : 
    \begin{align*}
        &\text{Journaux vendus} 
        = 
        \min\{d, x\}
        = 
        \begin{cases}
            d \quad \text{ si } d < x
            \\
            x \quad \text{ si } d > x
        \end{cases}
        \\
        &\text{Profit} : A(d, x) = -cx + p \min\{ d, x\}
    \end{align*}        

    Le profit espéré, la moyenne du profit sur le long terme est, est 
    simplement l'espérance de la variable aléatoire du profit, \( A(d,x)) \) :

    \begin{align*}
        E(D(D, x)) 
        &= 
        \sum_{d=\underline{d}}^{\overline{d}}A(d, x)P(D = d)
        = 
        \sum_{d=\underline{d}}^{\overline{d}}(-cx + p)P(D = d)
        \\
        &= 
        -cx \sum_{d=\underline{d}}^{\overline{d}}P(D = d) 
        +
        p \sum_{d=\underline{d}}^{\overline{d}}\min\{d, x\}P(D=d)
        \\
        &= 
        -cx(1)\phantom{\sum_{d=\underline{d}}^{\overline{d}}P(D=)}
        + 
        p \sum_{d=\underline{d}}^{\overline{d}}\min\{d, x\}P(D=d)
        \\
        &=
        -cx \phantom{\sum_{d=\underline{d}}^{\overline{d}}P(D=d1)}
        +
        p \sum_{d=\underline{d}}^{\overline{d}}\min\{d, x\}P(D=d)
    \end{align*}            


    On a alors la formule du profit espéré 
    \( A(x) \) :
    \begin{align*}
        \underbrace{E(A(D, x))}_{A(x)}
        =
        -cx 
        +
        p \sum_{d=\underline{d}}^{\overline{d}}\min{d, x}P(D=d)
    \end{align*}

    Pour maximiser le profit espérer, il faut donc déterminer la quantité 
    optimale \( x^* \) de journaux à commander :
    \begin{empheq}[box=\othermathbox]{align*}
        \max_{\underline{d} \leq x \leq \overline{d}} 
        \left\{A(x)\right\} 
        = 
        \max_{\underline{d} \leq x \leq \overline{d}} 
        \left\{ 
            -cx 
            +
            p \sum_{d=\underline{d}}^{\overline{d}}\min\{d, x\}P(D=d)
        \right\}
    \end{empheq}


    \newpage

    \chapter{Modèle inventaire}
    Il s'agit d'un problème où déterminer \textbf{quand acheter}  
    ou \textbf{combien} de lot il faut acheter, 
    en considérant les contraintes de coût d'inventaire, 
    de coût fixe par lot, de coût de lot par unité de produit 
    de demande, etc. 

   \begin{center}
    \begin{tikzpicture}[
        block/.style={rectangle, draw, minimum height=2em, minimum width=4em}, 
        arrow/.style={-{Triangle[angle=60:1.5mm]}, thick},
        dashedbox/.style={dashed, draw, inner sep=0.5cm, rounded corners},
        dot/.style={rectangle, draw, fill=black, minimum size=0.5em},
        client/.style={circle, draw, fill=black, minimum size=0.5em},
        redannotation/.style={rectangle, draw=red, thick, text=red, align=center},
        regularnnotation/.style={rectangle, draw=black, thick,align=center},
        node distance=3cm, auto
    ]

    % Nodes
    \node[block] (population) at (0, 0) {Usine};
    \node[block] (queue) at (4, 0) {Iventaire};
    \node[block, minimum height=6em, minimum width=3em] (service) at (5.75, -4) {};

    \node[regularnnotation] (clients) at (5.75, -5.5) {clients};
    % Dashed box
    \node[dashedbox, fit={(queue) (service) (clients)}, label=above:] {};


    % Servers inside the service node
    \node[dot] (server1) at (5.75, -3.25) {};
    \node[dot] (server2) at (5.75, -3.6) {};
    \node (dots) at (5.75, -3.95) {$\vdots$};
    \node[dot] (server3) at (5.75, -4.5) {};

    % Annotation for servers
    % Arrows
    \draw[arrow] (population) -- (queue);
    \draw[arrow] (queue) -- ++(1.75, 0) -- (service);


    \end{tikzpicture}
    \end{center}

    \section{Modèle de commande périodique en lot}
    On considère que la gestion se fait en considérant 
    le demande et en lots et en décidant quant à 
    la \textbf{quantité optimale}  \( Q^{*} \) 
    d'unité de produit dans un lot à acheter selon \textbf{contraintes de coût}  

    \subsection{Notation}
    \begin{itemize}
        \item [$\rhd$ ] \( a \) : Taux de \textbf{demande}---nb.
            unités de produit par unité de temps  
        \item [$\rhd$ ] \( K \) : Coût \textbf{fixe} d'aquisition d'\textbf{un lot}    
        \item [$\rhd$ ] \( c \) : Coût d'acquisition par unité du produit 
        \item [$\rhd$ ] \( h \) Coût d'inventaire par unité du produit par unité de temps 
        \item [$\rhd$ ] \( Q \) Nombre d'\textbf{unités de produit} das un lot  
    \end{itemize}   


    \begin{tikzpicture}[scale=1.5, >=stealth]

    % Axes
    \draw[->] (0,0) -- (4,0) node[below right] {temps};
    \draw[->] (0,0) -- (0,3) {};

    % Labels sur les axes
    \node[above left] at (0,2.5) {$Q$};
    \node[below] at (3,0) {$Q/\alpha$};

    % Approximation continue (ligne décroissante continue)
    \draw[thick] (0,2.5) -- (3,0);

    % Approximation discrète (marche en escalier)
    \draw[dashed, blue] (0,2.5) -- (0.5,2.5);
    \draw[dashed, blue] (0.5,2.5) -- (0.5,2);
    \draw[dashed, blue] (0.5,2) -- (1,2);
    \draw[dashed, blue] (1,2) -- (1,1.5);
    \draw[dashed, blue] (1,1.5) -- (1.5,1.5);
    \draw[dashed, blue] (1.5,1.5) -- (1.5,1);
    \draw[dashed, blue] (1.5,1) -- (2,1);
    \draw[dashed, blue] (2,1) -- (2,0.5);
    \draw[dashed, blue] (2,0.5) -- (2.5,0.5);
    \draw[dashed, blue] (2.5,0.5) -- (2.5,0);

    % Texte pour l'approximation continue
    \node[right] at (3.2,1.5) {\rotatebox{-90}{Approximation continue}};

    % Pointillés sur les axes
    \draw[dotted] (3,0) -- (3,2.5);
    \draw[dotted] (0,2.5) -- (3,2.5);

    \end{tikzpicture}
    La logeur d'un cycle est donnée par \( \dfrac{Q}{a} \). En effet, 
    si on commande $Q$ unité et vend un quantité $a = Q$, il restera
    $Q - Q = 0$ unités à la fin du cycle.  

    En considérant \( Q > 0\), le niveau \textbf{moyen d'inventaire } et 
    le \textbf{coût moyen d'inventaire} respectivement, son donné par 
    \begin{align*}
        \sfrac{Q}{2}, \quad \sfrac{hQ}{2}
    \end{align*}

    On a alors le \textcolor{myr}{\textbf{coût d'inventaire par cycle}} :
    \begin{align*}
        \sfrac{hQ}{2} \cdot \sfrac{Q}{a} = \sfrac{hQ^2}{2a}
    \end{align*}

    Soient le \textcolor{myr}{\textbf{coût d'inventaire par cycle }} 
    \( \sfrac{hQ^2}{2a} \), 
    et le  \textcolor{myr}{\textbf{coût d'acquisition par cycle}} 
    \(  K + cQ\), on a le coût total par cycle :
    \begin{align*}
        K + cQ + \sfrac{hQ^2}{2a}
    \end{align*}

    Le coût moyen par unité de temps est alors :
    \begin{align*}
        T(Q) 
        =
        \dfrac{K + cQ + \sfrac{hQ^2}{2a}}{\sfrac{Q}{a}}
    \end{align*}

    En dérivant paritiellement, on obtient :
    \begin{empheq}[box=\othermathbox]{align*}
        Q^{*} = \sqrt{\dfrac{2aK}{h}}
    \end{empheq}                    


    \newpage    
    \section{Modèle de demande variable}
    On divise le temps en \textbf{périodes} et suppose que la demande 
    peut varier selon la période.
    \subsection{Notation}
    \begin{itemize}
        \item [$\rhd$ ] \( r_n \) : Nombre d'\textbf{unités du produit} demandées.
            à la période \( n = 1, \dots, N \) 
        \item [$\rhd$ ] \( K \) : 
            \textbf{Coût fixe} d'acquisition d'unités du produit.    
        \item [$\rhd$ ] \( c \) : Coût d'acquisition \textbf{par unité du produit}.  
        \item [$\rhd$ ] \( h \) : \textbf{Coût d'inventaire} par unité de produit par 
            période. 
    \end{itemize}

    Il faut donc déterminer le nombre d'unité du produit 
    à obtenir \textcolor{myr}{\textbf{au début de chaque période}}
    afin de \textbf{minimiser le coût} sur \( N \)
    périodes \textcolor{myr}{\textbf{tout en satisfaisant la demande}}.  

    \subsection{Programmation dynamique}
    Soient \( n \) \textbf{périodes}, un état \( s_n \) correspondant au nombre 
    d'\textbf{unités en inventaire}   au début de la période \( n \), et 
    une variable \( x_n \) correspondant au \textbf{nombre d'unités à acquérir}
    au début de la périodes \( n \), la contribution directe 
    à la valeur de l'objectif est donnée par :
    \begin{align*}
        B_n(s_, x_n) = 
            \begin{cases}
                K + cx_n + h(s_n + x_n - r_n) \quad \text{ si } x_n > 0 
                \\ 
                h(s -r_n) \quad\quad\quad\quad\quad\quad\quad\;\;\, \text{ si } x_n = 0
            \end{cases}
    \end{align*}

    Le coût d'acquisition jusqu'à la fin 
    de la période \( \textcolor{myr}{\boldsymbol{N }  }  \) si 
    \( s_n \) unités sont en inventaire et la décision 
    est prise d'acquérir \( x_n \) nouvelles unités au début de la 
    période \( n \) est alors :
    \begin{align*}
        f_n(s_n, x_n) = B_n(s_n, x_n) + f^{n+1}(s_n + x_n -r_n)
    \end{align*}

    Le coût \textcolor{myr}{\textbf{coût minimal d'acquisition}}, \( f_n(s_n) \) 
    est donné par :
    \begin{align*}
        f_n^{s_n} 
        = 
        \min_{x_n \geq \max\{0, r_n -s_n \}}
        \Bigl\{ f_n (s_n, x_n) \Bigr\}
        = 
        f_n(s_n, x_n^{*})
    \end{align*}


    On a donc la formule de récurrence : 
    \begin{empheq}[box=\othermathbox]{align*}
        f_n^{*}(s_n) 
        = 
        \min_{x_n \geq \max\{0, r_n -s_n \}}
        \Bigl\{ 
            B_n(s_n, x_n) + f_{n+1}(s_n + x_n - r)
        \Bigr\}
    \end{empheq}                    
    



    

    








\end{multicols*}
\end{document}

                        

 
