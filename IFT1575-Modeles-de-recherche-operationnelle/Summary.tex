\documentclass{report}
%\usepackage[utopia]{mathdesign}
%\usepackage{amsmath, amsthm}


\usepackage{amsmath,amsfonts,amsthm,amssymb,mathtools}
%\usepackage[varbb]{newpxmath}
%\usepackage[osf,largesc,theoremfont]{newpxtext}
%\usepackage{coelacanth}
%\usepackage{beraserif} % Bitstream Vera Serif font
%\usepackage{berasans} % Bitstream Vera Sans font
%\usepackage{beramono} % Bitstream Vera Sans Mono font
%\usepackage{berasans}
%\usepackage{libertine}
%\usepackage{mathpazo}
%\usepackage{palatino}
%\usepackage{crimson}

% NEW ------- For pointilles lines
\usepackage{multido}

%% Choose one of the following (if not choosing the  
%% default, viz., Computer Modern, font family):
%\usepackage{lmodern}
\usepackage{bold-extra}
%%
%\usepackage{mathpazo}
% \usepackage{newpxmath}
%\usepackage{kpfonts} % Very good
%%
%\usepackage{mathptmx} %Very good
%\usepackage{stix} 
%\usepackage{txfonts} %Very good
\usepackage{newtxtext,newtxmath} %Very good
%%
%\usepackage{libertine} \usepackage[libertine]{newtxmath}
%\usepackage{libertine,libertinust1math} % added 2019/11/28
%%
%\usepackage{newpxtext} \usepackage[euler-digits]{eulervm}
%\usepackage{textcomp}
%\usepackage{bm}
\usepackage{contour}
\usepackage{adjustbox}





\input{/home/cryptopsy/Semesters/LaTeXTemplates/UniversalTeXTemplate/preamble.tex}
%From M275 "Topology" at SJSU
\newcommand{\id}{\mathrm{id}} % Identité
\newcommand{\taking}[1]{\xrightarrow{#1}} % Flèche avec annotation
\newcommand{\inv}{^{-1}} % Inverse

%From M170 "Introduction to Graph Theory" at SJSU
\DeclareMathOperator{\diam}{diam} % Diamètre
\DeclareMathOperator{\ord}{ord} % Ordre
\newcommand{\defeq}{\overset{\mathrm{def}}{=}} % Défini comme égal

%From the USAMO .tex files
\newcommand{\ts}{\textsuperscript} % Exposant
\newcommand{\dg}{^\circ} % Degré
\newcommand{\ii}{\item} % Item

% % From Math 55 and Math 145 at Harvard
% \newenvironment{subproof}[1][Proof]{%
% \begin{proof}[#1] \renewcommand{\qedsymbol}{$\blacksquare$}}%
% {\end{proof}}

\newcommand{\liff}{\leftrightarrow} % Si et seulement si
\newcommand{\lthen}{\rightarrow} % Implique
\newcommand{\opname}{\operatorname} % Opérateur générique
\newcommand{\surjto}{\twoheadrightarrow} % Flèche surjective
\newcommand{\injto}{\hookrightarrow} % Flèche injective
\newcommand{\On}{\mathrm{On}} % Ordinaux
\DeclareMathOperator{\img}{im} % Image
\DeclareMathOperator{\Img}{Im} % Image
\DeclareMathOperator{\coker}{coker} % Cokernel
\DeclareMathOperator{\Coker}{Coker} % Cokernel
\DeclareMathOperator{\Ker}{Ker} % Noyau
\DeclareMathOperator{\rank}{rank} % Rang
\DeclareMathOperator{\Spec}{Spec} % Spectre
\DeclareMathOperator{\Tr}{Tr} % Trace
\DeclareMathOperator{\pr}{pr} % Projection
\DeclareMathOperator{\ext}{ext} % Extension
\DeclareMathOperator{\pred}{pred} % Prédécesseur
\DeclareMathOperator{\dom}{dom} % Domaine
\DeclareMathOperator{\ran}{ran} % Image (range)
\DeclareMathOperator{\Hom}{Hom} % Homomorphisme
\DeclareMathOperator{\Mor}{Mor} % Morphismes
\DeclareMathOperator{\End}{End} % Endomorphisme

\newcommand{\eps}{\epsilon} % Épsilon
\newcommand{\veps}{\varepsilon} % Variance d'épsilon
\newcommand{\ol}{\overline} % Ligne au-dessus
\newcommand{\ul}{\underline} % Ligne en-dessous
\newcommand{\wt}{\widetilde} % Tilde large
\newcommand{\wh}{\widehat} % Chapeau large
\newcommand{\vocab}[1]{\textbf{\color{blue} #1}} % Texte en gras et bleu
\providecommand{\half}{\frac{1}{2}} % Fraction 1/2
\newcommand{\dang}{\measuredangle} % Angle dirigé
\newcommand{\ray}[1]{\overrightarrow{#1}} % Ray
\newcommand{\seg}[1]{\overline{#1}} % Segment
\newcommand{\arc}[1]{\wideparen{#1}} % Arc
\DeclareMathOperator{\cis}{cis} % cis
\DeclareMathOperator*{\lcm}{lcm} % Plus petit commun multiple
\DeclareMathOperator*{\argmin}{arg min} % Argument du minimum
\DeclareMathOperator*{\argmax}{arg max} % Argument du maximum
\newcommand{\cycsum}{\sum_{\mathrm{cyc}}} % Somme cyclique
\newcommand{\symsum}{\sum_{\mathrm{sym}}} % Somme symétrique
\newcommand{\cycprod}{\prod_{\mathrm{cyc}}} % Produit cyclique
\newcommand{\symprod}{\prod_{\mathrm{sym}}} % Produit symétrique
\newcommand{\Qed}{\begin{flushright}\qed\end{flushright}} % QED aligné à droite
\newcommand{\parinn}{\setlength{\parindent}{1cm}} % Indentation de paragraphe à 1 cm
\newcommand{\parinf}{\setlength{\parindent}{0cm}} % Pas d'indentation de paragraphe
% \newcommand{\norm}{\|\cdot\|} % Norme
\newcommand{\inorm}{\norm_{\infty}} % Norme infinie
\newcommand{\opensets}{\{V_{\alpha}\}_{\alpha\in I}} % Ensemble ouvert
\newcommand{\oset}{V_{\alpha}} % Ensemble ouvert V
\newcommand{\opset}[1]{V_{\alpha_{#1}}} % Ensemble ouvert V avec indice
\newcommand{\lub}{\text{lub}} % Plus petite borne supérieure
\newcommand{\del}[2]{\frac{\partial #1}{\partial #2}} % Dérivée partielle
\newcommand{\Del}[3]{\frac{\partial^{#1} #2}{\partial^{#1} #3}} % Dérivée partielle d'ordre élevé
\newcommand{\deld}[2]{\dfrac{\partial #1}{\partial #2}} % Dérivée partielle avec dfrac
\newcommand{\Deld}[3]{\dfrac{\partial^{#1} #2}{\partial^{#1} #3}} % Dérivée partielle d'ordre élevé avec dfrac
\newcommand{\lm}{\lambda} % Lambda
\newcommand{\uin}{\mathbin{\rotatebox[origin=c]{90}{$\in$}}} % Appartient, tourné de 90 degrés
\newcommand{\usubset}{\mathbin{\rotatebox[origin=c]{90}{$\subset$}}} % Sous-ensemble, tourné de 90 degrés
\newcommand{\lt}{\left} % Gauche
\newcommand{\rt}{\right} % Droite
\newcommand{\bs}[1]{\boldsymbol{#1}} % Symbole en gras
\newcommand{\exs}{\exists} % Il existe
\newcommand{\st}{\strut} % Strut
\newcommand{\dps}[1]{\displaystyle{#1}} % Disposition en ligne

\newcommand{\sol}{\setlength{\parindent}{0cm}\textbf{\textit{Solution:}}\setlength{\parindent}{1cm} } % Solution sans indentation initiale puis rétablie
\newcommand{\solve}[1]{\setlength{\parindent}{0cm}\textbf{\textit{Solution: }}\setlength{\parindent}{1cm}#1 \Qed}

\newcommand{\entoure}[1]{\fcolorbox{black}{gray!30}{\texttt{#1}}}

\renewcommand{\ttdefault}{cmtt}
\newcommand{\textttbf}[1]{\contour{yellow!45}{\texttt{#1}}}
\newcommand{\varitem}[3][black]{%
    \item [%
        \colorbox{#2}{\textcolor{#1}{\makebox(5.5,7){#3}}}%
    ]
}
% Allow you to do the non implication (implication barred)
\newcommand{\notimplies}{%
  \mathrel{{\ooalign{\hidewidth$\not\phantom{=}$\hidewidth\cr$\implies$}}}}


\newcommand*{\authorimg}[1]%
    { \raisebox{-1\baselineskip}{\includegraphics[width=\imagesize]{#1}}}
\newlength\imagesize 

\input{/home/cryptopsy/Semesters/LaTeXTemplates/UniversalTeXTemplate/letterfonts.tex}
% lstlistingsEnvs.tex

\usepackage{minted}


\lstset{
  basicstyle=\ttfamily, % Set
  columns=fullflexible,
  keepspaces=true,
  language=Python % You can specify the language if you want syntax highlighting
}

%%%%%%%%%%%%%%%%%%%%%%%%%%%%%%%%%%%%%%%%%%%%%%%%%%%%%%%%%%%%%%%%%%%%%%%%%%%%%%%%%%%%%%%%%%%%%%%%%
%                                 Custom lstlisting Environments
%%%%%%%%%%%%%%%%%%%%%%%%%%%%%%%%%%%%%%%%%%%%%%%%%%%%%%%%%%%%%%%%%%%%%%%%%%%%%%%%%%%%%%%%%%%%%%%%%
% Gruvbox style for Python
\definecolor{Pgruvbox-bg}{HTML}{282828}
\definecolor{Pgruvbox-fg}{HTML}{ebdbb2}
\definecolor{Pgruvbox-red}{HTML}{fb4934}
\definecolor{Pgruvbox-green}{HTML}{b8bb26}
\definecolor{Pgruvbox-yellow}{HTML}{fabd2f}
\definecolor{Pgruvbox-blue}{HTML}{83a598}
\definecolor{Pgruvbox-purple}{HTML}{d3869b}
\definecolor{Pgruvbox-aqua}{HTML}{8ec07c}
\definecolor{BBBlack}{rgb}{0.05, 0.06, 0.09}



% JAVA LSTLISTING STYLE IN Gruvbox Colorscheme
\definecolor{gruvbox-bg}{rgb}{0.282, 0.247, 0.204}
\definecolor{gruvbox-fg1}{rgb}{0.949, 0.898, 0.776}
\definecolor{gruvbox-fg2}{rgb}{0.871, 0.804, 0.671}
\definecolor{gruvbox-red}{rgb}{0.788, 0.255, 0.259}
\definecolor{gruvbox-green}{rgb}{0.518, 0.604, 0.239}
\definecolor{gruvbox-yellow}{rgb}{0.914, 0.808, 0.427}
\definecolor{gruvbox-blue}{rgb}{0.353, 0.510, 0.784}
\definecolor{gruvbox-purple}{rgb}{0.576, 0.412, 0.659}
\definecolor{gruvbox-aqua}{rgb}{0.459, 0.631, 0.737}
\definecolor{gruvbox-gray}{rgb}{0.518, 0.494, 0.471}

\definecolor{lst-bg}{RGB}{45, 45, 45}
\definecolor{lst-fg}{RGB}{220, 220, 204}
\definecolor{lst-keyword}{RGB}{215, 186, 125}
\definecolor{lst-comment}{RGB}{117, 113, 94}
\definecolor{lst-string}{RGB}{163, 190, 140}
\definecolor{lst-number}{RGB}{181, 206, 168}
\definecolor{lst-type}{RGB}{218, 142, 130}

\lstdefinestyle{PythonGruvbox}{
    language=Python,
    identifierstyle=\color{lst-fg},
    basicstyle=\ttfamily\color{Pgruvbox-fg},
    keywordstyle=\color{Pgruvbox-yellow},
    keywordstyle=[2]\color{Pgruvbox-blue},
    stringstyle=\color{Pgruvbox-green},
    commentstyle=\color{Pgruvbox-aqua},
    backgroundcolor=\color{BBBlack},
    rulecolor=\color{BBBlack},
    showstringspaces=false,
    keepspaces=true,
    captionpos=b,
    breaklines=true,
    tabsize=4,
    showspaces=false,
    numbers=left,
    numbersep=5pt,
    numberstyle=\tiny\color{gray},
    showtabs=false,
    columns=fullflexible,
    morekeywords={True,False,None},
    morekeywords=[2]{and,as,assert,break,class,continue,def,del,elif,else,except,exec,
    finally,for,from,global,if,import,in,is,lambda,nonlocal,not,or,pass,print,raise,
    return,try,while,with,yield},
    morecomment=[s]{"""}{"""},
    morecomment=[s]{'''}{'''},
    morecomment=[l]{\#},
    morestring=[b]",
    morestring=[b]',
    literate=
    {0}{{\textcolor{Pgruvbox-purple}{0}}}{1}
    {1}{{\textcolor{Pgruvbox-purple}{1}}}{1}
    {2}{{\textcolor{Pgruvbox-purple}{2}}}{1}
    {3}{{\textcolor{Pgruvbox-purple}{3}}}{1}
    {4}{{\textcolor{Pgruvbox-purple}{4}}}{1}
    {5}{{\textcolor{Pgruvbox-purple}{5}}}{1}
    {6}{{\textcolor{Pgruvbox-purple}{6}}}{1}
    {7}{{\textcolor{Pgruvbox-purple}{7}}}{1}
    {8}{{\textcolor{Pgruvbox-purple}{8}}}{1}
    {9}{{\textcolor{Pgruvbox-purple}{9}}}{1}
}

% Gruvbox style for Java
\definecolor{gruvbox-bg}{rgb}{0.282, 0.247, 0.204}
\definecolor{gruvbox-fg1}{rgb}{0.949, 0.898, 0.776}
\definecolor{gruvbox-fg2}{rgb}{0.871, 0.804, 0.671}
\definecolor{gruvbox-red}{rgb}{0.788, 0.255, 0.259}
\definecolor{gruvbox-green}{rgb}{0.518, 0.604, 0.239}
\definecolor{gruvbox-yellow}{rgb}{0.914, 0.808, 0.427}
\definecolor{gruvbox-blue}{rgb}{0.353, 0.510, 0.784}
\definecolor{gruvbox-purple}{rgb}{0.576, 0.412, 0.659}
\definecolor{gruvbox-aqua}{rgb}{0.459, 0.631, 0.737}
\definecolor{gruvbox-gray}{rgb}{0.518, 0.494, 0.471}

\lstdefinestyle{JavaGruvbox}{
    language=Java,
    basicstyle=\ttfamily\color{Pgruvbox-fg},
    keywordstyle=\color{Pgruvbox-yellow},
    keywordstyle=[2]\color{lst-type},
    commentstyle=\itshape\color{lst-comment},
    stringstyle=\color{lst-string},
    numberstyle=\color{lst-number},
    backgroundcolor=\color{BBBlack},
    rulecolor=\color{gruvbox-aqua},
    showstringspaces=false,
    keepspaces=true,
    captionpos=b,
    breaklines=true,
    tabsize=4,
    showspaces=false,
    showtabs=false,
    columns=fullflexible,
    morekeywords={var},
    morekeywords=[2]{boolean, byte, char, double, float, int, long, short, void},
    morecomment=[s]{/}{/},
    morecomment=[l]{//},
    morestring=[b]",
    morestring=[b]',
    numbers=left,
    numbersep=5pt,
    numberstyle=\tiny\color{gray},
}

% Dracula style for Java
\definecolor{draculawhite-background}{RGB}{237, 239, 252}
\definecolor{draculawhite-comment}{RGB}{98, 114, 164}
\definecolor{draculawhite-keyword}{RGB}{189, 147, 249}
\definecolor{draculawhite-string}{RGB}{152, 195, 121}
\definecolor{draculawhite-number}{RGB}{249, 189, 89}
\definecolor{draculawhite-operator}{RGB}{248, 248, 242}

\lstdefinestyle{JavaDraculaWhite}{
    language=Java,
    backgroundcolor=\color{draculawhite-background},
    commentstyle=\itshape\color{draculawhite-comment},
    keywordstyle=\color{draculawhite-keyword},
    stringstyle=\color{draculawhite-string},
    basicstyle=\ttfamily\footnotesize\color{black},
    identifierstyle=\color{black},
    keywordstyle=\color{draculawhite-keyword}\bfseries,
    morecomment=[s][\color{draculawhite-comment}]{/**}{*/},
    showstringspaces=false,
    showspaces=false,
    breaklines=true,
    %frame=single,
    rulecolor=\color{draculawhite-operator},
    tabsize=2,  
    numbers=left,
    numbersep=4pt,
    numberstyle=\ttfamily\tiny\color{gray}
}

% Dracula style for Python
\definecolor{draculawhite-bg}{HTML}{FAFAFA}
\definecolor{draculawhite-fg}{HTML}{282A36}
\definecolor{pdraculawhite-keyword}{HTML}{BD93F9}
\definecolor{pdraculawhite-comment}{HTML}{6272A4}
\definecolor{draculawhite-number}{HTML}{FF79C6}

\lstdefinestyle{PythonDraculaWhite}{
    language=Python,
    basicstyle=\ttfamily\small\color{draculawhite-fg},
    backgroundcolor=\color{draculawhite-background},
    keywordstyle=\color{orange}\bfseries,
    stringstyle=\color{draculawhite-string},
    commentstyle=\color{pdraculawhite-comment}\itshape,
    numberstyle=\color{draculawhite-number},
    showstringspaces=false,
    showspaces=false,
    breaklines=true,
    frame=single,
    rulecolor=\color{draculawhite-operator}, 
    tabsize=4,
    morekeywords={as,with,1,2,3,4, 5,6,7,8,9,True,False},
    numbers=left,
    numbersep=5pt,
    numberstyle=\small\bfseries\ttfamily\color{htmlcomment},
}

% Dracula Dark style for HTML
\definecolor{htmltag}{HTML}{ff79c6}
\definecolor{htmlattr}{HTML}{f1fa8c}
\definecolor{htmlvalue}{HTML}{bd93f9}
\definecolor{htmlcomment}{HTML}{6272a4}
\definecolor{htmltext}{HTML}{401E31}
\definecolor{htmlbackground}{HTML}{282a36}
\definecolor{comphtmlbackground}{HTML}{8093FF}

\lstdefinestyle{HTMLDraculaDark}{
    basicstyle=\normalsize\bfseries\ttfamily\color{htmltext},
    commentstyle=\itshape\color{htmlcomment},
    keywordstyle=\bfseries\color{htmltag},
    stringstyle=\color{htmlvalue},
    emph={DOCTYPE,html,head,body,div,span,a,script},
    emphstyle={\color{htmltag}\bfseries},
    sensitive=true,
    showstringspaces=false,
    backgroundcolor=\color{white},
    inputencoding=utf8,
    extendedchars=true,
    language=HTML,
    tabsize=4,
    breaklines=true,
    breakatwhitespace=true,
    numbers=left,
    numbersep=10pt,
    numberstyle=\small\bfseries\ttfamily\color{htmlcomment},
    escapeinside={<@}{@>},
    rulecolor=\color{htmlbackground},
    xleftmargin=10pt,
    frame=none, 
    breaklines=true,
    postbreak=\mbox{\textcolor{gray}{$\hookrightarrow$}\space},
    showlines=false,
    moredelim=[s][\itshape\color{htmlcomment}]{<!--}{-->},
    morekeywords={id,class,type,name,value,placeholder,checked,src,href,alt},
    literate={é}{{\'e}}1 {è}{{\`e}}1 {ê}{{\^e}}1 {ë}{{\"e}}1 {à}{{\`a}}1 {ù}{{\`u}}1 {û}{{\^u}}1 {ç}{{\c{c}}}1 {â}{{\^a}}1 {î}{{\^i}}1 {ï}{{\"i}}1
}


\lstdefinestyle{Haskell}{
  frame=none,
  xleftmargin=2pt,
  stepnumber=1,
  numbers=left,
  numbersep=5pt,
  numberstyle=\ttfamily\tiny\color[gray]{0.3},
  belowcaptionskip=\bigskipamount,
  captionpos=b,
  escapeinside={*'}{'*},
  language=haskell,
  tabsize=2,
  emphstyle={\bf},
  %commentstyle=\it,
  stringstyle=\mdseries\ttfamily,
  showspaces=false,
  keywordstyle=\bfseries\ttfamily,
  columns=flexible,
  basicstyle=\small\ttfamily,
  showstringspaces=false,
  morecomment=[l]\%,
}



\lstdefinestyle{CSSDraculaLight}{
    basicstyle=\bfseries\scriptsize\ttfamily\color{htmltext},
    commentstyle=\color{htmlcomment},
    keywordstyle=\bfseries\color{htmlvalue},
    stringstyle=\color{htmlvalue},
    emph={DOCTYPE,html,head,body,div,span,a,script},
    emphstyle={\color{htmltag}\bfseries},
    sensitive=true,
    showstringspaces=false,
    backgroundcolor=\color{white},
    inputencoding=utf8,
    extendedchars=true, % Support extended characters
    frame=none, 
    %frame=tb,
    tabsize=4,
    breaklines=true,
    breakatwhitespace=true,
    numbers=left,
    numbersep=10pt,
    numberstyle=\small\bfseries\ttfamily\color{htmlcomment},
    escapeinside={<@}{@>},
    rulecolor=\color{htmlbackground},
    xleftmargin=20pt,
    % Add a vertical line for opening and closing tags
    %frame=lines,
    framesep=2pt,
    framexleftmargin=4pt,
    % Add a vertical line for closing tags that go through multiple lines
    breaklines=true,
    postbreak=\mbox{\textcolor{gray}{$\hookrightarrow$}\space},
    showlines=true,
    % Add a rule to apply commentstyle to keywords inside comments
    moredelim=[s][\color{htmlcomment}]{/*}{*/},
    literate={é}{{\'e}}1
             {è}{{\`e}}1
             {ê}{{\^e}}1
             {ë}{{\"e}}1
             {à}{{\`a}}1
             {ù}{{\`u}}1
             {û}{{\^u}}1
             {ç}{{\c{c}}}1
             {â}{{\^a}}1
             {î}{{\^i}}1
             {ï}{{\"i}}1,
    morekeywords={color, background, background-color, font-size, font-weight, border, border-radius, padding, margin, display, position, top, right, bottom, left, flex, grid, width, height, min-width, max-width, min-height, max-height, transition, transform, animation, keyframes, content, z-index,id,class,type,name,value,placeholder,checked,src,href,alt},
    morestring=[s][\color{htmltag}]{:}{;},
}










\title{\huge{IFT1575}\\\Huge{Modèles de recherche opérationnelle}\\\vspace{2em} Introduction}
\author{\huge{Franz Girardin}}
\date{\today}


% NEW for point lines
\newcommand{\Pointilles}[1]{%
  \par\nobreak
  \noindent\rule{0pt}{1.5\baselineskip}% Provides a larger gap between the preceding paragraph and the dots
  \multido{}{#1}{\noindent\makebox[\linewidth]{\dotfill}\endgraf}% ... dotted lines ...
  \bigskip% Gap between dots and next paragraph
}
% Define a new command for dotted lines that works in align
\newcommand{\PointillesAlign}{%
  \multido{}{1}{\makebox[\linewidth]{\dotfill}}
}

   

\begin{document}

\maketitle
\newpage% or \cleardoublepage
% \pdfbookmark[<level>]{<title>}{<dest>}
\pdfbookmark[section]{\contentsname}{toc}
\tableofcontents
\pagebreak


\titleformat*{\section}{%
    \normalsize\bfseries%
}

\titleformat{\section}[block]{\normalsize\bfseries}{}{0pt}{}




\begin{multicols*}{3}
    \chapter{Modélisation}
    \footnotesize
\section{Définition de la prog. linéaire}

    Branche de la \textit{Recherche Opérationnelle} où on utilise 
    des \textbf{modèles} mathématiques qui font appel à 
    des fonction \textcolor{myb}{\textbf{linéaire}} pour 
    résoudre un problème. 

    \begin{align*}
        \texttt{prod}. \text{ vin blanc} \implies x \; (\text{litres}) \\ 
        \texttt{prod}. \text{ vin rouge} \implies y \; (\text{litres})
    \end{align*}
\section{Composantes du modèle}
    Il faut identifier \textbf{\textcolor{myb}{l'action}} sur 
    laquelle on doit prendre une décision et la représenter 
    par un \textbf{\textcolor{myb}{variable de décision}}. À cette 
    variable sera associée un constante qui représente 
    le \textbf{\textcolor{myb}{niveau}} de l'action.

\section{L'objectif} 
    S'exprime sous le forme d'une \textbf{fonction mathématique} 
    qui représente l'intention. P. ex. \textbf{maximiser le profit}
    correspondrait à : 
    \[%
        \text{Max} \; 5x + 3y 
    \]%

\section{Contraintes}
    Elles peuvent dépendre du \textit{contexte du problèmes} 
    ou peuvent être plus générales; p. ex. la 
    \textbf{non-négativité} d'une entité : 
    \begin{align*}
    &x \leq 50  \quad\quad\quad\;\; \texttt{prod}.\text{max vin blanc} \\ 
    &y \leq 60 \\
    &x +  y \leq 80  \quad\quad \texttt{prod}.\text{max totale de vin} \\ 
    &x \geq 0, y \geq 0 \quad\; \text{Non-nég.}
    \end{align*}

\section{Terminologie de base}
 Une \textbf{solution réalisable} est un \textbf{vecteur} 
 de valeurs $x = (x_1, x_2, \dots, x_n)$ appartenant à $X \subset \mathbb{R}^n$ tel 
 que chaque composante $x_1, x_2, \dots, x_n$ respecte \textit{toutes les contraintes} du problème.
 Dans ce contexte, $X$ est l'ensemble des solutions possibles, c'est-à-dire l'ensemble des points qui satisfont toutes des contraintes imposées au problème.
 Chaque point dans $X$ est un vecteur de longueur $n$ 
 Cet ensemble $X$ est un sous-ensemble de l'espace $\mathbb{R}^n$ 
 qui comprends tous les vecteurs possibles où 
 $n \in [1, n \longrightarrow \infty[$.

Une solution \textbf{optimale} est une solution réalisable qui maximise (ou minimise) la fonction objectif $z(x)$, en fonction de la nature du problème (maximisation ou minimisation). Autrement dit, c'est une solution réalisable qui produit la \textit{meilleure valeur possible} pour la fonction objectif $z$.

\vspace{2.5em}
\chapter{Maximisation et minimisation}

\begin{note}{}{}
    Un modèle peut présenter une \textbf{infinité} de 
    solution optimales. 
\end{note}       

\begin{theorem}{}{}
    La maximisation d'un objectif $f(w)$ est équivalent à l'\textbf{opposé}
    de la minimisation :

    \begin{align*}
        \text{Max} \; f(w) \leftrightarrow  \text{Min} -f(w)
    \end{align*}
\end{theorem}

\begin{Preuve}{}{}
    Considérons le problème :
    \begin{align*}
        \text{Max} \; f(w) \\ 
        \text{s.a} \;\; w \in X \subset \mathbb{R}^n
    \end{align*}
    Soit $w_{\text{opt}}$ un point de $X$ où $f(w)$ atteint son  
    \textbf{\textcolor{myb}{maximum}}. On a :  
    \begin{align*}
        f\left(w_{\text{opt}} \right) \geq &f(w) \;\; \forall w \in X \\
                                           &\updownarrow \\ 
        - f\left(w_{\text{opt}} \right) \leq - &f(w) \;\; \forall w \in X 
    \end{align*}
    Par conséquent, $-f(w_{\text{opt}})$ est la valeur minimale possible parmi 
    tous les $-f(w)$ possibles :

    \begin{align*}
        - \text{Max} f(w) = 
        - f\left(w_{\text{opt}}\right) = 
        \text{Min} -f(w)
    \end{align*}
    Ainsi, qu'on maximise $f(w)$ ou qu'on minimise $-f(w)$, on retrouve la 
    même solution optimale $w_{\text{opt}}$.
\end{Preuve}






\begin{Explication*}{}{}
    Soit la fonction objectif $f : X \rightarrow \mathbb{R}$  
Dans cette situation, le vecteur \( w \) joue deux rôles distincts : il est à la 
fois l'argument de la fonction objectif \( f(w) \) et un élément soumis aux 
contraintes. Ces deux aspects sont essentiels dans un problème de maximisation 
en programmation linéaire ou non linéaire.

\vspace{1em}%
\textbf{1. Vecteur dans la fonction objectif :}
La fonction objectif \( f(w) \) est la fonction que l'on cherche à maximiser. 
Elle prend un vecteur \( w \) comme entrée, qui est un point dans l'espace 
\( \mathbb{R}^n \). Ce vecteur \( w \) représente une solution candidate du 
problème, et \( f(w) \) retourne un nombre réel, par exemple un profit ou une 
performance, que l'on souhaite maximiser.

\vspace{1em}%
\textbf{2. Vecteur dans les contraintes :}
Le vecteur \( w \) doit aussi satisfaire un ensemble de contraintes qui 
définissent l'ensemble des solutions réalisables, noté \( X \subset \mathbb{R}^n \).
Ces contraintes peuvent être des inégalités ou des égalités (par exemple, 
\( g(w) \leq 0 \), \( h(w) = 0 \)), et elles définissent l'ensemble des vecteurs 
\( w \) admissibles.

\vspace{1em}%
\textbf{Synthèse des deux rôles :}
\begin{itemize}
    \item \textbf{Maximisation :} On cherche à maximiser la fonction \( f(w) \), 
    c'est-à-dire à trouver la meilleure valeur possible de \( f(w) \).
    \item \textbf{Contraintes :} Le vecteur \( w \) doit se situer dans un 
    ensemble admissible \( X \), défini par les contraintes du problème.
\end{itemize}

En résumé, \( w \) doit satisfaire les contraintes (ce qui garantit qu'il est 
une solution réalisable), tout en maximisant la valeur de \( f(w) \). Le problème
consiste à trouver le \( w \) optimal qui satisfait ces deux aspects à la fois.
\end{Explication*}                  
Soit le problème suivant. Un restaurateur dispose de \textbf{14} oursins, 
\textbf{24} crevettes, \textbf{18} huîtres. Deux types d'assiètes 
de fruit de mer sont offertes par le restaurateur :

\begin{align*}
    \textbf{8\$} \colon \text{5 oursins, 2 crevettes 1 huître}  \\
    \textbf{6\$} \colon \text{3 oursins, 3 crevettes 3 huîtres}   
\end{align*}

L'objectif est de déterminer le \textbf{nombres d'assiètes} de chaque 
type à préparer afin de \textit{maximiser} 
le revenu total du restaurateur tout en 
respectant les disponibilité de fruits de mer. 


\section{Contraintes}
\begin{align*}
    5x + 3y \leq 30& \quad \text{Oursins} \\
    2x + 3y \leq 24& \quad \text{crevettes} \\ 
    x  + 3y \leq 18& \quad \text{Huîtres}
    \\
    \\ 
    x, y \geq 0& \quad \text{Non-négativité}
    \\ 
    \\
    z = 8x + 6x& \quad \text{Objectif}
\end{align*}




En considérant que la maximisation d'un objectif est 
équivalente à la minimisation de l'opposé de la minimisation on a :


\begin{align*}
    \text{Min} \;\; -8x  -6y& \\
    \text{s.a.}
    \\
    \\
    5x + 3y \leq 30& \\
    2x + 3y \leq 24& \\
    x  + 3y \leq 18& 
    \\
    \\ 
    x, y \geq 0& 
    \\ 
    \\
    z = 8x + 6x& 
\end{align*}


\section{Forme standard}

\begin{align*}
    \text{Min} z = -8 -6y \\ 
    \text{s.a.} 
    \\
    5x + 3y& + \textcolor{myr}{\textbf{\textit{u}}}   &= 30 \\
    2x + 3y& \quad\quad + \textcolor{myr}{\textbf{\textit{p}}}   &= 24\\
     x + 3y& \quad\quad\quad\quad + \textcolor{myr}{\textbf{\textit{h}}}   &= 18 \\
     x, y&,  
     \textcolor{myr}{\textbf{\textit{u}}}, 
     \textcolor{myr}{\textbf{\textit{p}}},
     \textcolor{myr}{\textbf{\textit{h}}} \geq 0
\end{align*}

\section{Choisir les variables indépendantes}
Supposons que \
\textcolor{myb}{\textbf{\textit{x}}} et 
\textcolor{myb}{\textbf{\textit{y}}} sont des 
\textcolor{myb}{\textbf{\textit{variables indépendantes}}}. Exprimons 
les \textcolor{myr}{\textbf{\textit{variable dépendantes}}} 
\textcolor{myr}{\textbf{\textit{u}}},
\textcolor{myr}{\textbf{\textit{p}}},
\textcolor{myr}{\textbf{\textit{h}}},
\textcolor{myr}{\textbf{\textit{z}}} en fonction de 
\textcolor{myb}{\textbf{\textit{x}}} et \textcolor{myb}{\textbf{\textit{y}}}


\begin{align*}
    u \quad\quad\quad\quad\quad\quad &= 30 - 5x -3y& \\
    p \quad\quad\quad\quad  &= 24 -2x -3y& \\
    h \quad\quad &= 18 - x -3y&
\end{align*}
\vspace{-2.5em}    
\Pointilles{1}
\vspace{-2.5em}
\begin{align*}
    \quad\quad\quad\;\; z = 0 -8x -6y
\end{align*}
En supposant que 
\textcolor{myb}{\textbf{\textit{x}}} et 
\textcolor{myb}{\textbf{\textit{y}}}  sont \textbf{fixés à 0}, 
on a la \textbf{solution} : 

\begin{align*}
x = y = 0 \implies 
\textcolor{myr}{\textbf{\textit{u}}}  = 30, 
\textcolor{myr}{\textbf{\textit{p}}} = 24,
\textcolor{myr}{\textbf{\textit{h}}} = 18,
\textbf{\textit{z}} = 0  
\end{align*}




\end{multicols*}
            
\end{document}

