\documentclass{report}
%\usepackage[utopia]{mathdesign}
%\usepackage{amsmath, amsthm}


\usepackage{amsmath,amsfonts,amsthm,amssymb,mathtools}
\usepackage{nicefrac, xfrac}
%\usepackage[varbb]{newpxmath}
%\usepackage[osf,largesc,theoremfont]{newpxtext}
%\usepackage{coelacanth}
%\usepackage{beraserif} % Bitstream Vera Serif font
%\usepackage{berasans} % Bitstream Vera Sans font
%\usepackage{beramono} % Bitstream Vera Sans Mono font
%\usepackage{berasans}
%\usepackage{libertine}
%\usepackage{mathpazo}
%\usepackage{palatino}
%\usepackage{crimson}

% NEW ------- For pointilles lines
\usepackage{multido}

%% Choose one of the following (if not choosing the  
%% default, viz., Computer Modern, font family):
%\usepackage{lmodern}
\usepackage{bold-extra}
%%
%\usepackage{mathpazo}
% \usepackage{newpxmath}
%\usepackage{kpfonts} % Very good
%%
%\usepackage{mathptmx} %Very good
%\usepackage{stix} 
%\usepackage{txfonts} %Very good
\usepackage{newtxtext,newtxmath} %Very good
%%
%\usepackage{libertine} \usepackage[libertine]{newtxmath}
%\usepackage{libertine,libertinust1math} % added 2019/11/28
%%
%\usepackage{newpxtext} \usepackage[euler-digits]{eulervm}
%\usepackage{textcomp}
%\usepackage{bm}
\usepackage{contour}
\usepackage{adjustbox}





\input{/home/cryptopsy/Semesters/LaTeXTemplates/UniversalTeXTemplate/preamble.tex}
%From M275 "Topology" at SJSU
\newcommand{\id}{\mathrm{id}} % Identité
\newcommand{\taking}[1]{\xrightarrow{#1}} % Flèche avec annotation
\newcommand{\inv}{^{-1}} % Inverse

%From M170 "Introduction to Graph Theory" at SJSU
\DeclareMathOperator{\diam}{diam} % Diamètre
\DeclareMathOperator{\ord}{ord} % Ordre
\newcommand{\defeq}{\overset{\mathrm{def}}{=}} % Défini comme égal

%From the USAMO .tex files
\newcommand{\ts}{\textsuperscript} % Exposant
\newcommand{\dg}{^\circ} % Degré
\newcommand{\ii}{\item} % Item

% % From Math 55 and Math 145 at Harvard
% \newenvironment{subproof}[1][Proof]{%
% \begin{proof}[#1] \renewcommand{\qedsymbol}{$\blacksquare$}}%
% {\end{proof}}

\newcommand{\liff}{\leftrightarrow} % Si et seulement si
\newcommand{\lthen}{\rightarrow} % Implique
\newcommand{\opname}{\operatorname} % Opérateur générique
\newcommand{\surjto}{\twoheadrightarrow} % Flèche surjective
\newcommand{\injto}{\hookrightarrow} % Flèche injective
\newcommand{\On}{\mathrm{On}} % Ordinaux
\DeclareMathOperator{\img}{im} % Image
\DeclareMathOperator{\Img}{Im} % Image
\DeclareMathOperator{\coker}{coker} % Cokernel
\DeclareMathOperator{\Coker}{Coker} % Cokernel
\DeclareMathOperator{\Ker}{Ker} % Noyau
\DeclareMathOperator{\rank}{rank} % Rang
\DeclareMathOperator{\Spec}{Spec} % Spectre
\DeclareMathOperator{\Tr}{Tr} % Trace
\DeclareMathOperator{\pr}{pr} % Projection
\DeclareMathOperator{\ext}{ext} % Extension
\DeclareMathOperator{\pred}{pred} % Prédécesseur
\DeclareMathOperator{\dom}{dom} % Domaine
\DeclareMathOperator{\ran}{ran} % Image (range)
\DeclareMathOperator{\Hom}{Hom} % Homomorphisme
\DeclareMathOperator{\Mor}{Mor} % Morphismes
\DeclareMathOperator{\End}{End} % Endomorphisme

\newcommand{\eps}{\epsilon} % Épsilon
\newcommand{\veps}{\varepsilon} % Variance d'épsilon
\newcommand{\ol}{\overline} % Ligne au-dessus
\newcommand{\ul}{\underline} % Ligne en-dessous
\newcommand{\wt}{\widetilde} % Tilde large
\newcommand{\wh}{\widehat} % Chapeau large
\newcommand{\vocab}[1]{\textbf{\color{blue} #1}} % Texte en gras et bleu
\providecommand{\half}{\frac{1}{2}} % Fraction 1/2
\newcommand{\dang}{\measuredangle} % Angle dirigé
\newcommand{\ray}[1]{\overrightarrow{#1}} % Ray
\newcommand{\seg}[1]{\overline{#1}} % Segment
\newcommand{\arc}[1]{\wideparen{#1}} % Arc
\DeclareMathOperator{\cis}{cis} % cis
\DeclareMathOperator*{\lcm}{lcm} % Plus petit commun multiple
\DeclareMathOperator*{\argmin}{arg min} % Argument du minimum
\DeclareMathOperator*{\argmax}{arg max} % Argument du maximum
\newcommand{\cycsum}{\sum_{\mathrm{cyc}}} % Somme cyclique
\newcommand{\symsum}{\sum_{\mathrm{sym}}} % Somme symétrique
\newcommand{\cycprod}{\prod_{\mathrm{cyc}}} % Produit cyclique
\newcommand{\symprod}{\prod_{\mathrm{sym}}} % Produit symétrique
\newcommand{\Qed}{\begin{flushright}\qed\end{flushright}} % QED aligné à droite
\newcommand{\parinn}{\setlength{\parindent}{1cm}} % Indentation de paragraphe à 1 cm
\newcommand{\parinf}{\setlength{\parindent}{0cm}} % Pas d'indentation de paragraphe
% \newcommand{\norm}{\|\cdot\|} % Norme
\newcommand{\inorm}{\norm_{\infty}} % Norme infinie
\newcommand{\opensets}{\{V_{\alpha}\}_{\alpha\in I}} % Ensemble ouvert
\newcommand{\oset}{V_{\alpha}} % Ensemble ouvert V
\newcommand{\opset}[1]{V_{\alpha_{#1}}} % Ensemble ouvert V avec indice
\newcommand{\lub}{\text{lub}} % Plus petite borne supérieure
\newcommand{\del}[2]{\frac{\partial #1}{\partial #2}} % Dérivée partielle
\newcommand{\Del}[3]{\frac{\partial^{#1} #2}{\partial^{#1} #3}} % Dérivée partielle d'ordre élevé
\newcommand{\deld}[2]{\dfrac{\partial #1}{\partial #2}} % Dérivée partielle avec dfrac
\newcommand{\Deld}[3]{\dfrac{\partial^{#1} #2}{\partial^{#1} #3}} % Dérivée partielle d'ordre élevé avec dfrac
\newcommand{\lm}{\lambda} % Lambda
\newcommand{\uin}{\mathbin{\rotatebox[origin=c]{90}{$\in$}}} % Appartient, tourné de 90 degrés
\newcommand{\usubset}{\mathbin{\rotatebox[origin=c]{90}{$\subset$}}} % Sous-ensemble, tourné de 90 degrés
\newcommand{\lt}{\left} % Gauche
\newcommand{\rt}{\right} % Droite
\newcommand{\bs}[1]{\boldsymbol{#1}} % Symbole en gras
\newcommand{\exs}{\exists} % Il existe
\newcommand{\st}{\strut} % Strut
\newcommand{\dps}[1]{\displaystyle{#1}} % Disposition en ligne

\newcommand{\sol}{\setlength{\parindent}{0cm}\textbf{\textit{Solution:}}\setlength{\parindent}{1cm} } % Solution sans indentation initiale puis rétablie
\newcommand{\solve}[1]{\setlength{\parindent}{0cm}\textbf{\textit{Solution: }}\setlength{\parindent}{1cm}#1 \Qed}

\newcommand{\entoure}[1]{\fcolorbox{black}{gray!30}{\texttt{#1}}}

\renewcommand{\ttdefault}{cmtt}
\newcommand{\textttbf}[1]{\contour{yellow!45}{\texttt{#1}}}
\newcommand{\varitem}[3][black]{%
    \item [%
        \colorbox{#2}{\textcolor{#1}{\makebox(5.5,7){#3}}}%
    ]
}
% Allow you to do the non implication (implication barred)
\newcommand{\notimplies}{%
  \mathrel{{\ooalign{\hidewidth$\not\phantom{=}$\hidewidth\cr$\implies$}}}}


\newcommand*{\authorimg}[1]%
    { \raisebox{-1\baselineskip}{\includegraphics[width=\imagesize]{#1}}}
\newlength\imagesize 

\input{/home/cryptopsy/Semesters/LaTeXTemplates/UniversalTeXTemplate/letterfonts.tex}
% lstlistingsEnvs.tex

\usepackage{minted}


\lstset{
  basicstyle=\ttfamily, % Set
  columns=fullflexible,
  keepspaces=true,
  language=Python % You can specify the language if you want syntax highlighting
}

%%%%%%%%%%%%%%%%%%%%%%%%%%%%%%%%%%%%%%%%%%%%%%%%%%%%%%%%%%%%%%%%%%%%%%%%%%%%%%%%%%%%%%%%%%%%%%%%%
%                                 Custom lstlisting Environments
%%%%%%%%%%%%%%%%%%%%%%%%%%%%%%%%%%%%%%%%%%%%%%%%%%%%%%%%%%%%%%%%%%%%%%%%%%%%%%%%%%%%%%%%%%%%%%%%%
% Gruvbox style for Python
\definecolor{Pgruvbox-bg}{HTML}{282828}
\definecolor{Pgruvbox-fg}{HTML}{ebdbb2}
\definecolor{Pgruvbox-red}{HTML}{fb4934}
\definecolor{Pgruvbox-green}{HTML}{b8bb26}
\definecolor{Pgruvbox-yellow}{HTML}{fabd2f}
\definecolor{Pgruvbox-blue}{HTML}{83a598}
\definecolor{Pgruvbox-purple}{HTML}{d3869b}
\definecolor{Pgruvbox-aqua}{HTML}{8ec07c}
\definecolor{BBBlack}{rgb}{0.05, 0.06, 0.09}



% JAVA LSTLISTING STYLE IN Gruvbox Colorscheme
\definecolor{gruvbox-bg}{rgb}{0.282, 0.247, 0.204}
\definecolor{gruvbox-fg1}{rgb}{0.949, 0.898, 0.776}
\definecolor{gruvbox-fg2}{rgb}{0.871, 0.804, 0.671}
\definecolor{gruvbox-red}{rgb}{0.788, 0.255, 0.259}
\definecolor{gruvbox-green}{rgb}{0.518, 0.604, 0.239}
\definecolor{gruvbox-yellow}{rgb}{0.914, 0.808, 0.427}
\definecolor{gruvbox-blue}{rgb}{0.353, 0.510, 0.784}
\definecolor{gruvbox-purple}{rgb}{0.576, 0.412, 0.659}
\definecolor{gruvbox-aqua}{rgb}{0.459, 0.631, 0.737}
\definecolor{gruvbox-gray}{rgb}{0.518, 0.494, 0.471}

\definecolor{lst-bg}{RGB}{45, 45, 45}
\definecolor{lst-fg}{RGB}{220, 220, 204}
\definecolor{lst-keyword}{RGB}{215, 186, 125}
\definecolor{lst-comment}{RGB}{117, 113, 94}
\definecolor{lst-string}{RGB}{163, 190, 140}
\definecolor{lst-number}{RGB}{181, 206, 168}
\definecolor{lst-type}{RGB}{218, 142, 130}

\lstdefinestyle{PythonGruvbox}{
    language=Python,
    identifierstyle=\color{lst-fg},
    basicstyle=\ttfamily\color{Pgruvbox-fg},
    keywordstyle=\color{Pgruvbox-yellow},
    keywordstyle=[2]\color{Pgruvbox-blue},
    stringstyle=\color{Pgruvbox-green},
    commentstyle=\color{Pgruvbox-aqua},
    backgroundcolor=\color{BBBlack},
    rulecolor=\color{BBBlack},
    showstringspaces=false,
    keepspaces=true,
    captionpos=b,
    breaklines=true,
    tabsize=4,
    showspaces=false,
    numbers=left,
    numbersep=5pt,
    numberstyle=\tiny\color{gray},
    showtabs=false,
    columns=fullflexible,
    morekeywords={True,False,None},
    morekeywords=[2]{and,as,assert,break,class,continue,def,del,elif,else,except,exec,
    finally,for,from,global,if,import,in,is,lambda,nonlocal,not,or,pass,print,raise,
    return,try,while,with,yield},
    morecomment=[s]{"""}{"""},
    morecomment=[s]{'''}{'''},
    morecomment=[l]{\#},
    morestring=[b]",
    morestring=[b]',
    literate=
    {0}{{\textcolor{Pgruvbox-purple}{0}}}{1}
    {1}{{\textcolor{Pgruvbox-purple}{1}}}{1}
    {2}{{\textcolor{Pgruvbox-purple}{2}}}{1}
    {3}{{\textcolor{Pgruvbox-purple}{3}}}{1}
    {4}{{\textcolor{Pgruvbox-purple}{4}}}{1}
    {5}{{\textcolor{Pgruvbox-purple}{5}}}{1}
    {6}{{\textcolor{Pgruvbox-purple}{6}}}{1}
    {7}{{\textcolor{Pgruvbox-purple}{7}}}{1}
    {8}{{\textcolor{Pgruvbox-purple}{8}}}{1}
    {9}{{\textcolor{Pgruvbox-purple}{9}}}{1}
}

% Gruvbox style for Java
\definecolor{gruvbox-bg}{rgb}{0.282, 0.247, 0.204}
\definecolor{gruvbox-fg1}{rgb}{0.949, 0.898, 0.776}
\definecolor{gruvbox-fg2}{rgb}{0.871, 0.804, 0.671}
\definecolor{gruvbox-red}{rgb}{0.788, 0.255, 0.259}
\definecolor{gruvbox-green}{rgb}{0.518, 0.604, 0.239}
\definecolor{gruvbox-yellow}{rgb}{0.914, 0.808, 0.427}
\definecolor{gruvbox-blue}{rgb}{0.353, 0.510, 0.784}
\definecolor{gruvbox-purple}{rgb}{0.576, 0.412, 0.659}
\definecolor{gruvbox-aqua}{rgb}{0.459, 0.631, 0.737}
\definecolor{gruvbox-gray}{rgb}{0.518, 0.494, 0.471}

\lstdefinestyle{JavaGruvbox}{
    language=Java,
    basicstyle=\ttfamily\color{Pgruvbox-fg},
    keywordstyle=\color{Pgruvbox-yellow},
    keywordstyle=[2]\color{lst-type},
    commentstyle=\itshape\color{lst-comment},
    stringstyle=\color{lst-string},
    numberstyle=\color{lst-number},
    backgroundcolor=\color{BBBlack},
    rulecolor=\color{gruvbox-aqua},
    showstringspaces=false,
    keepspaces=true,
    captionpos=b,
    breaklines=true,
    tabsize=4,
    showspaces=false,
    showtabs=false,
    columns=fullflexible,
    morekeywords={var},
    morekeywords=[2]{boolean, byte, char, double, float, int, long, short, void},
    morecomment=[s]{/}{/},
    morecomment=[l]{//},
    morestring=[b]",
    morestring=[b]',
    numbers=left,
    numbersep=5pt,
    numberstyle=\tiny\color{gray},
}

% Dracula style for Java
\definecolor{draculawhite-background}{RGB}{237, 239, 252}
\definecolor{draculawhite-comment}{RGB}{98, 114, 164}
\definecolor{draculawhite-keyword}{RGB}{189, 147, 249}
\definecolor{draculawhite-string}{RGB}{152, 195, 121}
\definecolor{draculawhite-number}{RGB}{249, 189, 89}
\definecolor{draculawhite-operator}{RGB}{248, 248, 242}

\lstdefinestyle{JavaDraculaWhite}{
    language=Java,
    backgroundcolor=\color{draculawhite-background},
    commentstyle=\itshape\color{draculawhite-comment},
    keywordstyle=\color{draculawhite-keyword},
    stringstyle=\color{draculawhite-string},
    basicstyle=\ttfamily\footnotesize\color{black},
    identifierstyle=\color{black},
    keywordstyle=\color{draculawhite-keyword}\bfseries,
    morecomment=[s][\color{draculawhite-comment}]{/**}{*/},
    showstringspaces=false,
    showspaces=false,
    breaklines=true,
    %frame=single,
    rulecolor=\color{draculawhite-operator},
    tabsize=2,  
    numbers=left,
    numbersep=4pt,
    numberstyle=\ttfamily\tiny\color{gray}
}

% Dracula style for Python
\definecolor{draculawhite-bg}{HTML}{FAFAFA}
\definecolor{draculawhite-fg}{HTML}{282A36}
\definecolor{pdraculawhite-keyword}{HTML}{BD93F9}
\definecolor{pdraculawhite-comment}{HTML}{6272A4}
\definecolor{draculawhite-number}{HTML}{FF79C6}

\lstdefinestyle{PythonDraculaWhite}{
    language=Python,
    basicstyle=\ttfamily\small\color{draculawhite-fg},
    backgroundcolor=\color{draculawhite-background},
    keywordstyle=\color{orange}\bfseries,
    stringstyle=\color{draculawhite-string},
    commentstyle=\color{pdraculawhite-comment}\itshape,
    numberstyle=\color{draculawhite-number},
    showstringspaces=false,
    showspaces=false,
    breaklines=true,
    frame=single,
    rulecolor=\color{draculawhite-operator}, 
    tabsize=4,
    morekeywords={as,with,1,2,3,4, 5,6,7,8,9,True,False},
    numbers=left,
    numbersep=5pt,
    numberstyle=\small\bfseries\ttfamily\color{htmlcomment},
}

% Dracula Dark style for HTML
\definecolor{htmltag}{HTML}{ff79c6}
\definecolor{htmlattr}{HTML}{f1fa8c}
\definecolor{htmlvalue}{HTML}{bd93f9}
\definecolor{htmlcomment}{HTML}{6272a4}
\definecolor{htmltext}{HTML}{401E31}
\definecolor{htmlbackground}{HTML}{282a36}
\definecolor{comphtmlbackground}{HTML}{8093FF}

\lstdefinestyle{HTMLDraculaDark}{
    basicstyle=\normalsize\bfseries\ttfamily\color{htmltext},
    commentstyle=\itshape\color{htmlcomment},
    keywordstyle=\bfseries\color{htmltag},
    stringstyle=\color{htmlvalue},
    emph={DOCTYPE,html,head,body,div,span,a,script},
    emphstyle={\color{htmltag}\bfseries},
    sensitive=true,
    showstringspaces=false,
    backgroundcolor=\color{white},
    inputencoding=utf8,
    extendedchars=true,
    language=HTML,
    tabsize=4,
    breaklines=true,
    breakatwhitespace=true,
    numbers=left,
    numbersep=10pt,
    numberstyle=\small\bfseries\ttfamily\color{htmlcomment},
    escapeinside={<@}{@>},
    rulecolor=\color{htmlbackground},
    xleftmargin=10pt,
    frame=none, 
    breaklines=true,
    postbreak=\mbox{\textcolor{gray}{$\hookrightarrow$}\space},
    showlines=false,
    moredelim=[s][\itshape\color{htmlcomment}]{<!--}{-->},
    morekeywords={id,class,type,name,value,placeholder,checked,src,href,alt},
    literate={é}{{\'e}}1 {è}{{\`e}}1 {ê}{{\^e}}1 {ë}{{\"e}}1 {à}{{\`a}}1 {ù}{{\`u}}1 {û}{{\^u}}1 {ç}{{\c{c}}}1 {â}{{\^a}}1 {î}{{\^i}}1 {ï}{{\"i}}1
}


\lstdefinestyle{Haskell}{
  frame=none,
  xleftmargin=2pt,
  stepnumber=1,
  numbers=left,
  numbersep=5pt,
  numberstyle=\ttfamily\tiny\color[gray]{0.3},
  belowcaptionskip=\bigskipamount,
  captionpos=b,
  escapeinside={*'}{'*},
  language=haskell,
  tabsize=2,
  emphstyle={\bf},
  %commentstyle=\it,
  stringstyle=\mdseries\ttfamily,
  showspaces=false,
  keywordstyle=\bfseries\ttfamily,
  columns=flexible,
  basicstyle=\small\ttfamily,
  showstringspaces=false,
  morecomment=[l]\%,
}



\lstdefinestyle{CSSDraculaLight}{
    basicstyle=\bfseries\scriptsize\ttfamily\color{htmltext},
    commentstyle=\color{htmlcomment},
    keywordstyle=\bfseries\color{htmlvalue},
    stringstyle=\color{htmlvalue},
    emph={DOCTYPE,html,head,body,div,span,a,script},
    emphstyle={\color{htmltag}\bfseries},
    sensitive=true,
    showstringspaces=false,
    backgroundcolor=\color{white},
    inputencoding=utf8,
    extendedchars=true, % Support extended characters
    frame=none, 
    %frame=tb,
    tabsize=4,
    breaklines=true,
    breakatwhitespace=true,
    numbers=left,
    numbersep=10pt,
    numberstyle=\small\bfseries\ttfamily\color{htmlcomment},
    escapeinside={<@}{@>},
    rulecolor=\color{htmlbackground},
    xleftmargin=20pt,
    % Add a vertical line for opening and closing tags
    %frame=lines,
    framesep=2pt,
    framexleftmargin=4pt,
    % Add a vertical line for closing tags that go through multiple lines
    breaklines=true,
    postbreak=\mbox{\textcolor{gray}{$\hookrightarrow$}\space},
    showlines=true,
    % Add a rule to apply commentstyle to keywords inside comments
    moredelim=[s][\color{htmlcomment}]{/*}{*/},
    literate={é}{{\'e}}1
             {è}{{\`e}}1
             {ê}{{\^e}}1
             {ë}{{\"e}}1
             {à}{{\`a}}1
             {ù}{{\`u}}1
             {û}{{\^u}}1
             {ç}{{\c{c}}}1
             {â}{{\^a}}1
             {î}{{\^i}}1
             {ï}{{\"i}}1,
    morekeywords={color, background, background-color, font-size, font-weight, border, border-radius, padding, margin, display, position, top, right, bottom, left, flex, grid, width, height, min-width, max-width, min-height, max-height, transition, transform, animation, keyframes, content, z-index,id,class,type,name,value,placeholder,checked,src,href,alt},
    morestring=[s][\color{htmltag}]{:}{;},
}










\title{\huge{IFT-1575}\\\Huge{Modèles de recherche opérationnelle}\\\vspace{2em} Introduction}
\author{\huge{Franz Girardin}}
\date{\today}


% NEW for point lines
\newcommand{\Pointilles}[1]{%
  \par\nobreak
  \noindent\rule{0pt}{1.5\baselineskip}% Provides a larger gap between the preceding paragraph and the dots
  \multido{}{#1}{\noindent\makebox[\linewidth]{\dotfill}\endgraf}% ... dotted lines ...
  \bigskip% Gap between dots and next paragraph
}
% Define a new command for dotted lines that works in align
\newcommand{\PointillesAlign}{%
  \multido{}{1}{\makebox[\linewidth]{\dotfill}}
}

   

\begin{document}

\maketitle
\newpage% or \cleardoublepage
% \pdfbookmark[<level>]{<title>}{<dest>}
\pdfbookmark[section]{\contentsname}{toc}
\tableofcontents
\pagebreak


\titleformat*{\section}{%
    \normalsize\bfseries%
}

\titleformat{\section}[block]{\normalsize\bfseries}{}{0pt}{}

\titleformat{\subsection}[block]{\small\bfseries}{}{0pt}{}



\begin{multicols*}{3}
    \chapter{Modélisation}
    \footnotesize
\section{Définition de la prog. linéaire}

    Branche de la \textit{Recherche Opérationnelle} où on utilise 
    des \textbf{modèles} mathématiques qui font appel à 
    des fonction \textcolor{myb}{\textbf{linéaire}} pour 
    résoudre un problème. 

    \begin{align*}
        \texttt{prod}. \text{ vin blanc} \implies x \; (\text{litres}) \\ 
        \texttt{prod}. \text{ vin rouge} \implies y \; (\text{litres})
    \end{align*}
\section{Composantes du modèle}
    Il faut identifier \textbf{\textcolor{myb}{l'action}} sur 
    laquelle on doit prendre une décision et la représenter 
    par un \textbf{\textcolor{myb}{variable de décision}}. À cette 
    variable sera associée un constante qui représente 
    le \textbf{\textcolor{myb}{niveau}} de l'action.

\section{L'objectif} 
    S'exprime sous le forme d'une \textbf{fonction mathématique} 
    qui représente l'intention. P. ex. \textbf{maximiser le profit}
    correspondrait à : 
    \[%
        \text{Max} \; 5x + 3y 
    \]%

\section{Contraintes}
    Elles peuvent dépendre du \textit{contexte du problèmes} 
    ou peuvent être plus générales; p. ex. la 
    \textbf{non-négativité} d'une entité : 
    \begin{align*}
    &x \leq 50  \quad\quad\quad\;\; \texttt{prod}.\text{max vin blanc} \\ 
    &y \leq 60 \\
    &x +  y \leq 80  \quad\quad \texttt{prod}.\text{max totale de vin} \\ 
    &x \geq 0, y \geq 0 \quad\; \text{Non-nég.}
    \end{align*}

\section{Terminologie de base}
 Une \textbf{solution réalisable} est un \textbf{vecteur} 
 de valeurs $x = (x_1, x_2, \dots, x_n)$ appartenant à $X \subset \mathbb{R}^n$ tel 
 que chaque composante $x_1, x_2, \dots, x_n$ respecte \textit{toutes les contraintes} du problème.
 Dans ce contexte, $X$ est l'ensemble des solutions possibles, c'est-à-dire l'ensemble des points qui satisfont toutes des contraintes imposées au problème.
 Chaque point dans $X$ est un vecteur de longueur $n$ 
 Cet ensemble $X$ est un sous-ensemble de l'espace $\mathbb{R}^n$ 
 qui comprends tous les vecteurs possibles où 
 $n \in [1, n \longrightarrow \infty[$.

Une solution \textbf{optimale} est une solution réalisable qui maximise (ou minimise) la fonction objectif $z(x)$, en fonction de la nature du problème (maximisation ou minimisation). Autrement dit, c'est une solution réalisable qui produit la \textit{meilleure valeur possible} pour la fonction objectif $z$.

\vspace{2.5em}
\chapter{Maximisation et minimisation}

\begin{note}{}{}
    Un modèle peut présenter une \textbf{infinité} de 
    solution optimales. 
\end{note}       

\begin{theorem}{}{}
    La maximisation d'un objectif $f(w)$ est équivalent à l'\textbf{opposé}
    de la minimisation :

    \begin{align*}
        \text{Max} \; f(w) \leftrightarrow  \text{Min} -f(w)
    \end{align*}
\end{theorem}

\begin{Preuve}{}{}
    Considérons le problème :
    \begin{align*}
        \text{Max} \; f(w) \\ 
        \text{s.a} \;\; w \in X \subset \mathbb{R}^n
    \end{align*}
    Soit $w_{\text{opt}}$ un point de $X$ où $f(w)$ atteint son  
    \textbf{\textcolor{myb}{maximum}}. On a :  
    \begin{align*}
        f\left(w_{\text{opt}} \right) \geq &f(w) \;\; \forall w \in X \\
                                           &\updownarrow \\ 
        - f\left(w_{\text{opt}} \right) \leq - &f(w) \;\; \forall w \in X 
    \end{align*}
    Par conséquent, $-f(w_{\text{opt}})$ est la valeur minimale possible parmi 
    tous les $-f(w)$ possibles :

    \begin{align*}
        - \text{Max} f(w) = 
        - f\left(w_{\text{opt}}\right) = 
        \text{Min} -f(w)
    \end{align*}
    Ainsi, qu'on maximise $f(w)$ ou qu'on minimise $-f(w)$, on retrouve la 
    même solution optimale $w_{\text{opt}}$.
\end{Preuve}






\begin{Explication*}{}{}
    Soit la fonction objectif $f : X \rightarrow \mathbb{R}$  
Dans cette situation, le vecteur \( w \) joue deux rôles distincts : il est à la 
fois l'argument de la fonction objectif \( f(w) \) et un élément soumis aux 
contraintes. Ces deux aspects sont essentiels dans un problème de maximisation 
en programmation linéaire ou non linéaire.

\vspace{1em}%
\textbf{1. Vecteur dans la fonction objectif :}
La fonction objectif \( f(w) \) est la fonction que l'on cherche à maximiser. 
Elle prend un vecteur \( w \) comme entrée, qui est un point dans l'espace 
\( \mathbb{R}^n \). Ce vecteur \( w \) représente une solution candidate du 
problème, et \( f(w) \) retourne un nombre réel, par exemple un profit ou une 
performance, que l'on souhaite maximiser.

\vspace{1em}%
\textbf{2. Vecteur dans les contraintes :}
Le vecteur \( w \) doit aussi satisfaire un ensemble de contraintes qui 
définissent l'ensemble des solutions réalisables, noté \( X \subset \mathbb{R}^n \).
Ces contraintes peuvent être des inégalités ou des égalités (par exemple, 
\( g(w) \leq 0 \), \( h(w) = 0 \)), et elles définissent l'ensemble des vecteurs 
\( w \) admissibles.

\vspace{1em}%
\textbf{Synthèse des deux rôles :}
\begin{itemize}
    \item \textbf{Maximisation :} On cherche à maximiser la fonction \( f(w) \), 
    c'est-à-dire à trouver la meilleure valeur possible de \( f(w) \).
    \item \textbf{Contraintes :} Le vecteur \( w \) doit se situer dans un 
    ensemble admissible \( X \), défini par les contraintes du problème.
\end{itemize}

En résumé, \( w \) doit satisfaire les contraintes (ce qui garantit qu'il est 
une solution réalisable), tout en maximisant la valeur de \( f(w) \). Le problème
consiste à trouver le \( w \) optimal qui satisfait ces deux aspects à la fois.
\end{Explication*}                  

\chapter{Algorithme du Simplexe}
Soit le problème suivant. Un restaurateur dispose de \textbf{14} oursins, 
\textbf{24} crevettes, \textbf{18} huîtres. Deux types d'assiètes 
de fruit de mer sont offertes par le restaurateur :

\begin{align*}
    \textbf{8\$} \colon \text{5 oursins, 2 crevettes 1 huître}  \\
    \textbf{6\$} \colon \text{3 oursins, 3 crevettes 3 huîtres}   
\end{align*}

L'objectif est de déterminer le \textbf{nombres d'assiètes} de chaque 
type à préparer afin de \textit{maximiser} 
le revenu total du restaurateur tout en 
respectant les disponibilité de fruits de mer. 


\section{Contraintes}
\begin{align*}
    5x + 3y \leq 30& \quad \text{Oursins} \\
    2x + 3y \leq 24& \quad \text{crevettes} \\ 
    x  + 3y \leq 18& \quad \text{Huîtres}
    \\
    \\ 
    x, y \geq 0& \quad \text{Non-négativité}
    \\ 
    \\
    z = 8x + 6x& \quad \text{Objectif}
\end{align*}




En considérant que la maximisation d'un objectif est 
équivalente à la minimisation de l'opposé de la minimisation on a :


\begin{align*}
    \text{Min} \;\; -8x  -6y& \\
    \text{s.a.}
    \\
    \\
    5x + 3y \leq 30& \\
    2x + 3y \leq 24& \\
    x  + 3y \leq 18& 
    \\
    \\ 
    x, y \geq 0& 
    \\ 
    \\
    z = 8x + 6x& 
\end{align*}


\section{Forme standard}

\begin{align*}
    \text{Min } z = -8 -6y \\ 
    \text{s.a.} 
    \\
    5x + 3y& + \textcolor{myr}{\textbf{\textit{u}}}   &= 30 \\
    2x + 3y& \quad\quad + \textcolor{myr}{\textbf{\textit{p}}}   &= 24\\
     x + 3y& \quad\quad\quad\quad + \textcolor{myr}{\textbf{\textit{h}}}   &= 18 \\
     x, y&,  
     \textcolor{myr}{\textbf{\textit{u}}}, 
     \textcolor{myr}{\textbf{\textit{p}}},
     \textcolor{myr}{\textbf{\textit{h}}} \geq 0
\end{align*}

\section{Choisir les variables indépendantes}
Supposons que 
\textcolor{myb}{\textbf{\textit{x}}} et 
\textcolor{myb}{\textbf{\textit{y}}} sont des 
\textcolor{myb}{\textbf{\textit{variables indépendantes}}}. Exprimons 
les \textcolor{myr}{\textbf{\textit{variables dépendantes}}} 
\textcolor{myr}{\textbf{\textit{u}}},
\textcolor{myr}{\textbf{\textit{p}}},
\textcolor{myr}{\textbf{\textit{h}}},
\textcolor{myr}{\textbf{\textit{z}}} en fonction de 
\textcolor{myb}{\textbf{\textit{x}}} et \textcolor{myb}{\textbf{\textit{y}}}


\begin{align*}
    u \quad\quad\quad\quad\quad\quad &= 30 - 5x -3y& \\
    p \quad\quad\quad\quad  &= 24 -2x -3y& \\
    h \quad\quad &= 18 - x -3y&
\end{align*}
\vspace{-2.5em}    
\Pointilles{1}
\vspace{-2.5em}
\begin{align*}
    \quad\quad\quad\;\; z = 0 -8x -6y
\end{align*}
En supposant que 
\textcolor{myb}{\textbf{\textit{x}}} et 
\textcolor{myb}{\textbf{\textit{y}}}  sont \textbf{fixés à 0}, 
on a la \textbf{solution} : 

\begin{align*}
\boxed{x = y = 0 \implies 
\textcolor{myr}{\textbf{\textit{u}}}  = 30, 
\textcolor{myr}{\textbf{\textit{p}}} = 24,
\textcolor{myr}{\textbf{\textit{h}}} = 18,
\textbf{\textit{z}} = 0}
\end{align*}

En observant le système, on constate qu'il possible de 
\textbf{minimiser} $z$ en augmentant $x$ ; on considère 
donc $x$ comme \textcolor{myb}{\textbf{\textit{variable d'entrée}}}. Soit 
$y = 0$, on a :

\begin{align*}
    u &= 30 - 5x -3y& 
    \implies \quad 
    & x \leq 30/5 = \textcolor{myr}{\textbf{6}}  
    \\
    p &= 24 -2x -3y 
    & \implies \quad  &  x \leq 24/2 = \textbf{12}  
    \\
    h &= 18 - x -3y&
    \implies \quad & x \leq \textbf{18}  
\end{align*}

\textbf{Explication} : Soit $u = 30 - 5x -3\cancelto{0}{y}$, on sait que pour respecter 
la contrainte de non négativité, la quantité $u = 30 -5x \geq$. Cela se simplifie par 
\begin{align*}
    x \leq 6
\end{align*}


\section{Pivot}

Le système est limité par la diminution de $x$ jusqu'à \textcolor{myr}{\textbf{6}} 
L'équation limitante est celle qui implique $u$; on dit alors que 
\textcolor{myr}{\textbf{\textit{u}}} est le 
\textcolor{myr}{\textbf{\textit{variable de sortie}}}. On 
\textbf{pivote}. La variable $u$ prend la place de $x$ en tant que
variable indépendante;
$\textcolor{myb}{\textbf{\textit{u}}}$ et  
\textcolor{myb}{\textbf{\textit{y}}} sont les 
\textcolor{myb}{\textbf{\textit{nouvelles variables indépendantes}}}.   

On peut utiliser les équations du système pour extrapoler 
$x$, $p$, et $h$ en fonction de $u$ et $y$ :


\begin{align*}
    u \quad = 30 - 5x &- 3y 
    \textbf{ et } \textcolor{myb}{u}, \textcolor{myb}{y} = 0 
    \\
                         &\Downarrow
\end{align*}
      \[% 
          \boxed{x = \textcolor{myg}{\textit{\textbf{6 $-$ 1/5u $-$ 3/5y}}}}
      \]%


\begin{align*}
    p \quad = 24 - 2x &- 3y
    \textbf{ et } \textcolor{myb}{u}, \textcolor{myb}{y} = 0 
    \\
                 &\Downarrow
\end{align*}
      \[% 
    \boxed{p = 12 +2/5u -9/5y}
      \]%

\begin{align*}
    h \quad = 18 - x &- 3y
    \textbf{ et } \textcolor{myb}{u}, \textcolor{myb}{y} = 0 
    \\
                     &\Downarrow
\end{align*}
      \[% 
    \boxed{h = 12 +1/5u -12/5y}
      \]%
\begin{align*}
    z \quad = 0 - 8x &- 6y
    \textbf{ et } \textcolor{myb}{u}, \textcolor{myb}{y} = 0 
    \\
                     &\Downarrow
    \\
    z \quad = 0 - 8(&\textcolor{myg}{\textit{\textbf{6 $-$ 1/5u $-$ 3/5y}}})
 - 6y
    \\ 
                    &\Downarrow
\end{align*}
      \[% 
    \boxed{z = -48 +8/5u -6/5y}
      \]%

\section{Système 2}
\begin{align*}
    x \quad\quad\quad\quad\quad\quad &= 6 - 1/5u -3/5y& \\
    p \quad\quad\quad\quad  &= 12 -2/5u -9/5y& \\
    h \quad\quad &= 12 - 1/5u -12/5y&
\end{align*}
\vspace{-2.5em}    
\Pointilles{1}
\vspace{-2.5em}
\begin{align*}
    \quad\quad\quad\;\; z = -48 +8/5x -6/5y
\end{align*}
Sachant que 
\textcolor{myb}{\textbf{\textit{u}}} et 
\textcolor{myb}{\textbf{\textit{y}}}  sont \textbf{fixés à 0}, 
on a la \textbf{solution} : 

\begin{align*}
\boxed{u = y = 0 \implies 
\textcolor{myr}{\textbf{\textit{x}}}  = 6, 
\textcolor{myr}{\textbf{\textit{p}}} = 12,
\textcolor{myr}{\textbf{\textit{h}}} = 12,
\textbf{\textit{z}} = -48}
\end{align*}

En observant le système, on constate qu'il possible de 
\textbf{minimiser} $z$ en augmentant $y$ ; on considère 
donc $y$ comme \textcolor{myb}{\textbf{\textit{variable d'entrée}}}. Soit 
$u = 0$, on a :
\begin{align*}
    x &= 6 - 1/5u -3/5y& 
    \implies \quad 
    & y \leq \textbf{10}  
    \\
    p &= 12 - 2/5u -9/5y 
      & \implies \quad  &  y \leq \textbf{20/3}
    \\
    h &= 12 - 1/5u -12/5y&
    \implies \quad & y \leq \textbf{\textcolor{myr}{5}} 
\end{align*}

\section{Pivot}
Le système est limité par la diminution de $y$ jusqu'à \textcolor{myr}{\textbf{5}} 
L'équation limitante est celle qui implique $h$; on dit alors que 
\textcolor{myr}{\textbf{\textit{h}}} est le 
\textcolor{myr}{\textbf{\textit{variable de sortie}}}. On 
\textbf{pivote}. La variable $h$ prend la place de $u$ en tant que
variable indépendante;
$\textcolor{myb}{\textbf{\textit{h}}}$ et  
\textcolor{myb}{\textbf{\textit{y}}} sont les 
\textcolor{myb}{\textbf{\textit{nouvelles variables indépendantes}}}.   

On peut utiliser les équations du système pour extrapoler 
$x$, $p$, et $h$ en fonction de $h$ et $y$ :


\begin{align*}
    h \quad = 12 - 1/5u &-12/5y
    \textbf{ et } \textcolor{myb}{h}, \textcolor{myb}{y} = 0 
    \\
                 &\Downarrow
\end{align*}
      \[% 
    \boxed{y = \textcolor{myg}{\textbf{\textit{5 + 1/12u $-$ 5/12h}}} }
      \]%


\begin{align*}
    x = 6 - 1/5u &-3/5y 
    \textbf{ et } \textcolor{myb}{h}, \textcolor{myb}{y} = 0 
    \\
      &\Downarrow
    \\ 
    x = 6 - 1/5u -3/&5(\textcolor{myg}{\textbf{\textit{5 + 1/12u $-$ 5/12h}}}) 
    \\
                    &\Downarrow
\end{align*}

\vspace{-1em}
      \[% 
          \boxed{x = 3 -1/4u -1/4y}
      \]%

\begin{align*}
    p = 12 - 2/5u &-9/5y 
    \textbf{ et } \textcolor{myb}{h}, \textcolor{myb}{y} = 0 
    \\
      &\Downarrow
    \\ 
    p = 12 - 2/5u -9/&5(\textcolor{myg}{\textbf{\textit{5 + 1/12u $-$ 5/12h}}}) 
    \\
                    &\Downarrow
\end{align*}
\vspace{-1em}
      \[% 
          \boxed{p = 3 + 1/4u -3/4y}
      \]%


\begin{align*}
    z \quad = -48 +8/5u &- 6/5y
    \textbf{ et } \textcolor{myb}{h}, \textcolor{myb}{y} = 0 
    \\
                      &\Downarrow \\
    z = -48 +8/5u -6&(5 +1/12u -5/12h)
    \\
                    &\Downarrow
\end{align*}

\vspace{-1.25em}
      \[% 
    \boxed{z = -54 +3/2u -1/2h}
      \]%

\section{Système 3}
\begin{align*}
    x \quad\quad\quad\quad\quad\quad &= 3 - 1/4u -1/4h& \\
    p \quad\quad\quad\quad  &= 3 +1/4u +3/5h& \\
    y \quad\quad &= 5 + 1/12u -5/12h&
\end{align*}
\vspace{-2.5em}    
\Pointilles{1}
\vspace{-2.5em}
\begin{align*}
    \quad\quad\quad\;\; z = -54 +3/2u -1/2h
\end{align*}
Sachant que 
\textcolor{myb}{\textbf{\textit{u}}} et 
\textcolor{myb}{\textbf{\textit{h}}}  sont \textbf{fixés à 0}, 
on a la \textbf{solution} : 

\begin{align*}
\boxed{u = h = 0 \implies 
\textcolor{myr}{\textbf{\textit{x}}}  = 3, 
\textcolor{myr}{\textbf{\textit{p}}} = 3,
\textcolor{myr}{\textbf{\textit{y}}} = 5,
\textbf{\textit{z}} = -54}
\end{align*}

Il n’est pas intéressant d’augmenter ni la valeur de $u$, ni la valeur de $h$car la valeur 
de z augmente. Ainsi, nous sommes \textbf{à l’optimum} !

\end{multicols*}

\section{Résumé de l'algorithme}

\noindent 
$\blacktriangleright$ \textbf{Compter} le nombre d'équations. Un système 
en \textit{forme standard} de $N$ équations nécessaite une 
\textcolor{myb}{\textbf{variables d'équart}} par équation, pour un total de 
$N$ variable d'équart. 

\begin{note}{}{}
    Si le système sous sa forme standard possède $n$ variables inconnues, ce système 
    doit avoir \textit{au minimum} $n$ équations pour pouvoir trouver une 
    \textcolor{myb}{\textbf{solution unique}}. Autrement, le sytème pourrait 
    admettre \textbf{une infinité de solutions}.   
\end{note}


\noindent 
$\blacktriangleright$ \textbf{Exprimer} les \textcolor{myb}{\textbf{variables d'équart}} 
$u, p, h$, etc. et l'objectif $\textcolor{myr}{z}$ en fonction des des 
\textbf{variables fixées à zéro}, c'est-à-dire les variables inconnues 
$x_1, x_2, \cdots,  x_n$.  

\noindent 
$\blacktriangleright$ \textbf{Identifier} la variable dont l'$\uparrow$ réduit le plus 
$z$; cette variable minimise l'objectif et est donc la 
\textcolor{myb}{\textbf{variable d'entrée}} $x_e$.

\noindent
$\rhd$ Soit la variable $x_e$, trouver la variable $x_s$ qui limite le plus l'augmentation 
de $x_e$. Il s'agit de la \textcolor{myr}{\textbf{variable de sortie}} $x_s$. 


\noindent 
$\blacktriangleright$ \textbf{Fixer} $x_s$ à zéro. La variable de sortie $x_s$ 
prend la place de la variable d'entrée $x_e$. Exprimer $x_e$ et les autres variables 
$x_1, x_2, \cdots, x_n$, ainsi que l'objectif $z$ en fonction de $x_s$. 


\noindent 
$\blacktriangleright$ \textbf{Évaluer} le nouveau système d'équation obtenu. 

\noindent 
$\rhd$ \textbf{Si l'augmentation } 
\textcolor{myr}{\textbf{d'aucune}} variable dans la fonction  objectif n'entraîne une  
diminution de l'objectif, on a la \textbf{solution optimale}. L'algorithme est terminé.  

\noindent 
$\rhd$ Si l'augmentation \textcolor{myr}{\textbf{d'au moins une}} variable entraîne  
la diminution de l'objectif, la solution n'est pas optiomale et il faut poursuivre avec 
\textbf{une autre itération} de l'algorithme.       
\pagebreak 



\begin{multicols*}{2}

\chapter{Représentation forme tableau }

\section{Forme standard}

Soit le sytème suivant sous la forme standard, nous allons 
utiliser la représentation sous forme de tableau pour appliquer
l'algorithme du simplexe. 

\begin{align*}
    \text{Min } z = -8 -6y \\ 
    \text{s.a.} 
    \\
    5x + 3y& + \textcolor{myr}{\textbf{\textit{u}}}   &= 30 \\
    2x + 3y& \quad\quad + \textcolor{myr}{\textbf{\textit{p}}}   &= 24\\
     x + 3y& \quad\quad\quad\quad + \textcolor{myr}{\textbf{\textit{h}}}   &= 18 \\
     x, y&,  
     \textcolor{myr}{\textbf{\textit{u}}}, 
     \textcolor{myr}{\textbf{\textit{p}}},
     \textcolor{myr}{\textbf{\textit{h}}} \geq 0
\end{align*}


        \begin{table}[H]
                \begin{center}
                    \renewcommand{\arraystretch}{1.5}
                    \fontfamily{flr}\selectfont
                    \footnotesize
                        \begin{tabular}{|l|l l l l l |l|l|}
                        \arrayrulecolor{blue}
                        \hline
                        v.d. & $x$
                             & $y$ & $u$ & $p$ & $h$ & $-z$ & t.d 
                        \\
                        \hline
                        \arrayrulecolor{black}
                        \rowcolor{myr!40}
                        $\textcolor{myr}{u} 
                        $     & 5
                                & 3
                                & 1
                                & 
                                & &  &  30
                        \\
                        $\textcolor{myr}{p} $     
                               & 2
                               & 3
                               & & 1
                               & & & 24 
                        \\
                        $\textcolor{myr}{h} $     
                               & 1
                               & 3 
                               &  & 
                               &  1
                               & & 18 
                        \\ 
                        \hline
                        $\textcolor{myr}{-z}$ 
                                & -8
                                & -6
                                & 
                                & 
                                & 
                                & 1 & 0 
                        \\
                        \hline 



                        \end{tabular}
                \end{center}
        \caption{État initial de la forme \texttt{STD}}
        \end{table}         
Le tableau doit avoir la même allure que la forme standard ; 
il doit être possible d'identifier une \textbf{variable} par 
ligne du tableau. \textbf{Par exemple}, Le ligne surlignée en rouge engendre  
l'équation : 
\begin{align*}
        5x  + 3y + u = 30
\end{align*}

Dans ce contexte, on sait que $u$ est une \textbf{variable dépendante} (v.d.)  
et le terme de droite qui lui est associé (t.d.) est $30$. 



Tout comme durant la résolution algébrique, on constate que si $u, p$ et $h$ sont 
des \textbf{variables dépendantes} et que $x$ et $y$ sont des 
variables indépendantes \textbf{fixées à zéro}, la solution associée à ce système 
est : 
\begin{align*}
    \boxed{\left[ x, y = 0 \right] \implies u = 30, p = 24, h = 18, z = 0} 
\end{align*}

\section{Choix de la variable d'entrée}

        \begin{table}[H]
                \begin{center}
                    \renewcommand{\arraystretch}{1.5}
                    \fontfamily{flr}\selectfont
                    \footnotesize
                        \begin{tabular}{|l|l l l l l |l|l|}
                        \arrayrulecolor{blue}
                        \hline
                        v.d. & $x$
                             & $y$ & $u$ & $p$ & $h$ & $-z$ & t.d 
                        \\
                        \hline
                        \arrayrulecolor{black}
                        $\textcolor{myr}{u} 
                        $     & 5
                                & 3
                                & 1
                                & 
                                & &  &  30
                        \\
                        $\textcolor{myr}{p} $     
                               & 2
                               & 3
                               & & 1
                               & & & 24 
                        \\
                        $\textcolor{myr}{h} $     
                               & 1
                               & 3 
                               &  & 
                               &  1
                               & & 18 
                        \\ 
                        \hline
                        $\textcolor{myr}{-z}$ 
                               & \cellcolor{myr!40}-8
                                & -6
                                & 
                                & 
                                & 
                                & 1 & 0 
                        \\
                        \hline 



                        \end{tabular}
                \end{center}
        \caption{Variable d'entrée choisie}
        \end{table}         
 On choisit la variable dont l'augmentation rend l'objectif $z$ 
plus petit et négatif ; il sagit de \textcolor{myr}{\textbf{$x$ }} qui 
est la \textcolor{myb}{\textbf{variable d'entrée}}.


\section{Choix de la variable de sortie}
Soit une ligne du tableau $l$, et la variable d'entrée 
$x_e$, pour trouver la variable dépendante 
\textbf{limite davantage l'augmentation} de la 
\textcolor{myb}{\textbf{variable d'entrée}}, il suffit d'utiliser la formule  : 


\begin{align*}
    \boxed{\text{lim } = \text{t.d.}(l) \; \div \; x_e(l)}
\end{align*}


\noindent
\textbf{Exemple} 
\begin{align*}
    &\text{lim}(\textcolor{myr}{\textbf{u}} ) = 
    \text{t.d.(\textcolor{myr}{\textbf{u}} )} \div 
    x(\textcolor{myr}{\textbf{u}} ) = 30 \div 5 = \textcolor{myr}{6}
    \\
    &\text{lim}(\textcolor{myr}{\textbf{p}} ) = 
    \text{t.d.(\textcolor{myr}{\textbf{p}} )} \div 
    x(\textcolor{myr}{\textbf{p}} ) = 24 \div 2 = 12 
    \\
    &\text{lim}(\textcolor{myr}{\textbf{h}} ) = 
    \text{t.d.(\textcolor{myr}{\textbf{h}} )} \div 
    x(\textcolor{myr}{\textbf{h}} ) = 18 \div 1 = 12 
\end{align*}

Dans ce cas-ci, $\textcolor{myr}{\boldsymbol{u}}$ limite davantage l'augementation de 
$\textcolor{myb}{\boldsymbol{x}}$ ; $\textcolor{myr}{\boldsymbol{u}}$ est donc la variable de sortie.

\section{Pivot}

        \begin{table}[H]
                \begin{center}
                    \renewcommand{\arraystretch}{1.5}
                    \fontfamily{flr}\selectfont
                    \footnotesize
                        \begin{tabular}{|l|l l l l l |l|l|}
                        \arrayrulecolor{blue}
                        \hline
                        v.d. & $\textcolor{myb}{\boldsymbol{x}}$
                             & $y$ & $\textcolor{myr}{\boldsymbol{u}}$ & $p$ & $h$ & $-z$ & t.d 
                        \\
                        \hline
                        \arrayrulecolor{black}
                        $\textcolor{myr}{\boldsymbol{u}} 
                        $     & \cellcolor{myg!40} $\textcolor{myg}{\boldsymbol{5}}$ 
                                & 3
                                & 1
                                & 
                                & &  &  30
                        \\
                        $\textcolor{myr}{p} $     
                               & 2
                               & 3
                               & & 1
                               & & & 24 
                        \\
                        $\textcolor{myr}{h} $     
                               & 1
                               & 3 
                               &  & 
                               &  1
                               & & 18 
                        \\ 
                        \hline
                        $\textcolor{myr}{-z}$ 
                               & -8
                                & -6
                                & 
                                & 
                                & 
                                & 1 & 0 
                        \\
                        \hline 



                        \end{tabular}
                \end{center}
        \caption{Coefficient pivot}
        \end{table}

    Le coefficient à l'intersection de la colonne de la variable d'entrée 
    ($\textcolor{myb}{\boldsymbol{x}}$) et de la rangée de la variable de sortie 
    ($\textcolor{myr}{\boldsymbol{u}}$) est le \textcolor{myg}{\textit{coefficient pivot}}. 
    Le \textbf{pivot} est l'action par laquelle on modifie le tableau pour que la variable 
    d'entrée prenne la place de la variable de sortie. 


        \begin{table}[H]
                \begin{center}
                    \renewcommand{\arraystretch}{1.5}
                    \fontfamily{flr}\selectfont
                    \footnotesize
                        \begin{tabular}{|l|l l l l l |l|l|}
                        \arrayrulecolor{blue}
                        \hline
                        v.d. & $x$
                             & $y$ & $u$ & $p$ & $h$ & $-z$ & t.d 
                        \\
                        \hline
                        \arrayrulecolor{black}
                        $\textcolor{myb}{\boldsymbol{x}} 
                        $     & $\textcolor{myg}{\boldsymbol{5}}$ 
                                & 3
                                & 1
                                & 
                                & &  &  30
                        \\
                        $\textcolor{myr}{p} $     
                               & 2
                               & 3
                               & & 1
                               & & & 24 
                        \\
                        $\textcolor{myr}{h} $     
                               & 1
                               & 3 
                               &  & 
                               &  1
                               & & 18 
                        \\ 
                        \hline
                        $\textcolor{myr}{-z}$ 
                               & -8
                                & -6
                                & 
                                & 
                                & 
                                & 1 & 0 
                        \\
                        \hline 



                        \end{tabular}
                \end{center}
        \caption{État suite au pivot}
        \end{table}
    Pour que $x$ soit \textcolor{myr}{variable dépendante } à la place de u dans la première ligne, il faut un 
    coefficient de 1 sous $x$ dans la première ligne et des coefficients de 0 dans les autres 
    lignes. Cela revient à diviser la ligne de la variable de sortie par le coefficient du pivot.

        \begin{table}[H]
                \begin{center}
                    \renewcommand{\arraystretch}{1.5}
                    \fontfamily{flr}\selectfont
                    \footnotesize
                        \begin{tabular}{|l|l l l l l |l|l|}
                        \arrayrulecolor{blue}
                        \hline
                        v.d. & $x$
                             & $y$ & $u$ & $p$ & $h$ & $-z$ & t.d 
                        \\
                        \hline
                        \arrayrulecolor{black}
                        $\textcolor{myg}{\boldsymbol{x}} 
                        $     & $\textcolor{myg}{\boldsymbol{1}}$ 
                              & $\textcolor{myg}{\boldsymbol{\nicefrac{3}{5}}}$
                                & $\textcolor{myg}{\boldsymbol{\nicefrac{1}{5}}}$
                                & 
                                & &  &  $\textcolor{myg}{\boldsymbol{\nicefrac{30}{5}}}$
                        \\
                        $\textcolor{myr}{p} $     
                               & 2
                               & 3
                               & & 1
                               & & & 24 
                        \\
                        $\textcolor{myr}{h} $     
                               & 1
                               & 3 
                               &  & 
                               &  1
                               & & 18 
                        \\ 
                        \hline
                        $\textcolor{myr}{-z}$ 
                               & -8
                                & -6
                                & 
                                & 
                                & 
                                & 1 & 0 
                        \\
                        \hline 



                        \end{tabular}
                \end{center}
        \caption{Division par le cofficient du pivot}
        \end{table}

        Pour exprimer chaque variable dépendante en terme de la variable d'entrée ($x$)
        par le coefficient $c_e$ sous la colonne de la variable d'entrée. 


        \noindent 
        \textbf{Exemple} 
        \begin{align*}
            \boxed{\textcolor{myg}{x = 6 - \nicefrac{3}{5}y  - \nicefrac{1}{5}u } }
        \end{align*}


    \begin{align*}
        &p = 24 - 2x -3y \\ &\leftrightarrow p = 24 - 2(\textcolor{myg}{6 - \nicefrac{3}{5}y  - \nicefrac{1}{5}u}) - 3y \\
        &\leftrightarrow p = 24 - 2(\textcolor{myg}{6 - \nicefrac{3}{5}y  - \nicefrac{1}{5}u}) 
        \; \textcolor{myp}{ + \; 2x } \; \textcolor{myr}{- \;2x} - 3y \\
        &\leftrightarrow \textcolor{myr}{2x}   
        +3y + p + 2(\textcolor{myg}{6}  ) 
        -2(\nicefrac{3}{5}y  + \nicefrac{1}{5}u) \textcolor{myp}{-2x} = 24 \\
        &\leftrightarrow \textcolor{myr}{2x}   
        +3y + p   
        -2(\textcolor{myp}{x} + \nicefrac{3}{5}y  + \nicefrac{1}{5}u) = 24  -2(\textcolor{myg}{6}) 
    \end{align*}
        Soit $ligne(p) : 2x + 3y + p = 24$ 
        et $ligne(x) : x + \nicefrac{3}{5} + \nicefrac{1}{5} = 6$, 
        Cela revient à effectuer $ligne(p) \; - \; 2 \; \times \; ligne(x)$. On obtient alors 
        le tableau suivant 

\begin{table}[H]
                \begin{center}
                    \renewcommand{\arraystretch}{1.5}
                    \fontfamily{flr}\selectfont
                    \footnotesize
                        \begin{tabular}{|l|l l l l l |l|l|}
                        \arrayrulecolor{blue}
                        \hline
                        v.d. & $x$
                             & $y$ & $u$ & $p$ & $h$ & $-z$ & t.d 
                        \\
                        \hline
                        \arrayrulecolor{black}
                        $\textcolor{myg}{\boldsymbol{x}} 
                        $     & $\textcolor{myg}{\boldsymbol{1}}$ 
                              & $\textcolor{myg}{\boldsymbol{\nicefrac{3}{5}}}$
                                & $\textcolor{myg}{\boldsymbol{\nicefrac{1}{5}}}$
                                & 
                                & &  &  $\textcolor{myg}{\boldsymbol{6}}$
                        \\
                        $\textcolor{myg}{\boldsymbol{p}} $     
                                & $\textcolor{myg}{\boldsymbol{0}}$             
                                & $\textcolor{myg}{\boldsymbol{\nicefrac{9}{5}}}$
                                & $\textcolor{myg}{\boldsymbol{-\nicefrac{2}{5}}}$ & 
                                1 & & & $\textcolor{myg}{\boldsymbol{12}}$

                        \\
                        $\textcolor{myr}{h} $     
                               & 1
                               & 3 
                               &  & 
                               &  1
                               & & 18 
                        \\ 
                        \hline
                        $\textcolor{myr}{-z}$ 
                               & -8
                                & -6
                                & 
                                & 
                                & 
                                & 1 & 0 
                        \\
                        \hline 



                        \end{tabular}
                \end{center}
        \caption{$ligne(p) \; - \; 2 \; \times \; ligne(x)$}
        \end{table}


\begin{table}[H]
                \begin{center}
                    \renewcommand{\arraystretch}{1.5}
                    \fontfamily{flr}\selectfont
                    \footnotesize
                        \begin{tabular}{|l|l l l l l |l|l|}
                        \arrayrulecolor{blue}
                        \hline
                        v.d. & $x$
                             & $y$ & $u$ & $p$ & $h$ & $-z$ & t.d 
                        \\
                        \hline
                        \arrayrulecolor{black}
                        $\textcolor{black}{{x}} 
                        $     & $\textcolor{black}{{1}}$ 
                              & $\textcolor{black}{{\nicefrac{3}{5}}}$
                                & $\textcolor{black}{{\nicefrac{1}{5}}}$
                                & 
                                & &  &  $\textcolor{black}{{6}}$
                        \\
                        $\textcolor{black}{{p}} $     
                                & $\textcolor{black}{{0}}$  
                                & $\textcolor{black}{{\nicefrac{9}{5}}}$
                               & $\textcolor{black}{{-\nicefrac{2}{5}}}$ & 1 
                               & & & $\textcolor{black}{{12}}$

                        \\
                    $\textcolor{black}{{h}} $     
                                & $\textcolor{black}{{0}}$  
                                & $\textcolor{black}{{\nicefrac{12}{5}}}$
                                & $\textcolor{black}{{-\nicefrac{1}{5}}}$ &  
                                & 1 & & $\textcolor{black}{{12}}$
                        \\ 
                        \hline
                        $\textcolor{myr}{-z}$ 
                               & -8
                                & -6
                                & 
                                & 
                                & 
                                & 1 & 0 
                        \\
                        \hline 



                        \end{tabular}
                \end{center}
        \caption{$ligne(h) \; - \; 1 \; \times \; ligne(x)$}
        \end{table}

\begin{table}[H]
                \begin{center}
                    \renewcommand{\arraystretch}{1.5}
                    \fontfamily{flr}\selectfont
                    \footnotesize
                        \begin{tabular}{|l|l l l l l |l|l|}
                        \arrayrulecolor{blue}
                        \hline
                        v.d. & $x$
                             & $y$ & $u$ & $p$ & $h$ & $-z$ & t.d 
                        \\
                        \hline
                        \arrayrulecolor{black}
                        $\textcolor{black}{{x}} 
                        $     & $\textcolor{black}{{1}}$ 
                              & $\textcolor{black}{{\nicefrac{3}{5}}}$
                                & $\textcolor{black}{{\nicefrac{1}{5}}}$
                                & 
                                & &  &  $\textcolor{black}{{6}}$
                        \\
                        $\textcolor{black}{{p}} $     
                                & $\textcolor{black}{{0}}$  
                                & $\textcolor{black}{{\nicefrac{9}{5}}}$
                               & $\textcolor{black}{{-\nicefrac{2}{5}}}$ & 1 
                               & & & $\textcolor{black}{{12}}$

                        \\
                    $\textcolor{black}{{h}} $     
                                & $\textcolor{black}{{0}}$  
                                & $\textcolor{black}{{\nicefrac{12}{5}}}$
                                & $\textcolor{black}{{-\nicefrac{1}{5}}}$ &  
                                & 1 & & $\textcolor{black}{{12}}$
                        \\ 
                        \hline
                        $\textcolor{black}{{-z}}$ 
                                &  $\textcolor{black}{{0}}$
                                & $\textcolor{black}{{-\nicefrac{6}{5}}}$
                                & $\textcolor{black}{{\nicefrac{8}{5}}}$
                                & 
                                & 
                                & 1 & $\textcolor{black}{{48}}$ 
                        \\
                        \hline 



                        \end{tabular}
                \end{center}
        \caption{$ligne(-z) \; + \; 8 \; \times \; ligne(x)$}
        \end{table}


        Nous avons complété \textbf{une itération} de l'algorithme.
        On obtien alors la solution suivante :
        \begin{align*}
            \boxed{\left[y, u = 0 \right] \implies x = 6, p = 12, h = 12, z = -48}
        \end{align*}

        \section{Choix de la variable d'entrée}

\begin{table}[H]
                \begin{center}
                    \renewcommand{\arraystretch}{1.5}
                    \fontfamily{flr}\selectfont
                    \footnotesize
                        \begin{tabular}{|l|l l l l l |l|l|}
                        \arrayrulecolor{blue}
                        \hline
                        v.d. & $x$
                             & $y$ & $u$ & $p$ & $h$ & $-z$ & t.d 
                        \\
                        \hline
                        \arrayrulecolor{black}
                        $\textcolor{black}{{x}} 
                        $     & $\textcolor{black}{{1}}$ 
                              & $\textcolor{black}{{\nicefrac{3}{5}}}$
                                & $\textcolor{black}{{\nicefrac{1}{5}}}$
                                & 
                                & &  &  $\textcolor{black}{{6}}$
                        \\
                        $\textcolor{black}{{p}} $     
                                & $\textcolor{black}{{0}}$  
                                & $\textcolor{black}{{\nicefrac{9}{5}}}$
                               & $\textcolor{black}{{-\nicefrac{2}{5}}}$ & 1 
                               & & & $\textcolor{black}{{12}}$

                        \\
                    $\textcolor{black}{{h}} $     
                                & $\textcolor{black}{{0}}$  
                                & $\textcolor{black}{{\nicefrac{12}{5}}}$
                                & $\textcolor{black}{{-\nicefrac{1}{5}}}$ &  
                                & 1 & & $\textcolor{black}{{12}}$
                        \\ 
                        \hline
                        $\textcolor{black}{{-z}}$ 
                                &  $\textcolor{black}{{0}}$
                                & \cellcolor{myr!40}$\textcolor{black}{{-\nicefrac{6}{5}}}$
                                & $\textcolor{black}{{\nicefrac{8}{5}}}$
                                & 
                                & 
                                & 1 & $\textcolor{black}{{48}}$ 
                        \\
                        \hline 



                        \end{tabular}
                \end{center}
        \end{table}

La variable de sortie est donc $y$
\section{Choix de la variable de sortie}
\textbf{Démonstration} 
\begin{align*}
    &\text{lim}(\textcolor{myr}{\textbf{x}} ) = 
    \text{t.d.(\textcolor{myr}{\textbf{x}} )} \div 
    y(\textcolor{myr}{\textbf{x}} ) = 6  \div \nicefrac{3}{5} = 10
    \\
    &\text{lim}(\textcolor{myr}{\textbf{p}} ) = 
    \text{t.d.(\textcolor{myr}{\textbf{p}} )} \div 
    y(\textcolor{myr}{\textbf{p}} ) = 12 \div \nicefrac{9}{5} = \nicefrac{20}{3}
    \\
    &\text{lim}(\textcolor{myr}{\textbf{h}} ) = 
    \text{t.d.(\textcolor{myr}{\textbf{h}} )} \div 
    y(\textcolor{myr}{\textbf{h}} ) = 18 \div \nicefrac{12}{5} = \textcolor{myr}{5} 
\end{align*}

    La variable de sortie est donc $h$, puisqu'après 
    $y \leq 5$, $h$ devient négatif, ce qui viole la contrainte de \textbf{non-négativité}. 


\begin{table}[H]
                \begin{center}
                    \renewcommand{\arraystretch}{1.5}
                    \fontfamily{flr}\selectfont
                    \footnotesize
                        \begin{tabular}{|l|l l l l l |l|l|}
                        \arrayrulecolor{blue}
                        \hline
                        v.d. & $x$
                             & $y$ & $u$ & $p$ & $h$ & $-z$ & t.d 
                        \\
                        \hline
                        \arrayrulecolor{black}
                        $\textcolor{black}{{x}} 
                        $     & $\textcolor{black}{{1}}$ 
                              & $\textcolor{black}{{\nicefrac{3}{5}}}$
                                & $\textcolor{black}{{\nicefrac{1}{5}}}$
                                & 
                                & &  &  $\textcolor{black}{{6}}$
                        \\
                        $\textcolor{black}{{p}} $     
                                & $\textcolor{black}{{0}}$  
                                & $\textcolor{black}{{\nicefrac{9}{5}}}$
                               & $\textcolor{black}{{-\nicefrac{2}{5}}}$ & 1 
                               & & & $\textcolor{black}{{12}}$

                        \\
                    $\textcolor{black}{{h}} $     
                                & $\textcolor{black}{{0}}$  
                                & \cellcolor{myg!40}$\textcolor{myg}{{\nicefrac{12}{5}}}$
                                & $\textcolor{black}{{-\nicefrac{1}{5}}}$ &  
                                & 1 & & $\textcolor{black}{{12}}$
                        \\ 
                        \hline
                        $\textcolor{black}{{-z}}$ 
                                &  $\textcolor{black}{{0}}$
                                & $\textcolor{black}{{-\nicefrac{6}{5}}}$
                                & $\textcolor{black}{{\nicefrac{8}{5}}}$
                                & 
                                & 
                                & 1 & $\textcolor{black}{{48}}$ 
                        \\
                        \hline 
                        \end{tabular}
                \end{center}
                \caption{Coefficient de pivot}
        \end{table}



    Puisque le pivot est $\textcolor{myg}{\boldsymbol{\nicefrac{12}{5}}}$, 
    on divise $ligne(h)$ par cette même valeur pour que $y$ soit variable 
    dépendante à $ligne(h)$. 

    \begin{table}[H]
                    \begin{center}
                        \renewcommand{\arraystretch}{1.5}
                        \fontfamily{flr}\selectfont
                        \footnotesize
                            \begin{tabular}{|l|l l l l l |l|l|}
                            \arrayrulecolor{blue}
                            \hline
                            v.d. & $x$
                                 & $y$ & $u$ & $p$ & $h$ & $-z$ & t.d 
                            \\
                            \hline
                            \arrayrulecolor{black}
                            $\textcolor{black}{{x}} 
                            $     & $\textcolor{black}{{1}}$ 
                                  & $\textcolor{black}{{\nicefrac{3}{5}}}$
                                    & $\textcolor{black}{{\nicefrac{1}{5}}}$
                                    & 
                                    & &  &  $\textcolor{black}{{6}}$
                            \\
                            $\textcolor{black}{{p}} $     
                                    & $\textcolor{black}{{0}}$  
                                    & $\textcolor{black}{{\nicefrac{9}{5}}}$
                                   & $\textcolor{black}{{-\nicefrac{2}{5}}}$ & 1 
                                   & & & $\textcolor{black}{{12}}$

                            \\

                        \rowcolor{myg!40}
                        \cellcolor{myr!40}
                        $\textcolor{myr}{{y}} $ 
                                    & $\textcolor{myg}{{0}}$  
                                    & $\textcolor{myg}{{1}}$
                                    & $\textcolor{myg}{{-\nicefrac{1}{12}}}$ &  
                                    & $\textcolor{myg}{\nicefrac{5}{12}}$ & & $\textcolor{myg}{5}$
                            \\ 
                            \hline
                            $\textcolor{black}{{-z}}$ 
                                    &  $\textcolor{black}{{0}}$
                                    & $\textcolor{black}{{-\nicefrac{6}{5}}}$
                                    & $\textcolor{black}{{\nicefrac{8}{5}}}$
                                    & 
                                    & 
                                    & 1 & $\textcolor{black}{{48}}$ 
                            \\
                            \hline 
                            \end{tabular}
                    \end{center}
                    \caption{$ligne(h) \div \nicefrac{12}{5}$}
            \end{table}

            On soustrait maintenant $ligne(h)$ à $ligne(p)$ et $ligne(x)$ 
            de façon à avoir un coefficient de 0 sous $y$. 
    \begin{table}[H]
                        \begin{center}
                            \renewcommand{\arraystretch}{1.5}
                            \fontfamily{flr}\selectfont
                            \footnotesize
                                \begin{tabular}{|l|l l l l l |l|l|}
                                \arrayrulecolor{blue}
                                \hline
                                v.d. & $x$
                                     & $y$ & $u$ & $p$ & $h$ & $-z$ & t.d 
                                \\
                                \hline
                                \arrayrulecolor{black}
                                $\textcolor{black}{{x}} 
                                $     & $\textcolor{black}{{1}}$ 
                                      & $\textcolor{black}{{\nicefrac{3}{5}}}$
                                        & $\textcolor{black}{{\nicefrac{1}{5}}}$
                                        & 
                                        & &  &  $\textcolor{black}{{6}}$
                                \\
                                \rowcolor{myg!40}
                            $\textcolor{myg}{\boldsymbol{p}} $     
                                        & $\textcolor{myg}{\boldsymbol{0}}$  
                                        & $\textcolor{myg}{\boldsymbol{0}}$
                                        & $\textcolor{myg}{{\boldsymbol{-\nicefrac{1}{4}}}}$ 
                                        & $\textcolor{myg}{\boldsymbol{1}}$
                                        & $\textcolor{myg}{\boldsymbol{\nicefrac{-3}{4}}}$      
                                        & & $\textcolor{myg}{\boldsymbol{3}}$

                                \\

                            \rowcolor{myg!40}
                            $\textcolor{myg}{{y}} $ 
                                        & $\textcolor{myg}{{0}}$  
                                        & $\textcolor{myg}{{1}}$
                                        & $\textcolor{myg}{{-\nicefrac{1}{12}}}$ &  
                                        & $\textcolor{myg}{\nicefrac{5}{12}}$ & & $\textcolor{myg}{5}$
                                \\ 
                                \hline
                                $\textcolor{black}{{-z}}$ 
                                        &  $\textcolor{black}{{0}}$
                                        & $\textcolor{black}{{-\nicefrac{6}{5}}}$
                                        & $\textcolor{black}{{\nicefrac{8}{5}}}$
                                        & 
                                        & 
                                        & 1 & $\textcolor{black}{{48}}$ 
                                \\
                                \hline 
                                \end{tabular}
                        \end{center}
                        \caption{$ligne(p) \; -  \; \nicefrac{9}{5} \; ligne(h) $}
                \end{table}


    \begin{table}[H]
                        \begin{center}
                            \renewcommand{\arraystretch}{1.5}
                            \fontfamily{flr}\selectfont
                            \footnotesize
                                \begin{tabular}{|l|l l l l l |l|l|}
                                \arrayrulecolor{blue}
                                \hline
                                v.d. & $x$
                                     & $y$ & $u$ & $p$ & $h$ & $-z$ & t.d 
                                \\
                                \hline
                                \arrayrulecolor{black}
                                \rowcolor{myg!40}
                            $\textcolor{myg}{\boldsymbol{x}} $     
                                        & $\textcolor{myg}{\boldsymbol{1}}$  
                                        & $\textcolor{myg}{\boldsymbol{0}}$
                                        & $\textcolor{myg}{{\boldsymbol{\nicefrac{1}{4}}}}$ 
                                        & $\textcolor{myg}{\boldsymbol{0}}$
                                        & $\textcolor{myg}{\boldsymbol{\nicefrac{-1}{4}}}$      
                                        & & $\textcolor{myg}{\boldsymbol{3}}$
                                \\
                                \rowcolor{myg!40}
                            $\textcolor{myg}{\boldsymbol{p}} $     
                                        & $\textcolor{myg}{\boldsymbol{0}}$  
                                        & $\textcolor{myg}{\boldsymbol{0}}$
                                        & $\textcolor{myg}{{\boldsymbol{-\nicefrac{1}{4}}}}$ 
                                        & $\textcolor{myg}{\boldsymbol{1}}$
                                        & $\textcolor{myg}{\boldsymbol{\nicefrac{-3}{4}}}$      
                                        & & $\textcolor{myg}{\boldsymbol{3}}$

                                \\

                            \rowcolor{myg!40}
                            $\textcolor{myg}{{y}} $ 
                                        & $\textcolor{myg}{{0}}$  
                                        & $\textcolor{myg}{{1}}$
                                        & $\textcolor{myg}{{-\nicefrac{1}{12}}}$ &  
                                        & $\textcolor{myg}{\nicefrac{5}{12}}$ & & $\textcolor{myg}{5}$
                                \\ 
                                \hline
                                $\textcolor{black}{{-z}}$ 
                                        &  $\textcolor{black}{{0}}$
                                        & $\textcolor{black}{{-\nicefrac{6}{5}}}$
                                        & $\textcolor{black}{{\nicefrac{8}{5}}}$
                                        & 
                                        & 
                                        & 1 & $\textcolor{black}{{48}}$ 
                                \\
                                \hline 
                                \end{tabular}
                        \end{center}
                        \caption{$ligne(x) \; -  \; \nicefrac{3}{5} \; ligne(h) $}
                \end{table}


    \begin{table}[H]
                            \begin{center}
                                \renewcommand{\arraystretch}{1.5}
                                \fontfamily{flr}\selectfont
                                \footnotesize
                                    \begin{tabular}{|l|l l l l l |l|l|}
                                    \arrayrulecolor{blue}
                                    \hline
                                    v.d. & $x$
                                         & $y$ & $u$ & $p$ & $h$ & $-z$ & t.d 
                                    \\
                                    \hline
                                    \arrayrulecolor{black}
                                    \rowcolor{myg!40}
                                $\textcolor{myg}{\boldsymbol{x}} $     
                                            & $\textcolor{myg}{\boldsymbol{1}}$  
                                            & $\textcolor{myg}{\boldsymbol{0}}$
                                            & $\textcolor{myg}{{\boldsymbol{\nicefrac{1}{4}}}}$ 
                                            & $\textcolor{myg}{\boldsymbol{0}}$
                                            & $\textcolor{myg}{\boldsymbol{\nicefrac{-1}{4}}}$      
                                            & & $\textcolor{myg}{\boldsymbol{3}}$
                                    \\
                                    \rowcolor{myg!40}
                                $\textcolor{myg}{\boldsymbol{p}} $     
                                            & $\textcolor{myg}{\boldsymbol{0}}$  
                                            & $\textcolor{myg}{\boldsymbol{0}}$
                                            & $\textcolor{myg}{{\boldsymbol{-\nicefrac{1}{4}}}}$ 
                                            & $\textcolor{myg}{\boldsymbol{1}}$
                                            & $\textcolor{myg}{\boldsymbol{\nicefrac{-3}{4}}}$      
                                            & & $\textcolor{myg}{\boldsymbol{3}}$

                                    \\
                                \rowcolor{myg!40}
                                $\textcolor{myg}{{y}} $ 
                                            & $\textcolor{myg}{{0}}$  
                                            & $\textcolor{myg}{{1}}$
                                            & $\textcolor{myg}{{-\nicefrac{1}{12}}}$ &  
                                            & $\textcolor{myg}{\nicefrac{5}{12}}$ & & $\textcolor{myg}{5}$
                                    \\ 
                                    \hline
                                    \rowcolor{myg!40}
                                $\textcolor{myg}{\boldsymbol{-z}} $     
                                            & $\textcolor{myg}{\boldsymbol{0}}$  
                                            & $\textcolor{myg}{\boldsymbol{0}}$
                                            & $\textcolor{myg}{{\boldsymbol{\nicefrac{3}{2}}}}$ 
                                            & $\textcolor{myg}{\boldsymbol{0}}$
                                            & $\textcolor{myg}{\boldsymbol{\nicefrac{1}{2}}}$      
                                            & $\textcolor{myg}{\boldsymbol{1}}$
                                            & $\textcolor{myg}{\boldsymbol{54}}$
                                    \\
                                    \hline 
                                    \end{tabular}
                            \end{center}
                            \caption{$ligne(-z) \; -  \; \nicefrac{6}{5} \; ligne(h) $}
                    \end{table}

    Nous avons complété \textbf{une seconde itération} de l'algorithme.         
    Le système est associé à la solution :

    \begin{align*}
        \boxed{\left[u, h = 0 \right] \implies x = 3, p = 3, y = 5, z = -54}
    \end{align*}

    Il n’est pas intéressant d’augmenter ni la valeur de $u$, 
    ni la valeur de $h$ car la valeur de $z$ augmente. 
    Ainsi, nous sommes à l’optimum.


    \section{Concept de solutions de base}
    Soit un système de $m$ équations et de $n$ variables, l'algorithme 
    du simplexes considère les solutions où $n - m$ variables sont 
    fixées à zéro 
    et sont donc des \textbf{variables indépantes}. 
    Ces solutions sont appeleés 
    \textbf{solutions de bases}. 
    Avec $n$ variables et $n - m$ \textit{variables fixées à zéro}, il y a  
    le nombre suivant de solutions de bases possibles :
    \begin{align*}
        \boxed{{n \choose n - m} = \dfrac{n!}{(n - m )! m!}}
    \end{align*}

    
    \noindent
    \textbf{Exemple} 
    Soit un système de $n = 5$ variables et $m = 3$ équations, 
    il y aura le nombre suivant de solutions de base possibles :
     
    \begin{align*}
            {5 \choose 2} = \dfrac{5!}{2! \times 3!} = \dfrac{5 \times 4 \times 3!}{2! \times 3!} = \dfrac{20}{2} = 10 
    \end{align*}
\end{multicols*} 



\section{Résumé de l'algorithme en forme tableau}

\noindent 
    $\blacktriangleright$  \textbf{Organiser} les équations présentées sous 
    la \textit{forme standard} en un tableau. La 1\up{ere} colonne 
    contient \entoure{v.d.}, puis, dans chaque rangée subséquente 
    de cette colonne, chacune des variable dépendante et, finalement, 
    l'objectif négatif $-z$. La \textbf{dernière colonne} contient 
    \entoure{t.d.} puis, dans chacune des rangées subséquentes de cette 
    colonne, les termes de droite des équations correspondantes. 


    \noindent 
    $\blacktriangleright$
    \textbf{Choisir} la variable d'entrée $\boldsymbol{x_e}$
    ; il s'agit de la variables
    dépendante dont l'augmentation minimise davantage l'objectif. 

    \begin{align*}
        \boxed{
        x_e = x : -z \underset{x \to \infty}{\longrightarrow} -\infty
        }
    \end{align*}                    


    \noindent
    $\blacktriangleright$ 
    \textbf{Choisir} la variable de sortie 
    $\boldsymbol{x_s}$. Il s'agit de la variable 
    qui limite davantage l'augmentation de la variable de sortie, 
    puisqu'elle atteint 0 plus tôt que toutes les autres variable 
    lorsque $x_e$ augmente. 

    \noindent
    $\blacktriangleright$ 
    \textbf{Identifier} le coefficient pivot. Il s'agit du 
    coefficient qui se trouve à l'intersection de la colonne de 
    la variable d'entrée et la rangée de la variable de sortie. 


    \noindent
    $\blacktriangleright$ 
    \textbf{Diviser} la rangée de la variable de sortie par 
    le coefficient du pivot.


    \noindent
    $\blacktriangleright$ 
    \textbf{Soustraire} chacune des autres rangées par un facteur 
    $f \times ligne(x_s)$, de manière à obtenir 0 dans la colonne 
    de la variable de sortie (la même colonne que le coefficient pivot). 

\begin{multicols*}{2}

\chapter{Résolution Graphique}


\section{Représentation graphique selon l'équation}

Considérons le système d'équation suivant sous 
la forme standard :
\begin{align*}
    \text{Min} -z =  \;\; -8x  -6y& \\
    \text{s.a.}
    \\
    \\
    5x + 3y \leq 30& \\
    2x + 3y \leq 24& \\
    x  + 3y \leq 18& 
    \\
    \\ 
    x, y \geq 0& 
\end{align*}


Nous pouvons choisir chacune des équations  pour 
la représenter par une courbe. Soit l'équation suivante, nous avons la 
droite correspondante :

\begin{align*}
            5x + 3y = 30 
\end{align*}                    






\begin{center}
\begin{tikzpicture}[scale=0.35]
    % Axe des x et des y
    \draw[->] (-1,0) -- (18,0) node[right] {$x$};
    \draw[->] (0,-1) -- (0,12.5) node[above] {$y$};
    
    % Tracer la droite 5x + 3y = 30 en rouge
    \draw[red, thick] (0,10) -- (6,0) 
     node at (3, 11) {$5x + 3y = 30$};
        
    % Tracer la droite x + 3y = 18
    \draw[myg, thick] (0,18/3) -- (18,0) 
     node at (6.5, 6) {$x + 3y = 18$};



    % Tracer la droite 2x + 3y = 24 en bleu
    \draw[blue, thick] (0,8) -- (12,0) 
    node at (4.5,8.5) {$2x + 3y = 24$};
    
    
    % Hachures pour la région de la solution
    \fill[red!20,opacity=0.6] (0,10) -- (6,0) -- (0,0) -- cycle;
    \fill[blue!20,opacity=0.6] (0,8) -- (12,0) -- (0,0) -- cycle;
    \fill[myg!20,opacity=0.6] (0,18/3) -- (18,0) -- (0,0) -- cycle;


    
    % Ajouter des labels
\end{tikzpicture}
\end{center}
    Puisque l'équation contient une inégalité, nous savons que l'ensemble des 
    points qui satisfont la contrainte se trouve nécessairement sous la 
    courbe.


\begin{center}
\begin{tikzpicture}[scale=0.35]
    % Axe des x et des y
    \draw[->] (-1,0) -- (18,0) node[right] {$x$};
    \draw[->] (0,-1) -- (0,12.5) node[above] {$y$};
    
    % Tracer la droite 5x + 3y = 30 en rouge
    \draw[red, thick] (0,10) -- (6,0) 
     node at (3, 11) {$5x + 3y = 30$};
        
    % Tracer la droite x + 3y = 18
    \draw[myg, thick] (0,18/3) -- (18,0) 
     node at (6.5, 6) {$x + 3y = 18$};



    % Tracer la droite 2x + 3y = 24 en bleu
    \draw[blue, thick] (0,8) -- (12,0) 
    node at (4.5,8.5) {$2x + 3y = 24$};
    
    
    % Hachures pour la région de la solution
    \fill[red!20,opacity=0.6] (0,10) -- (6,0) -- (0,0) -- cycle;
    \fill[blue!20,opacity=0.6] (0,8) -- (12,0) -- (0,0) -- cycle;
    \fill[myg!20,opacity=0.6] (0,18/3) -- (18,0) -- (0,0) -- cycle;


 % Tracer les lignes d'intersection plus épaisses
    \draw[myp!70!blue, ultra thick] (0,6) -- (3,5); % Intersection rouge-bleu
    \draw[myp!70!blue, ultra thick] (3,5) -- (6,0); % Intersection rouge-bleu
    \draw[myp!70!blue, ultra thick] (6,0) -- (0,0); % Intersection rouge-bleu
    \draw[myp!70!blue, ultra thick] (0,0) -- (0,6); % Intersection rouge-bleu

    % Hachures pour la région de solution commune, avec une couleur plus foncée
    \fill[red!50!blue!50!myg!50, opacity=0.8] (0,6) -- (3,5) -- (6,0) -- (0,0) -- cycle;
    
    % Ajouter des labels
\end{tikzpicture}
\end{center}


    Les solutions possibles se trouvent donc dans la zone délimitée par 
    les courbes. La solution 
    \textbf{optimale} est se trouve au sommet de polygôme 
    définissant la zone.  

    pour la solution optimale $\hat{x} : (a, b) \in \mathbb{R}^2$ 
    nous avons : 


    \begin{align*}
        \hat{x} \in \Bigl\{ (0,6), \; (3,5), \;  (6, 0), \; (0, 0)\Bigr\}
    \end{align*}


    Selon les options possibles, la solution optimale est donc 
    $\hat{x} = (3, 5)$ et on a  

    \begin{align*}
        \boxed{-z = -8(3) - 6(5) = -24 - 30 = -54}
    \end{align*}


    Selon la théorie vue précedemment, nous savons 
    que le nombre de solutions de bases 
    différentes dans l'espace est donné par :

    \begin{align*}
        {5 \choose 3} = 
        \dfrac{5!}{3! \cdot 2!} 
        = \textcolor{myr}{10}  
    \end{align*}

    Or, le polyptote ne contient que 
    $\boldsymbol{4}$ \textit{coins} et donc               $\boldsymbol{4}$ soolution optimales. Cela est 
    dû au fait que l'algorithme du simplexe ne peut 
    visiter que 4 solutions de base 
    \textbf{réalisables}. Les solutions restantes 
    sont des solutions de bases 
    \textbf{non réalisables}. 

    \chapter{Simplexe en formulation générale}



    \section{Définitions et forme standard}
    Soit un système de programmation linéaire qui comprend 
    $\textcolor{myr}{\boldsymbol{n}}$ \textbf{variables} 
    et $\textcolor{myr}{\boldsymbol{m}}$ \textbf{contraintes} et 
    $n \geq m$, nous avons les entités suivantes :

    \begin{align*}
        &\boldsymbol{x_j} \Coloneqq \textit{variables de décision} \\   
        &\boldsymbol{c_j} \Coloneqq \textit{coeff. de } \boldsymbol{x_j}  \textit{ dans l'objectif}  \\  
        &\boldsymbol{a_{ij}} \Coloneqq \textit{coeff. de } \boldsymbol{x_j} \textit{ dans la contrainte } \boldsymbol{i} \\  
        &\boldsymbol{b_i} \Coloneqq \textit{terme de droite dans la contrainte } \boldsymbol{i}
    \end{align*}

    Les \textbf{variables de décisions} sont donc $x_j, j = 1, 2, \cdots n$ et chacune 
    d'elle est 
    présente dans une colonne $\boldsymbol{j}$ du tableau. Il y a 
    $\boldsymbol{m}$ (ou $\boldsymbol{i}$) rangées d'équations 
    dans le tableau, excluant la rangée de l'objectif. 



    \begin{align*}
        &\text{Min } \quad z = c_1x_1 + c_2x_2 + \cdots c_nx_n \\ 
        &\text{s.a.}   \\\\ 
        &a_{11}x_1 + a{12}x_2 + \cdots a_{1n}x_n = b_1 \\
        &a_{21}x_1 + a{22}x_2 + \cdots a_{2n}x_n = b_2 \\
        &\quad\quad\quad\quad\quad \cdots\cdots  \\ 
        &a_{m1}x_1 + a{m2}x_2 + \cdots a_{mn}x_n = b_m \\
        &x_1, x_2, \cdots, x_n \geq 0 
     \end{align*}

     Cela revient à la simplification suivante :

     \begin{align*}
        &\text{Min } z = \sum_{j=1}^{n }c_jx_j, \; \textbf{s}.\textbf{a}.  
        \sum_{n=j}^{n }a_{ij} = b_i, i = 1, \cdots,  m \\ 
        &x_j \geq 0, j = 1, \cdots, n 
     \end{align*} 


\section{Représentation tableau de la forme générale}


\begin{table}[H]
    \centering
    \renewcommand{\arraystretch}{1.5}
    \fontfamily{flr}\selectfont
    \footnotesize
    \resizebox{1\linewidth}{!}{
    \begin{tabular}{|l|*{12}{c}|c|c|}
    \arrayrulecolor{blue}\hline
    v.d. & $x_1$ & $x_2$ & $\cdots$ & $x_r$ & $\cdots$ & $x_m$ &
    $x_{m+1}$ & $\cdots$ & $x_s$ & $\cdots$ & $x_n$ & $-z$ & & t.d. \\
    \hline\arrayrulecolor{black}
    $x_1$ & $1$ & & & & & & $\overline{a}_{1,m+1}$ & $\cdots$ &
    $\overline{a}_{1s}$ & $\cdots$ & $\overline{a}_{1n}$ & & & $\overline{b}_1$ \\
    $x_2$ & & $1$ & & & & & $\overline{a}_{2,m+1}$ & $\cdots$ &
    $\overline{a}_{2s}$ & $\cdots$ & $\overline{a}_{2n}$ & & & $\overline{b}_2$ \\
    $\vdots$ & & & $\ddots$ & & & & & $\ddots$ & & $\ddots$ & & & & $\vdots$ \\
    $x_r$ & & & & $1$ & & & $\overline{a}_{r,m+1}$ & $\cdots$ &
    $\overline{a}_{rs}$ & $\cdots$ & $\overline{a}_{rn}$ & & & $\overline{b}_r$ \\
    $\vdots$ & & & & & $\ddots$ & & $\ddots$ & $\ddots$ & & $\ddots$ &
    $\overline{a}_{1n}$ & & & $\vdots$ \\
    $x_m$ & & & & & & $1$ & $\overline{a}_{m,m+1}$ & $\cdots$ &
    $\overline{a}_{ms}$ & $\cdots$ & $\overline{a}_{mn}$ & & & $\overline{b}_m$ \\
    \hline
    $-z$ & & & & & & & $\overline{c}_{m+1}$ & $\cdots$ &
    $\overline{c}_{s}$ & $\cdots$ & $\overline{c}_{n}$ & $1$ & & $\overline{z}$ \\
    \hline
    \end{tabular}}
\end{table}

La solution associée est :
\begin{align*}
    &x_1 = \overline{b}_1,\quad x_2 = \overline{b}_2,\quad \dots,\quad
    x_m = \overline{b}_m, \\
    &x_{m+1} = x_{m+2} = \dots = x_n = 0,\quad z = \overline{z}
\end{align*}

\section{Choix de la variable d'entrée}

La variable d'entrée $x_j$ est celle qui minimise $c_j x_j$ dans l'objectif $z$.
Si tous les coefficients $\overline{c}_j \geq 0$, l'algorithme se termine car l'objectif
ne peut plus être amélioré.

\subsection{Scénario 1 : Tous les $\overline{c}_j x_j$ sont positifs}
\begin{align*}
    \left[ \forall j,\ \overline{c}_j x_j \geq 0 \right] \implies
    z \longrightarrow \infty
\end{align*}

\subsection{Scénario 2 : Au moins un $\overline{c}_j x_j$ est négatif}

Si $\overline{c}_j < 0$ pour au moins un $j$, la variable d'entrée $x_s$ est choisie telle que :
\begin{align*}
    \overline{c}_s = \min\left\{ \overline{c}_j : j = 1, \dots, n \right\}
\end{align*}

Pour chaque ligne, on a :
\begin{align*}
    x_i = \overline{b}_i - \overline{a}_{is} x_s,\quad i = 1, \dots, m
\end{align*}

\section{Augmentation de la variable d'entrée}

\subsection{Scénario 1 : Tous les $\overline{a}_{is}$ sont négatifs}

Si $\overline{a}_{is} < 0$ pour tout $i$, l'augmentation de $x_s$ entraîne une augmentation
des $x_i$, rendant le problème non borné inférieurement.

\subsection{Scénario 2 : Au moins un $\overline{a}_{is}$ est positif}

Si $\overline{a}_{is} > 0$ pour au moins un $i$, $x_s$ est limité par :
\begin{align*}
    x_r = \min\left\{ \dfrac{\overline{b}_i}{\overline{a}_{is}} :
    \overline{a}_{is} > 0 \right\}
\end{align*}
La variable $x_r$ correspondante devient la variable de sortie.

\section{Pivot}

Le pivot échange la variable d'entrée $x_s$ avec la variable de sortie $x_r$. La ligne pivot
(ligne $r$) est divisée par le coefficient de pivot $\overline{a}_{rs}$, puis les autres
lignes sont ajustées pour obtenir des zéros dans la colonne $x_s$.

\subsection{Division de la ligne pivot}

Les nouveaux coefficients de la ligne pivot sont :
\begin{align*}
    \overline{\tilde{a}}_{rj} = \dfrac{\overline{a}_{rj}}{\overline{a}_{rs}},\quad
    j = 1, \dots, n
\end{align*}
et
\begin{align*}
    \overline{\tilde{b}}_r = \dfrac{\overline{b}_r}{\overline{a}_{rs}}
\end{align*}

\subsection{Mise à jour des autres lignes}

Pour chaque ligne $\boldsymbol{\textcolor{myr}{i}} \neq r$, on ajuste :
\begin{align*}
    \overline{\tilde{a}}_{\boldsymbol{\textcolor{myr}{i} }j} = 
    \overline{\tilde{a}}_{\boldsymbol{\textcolor{myr}{i} }j} -  
    \overline{a}_{is} \times \overline{\tilde{a}}_{rj} \quad
    \forall i \neq r, \; j = 1, \dots, n
\end{align*}
et
\begin{align*}
    \overline{\tilde{b}}_{\boldsymbol{\textcolor{myr}{i} }} = 
    \overline{b}_{\boldsymbol{\textcolor{myr}{i} }} 
    - \overline{a}_{is} \; \times \;  \overline{\tilde{b}}_r
\end{align*}

Pour la fonction objectif :
\begin{align*}
    \overline{\tilde{c}}_{j} = 
    \overline{c}_{j} 
    - \overline{c}_{s} \; \times \; \overline{\tilde{a}}_{rj} 
\end{align*}
et
\begin{align*}
    -\overline{\tilde{z}} = -\overline{z} - \overline{c}_s  \; \times \;
    \overline{\tilde{b}}_{rj}
\end{align*}

\section{Exemple}

Considérons le problème du restaurateur :


\begin{alignat*}{7}
    \text{Min } z \quad & = \quad & -8x & {} - {} & 6y & & & & & & \\
    \text{s.a.} \quad
                        &        & 5x & {} + {} & 3y & {} + {} & \textcolor{myr}{\textbf{\textit{u}}} & & &= & 30 \\
    &        & 2x & {} + {} & 3y & {} + {} & & \textcolor{myr}{\textbf{\textit{p}}}  & & = & 24 \\
    &        &  x & {} + {} & 3y & {} + {} & & &\textcolor{myr}{\textbf{\textit{h}}} &  = & 18 \\
    &        &  x,\ y,\ \textcolor{myr}{\textbf{\textit{u}}},\ \textcolor{myr}{\textbf{\textit{p}}},\ \textcolor{myr}{\textbf{\textit{h}}}  \geq & 0 
\end{alignat*}


En remplaçant les variables par $x_1$, $x_2$, $x_3$, $x_4$, $x_5$ :

\begin{alignat*}{7}
    \text{Min } z \quad & = \quad & -8x_1 & {} - {} & 6x_2 & & & & & & \\
    \text{s.a.} \quad
                        &        & 5x_1 & {} + {} & 3x_2 & {} + {} & \textcolor{myr}{\boldsymbol{x_3}} & & &= & 30 \\
                        &        & 2x_1 & {} + {} & 3x_2 & {} + {} & & \textcolor{myr}{\boldsymbol{x_4}}  & & = & 24 \\
                        &        &  x_1 & {} + {} & 3x_2 & {} + {} & & &\textcolor{myr}{\boldsymbol{x_5}} &  = & 18 \\
    &        &  x_1,\ x_2,\ \textcolor{myr}{\boldsymbol{x_3}},\ \textcolor{myr}{\boldsymbol{x_4}},\ 
    \textcolor{myr}{\boldsymbol{x_5}}  \geq & 0 
\end{alignat*}

Le tableau initial du simplexe est :

\begin{table}[H]
    \centering
    \renewcommand{\arraystretch}{1.5}
    \fontfamily{flr}\selectfont
    \footnotesize
    \begin{tabular}{|l|c c c c c|c|c|}
    \arrayrulecolor{blue}\hline
    v.d. & $x_1$ & $x_2$ & $x_3$ & $x_4$ & $x_5$ & $-z$ & t.d. \\
    \hline\arrayrulecolor{black}
    $x_3$ & $5$ & $3$ & $1$ & $$ & $$ & $$ & $30$ \\
    $x_4$ & $2$ & $3$ & $$ & $1$ & $$ & $$ & $24$ \\
    $x_5$ & $1$ & $3$ & $$ & $$ & $1$ & $$ & $18$ \\
    \hline
    $-z$ & $-8$ & $-6$ & $$ & $$ & $$ & $1$ & $$ \\
    \hline
    \end{tabular}
\end{table}

\subsection{Étape 1 : Choix de la variable d'entrée}
On voit que $\overline{c_j} < 0$ pour au moins un $j$. On a alors :
\begin{align*}
    \resizebox{1\linewidth}{!}{$ 
    \overline{c}_s = \min\Bigl\{\overline{c}_j : j = 1, \dots, n \Bigr\} 
    = \min\Bigl\{\overline{c}_1, \overline{c}_2 \Bigr\} 
    = \min\Bigl\{\boldsymbol{-8}, -6 \Bigr\}  = \boldsymbol{-8} 
$}
\end{align*}                
On identifie que $\overline{c}_1 = -8$ est le plus négatif, donc $x_1$ est la variable
d'entrée.

\subsection{Étape 2 : Détermination de la variable de sortie}
On fait face au \textbf{scénario 2}  : au moins un 
$a_{is}$ est positif. Dans ce cas-ci, $a_{is} = a_{i1}$; on fait 
référence aux coefficients sous la colonne de la variable d'entrée $x_1$ 
et donc $ s = 1$. 
\vspace{1em}\\
\indent On calcule les rapports pour déterminer la variable de sortie :

\begin{align*}
    \resizebox{1\linewidth}{!}{$ 
    \left[ x_r = \min \Bigl\{ \dfrac{\overline{b}_i}{\overline{a}_{is}} : \overline{a}_{is} > 0 \Bigr\} \right]
    \implies 
    x_r = \min \Bigl\{ \frac{30}{5} = \boldsymbol{6}, \; \frac{24}{2} = 12, \; \frac{18}{1} = 18   \Bigr\} 
$}
\end{align*}
Le minimum est $\boldsymbol{6}$, donc $x_r = x_3$ est la variable de sortie.

\subsection{Étape 3 : Pivot}

On divise la ligne de $x_3$ par le coefficient pivot $5$, donc on a applique la 
formule 
\begin{align*}
    \left[ \forall j, \;
        \overline{\tilde{a}}_{rj} = \overline{a}_{rj} -  
    \overline{a}_{is} \times \dfrac{\overline{a}_{rj}}{\overline{a}_{rs}} \right]  
    \implies 
    \forall j, \;
        \overline{\tilde{a}}_{3j} = \overline{\tilde{a}}_{3j} -  
        \overline{a}_{is} \times \dfrac{\overline{a}_{3j}}{\overline{a}_{3s}}, 
\end{align*}                

\begin{table}[H]
    \centering
    \renewcommand{\arraystretch}{1.5}
    \fontfamily{flr}\selectfont
    \footnotesize
    \begin{tabular}{|l|c c c c c|c|c|}
    \arrayrulecolor{blue}\hline
    v.d. & $x_1$ & $x_2$ & $x_3$ & $x_4$ & $x_5$ & $-z$ & t.d. \\
    \hline\arrayrulecolor{black}
     $x_1$ & $1$ & $\nicefrac{3}{5}$ & $\nicefrac{1}{5}$ & & & & $6$ \\
    $x_4$ & $2$ & $3$ & $$ & $1$ & $$ & $$ & $24$ \\
    $x_5$ & $1$ & $3$ & $$ & $$ & $1$ & $$ & $18$ \\
    \hline
    $-z$ & $-8$ & $-6$ & $$ & $$ & $$ & $1$ & $$ \\
    \hline
    \end{tabular}
\end{table}

Durant le pivot, la variable d'entrée $x_1$ prend la place de la variable de sortie 
$x_3$. 


\begin{table}[H]
    \centering
    \renewcommand{\arraystretch}{1.5}
    \fontfamily{flr}\selectfont
    \footnotesize
    \begin{tabular}{|l|c c c c c|c|c|}
    \arrayrulecolor{blue}\hline
    v.d. & $x_1$ & $x_2$ & $x_3$ & $x_4$ & $x_5$ & $-z$ & t.d. \\
    \hline\arrayrulecolor{black}
    $\boldsymbol{\textcolor{myr}{x_1}}$  & $1$ & $\nicefrac{3}{5}$ & $\nicefrac{1}{5}$ & $$ & $$ & $$ & $6$ \\    
    $x_4$ & $2$ & $3$ & $$ & $1$ & $$ & $$ & $24$ \\
    $x_5$ & $1$ & $3$ & $$ & $$ & $1$ & $$ & $18$ \\
    \hline
    $-z$ & $-8$ & $-6$ & $$ & $$ & $$ & $1$ & $$ \\
    \hline
    \end{tabular}
\end{table}


\subsection{Mise à jour des autres lignes}

Pour chaque ligne $i \neq r = x_3 (\text{nouvellement } x_1$, on ajuste :
\begin{align*}
    \forall (\boldsymbol{\textcolor{myr}{i}}, \boldsymbol{\textcolor{myr}{j}}), \; 
    \left[ 
        \overline{\tilde{a}}_{\boldsymbol{\textcolor{myr}{i}\textcolor{myr}{j}}} = 
    \overline{a}_{\boldsymbol{\textcolor{myr}{i}\textcolor{myr}{j}}} -  
    \overline{a}_{is} \times \overline{\tilde{a}}_{rj} \quad
    \right]
    \implies 
\end{align*}
\vspace{-3em} 

\begin{align*}
    &\overline{\tilde{a}}_{41} = 2 - 2 \times 1 = 0, \\
    &\overline{\tilde{a}}_{42} = 3 - 2 \times \dfrac{3}{5} = \dfrac{9}{5} \\ 
    &\overline{\tilde{a}}_{43} = 0 - 2 \times \dfrac{1}{5} = -\dfrac{2}{5} \\ 
    &\overline{\tilde{a}}_{44} = 1 - 2 \times 0  = 1 \\ 
    &\overline{\tilde{a}}_{45} = 0 - 2 \times 0 = 0 \\
    &\overline{\tilde{a}}_{51} = 1 - 1 \times 1 = 0, \\
    &\overline{\tilde{a}}_{52} = 3 - 1 \times \dfrac{3}{5} = \dfrac{12}{5} \\ 
    &\overline{\tilde{a}}_{53} = 0 - 1 \times \dfrac{1}{5} = -\dfrac{1}{5} \\ 
    &\overline{\tilde{a}}_{54} = 0 - 1 \times 0  = 0 \\ 
    &\overline{\tilde{a}}_{55} = 1 - 1 \times 0 = 1 
\end{align*}                    
et
\begin{align*}
    \left[ \overline{\tilde{b}}_i = \overline{b}_i - \overline{a}_{is} \overline{\tilde{b}}_r \right] 
    \implies
\end{align*}
\vspace{-3em} %

\begin{align*}
    &\overline{\tilde{b}}_4 = 24 - 2 \times 6 =  12  \\
    &\overline{\tilde{b}}_5 = 18 - 2 \times 6 =  6  \\
\end{align*}
et 
\begin{align*}
    \left[ \overline{\tilde{c}}_{j} = 
    \overline{c}_{j} 
- \overline{c}_{s} \; \times \; \overline{\tilde{a}}_{rj}  \right] 
\implies
\end{align*}
\vspace{-3em}% 

\begin{align*}
    &\overline{\tilde{c}}_1 = -8  - (-8)\times 1 = 0 , \\ 
    &\overline{\tilde{c}}_2 = -6  - (-8)\times \dfrac{3}{5} = -\dfrac{6}{5} , \\ 
    &\overline{\tilde{c}}_3 = 0  - (-8)\times \dfrac{3}{5} = \dfrac{8}{5} , \\ 
\end{align*}

et 

\begin{align*}
    \left[ -\overline{\tilde{z}} = -\overline{z} - \overline{c}_s  \; \times \;
    \overline{\tilde{b}}_{rj} \right] 
    \implies
    -\overline{\tilde{z}} = 1 - (-8) \times 6 = 48
\end{align*}


Selon les résultats présentés, on optient alors le tableau suivant :

\begin{table}[H]
                \begin{center}
                    \renewcommand{\arraystretch}{1.5}
                    \fontfamily{flr}\selectfont
                    \footnotesize
                        \begin{tabular}{|l|l l l l l |l|l|}
                        \arrayrulecolor{blue}
                        \hline
                        v.d. & $x$
                             & $y$ & $u$ & $p$ & $h$ & $-z$ & t.d 
                        \\
                        \hline
                        \arrayrulecolor{black}
                        $\textcolor{black}{{x}} 
                        $     & $\textcolor{black}{{1}}$ 
                              & $\textcolor{black}{{\nicefrac{3}{5}}}$
                                & $\textcolor{black}{{\nicefrac{1}{5}}}$
                                & 
                                & &  &  $\textcolor{black}{{6}}$
                        \\
                        $\textcolor{black}{{p}} $     
                                & $\textcolor{black}{{0}}$  
                                & $\textcolor{black}{{\nicefrac{9}{5}}}$
                               & $\textcolor{black}{{-\nicefrac{2}{5}}}$ & 1 
                               & & & $\textcolor{black}{{12}}$

                        \\
                    $\textcolor{black}{{h}} $     
                                & $\textcolor{black}{{0}}$  
                                & $\textcolor{black}{{\nicefrac{12}{5}}}$
                                & $\textcolor{black}{{-\nicefrac{1}{5}}}$ &  
                                & 1 & & $\textcolor{black}{{12}}$
                        \\ 
                        \hline
                        $\textcolor{black}{{-z}}$ 
                                &  $\textcolor{black}{{0}}$
                                & $\textcolor{black}{{-\nicefrac{6}{5}}}$
                                & $\textcolor{black}{{\nicefrac{8}{5}}}$
                                & 
                                & 
                                & 1 & $\textcolor{black}{{48}}$ 
                        \\
                        \hline 



                        \end{tabular}
                \end{center}
        \end{table}

    \pagebreak
    \chapter{Algorithme du Simplexe Analyse}

    \section{Définitions}

    Soit $x, y \in \mathbb{R}^n$, nous avons les quantités suivantes :

    \begin{align*}
        &\text{Produit scalaire } : \quad  x^{T}y = y^Yx = \sum_{j=1}^{n }x_jy_j \\ 
        &\text{Vecteur} : \quad
        \begin{pmatrix} x_1 \\ \vdots \\ x_n \end{pmatrix} \;
    \end{align*}            


    \begin{note}{}{}
        Un ensemble de vecteur $x^1, \dots, x^k \in \mathbb{R}^n$ 
        est dit \textbf{linéairement dépendant} s'il existe  
        $k$ scalaires $\lambda_{1}, \lambda_{2}, \dots \lambda_{k} 
        \in \mathbb{R}$ \textit{non nul} tel que :  
        \begin{align*}
            \lambda_{1}x^1 + \lambda_{2}x^2  + 
            \cdots 
            + \lambda_{k}x^k = 
        \begin{pmatrix} 0 \\ \vdots \\ 0 \end{pmatrix}  = 0
        \end{align*}
    \end{note}  

    Soit une matrice $A$ : 
        \begin{align*}
                        \begin{pmatrix} 
            a_{11} &  \cdots & a_{1n}  \\ 
            \vdots &  \ldots & \ldots \\ 
            a_{m1} &  \cdots & a_{mn}
            \end{pmatrix}
        \end{align*}

    \textbf{alors},  on dit que l'élément $a_{ij}$ est à
    \textbf{l'intersection} \texttt{ligne } $i$, \texttt{col. } $j$      
    de la matrice.  Une \textbf{matrice carrée} est une 
    matrice $A_{m \times n}$ où $m = n$ et un vecteur de dimension $n$ 
    peut être vue comme une matrice de dimension $n \times 1$. 
    La \textbf{matrice identité} $I$ est 
    une \textbf{matrice carrée} telle qu'on a : 

    \begin{align*}
        \boxed{\forall a_{ij} \in A_{m \times n} = I, \; 
        \left[  i = j \right] \implies a_{ij} = 1,  \; \textbf{et}  
        \left[ i \neq j \right] a_{ij} = 0 
    }
    \end{align*}


    
    \end{multicols*} 




        

        
            


\end{document}

