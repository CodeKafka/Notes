\documentclass{report}
%\usepackage[utopia]{mathdesign}
%\usepackage{amsmath, amsthm}


\usepackage{amsmath,amsfonts,amsthm,amssymb,mathtools}
%\usepackage[varbb]{newpxmath}
%\usepackage[osf,largesc,theoremfont]{newpxtext}
%\usepackage{coelacanth}
%\usepackage{beraserif} % Bitstream Vera Serif font
%\usepackage{berasans} % Bitstream Vera Sans font
%\usepackage{beramono} % Bitstream Vera Sans Mono font
%\usepackage{berasans}
%\usepackage{libertine}
%\usepackage{mathpazo}
%\usepackage{palatino}
%\usepackage{crimson}


%% Choose one of the following (if not choosing the  
%% default, viz., Computer Modern, font family):
%\usepackage{lmodern}
\usepackage{bold-extra}
%%
%\usepackage{mathpazo}
% \usepackage{newpxmath}
%\usepackage{kpfonts} % Very good
%%
%\usepackage{mathptmx} %Very good
%\usepackage{stix} 
%\usepackage{txfonts} %Very good
\usepackage{newtxtext,newtxmath} %Very good
%%
%\usepackage{libertine} \usepackage[libertine]{newtxmath}
%\usepackage{libertine,libertinust1math} % added 2019/11/28
%%
%\usepackage{newpxtext} \usepackage[euler-digits]{eulervm}
%\usepackage{textcomp}
%\usepackage{bm}
\usepackage{contour}
\usepackage{adjustbox}





\input{/home/cryptopsy/Semesters/LaTeXTemplates/UniversalTeXTemplate/preamble.tex}
%From M275 "Topology" at SJSU
\newcommand{\id}{\mathrm{id}} % Identité
\newcommand{\taking}[1]{\xrightarrow{#1}} % Flèche avec annotation
\newcommand{\inv}{^{-1}} % Inverse

%From M170 "Introduction to Graph Theory" at SJSU
\DeclareMathOperator{\diam}{diam} % Diamètre
\DeclareMathOperator{\ord}{ord} % Ordre
\newcommand{\defeq}{\overset{\mathrm{def}}{=}} % Défini comme égal

%From the USAMO .tex files
\newcommand{\ts}{\textsuperscript} % Exposant
\newcommand{\dg}{^\circ} % Degré
\newcommand{\ii}{\item} % Item

% % From Math 55 and Math 145 at Harvard
% \newenvironment{subproof}[1][Proof]{%
% \begin{proof}[#1] \renewcommand{\qedsymbol}{$\blacksquare$}}%
% {\end{proof}}

\newcommand{\liff}{\leftrightarrow} % Si et seulement si
\newcommand{\lthen}{\rightarrow} % Implique
\newcommand{\opname}{\operatorname} % Opérateur générique
\newcommand{\surjto}{\twoheadrightarrow} % Flèche surjective
\newcommand{\injto}{\hookrightarrow} % Flèche injective
\newcommand{\On}{\mathrm{On}} % Ordinaux
\DeclareMathOperator{\img}{im} % Image
\DeclareMathOperator{\Img}{Im} % Image
\DeclareMathOperator{\coker}{coker} % Cokernel
\DeclareMathOperator{\Coker}{Coker} % Cokernel
\DeclareMathOperator{\Ker}{Ker} % Noyau
\DeclareMathOperator{\rank}{rank} % Rang
\DeclareMathOperator{\Spec}{Spec} % Spectre
\DeclareMathOperator{\Tr}{Tr} % Trace
\DeclareMathOperator{\pr}{pr} % Projection
\DeclareMathOperator{\ext}{ext} % Extension
\DeclareMathOperator{\pred}{pred} % Prédécesseur
\DeclareMathOperator{\dom}{dom} % Domaine
\DeclareMathOperator{\ran}{ran} % Image (range)
\DeclareMathOperator{\Hom}{Hom} % Homomorphisme
\DeclareMathOperator{\Mor}{Mor} % Morphismes
\DeclareMathOperator{\End}{End} % Endomorphisme

\newcommand{\eps}{\epsilon} % Épsilon
\newcommand{\veps}{\varepsilon} % Variance d'épsilon
\newcommand{\ol}{\overline} % Ligne au-dessus
\newcommand{\ul}{\underline} % Ligne en-dessous
\newcommand{\wt}{\widetilde} % Tilde large
\newcommand{\wh}{\widehat} % Chapeau large
\newcommand{\vocab}[1]{\textbf{\color{blue} #1}} % Texte en gras et bleu
\providecommand{\half}{\frac{1}{2}} % Fraction 1/2
\newcommand{\dang}{\measuredangle} % Angle dirigé
\newcommand{\ray}[1]{\overrightarrow{#1}} % Ray
\newcommand{\seg}[1]{\overline{#1}} % Segment
\newcommand{\arc}[1]{\wideparen{#1}} % Arc
\DeclareMathOperator{\cis}{cis} % cis
\DeclareMathOperator*{\lcm}{lcm} % Plus petit commun multiple
\DeclareMathOperator*{\argmin}{arg min} % Argument du minimum
\DeclareMathOperator*{\argmax}{arg max} % Argument du maximum
\newcommand{\cycsum}{\sum_{\mathrm{cyc}}} % Somme cyclique
\newcommand{\symsum}{\sum_{\mathrm{sym}}} % Somme symétrique
\newcommand{\cycprod}{\prod_{\mathrm{cyc}}} % Produit cyclique
\newcommand{\symprod}{\prod_{\mathrm{sym}}} % Produit symétrique
\newcommand{\Qed}{\begin{flushright}\qed\end{flushright}} % QED aligné à droite
\newcommand{\parinn}{\setlength{\parindent}{1cm}} % Indentation de paragraphe à 1 cm
\newcommand{\parinf}{\setlength{\parindent}{0cm}} % Pas d'indentation de paragraphe
% \newcommand{\norm}{\|\cdot\|} % Norme
\newcommand{\inorm}{\norm_{\infty}} % Norme infinie
\newcommand{\opensets}{\{V_{\alpha}\}_{\alpha\in I}} % Ensemble ouvert
\newcommand{\oset}{V_{\alpha}} % Ensemble ouvert V
\newcommand{\opset}[1]{V_{\alpha_{#1}}} % Ensemble ouvert V avec indice
\newcommand{\lub}{\text{lub}} % Plus petite borne supérieure
\newcommand{\del}[2]{\frac{\partial #1}{\partial #2}} % Dérivée partielle
\newcommand{\Del}[3]{\frac{\partial^{#1} #2}{\partial^{#1} #3}} % Dérivée partielle d'ordre élevé
\newcommand{\deld}[2]{\dfrac{\partial #1}{\partial #2}} % Dérivée partielle avec dfrac
\newcommand{\Deld}[3]{\dfrac{\partial^{#1} #2}{\partial^{#1} #3}} % Dérivée partielle d'ordre élevé avec dfrac
\newcommand{\lm}{\lambda} % Lambda
\newcommand{\uin}{\mathbin{\rotatebox[origin=c]{90}{$\in$}}} % Appartient, tourné de 90 degrés
\newcommand{\usubset}{\mathbin{\rotatebox[origin=c]{90}{$\subset$}}} % Sous-ensemble, tourné de 90 degrés
\newcommand{\lt}{\left} % Gauche
\newcommand{\rt}{\right} % Droite
\newcommand{\bs}[1]{\boldsymbol{#1}} % Symbole en gras
\newcommand{\exs}{\exists} % Il existe
\newcommand{\st}{\strut} % Strut
\newcommand{\dps}[1]{\displaystyle{#1}} % Disposition en ligne

\newcommand{\sol}{\setlength{\parindent}{0cm}\textbf{\textit{Solution:}}\setlength{\parindent}{1cm} } % Solution sans indentation initiale puis rétablie
\newcommand{\solve}[1]{\setlength{\parindent}{0cm}\textbf{\textit{Solution: }}\setlength{\parindent}{1cm}#1 \Qed}

\newcommand{\entoure}[1]{\fcolorbox{black}{gray!30}{\texttt{#1}}}

\renewcommand{\ttdefault}{cmtt}
\newcommand{\textttbf}[1]{\contour{yellow!45}{\texttt{#1}}}
\newcommand{\varitem}[3][black]{%
    \item [%
        \colorbox{#2}{\textcolor{#1}{\makebox(5.5,7){#3}}}%
    ]
}
% Allow you to do the non implication (implication barred)
\newcommand{\notimplies}{%
  \mathrel{{\ooalign{\hidewidth$\not\phantom{=}$\hidewidth\cr$\implies$}}}}


\newcommand*{\authorimg}[1]%
    { \raisebox{-1\baselineskip}{\includegraphics[width=\imagesize]{#1}}}
\newlength\imagesize 

\input{/home/cryptopsy/Semesters/LaTeXTemplates/UniversalTeXTemplate/letterfonts.tex}
% lstlistingsEnvs.tex

\usepackage{minted}


\lstset{
  basicstyle=\ttfamily, % Set
  columns=fullflexible,
  keepspaces=true,
  language=Python % You can specify the language if you want syntax highlighting
}

%%%%%%%%%%%%%%%%%%%%%%%%%%%%%%%%%%%%%%%%%%%%%%%%%%%%%%%%%%%%%%%%%%%%%%%%%%%%%%%%%%%%%%%%%%%%%%%%%
%                                 Custom lstlisting Environments
%%%%%%%%%%%%%%%%%%%%%%%%%%%%%%%%%%%%%%%%%%%%%%%%%%%%%%%%%%%%%%%%%%%%%%%%%%%%%%%%%%%%%%%%%%%%%%%%%
% Gruvbox style for Python
\definecolor{Pgruvbox-bg}{HTML}{282828}
\definecolor{Pgruvbox-fg}{HTML}{ebdbb2}
\definecolor{Pgruvbox-red}{HTML}{fb4934}
\definecolor{Pgruvbox-green}{HTML}{b8bb26}
\definecolor{Pgruvbox-yellow}{HTML}{fabd2f}
\definecolor{Pgruvbox-blue}{HTML}{83a598}
\definecolor{Pgruvbox-purple}{HTML}{d3869b}
\definecolor{Pgruvbox-aqua}{HTML}{8ec07c}
\definecolor{BBBlack}{rgb}{0.05, 0.06, 0.09}



% JAVA LSTLISTING STYLE IN Gruvbox Colorscheme
\definecolor{gruvbox-bg}{rgb}{0.282, 0.247, 0.204}
\definecolor{gruvbox-fg1}{rgb}{0.949, 0.898, 0.776}
\definecolor{gruvbox-fg2}{rgb}{0.871, 0.804, 0.671}
\definecolor{gruvbox-red}{rgb}{0.788, 0.255, 0.259}
\definecolor{gruvbox-green}{rgb}{0.518, 0.604, 0.239}
\definecolor{gruvbox-yellow}{rgb}{0.914, 0.808, 0.427}
\definecolor{gruvbox-blue}{rgb}{0.353, 0.510, 0.784}
\definecolor{gruvbox-purple}{rgb}{0.576, 0.412, 0.659}
\definecolor{gruvbox-aqua}{rgb}{0.459, 0.631, 0.737}
\definecolor{gruvbox-gray}{rgb}{0.518, 0.494, 0.471}

\definecolor{lst-bg}{RGB}{45, 45, 45}
\definecolor{lst-fg}{RGB}{220, 220, 204}
\definecolor{lst-keyword}{RGB}{215, 186, 125}
\definecolor{lst-comment}{RGB}{117, 113, 94}
\definecolor{lst-string}{RGB}{163, 190, 140}
\definecolor{lst-number}{RGB}{181, 206, 168}
\definecolor{lst-type}{RGB}{218, 142, 130}

\lstdefinestyle{PythonGruvbox}{
    language=Python,
    identifierstyle=\color{lst-fg},
    basicstyle=\ttfamily\color{Pgruvbox-fg},
    keywordstyle=\color{Pgruvbox-yellow},
    keywordstyle=[2]\color{Pgruvbox-blue},
    stringstyle=\color{Pgruvbox-green},
    commentstyle=\color{Pgruvbox-aqua},
    backgroundcolor=\color{BBBlack},
    rulecolor=\color{BBBlack},
    showstringspaces=false,
    keepspaces=true,
    captionpos=b,
    breaklines=true,
    tabsize=4,
    showspaces=false,
    numbers=left,
    numbersep=5pt,
    numberstyle=\tiny\color{gray},
    showtabs=false,
    columns=fullflexible,
    morekeywords={True,False,None},
    morekeywords=[2]{and,as,assert,break,class,continue,def,del,elif,else,except,exec,
    finally,for,from,global,if,import,in,is,lambda,nonlocal,not,or,pass,print,raise,
    return,try,while,with,yield},
    morecomment=[s]{"""}{"""},
    morecomment=[s]{'''}{'''},
    morecomment=[l]{\#},
    morestring=[b]",
    morestring=[b]',
    literate=
    {0}{{\textcolor{Pgruvbox-purple}{0}}}{1}
    {1}{{\textcolor{Pgruvbox-purple}{1}}}{1}
    {2}{{\textcolor{Pgruvbox-purple}{2}}}{1}
    {3}{{\textcolor{Pgruvbox-purple}{3}}}{1}
    {4}{{\textcolor{Pgruvbox-purple}{4}}}{1}
    {5}{{\textcolor{Pgruvbox-purple}{5}}}{1}
    {6}{{\textcolor{Pgruvbox-purple}{6}}}{1}
    {7}{{\textcolor{Pgruvbox-purple}{7}}}{1}
    {8}{{\textcolor{Pgruvbox-purple}{8}}}{1}
    {9}{{\textcolor{Pgruvbox-purple}{9}}}{1}
}

% Gruvbox style for Java
\definecolor{gruvbox-bg}{rgb}{0.282, 0.247, 0.204}
\definecolor{gruvbox-fg1}{rgb}{0.949, 0.898, 0.776}
\definecolor{gruvbox-fg2}{rgb}{0.871, 0.804, 0.671}
\definecolor{gruvbox-red}{rgb}{0.788, 0.255, 0.259}
\definecolor{gruvbox-green}{rgb}{0.518, 0.604, 0.239}
\definecolor{gruvbox-yellow}{rgb}{0.914, 0.808, 0.427}
\definecolor{gruvbox-blue}{rgb}{0.353, 0.510, 0.784}
\definecolor{gruvbox-purple}{rgb}{0.576, 0.412, 0.659}
\definecolor{gruvbox-aqua}{rgb}{0.459, 0.631, 0.737}
\definecolor{gruvbox-gray}{rgb}{0.518, 0.494, 0.471}

\lstdefinestyle{JavaGruvbox}{
    language=Java,
    basicstyle=\ttfamily\color{Pgruvbox-fg},
    keywordstyle=\color{Pgruvbox-yellow},
    keywordstyle=[2]\color{lst-type},
    commentstyle=\itshape\color{lst-comment},
    stringstyle=\color{lst-string},
    numberstyle=\color{lst-number},
    backgroundcolor=\color{BBBlack},
    rulecolor=\color{gruvbox-aqua},
    showstringspaces=false,
    keepspaces=true,
    captionpos=b,
    breaklines=true,
    tabsize=4,
    showspaces=false,
    showtabs=false,
    columns=fullflexible,
    morekeywords={var},
    morekeywords=[2]{boolean, byte, char, double, float, int, long, short, void},
    morecomment=[s]{/}{/},
    morecomment=[l]{//},
    morestring=[b]",
    morestring=[b]',
    numbers=left,
    numbersep=5pt,
    numberstyle=\tiny\color{gray},
}

% Dracula style for Java
\definecolor{draculawhite-background}{RGB}{237, 239, 252}
\definecolor{draculawhite-comment}{RGB}{98, 114, 164}
\definecolor{draculawhite-keyword}{RGB}{189, 147, 249}
\definecolor{draculawhite-string}{RGB}{152, 195, 121}
\definecolor{draculawhite-number}{RGB}{249, 189, 89}
\definecolor{draculawhite-operator}{RGB}{248, 248, 242}

\lstdefinestyle{JavaDraculaWhite}{
    language=Java,
    backgroundcolor=\color{draculawhite-background},
    commentstyle=\itshape\color{draculawhite-comment},
    keywordstyle=\color{draculawhite-keyword},
    stringstyle=\color{draculawhite-string},
    basicstyle=\ttfamily\footnotesize\color{black},
    identifierstyle=\color{black},
    keywordstyle=\color{draculawhite-keyword}\bfseries,
    morecomment=[s][\color{draculawhite-comment}]{/**}{*/},
    showstringspaces=false,
    showspaces=false,
    breaklines=true,
    %frame=single,
    rulecolor=\color{draculawhite-operator},
    tabsize=2,  
    numbers=left,
    numbersep=4pt,
    numberstyle=\ttfamily\tiny\color{gray}
}

% Dracula style for Python
\definecolor{draculawhite-bg}{HTML}{FAFAFA}
\definecolor{draculawhite-fg}{HTML}{282A36}
\definecolor{pdraculawhite-keyword}{HTML}{BD93F9}
\definecolor{pdraculawhite-comment}{HTML}{6272A4}
\definecolor{draculawhite-number}{HTML}{FF79C6}

\lstdefinestyle{PythonDraculaWhite}{
    language=Python,
    basicstyle=\ttfamily\small\color{draculawhite-fg},
    backgroundcolor=\color{draculawhite-background},
    keywordstyle=\color{orange}\bfseries,
    stringstyle=\color{draculawhite-string},
    commentstyle=\color{pdraculawhite-comment}\itshape,
    numberstyle=\color{draculawhite-number},
    showstringspaces=false,
    showspaces=false,
    breaklines=true,
    frame=single,
    rulecolor=\color{draculawhite-operator}, 
    tabsize=4,
    morekeywords={as,with,1,2,3,4, 5,6,7,8,9,True,False},
    numbers=left,
    numbersep=5pt,
    numberstyle=\small\bfseries\ttfamily\color{htmlcomment},
}

% Dracula Dark style for HTML
\definecolor{htmltag}{HTML}{ff79c6}
\definecolor{htmlattr}{HTML}{f1fa8c}
\definecolor{htmlvalue}{HTML}{bd93f9}
\definecolor{htmlcomment}{HTML}{6272a4}
\definecolor{htmltext}{HTML}{401E31}
\definecolor{htmlbackground}{HTML}{282a36}
\definecolor{comphtmlbackground}{HTML}{8093FF}

\lstdefinestyle{HTMLDraculaDark}{
    basicstyle=\normalsize\bfseries\ttfamily\color{htmltext},
    commentstyle=\itshape\color{htmlcomment},
    keywordstyle=\bfseries\color{htmltag},
    stringstyle=\color{htmlvalue},
    emph={DOCTYPE,html,head,body,div,span,a,script},
    emphstyle={\color{htmltag}\bfseries},
    sensitive=true,
    showstringspaces=false,
    backgroundcolor=\color{white},
    inputencoding=utf8,
    extendedchars=true,
    language=HTML,
    tabsize=4,
    breaklines=true,
    breakatwhitespace=true,
    numbers=left,
    numbersep=10pt,
    numberstyle=\small\bfseries\ttfamily\color{htmlcomment},
    escapeinside={<@}{@>},
    rulecolor=\color{htmlbackground},
    xleftmargin=10pt,
    frame=none, 
    breaklines=true,
    postbreak=\mbox{\textcolor{gray}{$\hookrightarrow$}\space},
    showlines=false,
    moredelim=[s][\itshape\color{htmlcomment}]{<!--}{-->},
    morekeywords={id,class,type,name,value,placeholder,checked,src,href,alt},
    literate={é}{{\'e}}1 {è}{{\`e}}1 {ê}{{\^e}}1 {ë}{{\"e}}1 {à}{{\`a}}1 {ù}{{\`u}}1 {û}{{\^u}}1 {ç}{{\c{c}}}1 {â}{{\^a}}1 {î}{{\^i}}1 {ï}{{\"i}}1
}


\lstdefinestyle{Haskell}{
  frame=none,
  xleftmargin=2pt,
  stepnumber=1,
  numbers=left,
  numbersep=5pt,
  numberstyle=\ttfamily\tiny\color[gray]{0.3},
  belowcaptionskip=\bigskipamount,
  captionpos=b,
  escapeinside={*'}{'*},
  language=haskell,
  tabsize=2,
  emphstyle={\bf},
  %commentstyle=\it,
  stringstyle=\mdseries\ttfamily,
  showspaces=false,
  keywordstyle=\bfseries\ttfamily,
  columns=flexible,
  basicstyle=\small\ttfamily,
  showstringspaces=false,
  morecomment=[l]\%,
}



\lstdefinestyle{CSSDraculaLight}{
    basicstyle=\bfseries\scriptsize\ttfamily\color{htmltext},
    commentstyle=\color{htmlcomment},
    keywordstyle=\bfseries\color{htmlvalue},
    stringstyle=\color{htmlvalue},
    emph={DOCTYPE,html,head,body,div,span,a,script},
    emphstyle={\color{htmltag}\bfseries},
    sensitive=true,
    showstringspaces=false,
    backgroundcolor=\color{white},
    inputencoding=utf8,
    extendedchars=true, % Support extended characters
    frame=none, 
    %frame=tb,
    tabsize=4,
    breaklines=true,
    breakatwhitespace=true,
    numbers=left,
    numbersep=10pt,
    numberstyle=\small\bfseries\ttfamily\color{htmlcomment},
    escapeinside={<@}{@>},
    rulecolor=\color{htmlbackground},
    xleftmargin=20pt,
    % Add a vertical line for opening and closing tags
    %frame=lines,
    framesep=2pt,
    framexleftmargin=4pt,
    % Add a vertical line for closing tags that go through multiple lines
    breaklines=true,
    postbreak=\mbox{\textcolor{gray}{$\hookrightarrow$}\space},
    showlines=true,
    % Add a rule to apply commentstyle to keywords inside comments
    moredelim=[s][\color{htmlcomment}]{/*}{*/},
    literate={é}{{\'e}}1
             {è}{{\`e}}1
             {ê}{{\^e}}1
             {ë}{{\"e}}1
             {à}{{\`a}}1
             {ù}{{\`u}}1
             {û}{{\^u}}1
             {ç}{{\c{c}}}1
             {â}{{\^a}}1
             {î}{{\^i}}1
             {ï}{{\"i}}1,
    morekeywords={color, background, background-color, font-size, font-weight, border, border-radius, padding, margin, display, position, top, right, bottom, left, flex, grid, width, height, min-width, max-width, min-height, max-height, transition, transform, animation, keyframes, content, z-index,id,class,type,name,value,placeholder,checked,src,href,alt},
    morestring=[s][\color{htmltag}]{:}{;},
}


\renewcommand{\ttdefault}{cmtt}
\newcommand{\textttbf}[1]{\contour{yellow!45}{\texttt{#1}}}
\newcommand{\varitem}[3][black]{%
    \item[%
        \colorbox{#2}{\textcolor{#1}{\makebox(5.5,7){#3}}}%
    ]
}






\title{\huge{MATH1400}\\\Huge{Calcul à plusieurs variables}\\\vspace{2em}Introduction }
\author{\huge{Franz Girardin}}
\date{\today}

\begin{document}

\maketitle
\newpage% or \cleardoublepage
% \pdfbookmark[<level>]{<title>}{<dest>}
\pdfbookmark[section]{\contentsname}{toc}
\tableofcontents
\pagebreak


\titleformat*{\section}{%
    \large\bfseries%
}


\begin{multicols*}{3}
    \footnotesize

    \chapter{Fonctions et propriétés}

    \section{Fonction exponentielle}
    \begin{itemize}
        \item[$\rhd$]  \textbf{Domaine} : $\mathbb{R}$  
        \item[$\rhd$]  \textbf{Continuité} : Continue sur \textbf{domaine} 
        \item[$\rhd$]  \textbf{Croissance} : $0 < \textbf{base}   < 1 \implies $ 
            \textcolor{myr}{\textbf{Strict.}} $\downarrow$

        \item[$\rhd$]  \textbf{Croissance} :  $\textbf{base} > 1 \implies$  \textcolor{myr}{\textbf{Strict.}} $\uparrow$
        \item[$\rhd$]  \textbf{Ordonnée  à l'ori}. : $e^0 = 1$   
        \item[$\rhd$]  \textbf{Signe} : $\forall x \in \mathbb{R}, e^x > 0$  
        \item[$\rhd$]   $e^x : x\longrightarrow\infty +$ :  
            $\lim\limits_{x\to+\infty }e^x  = \infty$  
            \item[$\rhd$]   $e^x : x\longrightarrow\infty-$ :  
                $\lim\limits_{x\to-\infty}e^x  = 0$
    \end{itemize}


\section{Propriétés exponentielles}  
        \begin{enumerate}
            \item $e^{x+y} = e^{x}e^{y}k$
            \item $e^{xy} = (e^{x})^{y}$
            \item  $(e^x)^{\prime} = e^x$
            \item $(a^x)^{\prime} = a^x\log_e$

        \end{enumerate}



    \section{Fonction logarithmique}
    \begin{itemize}
        \item[$\rhd$ ] \textbf{Domaine} : $]0, \infty [$  
        \item[$\rhd$ ] \textbf{Continuité} : Continue sur son \textbf{domaine} 
        \item[$\rhd$ ] \textbf{Croissance}  $0 < \textbf{base} < 1$ : 
            \textcolor{myr}{\textbf{Strict.}} $\downarrow$

        \item[$\rhd$ ] \textbf{Croissance} : $\textbf{base}> 1$ : \textcolor{myr}{\textbf{Strict.}} $\uparrow$
        \item[$\rhd$ ] \textbf{Abscisse à l'ori.} : $\log_e(\textcolor{myr}{1}) = 0$ 
        \item[$\rhd$ ] \textbf{Signe} : $\forall x > \textcolor{myr}{1} \log_ax > 0$
        \item[$\rhd$ ] \textbf{Signe} : $\forall x, \; 0 < x < \textcolor{myr}{1}, \log_ax < 0$  
        \item[$\rhd$ ] $\lim\limits_{x\to+\infty}\log_ax  = \infty$  
        \item[$\rhd$ ] $\lim\limits_{x\to-\infty}\log_ax  = -\infty$
    \end{itemize}


\section{Propriétés logarithmiques}  
        \begin{enumerate}
            \item $\log(x+y) = \log x + \log y$
            \item $\log x^y = y\log x$ 
            \item  $\log_a b = \dfrac{\log_c a}{\log_c b} 
                \Leftrightarrow
                \dfrac{\ln a}{\ln b}$
            \item $(\log x)^{\prime} = \dfrac{1}{x}$
        \end{enumerate}



    \section{Optimisation}

    \begin{itemize}
        \item[$\rhd$]  \textbf{Maximum} : point $x \in \textbf{ dom } : 
            \forall y \in f, y \neq x, f(x) \geq f(y)$     
        \item[$\rhd$]  \textbf{Minimum} : point $x \in \textbf{ dom } : 
            \forall y \in f, y \neq x, f(x) \leq f(y)$    
        \item[$\rhd$]  \textbf{Point d'inflexion}: $\uparrow - \downarrow$ 
            ou $\downarrow - \uparrow$
        \item[$\rhd$]  \textbf{Potentiel max ou min} :   
        $\bigl( f^{\prime} \left( x \right) = 0 \lor f^{\prime} \left(x\right) \nexists \bigr) \implies \textbf{max}. \lor \textbf{min}.     $
    \end{itemize}

    \section{Test de la dérivé première}
    Soit $f(x)$, on peut considérer $f^{\prime}(x)$ 
    pour déduire des \textbf{informations} propres à $f$.   

    \begin{table}[H]
    \centering

      \caption {Test de la dérivé première pour une fonction hypothétique}

    \begin{adjustbox}{width=0.33\textwidth}
        \renewcommand{\arraystretch}{1.5}
        \fontfamily{flr}\selectfont
        \begin{tabular}{|l|l|l|l|l|l|l|l|l}
        \arrayrulecolor{blue}\hline
        \rowcolor{lightBlue}
        \textcolor{myb}{} & \textcolor{myb}{ $-\infty$ } & & -2 & & 1 & & 10
        \\
        \hline
        \hline
        \arrayrulecolor{black} 
        $f^{\prime} $ &  & + &  & + & $\nexists$  & - & 
        \\ 
        \hline 
        $f$ & $-\infty$  & $\nearrow$ & inflex. & 
        $\nearrow$ & max & $\searrow$ & 0 
        \\ 
        \hline
      \end{tabular}
    \end{adjustbox}
    \end{table}
    

    \begin{table}[h]
      \caption {Test de la dérivé première pour une fonction hypothétique}

      \begin{center}
        \renewcommand{\arraystretch}{1.5}
        \fontfamily{flr}\selectfont
        \footnotesize
        \begin{tabular}{|l|l|l|l|l|l|l|l|l}
        \arrayrulecolor{blue}\hline
        \rowcolor{lightBlue}
        \textcolor{myb}{} & \textcolor{myb}{ $-\infty$ } & & -2 & & 1 & & 10
        \\
        \hline
        \hline
        \arrayrulecolor{black} 
        $f^{\prime} $ &  & + &  & + & $\nexists$  & - & 
        \\ 
        \hline 
        $f$ & $-\infty$  & $\nearrow$ & inflex. & $\nearrow$ & max & $\searrow$ & 0 
        \\ 
        \hline
      \end{tabular}
    \end{center}
    \end{table}
\begin{EExample}{Interpréter un tableau de test de dérivé première}{}
  \textbf{1. Comportement à la frontière} 
  Appliquer une limite aux deux frontières de la fonction, dans ce cas-ci $x \rightarrow -\infty$ et 
  $x \rightarrow 10$. On a :
  \begin{align*}
    \lim\limits_{x\to\infty^{+}}f(x) = \infty  \textbf{ et }\lim\limits_{x\to 10}f(x) = 0        
  \end{align*}
  2. Calculer $f^{\prime}$. Trouver $x$ tels que :
  \begin{enumerate}
    \varitem{blue!40}{\textbf{1}} $f^{\prime}\left(x\right) = 0$
  \varitem{blue!40}{\textbf{2}} $f^{\prime}\left(x\right)$ n'existe pas  
  \end{enumerate}
  \textbf{Dans le contexte de l'exemple}, on a trouvé la valeur $-2$, qui correspond au moment ou 
  $f^{\prime}\left(x\right) = 0$. Et la valeur $1$ correspond au moment ou la dérivé n'existe pas. \\\\ 
  \textbf{3. Trouver le signe $f^{\prime}$} sur chacun des intervales entre nos points d'intérêts pour déterminer 
  le comportement de la fonction. \\ 
  \textbf{Entre $-\infty$ et $-2$}, la dérivé est positive; la fonction est donc \textbf{croissante} 
  sur cet interval. \\
  \textbf{Entre $-2$ et $1$}, la dérivé est positive; la fonction est donc \textbf{croissante} 
  sur cet interval. \\
  \textbf{Entre $1$ et $10$}, la dérivé est négative; la fonction est donc \textbf{décroissante} 
  sur cet interval. \\\\
Noter que pour déterminer le signe de la dérivé, il suffit d'évaluer $f^{\prime}\left(x\right)$ à n'importe 
quel endroit dans l'interval (e.g.  $f^{\prime}\left(1\right)$  pour l'intervale de $-\infty$ à $-2$)  
\end{EExample}



\section{Test de la dérivé seconde}
\begin{Concept}{Test de la dérivé seconde}{}
\textbf{Si et seulement si} on obtient un point d'intérêt ou la dérivée première est nulle, on peut trouver 
les maximums et minimums locaux, grâce au test de la dérivé seconde    
\end{Concept}

\begin{Definitionx*}{Maximum et minimum local}{}
  Soit $f^{\prime}\left(x\right) = 0$ \textcolor{myb}{\textbf{et}} $f^{\prime\prime}\left(x\right) < 0$, on a  
  un \textbf{maximum local} en x. \\\\
    Soit $f^{\prime}\left(x\right) = 0$ \textcolor{myb}{\textbf{et}} $f^{\prime\prime}\left(x\right) > 0$, on a  
  un \textbf{minimum local} en x.
\end{Definitionx*}

\section{Fonctions sinus et cosinus}
\begin{table}[H]
  \caption {Propriétés des fonctions sinus et cosinus}

  \begin{center}
    \renewcommand{\arraystretch}{1.5}
    \fontfamily{flr}\selectfont
    \footnotesize
    \begin{tabular}{l|l}
    \arrayrulecolor{blue}\hline
    \rowcolor{lightBlue}
    \textcolor{myb}{Propriété} & \textcolor{myb}{Descritpion}
    \\
    \hline
    \arrayrulecolor{black}
Domaine
& 
$\mathbb{R}$
\\
\hline
Continuité
&
Continue sur leur domaine  
\\
\hline
Croissance 
&
Toutes deux $2\pi$ périodiques. 
\\
\hline
\end{tabular}
\end{center}
\end{table}

\begin{Identite}{Cosinus pair et sinus impart}{}
    \begin{enumerate}
        \item   $\cos\left(-x\right) = \cos\left(x\right)$ 
        \item $\sin\left(-x\right) = -\sin\left(x\right)$
    \end{enumerate}
\end{Identite}

\begin{Identite}{Règle de dérivation de la fonction cosinus}{}
  $\dfrac{d}{dx}\cos\left(x\right) = -\sin\left(x\right)$
\end{Identite}

\begin{Identite}{Règle de dérivation de la fonction sinus}{}
  $\dfrac{d}{dx}\sin\left(x\right) = -\cos\left(x\right)$
\end{Identite}

\begin{Identite}{Rayon d'un cercle}{}
    $\cos^2\left(x\right) + \sin^2\left(x\right) = 1 $ 
\end{Identite}

\begin{Identite}{Sinus d'une addition}{}
    \begin{enumerate}
        \item $\cos\left(a +b \right) = \cos a \cos b -\sin a \sin b$
        \item $\sin\left(a +b \right) = -\sin a \cos b + \cos a \sin b$
    \end{enumerate}
 \end{Identite}

\chapter{Limites des fonctions}
\section{Limite}
\columnbreak
\begin{Definitionx*}{}{}
  $ \lim\limits_{x\to a}f(x)$ \textcolor{myb}{Converge vers L} si $f(x)$ est aussi proche que l'on veut de 
  $L$ \textbf{lorsque}   $x \to a$

\end{Definitionx*}
\vspace{-1em}

\section{Limite à droite et à gauche}
\begin{Definitionx*}{}{}
    Soit $f : D\rightarrow \mathbb{R}$ \\ 
    \textcolor{myb}{La limite à droite} est la limite lorsque $x$ s'approche de a, venant de la 
    \textbf{droite}   : \\ 
    \[ \lim\limits_{x\to a^{+}} f(x) \implies  x \in D \text{ et } x > a \] 

    \textcolor{myb}{La limite à droite} est la limite lorsque $x$ s'approche de a, venant de la \textbf{gauche} : \\ 
    \[ \lim\limits_{x\to a^{-}} f(x) \implies  x \in D \text{ et } x < a \]   
  \textbf{Lorsque les deux limites sont équivalente}, \textcolor{myb}{la limite existe } :   
  
  \begin{center}
    $\lim\limits_{x\to a^{-}}f(x) = L = \lim\limits_{x\to a^{+}}f(x) $\\ 
    $\big\Updownarrow$ \\
    $\lim\limits_{x\to a} = L$
  \end{center}

  \textbf{Si aucun} $x < a$ on considère uniquement la limite à droite, 
  puisque $f$ est définit sur un domaine $[x > a, b]$. Et si la limite 
  à droite existe, la limite de $f$ correspond à la limite à droite : 
  
  \begin{center}
    $\lim\limits_{x\to a^{+}}f(x) = L $ \\ 
    $\big\Updownarrow$ \\
    $\lim\limits_{x\to a}f(x) = L $
  \end{center}

  \textbf{Si aucun} $x > a$ :  
  
  \begin{center}
    $\lim\limits_{x\to a^{-}}f(x) = L $ \\ 
    $\big\Updownarrow$ \\
    $\lim\limits_{x\to a}f(x) = L $
  \end{center}

\end{Definitionx*}

\begin{Definitionx*}{Divergence d'une fonction}{}
  $\lim\limits_{x\to a}f(x) $ \textbf{diverge}   si la limite ne converge vers aucun $L \in \mathbb{R}$
  \\
  \\
  \textcolor{myb}{\textbf{Cas particulier}} si 
  {\begin{enumerate}
      \varitem{blue!40}{\textbf{1}} $\lim\limits_{x\to a^{-}}f(x) = L $
      \varitem{blue!40}{\textbf{2}} $\lim\limits_{x\to a^{+}}f(x) = M $ 
      \varitem{blue!40}{\textbf{3}} $ L \neq M$ 
  \end{enumerate}}
  \begin{center}
      \textbf{alors}, $\lim\limits_{x\to a}f(x) $ \textcolor{myr}{\textbf{diverge}}. 
  \end{center}
\end{Definitionx*}
\columnbreak


\section{Propriétés des limites}
\begin{Concept}{Addition, soustraction et multiplication de limite}{}
  Supposon que $\lim\limits_{x\to a}f(x) = L $ et $\lim\limits_{x\to a}g(x) \; | \; L, M \in \mathbb{R} $
  \begin{enumerate}
    \varitem{blue!40}{\textbf{1}} 
    $\lim\limits_{x\to a}\Bigl(f(x) + g(x) \Bigr) =$ \\\\
    $\lim\limits_{x\to a}f(x) +  \lim\limits_{x\to a}g(x) = L + M$

    \varitem{blue!40}{\textbf{2}} 
    $\lim\limits_{x\to a}\Bigl(f(x) - g(x) \Bigr) = $ \\\\ 
    $\lim\limits_{x\to a}f(x) - \lim\limits_{x\to a}g(x) = L - M$ 

    \varitem{blue!40}{\textbf{3}} \
    $\lim\limits_{x\to a}\Bigl(f(x)g(x) \Bigr) = $ \\\\
    $\lim\limits_{x\to a}f(x) \lim\limits_{x\to a}g(x) = LM$

    \varitem{blue!40}{\textbf{4}} $\lim\limits_{x\to a}\Bigl(c f(x)\Bigr) =  c\lim\limits_{x\to a}f(x)$ 

    \varitem{blue!40}{\textbf{5}} $\lim\limits_{x\to a} \dfrac{f(x)}{g(x)} $ 
    =  $\dfrac{\lim\limits_{x\to a}f(x)}{\lim\limits_{x\to a}g(x)} = \dfrac{L}{M} \; \textbf{\textcolor{myr}{si}} \; M \neq 0$

\end{enumerate}
\end{Concept}


\begin{Concept}{Comportement asymptotique et c > 0}{}
  Soit $\dfrac{\lim\limits_{x\to a}f(x)}{\lim\limits_{x\to a}g(x)}$ 
  \textcolor{myb}{\textbf{\quad et}} \\ 
  $\lim\limits_{x\to a}f(x) = c \neq 0, \pm \infty$ si c > 0, on a : 
\end{Concept}


\begin{Concept}{Comportement asymptotique et c > 0}{}
  Soit $\dfrac{\lim\limits_{x\to a}f(x)}{\lim\limits_{x\to a}g(x)}$, 
  \textcolor{myb}{\textbf{\quad et}} \\
  $\lim\limits_{x\to a}g(x) = 0$, \textcolor{myb}{\textbf{\quad mais}}$ \\
  $$\lim\limits_{x\to a}f(x) = c \neq 0, \pm \infty$.
  \vspace{1em}
  
  Si $c > 0$, on a alors un comportement asymptotique équivalent. En d'autres termes, 
  il existe une constante $C$ telle que pour $x$ proche de $a$, $f(x)$ et $g(x)$ 
  partagent la même "tendance" de croissance ou de décroissance, de sorte que :
  \[
  \lim\limits_{x \to a} \frac{f(x)}{g(x)} = c > 0.
  \]
  Cela signifie que $f(x)$ et $g(x)$ sont asymptotiquement comparables en $a$, et que l'écart entre eux reste proportionnel à $c$.
\end{Concept}
  \begin{table}[h]
    \begin{center}
      \renewcommand{\arraystretch}{1.5}
      \fontfamily{flr}\selectfont
      \footnotesize
      \begin{tabular}{|l|l|l|l|l|}
      \arrayrulecolor{blue}\hline
      \rowcolor{lightBlue}
      \textcolor{myb}{Forme} & \textcolor{myb}{$\dfrac{c}{0^{+}}$} & \textcolor{myb}{$\dfrac{c}{0^{-}}$}
                             & \textcolor{myb}{$\dfrac{c}{\infty}$} & \textcolor{myb}{$\dfrac{c}{-\infty}$}
      \\
      \hline
      \hline
      \arrayrulecolor{black}
      tend vers & $\infty$ & $-\infty$ & 0 & 0 
      \\
      \hline
      

  \end{tabular}
  \end{center}
  \end{table}



  \begin{table}[h]
    \begin{center}
      \renewcommand{\arraystretch}{1.5}
      \fontfamily{flr}\selectfont
      \footnotesize
      \begin{tabular}{|l|l|l|l|l|}
      \arrayrulecolor{blue}\hline
      \rowcolor{lightBlue}
      \textcolor{myb}{Forme} & \textcolor{myb}{$\dfrac{c}{0^{+}}$} & \textcolor{myb}{$\dfrac{c}{0^{-}}$}
                             & \textcolor{myb}{$\dfrac{c}{\infty}$} & \textcolor{myb}{$\dfrac{c}{-\infty}$}
      \\
      \hline
      \hline
      \arrayrulecolor{black}
      tend vers & $-\infty$ & $\infty$ & 0 & 0 
      \\
      \hline
      

  \end{tabular}
  \end{center}
  \end{table}
\section{Continuité}
\begin{Concept}{Fonction continue}{}
  $f\text{:} \; D \rightarrow \mathbb{R}$ est \textcolor{myb}{continue}   en $a \in D$ si $\lim\limits_{x\to a} f(x) = f(a)$. 
  Autrement dit, une fonction est continue sur sont domaine si pour chaque élément $a \in D$, la limite 
  lorsque $x \rightarrow a$ est égale à $f(a)$. Et donc, la limite à gauche et à droite est approche 
  la même valeur $f(a)$
\end{Concept}

\begin{Identite}{Conséquence de la continuité de deux fonctions}{}
    Si $f$ et $g$ sont continues en $a$, alors 
    \begin{enumerate}
      \varitem{blue!40}{\textbf{1}} $f+g$ et $fg$ sont continues en $a$ 
      \varitem{blue!40}{\textbf{2}} $\dfrac{f}{g}$ est $fg$ continues en $a$ si $g(a) \neq 0$
      \varitem{blue!40}{\textbf{3}} $f \circ g$ et $fg$ sont continues en $a$ si $f \circ g$ 
      est définie près de $a$
    \end{enumerate}
\end{Identite}


\section{Dérivée}
\begin{Definitionx*}{La dérivée d'une fonction}{}
  Soit $f^{\prime}(a) = \lim\limits_{h\to 0}\frac{f(a + h) - f(a)}{h} $ Si cette limite existe, on dit que 
  on dit qu'il s'agit de \textcolor{myb}{la dérivée de la fonction}   $f$ au point $a$. Géométriquement, 
  la valeur vers laquelle converge $f^{\prime}(a)$ correspond à la pente de la droite tengente en $a$.
\end{Definitionx*}

\section{Formule}
\begin{Concept}{Règles de dérivations pour des fonctions courantes}{}
  $ \textbf{1. }   (c)^{\prime} = 1$, c une constante \;\;
  $ \textbf{2. }   (x^r)^{\prime} = rx^{r -1}, \; \forall r \in \mathbb{R}$ \\
  $ \textbf{3. }   (a^x)^{\prime} = a^xln(a)$ \\ 
  $ \textbf{4. }   (e^x)^{\prime} = e^x$ \\ 
  $ \textbf{5. }   (\ln(x))^{\prime} = \dfrac{1}{x} $ \\
  $ \textbf{6. }   (\log_a(x))^{\prime} = \dfrac{1}{x\ln(a)}$ \\\\\\\\
  $ \textbf{7. }   (\sin x)^{\prime} = \cos x$ \\ 
  $ \textbf{8. }   (\cos x)^{\prime} = - \sin x$ \\
  $ \textbf{9. }   (\tan x)^{\prime} = -\sec^2 x$ \\\\
  $ \textbf{10. }   (\arctan x)^{\prime} = \dfrac{1}{x^2 +1}$ \\
  $ \textbf{11. }   (\text{arcsec}x)^{\prime} = \dfrac{1}{x\sqrt{x^2 -1}}$ \\
  $ \textbf{11. }   (\arcsin x)^{\prime} = \dfrac{1}{x\sqrt{1 - x^2}}$ \\
  $ \textbf{11. }   (\arcsin x)^{\prime} = -\dfrac{1}{x\sqrt{1 - x^2}}$ 
\end{Concept}

\columnbreak

\section{Propriétés d'addition et de multiplication}
\begin{Concept}{Propriétés de la dérivée}{}
  Soit $f, g \text{:} \; I \rightarrow \mathbb{R}$ deux fonctions dérivables \\\\
  $\textbf{1. } \Bigl(cf(x)\Bigr)^{\prime} = cf^{\prime}(x)$, ou c est une constante. \\\\
  $\textbf{2. } \Bigl(f(x) + g(x)\Bigr)^{\prime} = cf^{\prime}(x) +  g^{\prime}(x)$ \\\\
  $\textbf{3. } f^{\prime}(x) = 0$ si et seulement si $f$ est une constante. 
\end{Concept}



\section{Règles de différenciation}
\begin{Concept}{Règles de calcul}{}
\begin{adjustbox}{width=1\textwidth}
\textbf{1. Produit :}
  $\Bigl(f(x)g(x)\Bigr)^{\prime} = f^{\prime}(x)g(x) + f(x)g^{\prime}(x)$
\end{adjustbox}

\vspace{1em}

\begin{adjustbox}{width=1\textwidth}
  $\textbf{2. Quotient :} \Bigl(\dfrac{f(x)}{g(x)}\Bigr)^{\prime} = \dfrac{f^{\prime}(x)g(x) - f(x)g^{\prime}(x)}{g(x)^2}$ 
\end{adjustbox}

\vspace{1em}

\begin{adjustbox}{width=1\textwidth}
  $\textbf{3. Dérivation en chaîne :} \Bigl(f(g(x)) \Bigr)^{\prime} = f^{\prime}\Bigl(g(x)\Bigr)g^{\prime}(x)$
\end{adjustbox}
\end{Concept} 

\section{Les formes indéterminées}
Toute expression représenté par une des formes suivantes est dite \textcolor{myb}{indéterminée} :
\begin{align*}
  \textcolor{red}{\dfrac{0}{0}}, \;\; \textcolor{red}{\dfrac{\infty}{\infty}}, \;\; 
  \textcolor{black}{\infty - \infty} \;\; \textcolor{myyellow}{1^{\infty}}, \;\; 
  0 \times \infty, \;\; \textcolor{myyellow}\infty^{0}, \;\; \textcolor{myyellow}{0^0}   
\end{align*}
Ces formes indéterminées peuvent être simplifiées en utilisant différentes techniques :
\begin{enumerate}
  \varitem{black}{} \textbf{Manipulations algébriques} facorisation, multipliucation par le conjugué, simplification
  \varitem{red}{} Règle de l'Hôpital 
  \varitem{myyellow}{} Utilisation du logarithme
\end{enumerate}

\chapter{Intégration}
\section{Définition d'une intégrale}

\begin{Concept}{Intégrale et théorème fondamental du calcul}{}
  Soit $f \text{:} \; [a,b] \rightarrow \mathbb{R} $ une fonction continue, \textcolor{myb}{l'intégrale} 
  de $a$ à $b$ de $f$ est noté :  
  \[ \int_{a}^{b}f(x)dx \]
\end{Concept}

\columnbreak


\section{Propriétés de l'intégrale}

\begin{Concept}{}{}

$$\text{Soit } f\text{:} \; [a,b] \rightarrow \mathbb{R}$$ \textbf{Alors}, \\\\
$$\int_{a}^{a}f(x)dx = 0 \textbf{ et }$$
\\
$$\int_{b}^{a}f(x)dx = - \int_{a}^{b}f(x)dx$$


\vspace{2em}
$$\textbf{Si } a < c < b,  \textbf{alors}$$ 
\\ 
$$\int_{a}^{b}f(x)dx = \int_{a}^{c}f(x)dx + \int_{c}^{b}f(x)dx $$ 
\\
$$\int_{a}^{b}cf(x)dx = c\int_{a}^{b}f(x)dx$$ 
\\

$$\int_{a}^{b}\Bigl(f(x) + g(x)\Bigr)dx =$$ \\ 
$$\int_{a}^{b}f(x)dx \; +  \int_{a}^{b}g(x)dx$$
\end{Concept}

\begin{Definitionx*}{Théorème fondamental du calcul}{}
    Soit $f\text{:} \; [a,b] \rightarrow \mathbb{R}$ une fonction continue
    et dérivable sur l'intervale. 
    Si une fonction  $F(x)$ correspond à la dérivée de  $f(t)$ en fonction de $x$ 
    sur l'intervalle  $[a, b]$, autrement dit : 
    $$\textcolor{myb}{F}(x) = \int_{a}^{x}f(t)dt$$ 
    \textbf{Alors}, $F(x)$ est dérivable et on a :\\\\ 
    $$\textcolor{red}{F^{\prime}}(x) = f(t) \Leftrightarrow \dfrac{d}{dx}\int_{a}^{b}f(t)dt = f(x)$$

  \vspace{2em}
  Si $F$ est une fonction de la dérivée est $f$; autrement dit : 
  $$\textcolor{myb}{F}^{\prime}(x) = f(x)$$            
  \textbf{Alors}, l'intégrale définit de $f(x)$ sur l'intervale $[a, b]$ correspond à
  la différence entre les ordonnées de $a$ et $b$ sur $F$ :
  \\\\
  
  $$\int_{a}^{b}f(x)dx = F(x)  \bigg|_{a}^{b} = F(b) - F(a) $$
  $$\big\Updownarrow$$
  $$\int_{a}^{b}\dfrac{d}{dt}f(t)dt =  f(b) - f(a)$$ 
\end{Definitionx*}


\section{Trouver l'aire sous la courbe}


\chapter{Techniques de bases}
\section{Polynôme}
\begin{Definitionx*}{}{}
  Un polynôme a la forme $p(x) = a_0 + a_1x + \cdots + a_nx^n$ et la \textcolor{myb}{puissance} d'un polynôme 
  est l'exposant le plus elevé de l'expression.    
\end{Definitionx*}

\begin{note}{}{}
    Lorsque le degré du numérateur est plus grand ou égal au degré du dénominateur, on peut effectuer une division 
    polynomiale pour simplifier une expression : 
    \begin{align*}
      \dfrac{x^-1}{x^2 +1} = 1 - \dfrac{2}{x^2 +1}
    \end{align*}
\end{note}



\begin{Concept}{Complétion du carré}{}
  Soit un polynôme $p = ax^2 + bx + c$, on peut compléter le carré en considérant : 
  \begin{align*}
  &h = \left(\dfrac{b}{2}\right)^2 \\ 
  &p(x) = a\Bigl(x^2 + \dfrac{b}{a}x - \textcolor{blue}{\left(\dfrac{b}{2a}\right)^2} 
  - \textcolor{red}{\left(\dfrac{b}{2a}\right)^2}\Bigr)  + c               
  \\
  &p(x) = a\Bigl(x^2 + \dfrac{b}{a} x - \textcolor{blue}{\dfrac{b^2}{4a}}\Bigr) + \textcolor{red}{\dfrac{b^2}{4a}}  + c   
  \\
  &p(x) = a\Bigl(x - \textcolor{blue}{\dfrac{b}{2}}\Bigr)^2 +  
  \textcolor{red}{\dfrac{b^2}{4a}}  + c  
  \end{align*}


  
\end{Concept}

\end{multicols*}


\end{document}
