\documentclass{report}
%\usepackage[utopia]{mathdesign}
%\usepackage{amsmath, amsthm}


\usepackage{amsmath,amsfonts,amsthm,amssymb,mathtools}
%\usepackage[varbb]{newpxmath}
%\usepackage[osf,largesc,theoremfont]{newpxtext}
%\usepackage{coelacanth}
%\usepackage{beraserif} % Bitstream Vera Serif font
%\usepackage{berasans} % Bitstream Vera Sans font
%\usepackage{beramono} % Bitstream Vera Sans Mono font
%\usepackage{berasans}
%\usepackage{libertine}
%\usepackage{mathpazo}
%\usepackage{palatino}
%\usepackage{crimson}


%% Choose one of the following (if not choosing the  
%% default, viz., Computer Modern, font family):
%\usepackage{lmodern}
\usepackage{bold-extra}
\usepackage{pgfplots}
\usepackage{pgfplots}
\usepgfplotslibrary{external}
%%
%\usepackage{mathpazo}
% \usepackage{newpxmath}
%\usepackage{kpfonts} % Very good
%%
%\usepackage{mathptmx} %Very good
%\usepackage{stix} 
%\usepackage{txfonts} %Very good
\usepackage{newtxtext,newtxmath} %Very good
%%
%\usepackage{libertine} \usepackage[libertine]{newtxmath}
%\usepackage{libertine,libertinust1math} % added 2019/11/28
%%
%\usepackage{newpxtext} \usepackage[euler-digits]{eulervm}
%\usepackage{textcomp}
%\usepackage{bm}
\usepackage{contour}
\usepackage{adjustbox}



%%%%%%%%%%%%%%%%%%%%%%%%%%%%%%%%%%%%%%%%%%%%%%%%%%%%%%%%%%%%%%%%%%%%%%%%%%%%%%%%%%%%%%%%%%%%%%%%%
%                                 Additional Packages (commented)
%%%%%%%%%%%%%%%%%%%%%%%%%%%%%%%%%%%%%%%%%%%%%%%%%%%%%%%%%%%%%%%%%%%%%%%%%%%%%%%%%%%%%%%%%%%%%%%%%
%%%%%%%%%%%%%%%%%%%%%%%%%%%%%%%%%%%%%%%%%%%%%%%%%%%%%%%%%%%%%%%%%%%%%%%%%%%%%%%%%%%%%%%%%%%%%%%%%
%                                 Language and Encoding
%%%%%%%%%%%%%%%%%%%%%%%%%%%%%%%%%%%%%%%%%%%%%%%%%%%%%%%%%%%%%%%%%%%%%%%%%%%%%%%%%%%%%%%%%%%%%%%%%
\usepackage[french]{babel} % Utilisation du français pour la typographie et les hyphenations
\usepackage[T1]{fontenc} % Utilisation de l'encodage de police T1 pour une meilleure gestion des caractères accentués
\usepackage{titlesec}
\usepackage[utf8]{inputenc} % Permet l'utilisation de l'encodage UTF-8 dans le fichier source
\usepackage{csquotes} % Gestion avancée des guillemets
\usepackage{microtype}
\usepackage{listings}

%%%%%%%%%%%%%%%%%%%%%%%%%%%%%%%%%
% PACKAGE IMPORTS
%%%%%%%%%%%%%%%%%%%%%%%%%%%%%%%%%


\usepackage[tmargin=2cm,rmargin=1in,lmargin=1in,margin=0.85in,bmargin=2cm,footskip=.2in]{geometry}
\usepackage{xfrac}
\usepackage[makeroom]{cancel}
\usepackage{bookmark}
\usepackage{enumitem}
\usepackage{hyperref,theoremref}
\hypersetup{
	pdftitle={Assignment},
	colorlinks=true, linkcolor=black,
	bookmarksnumbered=true,
	bookmarksopen=true
}
\hypersetup{
    colorlinks=true,       % false: boxed links; true: colored links
    linkcolor=black,       % color of internal links
    citecolor=black,       % color of citation links
    filecolor=black,       % color of file links
    urlcolor=black         % color of external links (like \url and \href)
}
\usepackage[most,many,breakable]{tcolorbox}
\usepackage{xcolor}
\usepackage{varwidth}
\usepackage{varwidth}
\usepackage{etoolbox}
%\usepackage{authblk}
\usepackage{nameref}
\usepackage{multicol,array}
\usepackage{colortbl}
\usepackage{tikz-cd}
\usepackage[ruled,vlined,linesnumbered]{algorithm2e}
\usepackage{comment} % enables the use of multi-line comments (\ifx \fi) 
\usepackage{import}
\usepackage{xifthen}
\usepackage{pdfpages}
\usepackage{transparent}

\newcommand\mycommfont[1]{\footnotesize\ttfamily\textcolor{blue}{#1}}
\SetCommentSty{mycommfont}
\newcommand{\incfig}[1]{%
    \def\svgwidth{\columnwidth}
    \import{./figures/}{#1.pdf_tex}
}

\usepackage{tikzsymbols}
%\renewcommand\qedsymbol{$\Laughey$}


%\usepackage{import}
%\usepackage{xifthen}
%\usepackage{pdfpages}
%\usepackage{transparent}


%%%%%%%%%%%%%%%%%%%%%%%%%%%%%%
% SELF MADE COLORS
%%%%%%%%%%%%%%%%%%%%%%%%%%%%%%



\definecolor{lightBlue}{rgb}{0.88,1,1}
\definecolor{myg}{RGB}{56, 140, 70}
\definecolor{myb}{RGB}{45, 111, 177}
\definecolor{myr}{RGB}{199, 68, 64}
\definecolor{mytheorembg}{HTML}{F2F2F9}
\definecolor{mytheoremfr}{HTML}{00007B}
\definecolor{mylenmabg}{HTML}{FFFAF8}
\definecolor{mylenmafr}{HTML}{983b0f}
\definecolor{mypropbg}{HTML}{f2fbfc}
\definecolor{mypropfr}{HTML}{191971}
\definecolor{myexamplebg}{HTML}{F2FBF8}
\definecolor{myexamplefr}{HTML}{88D6D1}
\definecolor{myexampleti}{HTML}{2A7F7F}
\definecolor{mydefinitbg}{HTML}{E5E5FF}
\definecolor{mydefinitfr}{HTML}{3F3FA3}
\definecolor{notesgreen}{RGB}{0,162,0}
\definecolor{myp}{RGB}{197, 92, 212}
\definecolor{mygr}{HTML}{2C3338}
\definecolor{myred}{RGB}{127,0,0}
\definecolor{myyellow}{RGB}{169,121,69}
\definecolor{myexercisebg}{HTML}{F2FBF8}
\definecolor{myexercisefg}{HTML}{88D6D1}
\definecolor{myblue}{RGB}{0,82,155}

\usepackage{forest}
\usepackage{adjustbox}



%%%%%%%%%%%%%%%%%%%%%%%%%%%%
% chapter format
%%%%%%%%%%%%%%%%%%%%%%%%%%%%

\titleformat{\chapter}[display]
  {\normalfont\bfseries\color{doc!60}}
  {\filleft%
    \begin{tikzpicture}
    \node[
      outer sep=-4pt,
      text width=0.75cm,
      minimum height=1cm,
      fill=doc!60,
      font=\color{white}\fontsize{20}{25}\selectfont\fontfamily{lmss},
      align=center
      ] (num) {\thechapter};
    \node[
      rotate=90,
      anchor=south,
      font=\color{black}\normalsize\fontfamily{lmss}
      ] at ([xshift=-5pt]num.west) {\textls[180]{\textsc{}}};  
    \end{tikzpicture}%
  }
  {10pt}
  {\titlerule[2.5pt]\vskip1.5pt\titlerule\vskip4pt\large\bfseries\sc}
\titlespacing*{\chapter}{0pt}{*0}{*0}

\makeatletter
\patchcmd{\chapter}{\if@openright\cleardoublepage\else\clearpage\fi}{}{}{}
\makeatother




%%%%%%%%%%%%%%%%%%%%%%%%%%%%
% TCOLORBOX SETUPS
%%%%%%%%%%%%%%%%%%%%%%%%%%%%

\setlength{\parindent}{1cm}
%================================
% THEOREM BOX
%================================

\tcbuselibrary{theorems,skins,hooks}
\newtcbtheorem[number within=section]{Theorem}{Theorem}
{%
	enhanced,
	breakable,
	colback = mytheorembg,
	frame hidden,
	boxrule = 0sp,
	borderline west = {2pt}{0pt}{mytheoremfr},
	sharp corners,
	detach title,
	before upper = \tcbtitle\par\smallskip,
	coltitle = mytheoremfr,
	fonttitle = \bfseries\sffamily,
	description font = \mdseries,
	separator sign none,
	segmentation style={solid, mytheoremfr},
}
{th}

\tcbuselibrary{theorems,skins,hooks}
\newtcbtheorem[number within=chapter]{theorem}{Theorem}
{%
	enhanced,
	breakable,
	colback = mytheorembg,
	frame hidden,
	boxrule = 0sp,
	borderline west = {2pt}{0pt}{mytheoremfr},
	sharp corners,
	detach title,
	before upper = \tcbtitle\par\smallskip,
	coltitle = mytheoremfr,
	fonttitle = \bfseries\sffamily,
	description font = \mdseries,
	separator sign none,
	segmentation style={solid, mytheoremfr},
}
{th}


\tcbuselibrary{theorems,skins,hooks}
\newtcolorbox{Theoremcon}
{%
	enhanced
	,breakable
	,colback = mytheorembg
	,frame hidden
	,boxrule = 0sp
	,borderline west = {2pt}{0pt}{mytheoremfr}
	,sharp corners
	,description font = \mdseries
	,separator sign none
}

%================================
% Preuve
%================================

% Crée un environnement "Preuve" numéroté en fonction du document
\tcbuselibrary{theorems,skins,hooks}
\newtcbtheorem[number within=chapter]{Preuve}{Preuve}
{
	enhanced,
	breakable,
	colback=white,
	frame hidden,
	boxrule = 0sp,
	borderline west = {2pt}{0pt}{mytheoremfr},
	sharp corners,
	detach title,
	before upper = \tcbtitle\par\smallskip,
	coltitle = mytheoremfr,
	description font=\fontfamily{lmss}\selectfont,
	fonttitle=\fontfamily{lmss}\selectfont\bfseries,
	separator sign none,
	segmentation style={solid, mytheoremfr},
}
{th}



%================================
% Corollery
%================================
\tcbuselibrary{theorems,skins,hooks}
\newtcbtheorem[number within=section]{Corollary}{Corollaire}
{%
	enhanced
	,breakable
	,colback = myp!10
	,frame hidden
	,boxrule = 0sp
	,borderline west = {2pt}{0pt}{myp!85!black}
	,sharp corners
	,detach title
	,before upper = \tcbtitle\par\smallskip
	,coltitle = myp!85!black
	,fonttitle = \bfseries\sffamily
	,description font = \mdseries
	,separator sign none
	,segmentation style={solid, myp!85!black}
}
{th}
\tcbuselibrary{theorems,skins,hooks}
\newtcbtheorem[number within=chapter]{corollary}{Corollaire}
{%
	enhanced
	,breakable
	,colback = myp!10
	,frame hidden
	,boxrule = 0sp
	,borderline west = {2pt}{0pt}{myp!85!black}
	,sharp corners
	,detach title
	,before upper = \tcbtitle\par\smallskip
	,coltitle = myp!85!black
	,fonttitle = \bfseries\sffamily
	,description font = \mdseries
	,separator sign none
	,segmentation style={solid, myp!85!black}
}
{th}


%================================
% LENMA
%================================

\tcbuselibrary{theorems,skins,hooks}
\newtcbtheorem[number within=section]{Lenma}{Lemme}
{%
	enhanced,
	breakable,
	colback = mylenmabg,
	frame hidden,
	boxrule = 0sp,
	borderline west = {2pt}{0pt}{mylenmafr},
	sharp corners,
	detach title,
	before upper = \tcbtitle\par\smallskip,
	coltitle = mylenmafr,
	fonttitle = \bfseries\sffamily,
	description font = \mdseries,
	separator sign none,
	segmentation style={solid, mylenmafr},
}
{th}

\tcbuselibrary{theorems,skins,hooks}
\newtcbtheorem[number within=chapter]{lenma}{Lemme}
{%
	enhanced,
	breakable,
	colback = mylenmabg,
	frame hidden,
	boxrule = 0sp,
	borderline west = {2pt}{0pt}{mylenmafr},
	sharp corners,
	detach title,
	before upper = \tcbtitle\par\smallskip,
	coltitle = mylenmafr,
	fonttitle = \bfseries\sffamily,
	description font = \mdseries,
	separator sign none,
	segmentation style={solid, mylenmafr},
}
{th}


%================================
% PROPOSITION
%================================

\tcbuselibrary{theorems,skins,hooks}
\newtcbtheorem[number within=section]{Prop}{Proposition}
{%
	enhanced,
	breakable,
	colback = mypropbg,
	frame hidden,
	boxrule = 0sp,
	borderline west = {2pt}{0pt}{mypropfr},
	sharp corners,
	detach title,
	before upper = \tcbtitle\par\smallskip,
	coltitle = mypropfr,
	fonttitle = \bfseries\sffamily,
	description font = \mdseries,
	separator sign none,
	segmentation style={solid, mypropfr},
}
{th}

\tcbuselibrary{theorems,skins,hooks}
\newtcbtheorem[number within=chapter]{prop}{Proposition}
{%
	enhanced,
	breakable,
	colback = mypropbg,
	frame hidden,
	boxrule = 0sp,
	borderline west = {2pt}{0pt}{mypropfr},
	sharp corners,
	detach title,
	before upper = \tcbtitle\par\smallskip,
	coltitle = mypropfr,
	fonttitle = \bfseries\sffamily,
	description font = \mdseries,
	separator sign none,
	segmentation style={solid, mypropfr},
}
{th}


%================================
% CLAIM
%================================

\tcbuselibrary{theorems,skins,hooks}
\newtcbtheorem[number within=section]{claim}{Affirmation}
{%
	enhanced
	,breakable
	,colback = myg!10
	,frame hidden
	,boxrule = 0sp
	,borderline west = {2pt}{0pt}{myg}
	,sharp corners
	,detach title
	,before upper = \tcbtitle\par\smallskip
	,coltitle = myg!85!black
	,fonttitle = \bfseries\sffamily
	,description font = \mdseries
	,separator sign none
	,segmentation style={solid, myg!85!black}
}
{th}



%================================
% Exercise
%================================

\tcbuselibrary{theorems,skins,hooks}
\newtcbtheorem[number within=section]{Exercise}{Exercice}
{%
	enhanced,
	breakable,
	colback = myexercisebg,
	frame hidden,
	boxrule = 0sp,
	borderline west = {2pt}{0pt}{myexercisefg},
	sharp corners,
	detach title,
	before upper = \tcbtitle\par\smallskip,
	coltitle = myexercisefg,
	fonttitle = \bfseries\sffamily,
	description font = \mdseries,
	separator sign none,
	segmentation style={solid, myexercisefg},
}
{th}

\tcbuselibrary{theorems,skins,hooks}
\newtcbtheorem[number within=chapter]{exercise}{Exercice}
{%
	enhanced,
	breakable,
	colback = myexercisebg,
	frame hidden,
	boxrule = 0sp,
	borderline west = {2pt}{0pt}{myexercisefg},
	sharp corners,
	detach title,
	before upper = \tcbtitle\par\smallskip,
	coltitle = myexercisefg,
	fonttitle = \bfseries\sffamily,
	description font = \mdseries,
	separator sign none,
	segmentation style={solid, myexercisefg},
}
{th}

%================================
% EXAMPLE BOX
%================================

\newtcbtheorem[number within=section]{Example}{Example}
{%
	colback = myexamplebg
	,breakable
	,colframe = myexamplefr
	,coltitle = myexampleti
	,boxrule = 1pt
	,sharp corners
	,detach title
	,before upper=\tcbtitle\par\smallskip
	,fonttitle = \bfseries
	,description font = \mdseries
	,separator sign none
	,description delimiters parenthesis
}
{ex}

\newtcbtheorem[number within=chapter]{example}{Example}
{%
	colback = myexamplebg
	,breakable
	,colframe = myexamplefr
	,coltitle = myexampleti
	,boxrule = 1pt
	,sharp corners
	,detach title
	,before upper=\tcbtitle\par\smallskip
	,fonttitle = \bfseries
	,description font = \mdseries
	,separator sign none
	,description delimiters parenthesis
}
{ex}

%================================
% DEFINITION BOX
%================================

\newtcbtheorem[number within=chapter]{Definition}{Définition}{enhanced,
	before skip=2mm,after skip=2mm, colback=red!5,colframe=red!80!black,boxrule=0.5mm,
	attach boxed title to top left={xshift=1cm,yshift*=1mm-\tcboxedtitleheight}, varwidth boxed title*=-3cm,
	boxed title style={frame code={
			\path[fill=tcbcolback!10!red]
			([yshift=-1mm,xshift=-1mm]frame.north west)
			arc[start angle=0,end angle=180,radius=1mm]
			([yshift=-1mm,xshift=1mm]frame.north east)
			arc[start angle=180,end angle=0,radius=1mm];
			\path[left color=tcbcolback!10!myred,right color=tcbcolback!10!myred,
			middle color=tcbcolback!60!myred]
			([xshift=-2mm]frame.north west) -- ([xshift=2mm]frame.north east)
			[rounded corners=1mm]-- ([xshift=1mm,yshift=-1mm]frame.north east)
			-- (frame.south east) -- (frame.south west)
			-- ([xshift=-1mm,yshift=-1mm]frame.north west)
			[sharp corners]-- cycle;
		},interior engine=empty,
	},
	fonttitle=\bfseries,
	title={#2},#1}{def}
\newtcbtheorem[number within=chapter]{definition}{Definition}{enhanced,
	before skip=2mm,after skip=2mm, colback=red!5,colframe=red!80!black,boxrule=0.5mm,
	attach boxed title to top left={xshift=1cm,yshift*=1mm-\tcboxedtitleheight}, varwidth boxed title*=-3cm,
	boxed title style={frame code={
					\path[fill=tcbcolback]
					([yshift=-1mm,xshift=-1mm]frame.north west)
					arc[start angle=0,end angle=180,radius=1mm]
					([yshift=-1mm,xshift=1mm]frame.north east)
					arc[start angle=180,end angle=0,radius=1mm];
					\path[left color=tcbcolback!60!black,right color=tcbcolback!60!black,
						middle color=tcbcolback!80!black]
					([xshift=-2mm]frame.north west) -- ([xshift=2mm]frame.north east)
					[rounded corners=1mm]-- ([xshift=1mm,yshift=-1mm]frame.north east)
					-- (frame.south east) -- (frame.south west)
					-- ([xshift=-1mm,yshift=-1mm]frame.north west)
					[sharp corners]-- cycle;
				},interior engine=empty,
		},
	fonttitle=\bfseries,
	title={#2},#1}{def}


    \newtcbtheorem{Definitionx}{Définition}
    {
    enhanced,
    breakable,
    colback=red!5,
      before upper=\tcbtitle\par\Hugeskip,
    frame hidden,
    boxrule = 0sp,
    borderline west = {2pt}{0pt}{red!80!black},
    sharp corners,
    detach title,
    before upper = \tcbtitle\par\smallskip,
    coltitle = red!80!black,
    description font=\mdseries\fontfamily{lmss}\selectfont,
    fonttitle=\fontfamily{lmss}\selectfont\bfseries,
    fontlower=\fontfamily{cmr}\selectfont,
      separator sign none,
    segmentation style={solid, mytheoremfr},
    }
    {th}



%================================
% Solution BOX
%================================

\makeatletter
\newtcbtheorem{question}{Question}{enhanced,
	breakable,
	colback=white,
	colframe=myb!80!black,
	attach boxed title to top left={yshift*=-\tcboxedtitleheight},
	fonttitle=\bfseries,
	title={#2},
	boxed title size=title,
	boxed title style={%
			sharp corners,
			rounded corners=northwest,
			colback=tcbcolframe,
			boxrule=0pt,
		},
	underlay boxed title={%
			\path[fill=tcbcolframe] (title.south west)--(title.south east)
			to[out=0, in=180] ([xshift=5mm]title.east)--
			(title.center-|frame.east)
			[rounded corners=\kvtcb@arc] |-
			(frame.north) -| cycle;
		},
	#1
}{def}
\makeatother

%================================
% SOLUTION BOX
%================================

\makeatletter
\newtcolorbox{solution}{enhanced,
	breakable,
	colback=white,
	colframe=myg!80!black,
	attach boxed title to top left={yshift*=-\tcboxedtitleheight},
	title=Solution,
	boxed title size=title,
	boxed title style={%
			sharp corners,
			rounded corners=northwest,
			colback=tcbcolframe,
			boxrule=0pt,
		},
	underlay boxed title={%
			\path[fill=tcbcolframe] (title.south west)--(title.south east)
			to[out=0, in=180] ([xshift=5mm]title.east)--
			(title.center-|frame.east)
			[rounded corners=\kvtcb@arc] |-
			(frame.north) -| cycle;
		},
}
\makeatother

%================================
% Question BOX
%================================

\makeatletter
\newtcbtheorem{qstion}{Question}{enhanced,
	breakable,
	colback=white,
	colframe=mygr,
	attach boxed title to top left={yshift*=-\tcboxedtitleheight},
	fonttitle=\bfseries,
	title={#2},
	boxed title size=title,
	boxed title style={%
			sharp corners,
			rounded corners=northwest,
			colback=tcbcolframe,
			boxrule=0pt,
		},
	underlay boxed title={%
			\path[fill=tcbcolframe] (title.south west)--(title.south east)
			to[out=0, in=180] ([xshift=5mm]title.east)--
			(title.center-|frame.east)
			[rounded corners=\kvtcb@arc] |-
			(frame.north) -| cycle;
		},
	#1
}{def}
\makeatother

\newtcbtheorem[number within=chapter]{wconc}{Wrong Concept}{
	breakable,
	enhanced,
	colback=white,
	colframe=myr,
	arc=0pt,
	outer arc=0pt,
	fonttitle=\bfseries\sffamily\large,
	colbacktitle=myr,
	attach boxed title to top left={},
	boxed title style={
			enhanced,
			skin=enhancedfirst jigsaw,
			arc=3pt,
			bottom=0pt,
			interior style={fill=myr}
		},
	#1
}{def}


%%%%%%%%%%%%%%%%%%%%%%%%%%%%%%%%%%%%%%%%%%%%%%%%%%%%%%%%%%%%%%%%%%%%%%%%%%%%%%%%%%%%%%%%%%%%%%%%%
%                                Environnement Explication
%%%%%%%%%%%%%%%%%%%%%%%%%%%%%%%%%%%%%%%%%%%%%%%%%%%%%%%%%%%%%%%%%%%%%%%%%%%%%%%%%%%%%%%%%%%%%%%%%
\newtcbtheorem{Explication}{Explication}
{
	enhanced,
	breakable,
	colback=white,
	frame hidden,
	boxrule = 0sp,
	borderline west = {2pt}{0pt}{mytheoremfr},
	sharp corners,
	detach title,
	before upper = \tcbtitle\par\smallskip,
	coltitle = mytheoremfr,
	description font=\fontfamily{lmss}\selectfont,
	fonttitle=\fontfamily{lmss}\selectfont\bfseries,
	separator sign none,
	segmentation style={solid, mytheoremfr},
}
{th}



%%%%%%%%%%%%%%%%%%%%%%%%%%%%%%%%%%%%%%%%%%%%%%%%%%%%%%%%%%%%%%%%%%%%%%%%%%%%%%%%%%%%%%%%%%%%%%%%%
%                                Environnement EExample
%%%%%%%%%%%%%%%%%%%%%%%%%%%%%%%%%%%%%%%%%%%%%%%%%%%%%%%%%%%%%%%%%%%%%%%%%%%%%%%%%%%%%%%%%%%%%%%%%
% Crée un environnement "EExample" numéroté en fonction du document
\newtcbtheorem{EExample}{Exemple}
{
	enhanced,
	breakable,
	colback=white,
	frame hidden,
	boxrule = 0sp,
	borderline west = {2pt}{0pt}{myb},
	sharp corners,
	detach title,
	before upper = \tcbtitle\par\smallskip,
	coltitle = myb,
	description font=\mdseries\fontfamily{lmss}\selectfont,
	fonttitle=\fontfamily{lmss}\selectfont\bfseries,
	separator sign none,
	segmentation style={solid, mytheoremfr},
}
{th}

%%%%%%%%%%%%%%%%%%%%%%%%%%%%%%%%%%%%%%%%%%%%%%%%%%%%%%%%%%%%%%%%%%%%%%%%%%%%%%%%%%%%%%%%%%%%%%%%%
%                                Environnement Liste
%%%%%%%%%%%%%%%%%%%%%%%%%%%%%%%%%%%%%%%%%%%%%%%%%%%%%%%%%%%%%%%%%%%%%%%%%%%%%%%%%%%%%%%%%%%%%%%%%

% Pour créer un environnement "Liste" 

\tcbuselibrary{theorems,skins,hooks}
\newtcbtheorem[number within=chapter]{Liste}{Liste}
{%
	enhanced
	,breakable
	,colback = myp!10
	,frame hidden
	,boxrule = 0sp
	,borderline west = {2pt}{0pt}{myp!85!black}
	,sharp corners
	,detach title
	,before upper = \tcbtitle\par\smallskip
	,coltitle = myp!85!black
	,fonttitle = \bfseries\sffamily
	,description font = \mdseries
	,separator sign none
	,segmentation style={solid, myp!85!black}

}
{th}


%%%%%%%%%%%%%%%%%%%%%%%%%%%%%%%%%%%%%%%%%%%%%%%%%%%%%%%%%%%%%%%%%%%%%%%%%%%%%%%%%%%%%%%%%%%%%%%%%
%                                Environnement Syntaxe
%%%%%%%%%%%%%%%%%%%%%%%%%%%%%%%%%%%%%%%%%%%%%%%%%%%%%%%%%%%%%%%%%%%%%%%%%%%%%%%%%%%%%%%%%%%%%%%%%
\tcbuselibrary{theorems,skins,hooks}
\newtcbtheorem{Syntaxe}{Syntaxe.}
{%
	enhanced
	,breakable
	,colback = myp!10
	,frame hidden
	,boxrule = 0sp
	,borderline west = {2pt}{0pt}{myp!85!black}
	,sharp corners
	,detach title
	,before upper = \tcbtitle\par\smallskip
	,coltitle = myp!85!black
	,fonttitle = \bfseries\fontfamily{lmss}\selectfont 
	,description font = \mdseries\fontfamily{lmss}\selectfont 
	,separator sign none
	,segmentation style={solid, myp!85!black}
}
{th}



%%%%%%%%%%%%%%%%%%%%%%%%%%%%%%%%%%%%%%%%%%%%%%%%%%%%%%%%%%%%%%%%%%%%%%%%%%%%%%%%%%%%%%%%%%%%%%%%%
%                                Environnement Concept
%%%%%%%%%%%%%%%%%%%%%%%%%%%%%%%%%%%%%%%%%%%%%%%%%%%%%%%%%%%%%%%%%%%%%%%%%%%%%%%%%%%%%%%%%%%%%%%%%
% Crée un environnement "Concept" numéroté en fonction du document
\tcbuselibrary{theorems,skins,hooks}
\newtcbtheorem{Concept}{Concept.}
{
	enhanced,
	breakable,
	colback=mylenmabg,
	frame hidden,
	boxrule = 0sp,
	borderline west = {2pt}{0pt}{mylenmafr},
	sharp corners,
	detach title,
	before upper = \tcbtitle\par\smallskip,
	coltitle = mylenmafr,
	description font=\mdseries\fontfamily{lmss}\selectfont,
	fonttitle=\fontfamily{lmss}\selectfont\bfseries,
	separator sign none,
	segmentation style={solid, mytheoremfr},
}
{th}



%%%%%%%%%%%%%%%%%%%%%%%%%%%%%%%%%%%%%%%%%%%%%%%%%%%%%%%%%%%%%%%%%%%%%%%%%%%%%%%%%%%%%%%%%%%%%%%%%
%                                Environnement codeEx
%%%%%%%%%%%%%%%%%%%%%%%%%%%%%%%%%%%%%%%%%%%%%%%%%%%%%%%%%%%%%%%%%%%%%%%%%%%%%%%%%%%%%%%%%%%%%%%%%
% Crée un environnement "codeEx" numéroté en fonction du document
\tcbuselibrary{theorems,skins,hooks}
\newtcbtheorem{codeEx}{Exemple}
{
	enhanced,
	breakable,
	colback=white,
	frame hidden,
	boxrule = 0sp,
	borderline west = {2pt}{0pt}{gruvbox-bg},
	sharp corners,
	detach title,
	before upper = \tcbtitle\par\smallskip,
	coltitle = gruvbox-bg,
	description font=\md:wqseries\fontfamily{lmss}\selectfont,
	fonttitle=\fontfamily{lmss}\selectfont\bfseries,
	separator sign none,
	segmentation style={solid, mytheoremfr},
}
{th}



%%%%%%%%%%%%%%%%%%%%%%%%%%%%%%%%%%%%%%%%%%%%%%%%%%%%%%%%%%%%%%%%%%%%%%%%%%%%%%%%%%%%%%%%%%%%%%%%%
%                                Environnement Remarque
%%%%%%%%%%%%%%%%%%%%%%%%%%%%%%%%%%%%%%%%%%%%%%%%%%%%%%%%%%%%%%%%%%%%%%%%%%%%%%%%%%%%%%%%%%%%%%%%%
% Crée un environnement "Remarque" numéroté en fonction du document
\tcbuselibrary{theorems,skins,hooks}
\newtcbtheorem{codeRem}{Remarque}
{
	enhanced,
	breakable,
	colback=white,
	frame hidden,
	boxrule = 0sp,
	borderline west = {2pt}{0pt}{gruvbox-bg},
	sharp corners,
	detach title,
	before upper = \tcbtitle\par\smallskip,
	coltitle = gruvbox-bg,
	description font=\mdseries\fontfamily{lmss}\selectfont,
	fonttitle=\fontfamily{lmss}\selectfont\bfseries,
	separator sign none,
	segmentation style={solid, mytheoremfr},
}
{th}


%%%%%%%%%%%%%%%%%%%%%%%%%%%%%%%%%%%%%%%%%%%%%%%%%%%%%%%%%%%%%%%%%%%%%%%%%%%%%%%%%%%%%%%%%%%%%%%%%
%                                Environnement Identité
%%%%%%%%%%%%%%%%%%%%%%%%%%%%%%%%%%%%%%%%%%%%%%%%%%%%%%%%%%%%%%%%%%%%%%%%%%%%%%%%%%%%%%%%%%%%%%%%%
\tcbuselibrary{theorems,skins,hooks}
\newtcbtheorem{Identite}{Identité}
{
	enhanced,
	breakable,
	colback=white,
  before upper=\tcbtitle\par\Hugeskip,
	frame hidden,
	boxrule = 0sp,
	borderline west = {2pt}{0pt}{gruvbox-bg},
	sharp corners,
	detach title,
	before upper = \tcbtitle\par\smallskip,
	coltitle = gruvbox-bg,
	description font=\mdseries\fontfamily{lmss}\selectfont,
	fonttitle=\fontfamily{lmss}\selectfont\bfseries,
	fontlower=\fontfamily{cmr}\selectfont,
  separator sign none,
	segmentation style={solid, mytheoremfr},
}
{th}



%%%%%%%%%%%%%%%%%%%%%%%%%%%%%%%%%%%%%%%%%%%%%%%%%%%%%%%%%%%%%%%%%%%%%%%%%%%%%%%%%%%%%%%%%%%%%%%%%
%                                Environnement Exercice 
%%%%%%%%%%%%%%%%%%%%%%%%%%%%%%%%%%%%%%%%%%%%%%%%%%%%%%%%%%%%%%%%%%%%%%%%%%%%%%%%%%%%%%%%%%%%%%%%%
\tcbuselibrary{theorems,skins,hooks}
\newtcbtheorem{Exercice}{Exercice}
{
	enhanced,
	breakable,
	colback=white,
  before upper=\tcbtitle\par\Hugeskip,
	frame hidden,
	boxrule = 0sp,
	borderline west = {2pt}{0pt}{gruvbox-green},
	sharp corners,
	detach title,
	before upper = \tcbtitle\par\smallskip,
	coltitle = gruvbox-green,
	description font=\mdseries\fontfamily{lmss}\selectfont,
	fonttitle=\fontfamily{lmss}\selectfont\bfseries,
	fontlower=\fontfamily{cmr}\selectfont,
  separator sign none,
	segmentation style={solid, mytheoremfr},
}
{th}


%%%%%%%%%%%%%%%%%%%%%%%%%%%%%%%%%%%%%%%%%%%%%%%%%%%%%%%%%%%%%%%%%%%%%%%%%%%%%%%%%%%%%%%%%%%%%%%%%
%                                Environnement Réponse
%%%%%%%%%%%%%%%%%%%%%%%%%%%%%%%%%%%%%%%%%%%%%%%%%%%%%%%%%%%%%%%%%%%%%%%%%%%%%%%%%%%%%%%%%%%%%%%%%
% Crée un environnement "Réponse" numéroté en fonction du document
\tcbuselibrary{theorems,skins,hooks}
\newtcbtheorem{Reponse}{Réponse}
{
	enhanced,
	breakable,
	colback=white,
	frame hidden,
	boxrule = 0sp,
	borderline west = {2pt}{0pt}{mytheoremfr},
	sharp corners,
	detach title,
	before upper = \tcbtitle\par\smallskip,
	coltitle = mytheoremfr,
	description font=\fontfamily{lmss}\selectfont,
	fonttitle=\fontfamily{lmss}\selectfont\bfseries,
	separator sign none,
	segmentation style={solid, mytheoremfr},
}
{th}

\newtcbtheorem{Remarque}{Remarque.}
{
	enhanced,
	breakable,
	colback=white,
	frame hidden,
	boxrule = 0sp,
	borderline west = {2pt}{0pt}{myb},
	sharp corners,
	detach title,
	before upper = \tcbtitle\par\smallskip,
	coltitle = myb,
	description font=\mdseries\fontfamily{lmss}\selectfont,
	fonttitle=\fontfamily{lmss}\selectfont\bfseries,
	separator sign none,
	segmentation style={solid, mytheoremfr},
}
{th}





%================================
% NOTE BOX
%================================

\usetikzlibrary{arrows,calc,shadows.blur}
\usetikzlibrary {arrows.meta,backgrounds,fit,positioning,petri}
\tcbuselibrary{skins}
\newtcolorbox{note}[1][]{%
	enhanced jigsaw,
	colback=gray!20!white,%
	colframe=gray!80!black,
	size=small,
	boxrule=1pt,
	title=\textbf{Note:-},
	halign title=flush center,
	coltitle=black,
	breakable,
	drop shadow=black!50!white,
	attach boxed title to top left={xshift=1cm,yshift=-\tcboxedtitleheight/2,yshifttext=-\tcboxedtitleheight/2},
	minipage boxed title=1.5cm,
	boxed title style={%
			colback=white,
			size=fbox,
			boxrule=1pt,
			boxsep=2pt,
			underlay={%
					\coordinate (dotA) at ($(interior.west) + (-0.5pt,0)$);
					\coordinate (dotB) at ($(interior.east) + (0.5pt,0)$);
					\begin{scope}
						\clip (interior.north west) rectangle ([xshift=3ex]interior.east);
						\filldraw [white, blur shadow={shadow opacity=60, shadow yshift=-.75ex}, rounded corners=2pt] (interior.north west) rectangle (interior.south east);
					\end{scope}
					\begin{scope}[gray!80!black]
						\fill (dotA) circle (2pt);
						\fill (dotB) circle (2pt);
					\end{scope}
				},
		},
	#1,
}

%%%%%%%%%%%%%%%%%%%%%%%%%%%%%%
% SELF MADE COMMANDS
%%%%%%%%%%%%%%%%%%%%%%%%%%%%%%


\newcommand{\thm}[2]{\begin{Theorem}{#1}{}#2\end{Theorem}}
\newcommand{\cor}[2]{\begin{Corollary}{#1}{}#2\end{Corollary}}
\newcommand{\mlenma}[2]{\begin{Lenma}{#1}{}#2\end{Lenma}}
\newcommand{\mprop}[2]{\begin{Prop}{#1}{}#2\end{Prop}}
\newcommand{\clm}[3]{\begin{claim}{#1}{#2}#3\end{claim}}
\newcommand{\wc}[2]{\begin{wconc}{#1}{}\setlength{\parindent}{1cm}#2\end{wconc}}
\newcommand{\thmcon}[1]{\begin{Theoremcon}{#1}\end{Theoremcon}}
\newcommand{\ex}[2]{\begin{Example}{#1}{}#2\end{Example}}
\newcommand{\dfn}[2]{\begin{Definition}[colbacktitle=red!75!black]{#1}{}#2\end{Definition}}
\newcommand{\dfnc}[2]{\begin{definition}[colbacktitle=red!75!black]{#1}{}#2\end{definition}}
\newcommand{\qs}[2]{\begin{question}{#1}{}#2\end{question}}
\newcommand{\pf}[2]{\begin{myproof}[#1]#2\end{myproof}}
\newcommand{\nt}[1]{\begin{note}#1\end{note}}

\newcommand*\circled[1]{\tikz[baseline=(char.base)]{
		\node[shape=circle,draw,inner sep=1pt] (char) {#1};}}
\newcommand\getcurrentref[1]{%
	\ifnumequal{\value{#1}}{0}
	{??}
	{\the\value{#1}}%
}
\newcommand{\getCurrentSectionNumber}{\getcurrentref{section}}
\newenvironment{myproof}[1][\proofname]{%
	\proof[\bfseries #1: ]%
}{\endproof}

\newcommand{\mclm}[2]{\begin{myclaim}[#1]#2\end{myclaim}}
\newenvironment{myclaim}[1][\claimname]{\proof[\bfseries #1: ]}{}

\newcounter{mylabelcounter}

\makeatletter
\newcommand{\setword}[2]{%
	\phantomsection
	#1\def\@currentlabel{\unexpanded{#1}}\label{#2}%
}
\makeatother




\tikzset{
	symbol/.style={
			draw=none,
			every to/.append style={
					edge node={node [sloped, allow upside down, auto=false]{$#1$}}}
		}
}


% deliminators
% Manually define paired delimiters
\newcommand{\abs}[1]{\lvert#1\rvert}
\newcommand{\norm}[1]{\lVert#1\rVert}
\newcommand{\ceil}[1]{\lceil#1\rceil}
\newcommand{\floor}[1]{\lfloor#1\rfloor}
\newcommand{\round}[1]{\lfloor#1\rceil}

%
\newsavebox\diffdbox
\newcommand{\slantedromand}{{\mathpalette\makesl{d}}}
\newcommand{\makesl}[2]{%
\begingroup
\sbox{\diffdbox}{$\mathsurround=0pt#1\mathrm{#2}$}%
\pdfsave
\pdfsetmatrix{1 0 0.2 1}%
\rlap{\usebox{\diffdbox}}%
\pdfrestore
\hskip\wd\diffdbox
\endgroup
}
\newcommand{\dd}[1][]{\ensuremath{\mathop{}\!\ifstrempty{#1}{%
\slantedromand\@ifnextchar^{\hspace{0.2ex}}{\hspace{0.1ex}}}%
{\slantedromand\hspace{0.2ex}^{#1}}}}
\ProvideDocumentCommand\dv{o m g}{%
  \ensuremath{%
    \IfValueTF{#3}{%
      \IfNoValueTF{#1}{%
        \frac{\dd #2}{\dd #3}%
      }{%
        \frac{\dd^{#1} #2}{\dd #3^{#1}}%
      }%
    }{%
      \IfNoValueTF{#1}{%
        \frac{\dd}{\dd #2}%
      }{%
        \frac{\dd^{#1}}{\dd #2^{#1}}%
      }%
    }%
  }%
}
\providecommand*{\pdv}[3][]{\frac{\partial^{#1}#2}{\partial#3^{#1}}}
%  - others
\DeclareMathOperator{\Lap}{\mathcal{L}}
\DeclareMathOperator{\Var}{Var} % varience
\DeclareMathOperator{\Cov}{Cov} % covarience
\DeclareMathOperator{\E}{E} % expected

% Since the amsthm package isn't loaded

% I prefer the slanted \leq
\let\oldleq\leq % save them in case they're every wanted
\let\oldgeq\geq
\renewcommand{\leq}{\leqslant}
\renewcommand{\geq}{\geqslant}

% % redefine matrix env to allow for alignment, use r as default
% \renewcommand*\env@matrix[1][r]{\hskip -\arraycolsep
%     \let\@ifnextchar\new@ifnextchar
%     \array{*\c@MaxMatrixCols #1}}


%\usepackage{framed}
%\usepackage{titletoc}
%\usepackage{etoolbox}
%\usepackage{lmodern}


%\patchcmd{\tableofcontents}{\contentsname}{\sffamily\contentsname}{}{}

%\renewenvironment{leftbar}
%{\def\FrameCommand{\hspace{6em}%
%		{\color{myyellow}\vrule width 2pt depth 6pt}\hspace{1em}}%
%	\MakeFramed{\parshape 1 0cm \dimexpr\textwidth-6em\relax\FrameRestore}\vskip2pt%
%}
%{\endMakeFramed}

%\titlecontents{chapter}
%[0em]{\vspace*{2\baselineskip}}
%{\parbox{4.5em}{%
%		\hfill\Huge\sffamily\bfseries\color{myred}\thecontentspage}%
%	\vspace*{-2.3\baselineskip}\leftbar\textsc{\small\chaptername~\thecontentslabel}\\\sffamily}
%{}{\endleftbar}
%\titlecontents{section}
%[8.4em]
%{\sffamily\contentslabel{3em}}{}{}
%{\hspace{0.5em}\nobreak\itshape\color{myred}\contentspage}
%\titlecontents{subsection}
%[8.4em]
%{\sffamily\contentslabel{3em}}{}{}  
%{\hspace{0.5em}\nobreak\itshape\color{myred}\contentspage}



%%%%%%%%%%%%%%%%%%%%%%%%%%%%%%%%%%%%%%%%%%%
% TABLE OF CONTENTS
%%%%%%%%%%%%%%%%%%%%%%%%%%%%%%%%%%%%%%%%%%%

\usepackage{tikz}
\usetikzlibrary{automata, positioning}
\definecolor{doc}{RGB}{0,60,110}
\usepackage{titletoc}
\contentsmargin{0cm}
\titlecontents{chapter}[4.9pc]
{\addvspace{40pt}%
	\begin{tikzpicture}[remember picture, overlay]%
		\draw[fill=doc!60,draw=doc!60] (-7,-.1) rectangle (-0.9,.5);%
		\pgftext[left,x=-3.5cm,y=0.2cm]{\color{white}\Large\sc\bfseries Section \ \thecontentslabel};%
	\end{tikzpicture}\color{doc!60}\large\sc\bfseries}%
{}
{}
{\;\titlerule\;\large\sc\bfseries Page \thecontentspage
	\begin{tikzpicture}[remember picture, overlay]
		\draw[fill=doc!60,draw=doc!60] (2pt,0) rectangle (4,0.1pt);
	\end{tikzpicture}}%
\titlecontents{section}[3.7pc]
{\addvspace{2pt}}
{\contentslabel[\thecontentslabel]{2pc}}
{}
{\hfill\small \thecontentspage}
[]
\titlecontents*{subsection}[3.7pc]
{\addvspace{-1pt}\small}
{}
{}
{\ --- \small\thecontentspage}
[ \textbullet\ ][]

\makeatletter
\renewcommand{\tableofcontents}{%
	\chapter*{%
	  \vspace*{-20\p@}%
	  \begin{tikzpicture}[remember picture, overlay]%
		  \pgftext[right,x=15cm,y=0.2cm]{\color{doc!60}\Huge\sc\bfseries \contentsname};%
		  \draw[fill=doc!60,draw=doc!60] (13,-.75) rectangle (20,2);%
		  \clip (13,-.75) rectangle (20,1);
		  \pgftext[right,x=15cm,y=0.2cm]{\color{white}\Huge\sc\bfseries \contentsname};%
	  \end{tikzpicture}}%
	\@starttoc{toc}}
\makeatother






%From M275 "Topology" at SJSU
\newcommand{\id}{\mathrm{id}} % Identité
\newcommand{\taking}[1]{\xrightarrow{#1}} % Flèche avec annotation
\newcommand{\inv}{^{-1}} % Inverse

%From M170 "Introduction to Graph Theory" at SJSU
\DeclareMathOperator{\diam}{diam} % Diamètre
\DeclareMathOperator{\ord}{ord} % Ordre
\newcommand{\defeq}{\overset{\mathrm{def}}{=}} % Défini comme égal

%From the USAMO .tex files
\newcommand{\ts}{\textsuperscript} % Exposant
\newcommand{\dg}{^\circ} % Degré
\newcommand{\ii}{\item} % Item

% % From Math 55 and Math 145 at Harvard
% \newenvironment{subproof}[1][Proof]{%
% \begin{proof}[#1] \renewcommand{\qedsymbol}{$\blacksquare$}}%
% {\end{proof}}

\newcommand{\liff}{\leftrightarrow} % Si et seulement si
\newcommand{\lthen}{\rightarrow} % Implique
\newcommand{\opname}{\operatorname} % Opérateur générique
\newcommand{\surjto}{\twoheadrightarrow} % Flèche surjective
\newcommand{\injto}{\hookrightarrow} % Flèche injective
\newcommand{\On}{\mathrm{On}} % Ordinaux
\DeclareMathOperator{\img}{im} % Image
\DeclareMathOperator{\Img}{Im} % Image
\DeclareMathOperator{\coker}{coker} % Cokernel
\DeclareMathOperator{\Coker}{Coker} % Cokernel
\DeclareMathOperator{\Ker}{Ker} % Noyau
\DeclareMathOperator{\rank}{rank} % Rang
\DeclareMathOperator{\Spec}{Spec} % Spectre
\DeclareMathOperator{\Tr}{Tr} % Trace
\DeclareMathOperator{\pr}{pr} % Projection
\DeclareMathOperator{\ext}{ext} % Extension
\DeclareMathOperator{\pred}{pred} % Prédécesseur
\DeclareMathOperator{\dom}{dom} % Domaine
\DeclareMathOperator{\ran}{ran} % Image (range)
\DeclareMathOperator{\Hom}{Hom} % Homomorphisme
\DeclareMathOperator{\Mor}{Mor} % Morphismes
\DeclareMathOperator{\End}{End} % Endomorphisme

\newcommand{\eps}{\epsilon} % Épsilon
\newcommand{\veps}{\varepsilon} % Variance d'épsilon
\newcommand{\ol}{\overline} % Ligne au-dessus
\newcommand{\ul}{\underline} % Ligne en-dessous
\newcommand{\wt}{\widetilde} % Tilde large
\newcommand{\wh}{\widehat} % Chapeau large
\newcommand{\vocab}[1]{\textbf{\color{blue} #1}} % Texte en gras et bleu
\providecommand{\half}{\frac{1}{2}} % Fraction 1/2
\newcommand{\dang}{\measuredangle} % Angle dirigé
\newcommand{\ray}[1]{\overrightarrow{#1}} % Ray
\newcommand{\seg}[1]{\overline{#1}} % Segment
\newcommand{\arc}[1]{\wideparen{#1}} % Arc
\DeclareMathOperator{\cis}{cis} % cis
\DeclareMathOperator*{\lcm}{lcm} % Plus petit commun multiple
\DeclareMathOperator*{\argmin}{arg min} % Argument du minimum
\DeclareMathOperator*{\argmax}{arg max} % Argument du maximum
\newcommand{\cycsum}{\sum_{\mathrm{cyc}}} % Somme cyclique
\newcommand{\symsum}{\sum_{\mathrm{sym}}} % Somme symétrique
\newcommand{\cycprod}{\prod_{\mathrm{cyc}}} % Produit cyclique
\newcommand{\symprod}{\prod_{\mathrm{sym}}} % Produit symétrique
\newcommand{\Qed}{\begin{flushright}\qed\end{flushright}} % QED aligné à droite
\newcommand{\parinn}{\setlength{\parindent}{1cm}} % Indentation de paragraphe à 1 cm
\newcommand{\parinf}{\setlength{\parindent}{0cm}} % Pas d'indentation de paragraphe
% \newcommand{\norm}{\|\cdot\|} % Norme
\newcommand{\inorm}{\norm_{\infty}} % Norme infinie
\newcommand{\opensets}{\{V_{\alpha}\}_{\alpha\in I}} % Ensemble ouvert
\newcommand{\oset}{V_{\alpha}} % Ensemble ouvert V
\newcommand{\opset}[1]{V_{\alpha_{#1}}} % Ensemble ouvert V avec indice
\newcommand{\lub}{\text{lub}} % Plus petite borne supérieure
\newcommand{\del}[2]{\frac{\partial #1}{\partial #2}} % Dérivée partielle
\newcommand{\Del}[3]{\frac{\partial^{#1} #2}{\partial^{#1} #3}} % Dérivée partielle d'ordre élevé
\newcommand{\deld}[2]{\dfrac{\partial #1}{\partial #2}} % Dérivée partielle avec dfrac
\newcommand{\Deld}[3]{\dfrac{\partial^{#1} #2}{\partial^{#1} #3}} % Dérivée partielle d'ordre élevé avec dfrac
\newcommand{\lm}{\lambda} % Lambda
\newcommand{\uin}{\mathbin{\rotatebox[origin=c]{90}{$\in$}}} % Appartient, tourné de 90 degrés
\newcommand{\usubset}{\mathbin{\rotatebox[origin=c]{90}{$\subset$}}} % Sous-ensemble, tourné de 90 degrés
\newcommand{\lt}{\left} % Gauche
\newcommand{\rt}{\right} % Droite
\newcommand{\bs}[1]{\boldsymbol{#1}} % Symbole en gras
\newcommand{\exs}{\exists} % Il existe
\newcommand{\st}{\strut} % Strut
\newcommand{\dps}[1]{\displaystyle{#1}} % Disposition en ligne

\newcommand{\sol}{\setlength{\parindent}{0cm}\textbf{\textit{Solution:}}\setlength{\parindent}{1cm} } % Solution sans indentation initiale puis rétablie
\newcommand{\solve}[1]{\setlength{\parindent}{0cm}\textbf{\textit{Solution: }}\setlength{\parindent}{1cm}#1 \Qed}

\newcommand{\entoure}[1]{\fcolorbox{black}{gray!30}{\texttt{#1}}}

\renewcommand{\ttdefault}{cmtt}
\newcommand{\textttbf}[1]{\contour{yellow!45}{\texttt{#1}}}
\newcommand{\varitem}[3][black]{%
    \item [%
        \colorbox{#2}{\textcolor{#1}{\makebox(5.5,7){#3}}}%
    ]
}
% Allow you to do the non implication (implication barred)
\newcommand{\notimplies}{%
  \mathrel{{\ooalign{\hidewidth$\not\phantom{=}$\hidewidth\cr$\implies$}}}}


\newcommand*{\authorimg}[1]%
    { \raisebox{-1\baselineskip}{\includegraphics[width=\imagesize]{#1}}}
\newlength\imagesize 

% Things Lie
\newcommand{\kb}{\mathfrak b}
\newcommand{\kg}{\mathfrak g}
\newcommand{\kh}{\mathfrak h}
\newcommand{\kn}{\mathfrak n}
\newcommand{\ku}{\mathfrak u}
\newcommand{\kz}{\mathfrak z}
\DeclareMathOperator{\Ext}{Ext} % Ext functor
\DeclareMathOperator{\Tor}{Tor} % Tor functor
\newcommand{\gl}{\opname{\mathfrak{gl}}} % frak gl group
\renewcommand{\sl}{\opname{\mathfrak{sl}}} % frak sl group chktex 6

% More script letters etc.
\newcommand{\SA}{\mathcal A}
\newcommand{\SB}{\mathcal B}
\newcommand{\SC}{\mathcal C}
\newcommand{\SF}{\mathcal F}
\newcommand{\SG}{\mathcal G}
\newcommand{\SH}{\mathcal H}
\newcommand{\OO}{\mathcal O}

\newcommand{\SCA}{\mathscr A}
\newcommand{\SCB}{\mathscr B}
\newcommand{\SCC}{\mathscr C}
\newcommand{\SCD}{\mathscr D}
\newcommand{\SCE}{\mathscr E}
\newcommand{\SCF}{\mathscr F}
\newcommand{\SCG}{\mathscr G}
\newcommand{\SCH}{\mathscr H}

% Mathfrak primes
\newcommand{\km}{\mathfrak m}
\newcommand{\kp}{\mathfrak p}
\newcommand{\kq}{\mathfrak q}

% number sets
\newcommand{\RR}[1][]{\ensuremath{\ifstrempty{#1}{\mathbb{R}}{\mathbb{R}^{#1}}}}
\newcommand{\NN}[1][]{\ensuremath{\ifstrempty{#1}{\mathbb{N}}{\mathbb{N}^{#1}}}}
\newcommand{\ZZ}[1][]{\ensuremath{\ifstrempty{#1}{\mathbb{Z}}{\mathbb{Z}^{#1}}}}
\newcommand{\QQ}[1][]{\ensuremath{\ifstrempty{#1}{\mathbb{Q}}{\mathbb{Q}^{#1}}}}
\newcommand{\CC}[1][]{\ensuremath{\ifstrempty{#1}{\mathbb{C}}{\mathbb{C}^{#1}}}}
\newcommand{\PP}[1][]{\ensuremath{\ifstrempty{#1}{\mathbb{P}}{\mathbb{P}^{#1}}}}
\newcommand{\HH}[1][]{\ensuremath{\ifstrempty{#1}{\mathbb{H}}{\mathbb{H}^{#1}}}}
\newcommand{\FF}[1][]{\ensuremath{\ifstrempty{#1}{\mathbb{F}}{\mathbb{F}^{#1}}}}
% expected value
\newcommand{\EE}{\ensuremath{\mathbb{E}}}
\newcommand{\charin}{\text{ char }}
\DeclareMathOperator{\sign}{sign}
\DeclareMathOperator{\Aut}{Aut}
\DeclareMathOperator{\Inn}{Inn}
\DeclareMathOperator{\Syl}{Syl}
\DeclareMathOperator{\Gal}{Gal}
\DeclareMathOperator{\GL}{GL} % General linear group
\DeclareMathOperator{\SL}{SL} % Special linear group

%---------------------------------------
% BlackBoard Math Fonts :-
%---------------------------------------

%Captital Letters
\newcommand{\bbA}{\mathbb{A}}	\newcommand{\bbB}{\mathbb{B}}
\newcommand{\bbC}{\mathbb{C}}	\newcommand{\bbD}{\mathbb{D}}
\newcommand{\bbE}{\mathbb{E}}	\newcommand{\bbF}{\mathbb{F}}
\newcommand{\bbG}{\mathbb{G}}	\newcommand{\bbH}{\mathbb{H}}
\newcommand{\bbI}{\mathbb{I}}	\newcommand{\bbJ}{\mathbb{J}}
\newcommand{\bbK}{\mathbb{K}}	\newcommand{\bbL}{\mathbb{L}}
\newcommand{\bbM}{\mathbb{M}}	\newcommand{\bbN}{\mathbb{N}}
\newcommand{\bbO}{\mathbb{O}}	\newcommand{\bbP}{\mathbb{P}}
\newcommand{\bbQ}{\mathbb{Q}}	\newcommand{\bbR}{\mathbb{R}}
\newcommand{\bbS}{\mathbb{S}}	\newcommand{\bbT}{\mathbb{T}}
\newcommand{\bbU}{\mathbb{U}}	\newcommand{\bbV}{\mathbb{V}}
\newcommand{\bbW}{\mathbb{W}}	\newcommand{\bbX}{\mathbb{X}}
\newcommand{\bbY}{\mathbb{Y}}	\newcommand{\bbZ}{\mathbb{Z}}

%---------------------------------------
% MathCal Fonts :-
%---------------------------------------

%Captital Letters
\newcommand{\mcA}{\mathcal{A}}	\newcommand{\mcB}{\mathcal{B}}
\newcommand{\mcC}{\mathcal{C}}	\newcommand{\mcD}{\mathcal{D}}
\newcommand{\mcE}{\mathcal{E}}	\newcommand{\mcF}{\mathcal{F}}
\newcommand{\mcG}{\mathcal{G}}	\newcommand{\mcH}{\mathcal{H}}
\newcommand{\mcI}{\mathcal{I}}	\newcommand{\mcJ}{\mathcal{J}}
\newcommand{\mcK}{\mathcal{K}}	\newcommand{\mcL}{\mathcal{L}}
\newcommand{\mcM}{\mathcal{M}}	\newcommand{\mcN}{\mathcal{N}}
\newcommand{\mcO}{\mathcal{O}}	\newcommand{\mcP}{\mathcal{P}}
\newcommand{\mcQ}{\mathcal{Q}}	\newcommand{\mcR}{\mathcal{R}}
\newcommand{\mcS}{\mathcal{S}}	\newcommand{\mcT}{\mathcal{T}}
\newcommand{\mcU}{\mathcal{U}}	\newcommand{\mcV}{\mathcal{V}}
\newcommand{\mcW}{\mathcal{W}}	\newcommand{\mcX}{\mathcal{X}}
\newcommand{\mcY}{\mathcal{Y}}	\newcommand{\mcZ}{\mathcal{Z}}


%---------------------------------------
% Bold Math Fonts :-
%---------------------------------------

%Captital Letters
\newcommand{\bmA}{\boldsymbol{A}}	\newcommand{\bmB}{\boldsymbol{B}}
\newcommand{\bmC}{\boldsymbol{C}}	\newcommand{\bmD}{\boldsymbol{D}}
\newcommand{\bmE}{\boldsymbol{E}}	\newcommand{\bmF}{\boldsymbol{F}}
\newcommand{\bmG}{\boldsymbol{G}}	\newcommand{\bmH}{\boldsymbol{H}}
\newcommand{\bmI}{\boldsymbol{I}}	\newcommand{\bmJ}{\boldsymbol{J}}
\newcommand{\bmK}{\boldsymbol{K}}	\newcommand{\bmL}{\boldsymbol{L}}
\newcommand{\bmM}{\boldsymbol{M}}	\newcommand{\bmN}{\boldsymbol{N}}
\newcommand{\bmO}{\boldsymbol{O}}	\newcommand{\bmP}{\boldsymbol{P}}
\newcommand{\bmQ}{\boldsymbol{Q}}	\newcommand{\bmR}{\boldsymbol{R}}
\newcommand{\bmS}{\boldsymbol{S}}	\newcommand{\bmT}{\boldsymbol{T}}
\newcommand{\bmU}{\boldsymbol{U}}	\newcommand{\bmV}{\boldsymbol{V}}
\newcommand{\bmW}{\boldsymbol{W}}	\newcommand{\bmX}{\boldsymbol{X}}
\newcommand{\bmY}{\boldsymbol{Y}}	\newcommand{\bmZ}{\boldsymbol{Z}}
%Small Letters
\newcommand{\bma}{\boldsymbol{a}}	\newcommand{\bmb}{\boldsymbol{b}}
\newcommand{\bmc}{\boldsymbol{c}}	\newcommand{\bmd}{\boldsymbol{d}}
\newcommand{\bme}{\boldsymbol{e}}	\newcommand{\bmf}{\boldsymbol{f}}
\newcommand{\bmg}{\boldsymbol{g}}	\newcommand{\bmh}{\boldsymbol{h}}
\newcommand{\bmi}{\boldsymbol{i}}	\newcommand{\bmj}{\boldsymbol{j}}
\newcommand{\bmk}{\boldsymbol{k}}	\newcommand{\bml}{\boldsymbol{l}}
\newcommand{\bmm}{\boldsymbol{m}}	\newcommand{\bmn}{\boldsymbol{n}}
\newcommand{\bmo}{\boldsymbol{o}}	\newcommand{\bmp}{\boldsymbol{p}}
\newcommand{\bmq}{\boldsymbol{q}}	\newcommand{\bmr}{\boldsymbol{r}}
\newcommand{\bms}{\boldsymbol{s}}	\newcommand{\bmt}{\boldsymbol{t}}
\newcommand{\bmu}{\boldsymbol{u}}	\newcommand{\bmv}{\boldsymbol{v}}
\newcommand{\bmw}{\boldsymbol{w}}	\newcommand{\bmx}{\boldsymbol{x}}
\newcommand{\bmy}{\boldsymbol{y}}	\newcommand{\bmz}{\boldsymbol{z}}

%---------------------------------------
% Scr Math Fonts :-
%---------------------------------------

\newcommand{\sA}{{\mathscr{A}}}   \newcommand{\sB}{{\mathscr{B}}}
\newcommand{\sC}{{\mathscr{C}}}   \newcommand{\sD}{{\mathscr{D}}}
\newcommand{\sE}{{\mathscr{E}}}   \newcommand{\sF}{{\mathscr{F}}}
\newcommand{\sG}{{\mathscr{G}}}   \newcommand{\sH}{{\mathscr{H}}}
\newcommand{\sI}{{\mathscr{I}}}   \newcommand{\sJ}{{\mathscr{J}}}
\newcommand{\sK}{{\mathscr{K}}}   \newcommand{\sL}{{\mathscr{L}}}
\newcommand{\sM}{{\mathscr{M}}}   \newcommand{\sN}{{\mathscr{N}}}
\newcommand{\sO}{{\mathscr{O}}}   \newcommand{\sP}{{\mathscr{P}}}
\newcommand{\sQ}{{\mathscr{Q}}}   \newcommand{\sR}{{\mathscr{R}}}
\newcommand{\sS}{{\mathscr{S}}}   \newcommand{\sT}{{\mathscr{T}}}
\newcommand{\sU}{{\mathscr{U}}}   \newcommand{\sV}{{\mathscr{V}}}
\newcommand{\sW}{{\mathscr{W}}}   \newcommand{\sX}{{\mathscr{X}}}
\newcommand{\sY}{{\mathscr{Y}}}   \newcommand{\sZ}{{\mathscr{Z}}}


%---------------------------------------
% Math Fraktur Font
%---------------------------------------

%Captital Letters
\newcommand{\mfA}{\mathfrak{A}}	\newcommand{\mfB}{\mathfrak{B}}
\newcommand{\mfC}{\mathfrak{C}}	\newcommand{\mfD}{\mathfrak{D}}
\newcommand{\mfE}{\mathfrak{E}}	\newcommand{\mfF}{\mathfrak{F}}
\newcommand{\mfG}{\mathfrak{G}}	\newcommand{\mfH}{\mathfrak{H}}
\newcommand{\mfI}{\mathfrak{I}}	\newcommand{\mfJ}{\mathfrak{J}}
\newcommand{\mfK}{\mathfrak{K}}	\newcommand{\mfL}{\mathfrak{L}}
\newcommand{\mfM}{\mathfrak{M}}	\newcommand{\mfN}{\mathfrak{N}}
\newcommand{\mfO}{\mathfrak{O}}	\newcommand{\mfP}{\mathfrak{P}}
\newcommand{\mfQ}{\mathfrak{Q}}	\newcommand{\mfR}{\mathfrak{R}}
\newcommand{\mfS}{\mathfrak{S}}	\newcommand{\mfT}{\mathfrak{T}}
\newcommand{\mfU}{\mathfrak{U}}	\newcommand{\mfV}{\mathfrak{V}}
\newcommand{\mfW}{\mathfrak{W}}	\newcommand{\mfX}{\mathfrak{X}}
\newcommand{\mfY}{\mathfrak{Y}}	\newcommand{\mfZ}{\mathfrak{Z}}
%Small Letters
\newcommand{\mfa}{\mathfrak{a}}	\newcommand{\mfb}{\mathfrak{b}}
\newcommand{\mfc}{\mathfrak{c}}	\newcommand{\mfd}{\mathfrak{d}}
\newcommand{\mfe}{\mathfrak{e}}	\newcommand{\mff}{\mathfrak{f}}
\newcommand{\mfg}{\mathfrak{g}}	\newcommand{\mfh}{\mathfrak{h}}
\newcommand{\mfi}{\mathfrak{i}}	\newcommand{\mfj}{\mathfrak{j}}
\newcommand{\mfk}{\mathfrak{k}}	\newcommand{\mfl}{\mathfrak{l}}
\newcommand{\mfm}{\mathfrak{m}}	\newcommand{\mfn}{\mathfrak{n}}
\newcommand{\mfo}{\mathfrak{o}}	\newcommand{\mfp}{\mathfrak{p}}
\newcommand{\mfq}{\mathfrak{q}}	\newcommand{\mfr}{\mathfrak{r}}
\newcommand{\mfs}{\mathfrak{s}}	\newcommand{\mft}{\mathfrak{t}}
\newcommand{\mfu}{\mathfrak{u}}	\newcommand{\mfv}{\mathfrak{v}}
\newcommand{\mfw}{\mathfrak{w}}	\newcommand{\mfx}{\mathfrak{x}}
\newcommand{\mfy}{\mathfrak{y}}	\newcommand{\mfz}{\mathfrak{z}}

% lstlistingsEnvs.tex

\usepackage{minted}


\lstset{
  basicstyle=\ttfamily, % Set
  columns=fullflexible,
  keepspaces=true,
  language=Python % You can specify the language if you want syntax highlighting
}

%%%%%%%%%%%%%%%%%%%%%%%%%%%%%%%%%%%%%%%%%%%%%%%%%%%%%%%%%%%%%%%%%%%%%%%%%%%%%%%%%%%%%%%%%%%%%%%%%
%                                 Custom lstlisting Environments
%%%%%%%%%%%%%%%%%%%%%%%%%%%%%%%%%%%%%%%%%%%%%%%%%%%%%%%%%%%%%%%%%%%%%%%%%%%%%%%%%%%%%%%%%%%%%%%%%
% Gruvbox style for Python
\definecolor{Pgruvbox-bg}{HTML}{282828}
\definecolor{Pgruvbox-fg}{HTML}{ebdbb2}
\definecolor{Pgruvbox-red}{HTML}{fb4934}
\definecolor{Pgruvbox-green}{HTML}{b8bb26}
\definecolor{Pgruvbox-yellow}{HTML}{fabd2f}
\definecolor{Pgruvbox-blue}{HTML}{83a598}
\definecolor{Pgruvbox-purple}{HTML}{d3869b}
\definecolor{Pgruvbox-aqua}{HTML}{8ec07c}
\definecolor{BBBlack}{rgb}{0.05, 0.06, 0.09}



% JAVA LSTLISTING STYLE IN Gruvbox Colorscheme
\definecolor{gruvbox-bg}{rgb}{0.282, 0.247, 0.204}
\definecolor{gruvbox-fg1}{rgb}{0.949, 0.898, 0.776}
\definecolor{gruvbox-fg2}{rgb}{0.871, 0.804, 0.671}
\definecolor{gruvbox-red}{rgb}{0.788, 0.255, 0.259}
\definecolor{gruvbox-green}{rgb}{0.518, 0.604, 0.239}
\definecolor{gruvbox-yellow}{rgb}{0.914, 0.808, 0.427}
\definecolor{gruvbox-blue}{rgb}{0.353, 0.510, 0.784}
\definecolor{gruvbox-purple}{rgb}{0.576, 0.412, 0.659}
\definecolor{gruvbox-aqua}{rgb}{0.459, 0.631, 0.737}
\definecolor{gruvbox-gray}{rgb}{0.518, 0.494, 0.471}

\definecolor{lst-bg}{RGB}{45, 45, 45}
\definecolor{lst-fg}{RGB}{220, 220, 204}
\definecolor{lst-keyword}{RGB}{215, 186, 125}
\definecolor{lst-comment}{RGB}{117, 113, 94}
\definecolor{lst-string}{RGB}{163, 190, 140}
\definecolor{lst-number}{RGB}{181, 206, 168}
\definecolor{lst-type}{RGB}{218, 142, 130}

\lstdefinestyle{PythonGruvbox}{
    language=Python,
    identifierstyle=\color{lst-fg},
    basicstyle=\ttfamily\color{Pgruvbox-fg},
    keywordstyle=\color{Pgruvbox-yellow},
    keywordstyle=[2]\color{Pgruvbox-blue},
    stringstyle=\color{Pgruvbox-green},
    commentstyle=\color{Pgruvbox-aqua},
    backgroundcolor=\color{BBBlack},
    rulecolor=\color{BBBlack},
    showstringspaces=false,
    keepspaces=true,
    captionpos=b,
    breaklines=true,
    tabsize=4,
    showspaces=false,
    numbers=left,
    numbersep=5pt,
    numberstyle=\tiny\color{gray},
    showtabs=false,
    columns=fullflexible,
    morekeywords={True,False,None},
    morekeywords=[2]{and,as,assert,break,class,continue,def,del,elif,else,except,exec,
    finally,for,from,global,if,import,in,is,lambda,nonlocal,not,or,pass,print,raise,
    return,try,while,with,yield},
    morecomment=[s]{"""}{"""},
    morecomment=[s]{'''}{'''},
    morecomment=[l]{\#},
    morestring=[b]",
    morestring=[b]',
    literate=
    {0}{{\textcolor{Pgruvbox-purple}{0}}}{1}
    {1}{{\textcolor{Pgruvbox-purple}{1}}}{1}
    {2}{{\textcolor{Pgruvbox-purple}{2}}}{1}
    {3}{{\textcolor{Pgruvbox-purple}{3}}}{1}
    {4}{{\textcolor{Pgruvbox-purple}{4}}}{1}
    {5}{{\textcolor{Pgruvbox-purple}{5}}}{1}
    {6}{{\textcolor{Pgruvbox-purple}{6}}}{1}
    {7}{{\textcolor{Pgruvbox-purple}{7}}}{1}
    {8}{{\textcolor{Pgruvbox-purple}{8}}}{1}
    {9}{{\textcolor{Pgruvbox-purple}{9}}}{1}
}

% Gruvbox style for Java
\definecolor{gruvbox-bg}{rgb}{0.282, 0.247, 0.204}
\definecolor{gruvbox-fg1}{rgb}{0.949, 0.898, 0.776}
\definecolor{gruvbox-fg2}{rgb}{0.871, 0.804, 0.671}
\definecolor{gruvbox-red}{rgb}{0.788, 0.255, 0.259}
\definecolor{gruvbox-green}{rgb}{0.518, 0.604, 0.239}
\definecolor{gruvbox-yellow}{rgb}{0.914, 0.808, 0.427}
\definecolor{gruvbox-blue}{rgb}{0.353, 0.510, 0.784}
\definecolor{gruvbox-purple}{rgb}{0.576, 0.412, 0.659}
\definecolor{gruvbox-aqua}{rgb}{0.459, 0.631, 0.737}
\definecolor{gruvbox-gray}{rgb}{0.518, 0.494, 0.471}

\lstdefinestyle{JavaGruvbox}{
    language=Java,
    basicstyle=\ttfamily\color{Pgruvbox-fg},
    keywordstyle=\color{Pgruvbox-yellow},
    keywordstyle=[2]\color{lst-type},
    commentstyle=\itshape\color{lst-comment},
    stringstyle=\color{lst-string},
    numberstyle=\color{lst-number},
    backgroundcolor=\color{BBBlack},
    rulecolor=\color{gruvbox-aqua},
    showstringspaces=false,
    keepspaces=true,
    captionpos=b,
    breaklines=true,
    tabsize=4,
    showspaces=false,
    showtabs=false,
    columns=fullflexible,
    morekeywords={var},
    morekeywords=[2]{boolean, byte, char, double, float, int, long, short, void},
    morecomment=[s]{/}{/},
    morecomment=[l]{//},
    morestring=[b]",
    morestring=[b]',
    numbers=left,
    numbersep=5pt,
    numberstyle=\tiny\color{gray},
}

% Dracula style for Java
\definecolor{draculawhite-background}{RGB}{237, 239, 252}
\definecolor{draculawhite-comment}{RGB}{98, 114, 164}
\definecolor{draculawhite-keyword}{RGB}{189, 147, 249}
\definecolor{draculawhite-string}{RGB}{152, 195, 121}
\definecolor{draculawhite-number}{RGB}{249, 189, 89}
\definecolor{draculawhite-operator}{RGB}{248, 248, 242}

\lstdefinestyle{JavaDraculaWhite}{
    language=Java,
    backgroundcolor=\color{draculawhite-background},
    commentstyle=\itshape\color{draculawhite-comment},
    keywordstyle=\color{draculawhite-keyword},
    stringstyle=\color{draculawhite-string},
    basicstyle=\ttfamily\footnotesize\color{black},
    identifierstyle=\color{black},
    keywordstyle=\color{draculawhite-keyword}\bfseries,
    morecomment=[s][\color{draculawhite-comment}]{/**}{*/},
    showstringspaces=false,
    showspaces=false,
    breaklines=true,
    %frame=single,
    rulecolor=\color{draculawhite-operator},
    tabsize=2,  
    numbers=left,
    numbersep=4pt,
    numberstyle=\ttfamily\tiny\color{gray}
}

% Dracula style for Python
\definecolor{draculawhite-bg}{HTML}{FAFAFA}
\definecolor{draculawhite-fg}{HTML}{282A36}
\definecolor{pdraculawhite-keyword}{HTML}{BD93F9}
\definecolor{pdraculawhite-comment}{HTML}{6272A4}
\definecolor{draculawhite-number}{HTML}{FF79C6}

\lstdefinestyle{PythonDraculaWhite}{
    language=Python,
    basicstyle=\ttfamily\small\color{draculawhite-fg},
    backgroundcolor=\color{draculawhite-background},
    keywordstyle=\color{orange}\bfseries,
    stringstyle=\color{draculawhite-string},
    commentstyle=\color{pdraculawhite-comment}\itshape,
    numberstyle=\color{draculawhite-number},
    showstringspaces=false,
    showspaces=false,
    breaklines=true,
    frame=single,
    rulecolor=\color{draculawhite-operator}, 
    tabsize=4,
    morekeywords={as,with,1,2,3,4, 5,6,7,8,9,True,False},
    numbers=left,
    numbersep=5pt,
    numberstyle=\small\bfseries\ttfamily\color{htmlcomment},
}

% Dracula Dark style for HTML
\definecolor{htmltag}{HTML}{ff79c6}
\definecolor{htmlattr}{HTML}{f1fa8c}
\definecolor{htmlvalue}{HTML}{bd93f9}
\definecolor{htmlcomment}{HTML}{6272a4}
\definecolor{htmltext}{HTML}{401E31}
\definecolor{htmlbackground}{HTML}{282a36}
\definecolor{comphtmlbackground}{HTML}{8093FF}

\lstdefinestyle{HTMLDraculaDark}{
    basicstyle=\normalsize\bfseries\ttfamily\color{htmltext},
    commentstyle=\itshape\color{htmlcomment},
    keywordstyle=\bfseries\color{htmltag},
    stringstyle=\color{htmlvalue},
    emph={DOCTYPE,html,head,body,div,span,a,script},
    emphstyle={\color{htmltag}\bfseries},
    sensitive=true,
    showstringspaces=false,
    backgroundcolor=\color{white},
    inputencoding=utf8,
    extendedchars=true,
    language=HTML,
    tabsize=4,
    breaklines=true,
    breakatwhitespace=true,
    numbers=left,
    numbersep=10pt,
    numberstyle=\small\bfseries\ttfamily\color{htmlcomment},
    escapeinside={<@}{@>},
    rulecolor=\color{htmlbackground},
    xleftmargin=10pt,
    frame=none, 
    breaklines=true,
    postbreak=\mbox{\textcolor{gray}{$\hookrightarrow$}\space},
    showlines=false,
    moredelim=[s][\itshape\color{htmlcomment}]{<!--}{-->},
    morekeywords={id,class,type,name,value,placeholder,checked,src,href,alt},
    literate={é}{{\'e}}1 {è}{{\`e}}1 {ê}{{\^e}}1 {ë}{{\"e}}1 {à}{{\`a}}1 {ù}{{\`u}}1 {û}{{\^u}}1 {ç}{{\c{c}}}1 {â}{{\^a}}1 {î}{{\^i}}1 {ï}{{\"i}}1
}


\lstdefinestyle{Haskell}{
  frame=none,
  xleftmargin=2pt,
  stepnumber=1,
  numbers=left,
  numbersep=5pt,
  numberstyle=\ttfamily\tiny\color[gray]{0.3},
  belowcaptionskip=\bigskipamount,
  captionpos=b,
  escapeinside={*'}{'*},
  language=haskell,
  tabsize=2,
  emphstyle={\bf},
  %commentstyle=\it,
  stringstyle=\mdseries\ttfamily,
  showspaces=false,
  keywordstyle=\bfseries\ttfamily,
  columns=flexible,
  basicstyle=\small\ttfamily,
  showstringspaces=false,
  morecomment=[l]\%,
}



\lstdefinestyle{CSSDraculaLight}{
    basicstyle=\bfseries\scriptsize\ttfamily\color{htmltext},
    commentstyle=\color{htmlcomment},
    keywordstyle=\bfseries\color{htmlvalue},
    stringstyle=\color{htmlvalue},
    emph={DOCTYPE,html,head,body,div,span,a,script},
    emphstyle={\color{htmltag}\bfseries},
    sensitive=true,
    showstringspaces=false,
    backgroundcolor=\color{white},
    inputencoding=utf8,
    extendedchars=true, % Support extended characters
    frame=none, 
    %frame=tb,
    tabsize=4,
    breaklines=true,
    breakatwhitespace=true,
    numbers=left,
    numbersep=10pt,
    numberstyle=\small\bfseries\ttfamily\color{htmlcomment},
    escapeinside={<@}{@>},
    rulecolor=\color{htmlbackground},
    xleftmargin=20pt,
    % Add a vertical line for opening and closing tags
    %frame=lines,
    framesep=2pt,
    framexleftmargin=4pt,
    % Add a vertical line for closing tags that go through multiple lines
    breaklines=true,
    postbreak=\mbox{\textcolor{gray}{$\hookrightarrow$}\space},
    showlines=true,
    % Add a rule to apply commentstyle to keywords inside comments
    moredelim=[s][\color{htmlcomment}]{/*}{*/},
    literate={é}{{\'e}}1
             {è}{{\`e}}1
             {ê}{{\^e}}1
             {ë}{{\"e}}1
             {à}{{\`a}}1
             {ù}{{\`u}}1
             {û}{{\^u}}1
             {ç}{{\c{c}}}1
             {â}{{\^a}}1
             {î}{{\^i}}1
             {ï}{{\"i}}1,
    morekeywords={color, background, background-color, font-size, font-weight, border, border-radius, padding, margin, display, position, top, right, bottom, left, flex, grid, width, height, min-width, max-width, min-height, max-height, transition, transform, animation, keyframes, content, z-index,id,class,type,name,value,placeholder,checked,src,href,alt},
    morestring=[s][\color{htmltag}]{:}{;},
}


\hypersetup{
    colorlinks=true,
    linkcolor=myb
    }









\title{\huge{MATH1400}\\\Huge{Calcul à plusieurs variables}\\\vspace{2em}Ensemble de théorèmes, lemmes et définitions }
\author{\huge{Franz Girardin}}
\date{\today}


   

\begin{document}


\begin{tikzpicture}[remember picture,overlay]

% Fill the background with BlueViolet color
\fill[myb] (current page.south west) rectangle (current page.north east);

% Hexagon pattern on the left
\foreach \i in {2.5,3,...,22} {
    \node[
        rounded corners,
        myb!90,
        draw,
        regular polygon,
        regular polygon sides=6,
        minimum size=\i cm,
        ultra thick
    ] at ($(current page.west)+(2.5,-5)$) {};
}

% Central hexagon with the year
\node[
    rounded corners,
    fill=myb!95,
    text=myb!5,
    regular polygon,
    regular polygon sides=6,
    minimum size=2.5cm,
    inner sep=0,
    ultra thick
] at ($(current page.west)+(2.5,-5)$) {\LARGE \bfseries 2024};

% Hexagon pattern at the top left
\foreach \i in {0.5,1,...,22} {
    \node[
        rounded corners,
        myb!90,
        draw,
        regular polygon,
        regular polygon sides=6,
        minimum size=\i cm,
        ultra thick
    ] at ($(current page.north west)+(2.5,0)$) {};
}

% Hexagon pattern at the top right
\foreach \i in {0.5,1,...,22} {
    \node[
        rounded corners,
        myb!98,
        draw,
        regular polygon,
        regular polygon sides=6,
        minimum size=\i cm,
        ultra thick
    ] at ($(current page.north east)+(0,-9.5)$) {};
}

% Hexagon pattern at the bottom right
\foreach \i in {12} {
    \node[
        fill=myb,
        rounded corners,
        draw=myb,
        regular polygon,
        regular polygon sides=6,
        minimum size=\i cm,
        ultra thick
    ] at ($(current page.south east)+(-0.2,-0.45)$) {};
}

\foreach \i in {21,20,...,6} {
    \node[
        myb!95,
        rounded corners,
        draw,
        regular polygon,
        regular polygon sides=6,
        minimum size=\i cm,
        ultra thick
    ] at ($(current page.south east)+(-0.2,-0.45)$) {};
}

% Title of the report
\node[
    left,
    myb!5,
    minimum width=0.625\paperwidth,
    minimum height=3cm,
    rounded corners
] at ($(current page.north east)+(0,-9.5)$) {\fontsize{25}{30}\selectfont \bfseries MATH-1400};

% Subtitle of the report
\node[
    left,
    myb!10,
    minimum width=0.625\paperwidth,
    minimum height=2cm,
    rounded corners
] at ($(current page.north east)+(0,-11)$) {\huge \textit{Calcul Mutivariables}};

% Author name
\node[
    left,
    myb!5,
    minimum width=0.625\paperwidth,
    minimum height=2cm,
    rounded corners
] at ($(current page.north east)+(0,-13)$) {\Large \textsc{Franz Girardin}};

\end{tikzpicture}
\newpage
\pdfbookmark[section]{\contentsname}{toc}
\tableofcontents
\pagebreak


\titleformat{\section}[block]{\normalsize\bfseries}{}{0pt}{}


\begin{multicols*}{3}
    \footnotesize

    \chapter{Suites numériques}
    \section{Définition d'une suite}
        \textbf{Fonction} $a \colon \mathbb{N}^* 
        \rightarrow \mathbb{R}$ qui accepte des valeurs entières 
        $n \in \mathbb{N}^*$ et engendre une \textbf{séquence ordonnée} de 
        \textbf{réels} $a_n$. L'image de $n$ est donnée par :
        \[%
           n \rightarrow a(n) = a_n 
        \]%
        \begin{center}
        \textbf{Notations équivalentes} :    
        \end{center}
        \[%
            \{a_1, a_2, a_3,\dots\} \leftrightarrow  \{ a_n \} 
                            \leftrightarrow  \{a_n\}_{ n = 1}^{\infty }
        \]%
        On dit que $a_n$ ou $\{a_n\}$ est le \textbf{terme général}; c'est la 
        règle qui définit la suite; \textit{la formule}  qui permet de calculer 
        n'importe quel terme de la suite en fonction de $n$. 

    \vspace{-1em}
    \section{Suite par récurrence}
    Est dite \textbf{récurrente} toute suite $a_n$ dont la 
    règle fait appel à des termes antérieurs, après avoir 
    établit certains termes de départ. L'\textbf{ordre de récurrence}
    dépend du nombre de termes auquel la formule du terme général 
    fait appel.

    \section{Définition d'une suite arithmétique}    
    \vspace{-1em}
    \vspace{-1em}
    \begin{align*}
        &a_n \Coloneqq 
        \begin{cases}
            a_1 = a_1 & \hspace{2.5em} \textbf{1\textsuperscript{er} Terme}\\  
            a_{n} = a_{n-1} + r & \hspace{2.5em} \text{Récurrence}
        \end{cases}
        \\\\
        &r = a_n - a_{n-1} \;\; \forall n \geq 2  
        \quad\quad\quad\quad\;\;% 
        \textbf{\hfill Raison}  
        \\
        &a_n = a_1 + \left(n - 1\right) \cdot r \quad \forall n \geq 1 
        \quad 
        \text{$n$\up{e} terme} \\
    \end{align*}

    \section{Somme $n$ termes d'une arithmétique}
    Nous pouvons \hyperlink{Somme des n premiers termes arithmétique}{\textbf{montrer}}
    que la somme des $n$ premiers termes d'une suites arithmétique
        $S_n = a_1 + a_2 + \cdots + a_n$ est donnée par :
    \[% 
        \boxed{S_n = \sum_{k=1}^{n }a_k = \dfrac{n(a_1 + a_n)}{2}}
    \]%



    \section{Définition d'une suite géométrique}
        \begin{align*}
                &a_n \Coloneqq 
                \begin{cases}
                    a_1 = a_1 &\omit\quad\quad\quad \textbf{1\textsuperscript{er} Terme} 
                \\  
                    a_{n} = a_{n-1} \cdot r &\omit\hfill \quad\quad Récurrence 
            \end{cases} 
           \\\\ &r = \frac{a_n}{a_{n-1}} \; \forall \; n \geq 2 
                 \quad\quad\quad\quad\quad\quad\;\; \textbf{Raison} 
            \\ &a_n = a_1r^{n - 1} \; \forall \; n \geq 1 
        \text{\quad\quad\quad\quad\;\; $n$\up{e} terme} 
        \end{align*} 

        \section{Somme $n$ termes d'une géométrique}
        Nous pouvons \hyperlink{Somme des n premiers termes géométrique}{\textbf{montrer}}
        que la somme des $n$ premiers termes d'une suites arithmétique
            $S_n = a_1 + a_2 + \cdots + a_n$ est donnée par :
    \[% 
        \boxed{S_n = \sum_{k=1}^{n }a_k = a_1\dfrac{r^{n} - 1}{r - 1}, 
            \;\; \textcolor{myr}{\boldsymbol{r \neq 1}}}    
        \]%

        \section{Convergence d'une suite géométrique}
        \vspace{-1em}
        \noindent       
        \begin{align*}
            \lim\limits_{n\to+\infty }a_1r^n  
                    \begin{cases}
                        \textcolor{myb}{\textbf{conv.}}  \quad \text{si $-1 < r \leq 1$} \\
                        \textcolor{myr}{\textbf{div.}}  \quad\;\; \text{autrement}
                    \end{cases}
        \end{align*}


        \begin{align*}
            \boxed{
            \lim\limits_{n\to+\infty }a_1r^n = 
                    \begin{cases}
                        0 \quad \text{si $-1 < r < 1$} \\
                        1 \quad \text{si $r = 1$}
                \end{cases}}
        \end{align*}


    \section{Convergence d'une série géométrique}
        Nous pouvons \hyperlink{Convergence série géométrique}{\textbf{montrer}}
        qu'une suite géométrique converge vers la valeur suivante :
        \begin{align*}
                \sum_{n=1}^{\infty }a_1r^n  
                    \begin{cases}
                    \textcolor{myb}{\textbf{conv.}}  \quad \text{si $|r| < 1$} \\
                    \textcolor{myr}{\textbf{div.}}  \quad\;\;\, \text{si $|r| \geq 1$}
                \end{cases}
        \end{align*}

        Si la série converge, on a :
        \begin{align*}
            \boxed{
            \sum_{n=0}^{\infty }a_1r^{n} = 
            \dfrac{a_1}{1 - r}, \; \forall r \colon |r| < 1 
        }
        \end{align*}

    \section{Monotonicité} 
    Soit une suite $a_n$, on dit que la suite est : 
    \begin{itemize} 
        \item[$\rhd$ ] \textbf{Strictement croissante} 
            \\ \textbf{si} $\forall n \geq 1, a_{n+1} \; \textcolor{myr}{\boldsymbol{>}} \; a_n$         
        \item[$\rhd$ ] \textbf{Croissante} \\ 
            \textbf{si} $\forall n \geq 1, a_{n+1}\;  \textcolor{myr}{\boldsymbol{\geq}} \; a_n$         
        \item[$\rhd$ ] \textbf{Strictement décroissante} \\ 
            \textbf{si} $\forall n \geq 1, a_{n+1} \; \textcolor{myr}{\boldsymbol{<}}  \;a_n$ 
        \item[$\rhd$ ] \textbf{Décroissante} \\ 
            \textbf{si} $\forall n \geq 1, a_{n+1} \;  \textcolor{myr}{\boldsymbol{\leq}}  \; a_n$ 
        \item[$\rhd$ ] \textbf{Stationnaire} ou \textbf{constante} \\ 
            \textbf{si} elle est telle que $a_{n+1} = a_n = a_{n-1}$
    \end{itemize}

    \section{Propriétés limites d'une suite}
    \begin{itemize}
        \item[$\rhd$] \textbf{Unicité.} La limite d'une suite convergente est \textit{unique}. Ainsi, les limite à gauche 
            et est à droite sont les mêmes. 
        
        \item[$\rhd$] Toute suite convergente est \textbf{bornée};  
            toute suite non bornée est \textbf{divergente}.        
        \item[$\rhd$] Toute suite croissante et majorée est convergente. 

            \textbf{Exemple} : \( a_n = 1 - 1/n \). 
        \[
        \lim_{n \to \infty} a_n = 1.
        \]
        
        \item[$\rhd$] Toute suite décroissante et minorée est convergente. 

            \textbf{Exemple} : \( b_n = 1/n \).
        \[
        \lim_{n \to \infty} b_n = 0.
        \]
        
        \item[$\rhd$] Toute suite monotone et bornée est convergente.
    \end{itemize}


     \section{Définitions de bornes d'une suite} 
     Une suite  est \textbf{minoriée} ou 
     \textbf{majorée} selon les conditions suivantes :

     \begin{align*}
        a_n 
        \begin{cases} 
            \textbf{Majorée} \quad \text{si } \forall n \in \mathbb{N}, a_n \leq M 
            \\ 
            \textbf{Minorée} \quad \text{si } \forall n \in \mathbb{N} a_n \geq m  
        \end{cases}
     \end{align*}

     
     On dit qu'elle est \textcolor{myr}{\textbf{bornée}} si elle est à la fois  
     majorée et minorée.
     \vspace{0.25em}

     \section{Théorème des suites monotones}
        Toute suite monotone et bornée est nécessairement \textbf{convergente}.  

    \begin{center}
        \begin{tikzpicture}[scale=0.5]
    \begin{axis}[
        domain=1:20, 
        axis lines=middle, 
        grid=none, 
        ytick={6,7,8},
        ymin=6.6, ymax=8.5, 
        xtick={2,4,6,...,20},
        width=12cm, 
        height=6cm,
    ]
        % Ajout des points de la suite
        \addplot[only marks,mark=*,myb] 
        plot[domain=1:20] (\x, {8 - (8/(\x*\x)) * (1 + sin(deg(\x*180/4)))});
        
        % Ajout de l'asymptote horizontale y = 9
        \addplot[dashed, myr] 
        coordinates {(1,8) (21,8)};
    \end{axis}
\end{tikzpicture}   
\end{center}
    

    \section{Définition formelle de convergence}
    \vspace{-1em}
    \begin{align*}
        \lim\limits_{n\to \infty}a_n = L 
    \end{align*}
    signifie qu'une suite est \textcolor{myb}{\textbf{convergente }} et tend vers la limite $L$ 
    \textbf{si et seulement si} :
    \vspace{1em}
    \[%
        \boxed{\resizebox{1\linewidth}{!}{$
        \forall\varepsilon > 0, \exists N\left( \varepsilon \right) > 0 : 
        n > N\left( \varepsilon \right) \implies |a_n -L| < \varepsilon
    $}}
    \]%

    \section{Vulgarisation}
    Soit un nombre positif $\varepsilon > 0$ \textbf{aussi petit que l'on souhaite}. 
    Il est toujours possible de trouver un entier 
    $N(\varepsilon)$ tel que, pour tous les termes de la suite dont l'indice $n$ est supérieur à 
    $N$, c'est-à-dire $n > N(\varepsilon)$, l'image $a_n$ sera suffisamment proche de $L$. 
    Autrement dit, la distance entre $a_n$ et $L$ sera inférieure à $\varepsilon$ et donc négligeable, 
    ce qui signifie que la suite se rapproche indéfiniment de la limite $L$ au fur et à mesure que $n$ augmente.
    \vspace{-2em}


    \begin{center}
        \begin{tikzpicture}[scale=.85]
    % Représentation de la suite avec l'intervalle L - epsilon à L + epsilon

    % Axe horizontal
    % \draw[->] (-1, 3.5) -- (6, 3.5) node[right] {};

    % Points a_n et labels
    % \fill[black] (0, 3.5) circle (1pt) node[below] {$a_1$};
    % \fill[black] (0.5, 3.5) circle (1pt) node[below] {$a_3$};
    % \fill[black] (1, 3.5) circle (1pt) node[below] {$a_2$};
    % \fill[black] (1.5, 3.5) circle (1pt) node[below] {$a_8$};
    % \fill[black] (2.3, 3.5) circle (1pt) node[below] {$a_N$};
    % \fill[myb] (3.5, 3.5) circle (1pt);
    % \fill[myr] (3.05, 3.5) circle (1pt) node[below] {\tiny{$a_{N+1}$}};
    % \fill[myr] (3.9, 3.5) circle (1pt) node[below] {\tiny{$a_{N+2}$}};
    % \fill[black] (4.5, 3.5) circle (1pt) node[below] {$a_9$};
    % \fill[black] (5, 3.5) circle (1pt) node[below] {$a_6$};
    % \fill[black] (5.5, 3.5) circle (1pt) node[below] {$a_7$};
    %
    % % Intervalle L - epsilon à L + epsilon
    % \draw[myr, thick] (2.75, 3.8) -- (3.75, 3.8);
    % \draw[myb, thick] (3.75, 3.8) -- (4.25, 3.8);
    %
    % % Parenthèses de l'intervalle
    % \draw[myg, thick] (2.75, 3.6) -- (2.75, 4) node[above] {$L - \varepsilon$};
    % \draw[myg, thick] (4.25, 3.6) -- (4.25, 4) node[above] {$L + \varepsilon$};
    % \node[below] at (3.5, 3.5) {\textcolor{myb}{$L$}};

            % Axes
            \draw[->] (0,0) -- (5.5,0) node[right] {$n$};
            \draw[->] (0,0) -- (0,3) node[above] {$y$};

            % Lignes horizontales pour L, L + epsilon, et L - epsilon
            \draw[myb, thin] (0,2) -- (5.5,2) node[right] {$L$};
            \draw[myyellow, thin] (0,2.5) -- (5.5,2.5) node[right] {$y = L + \varepsilon$};
            \draw[myyellow, thin] (0,1.5) -- (5.5,1.5) node[right] {$y = L - \varepsilon$};

            % Suite de points
            \foreach \x in {0.5, 0.75, 1, 1.25, 1.5, 1.75, 2, 2.25, 2.5, 2.75, 3, 3.25, 3.5, 3.75, 4, 4.25, 4.5, 4.75, 5, 5.25, 5.5} {
                \pgfmathsetmacro\y{2+sin(8 + pi +\x^2 r)/\x^2}
                \fill[myp] (\x,\y) circle (1pt);
            }

            % Labels
            \node at (-0.3,2) {$L$};
            \node at (1.4,0) {\tiny{$|$}};
            \node at (1.4,-0.3) {$N$};
            
        \end{tikzpicture}
    \end{center}                

    \section{Définition formelle de la divergence}
    \vspace{-1em}
    \begin{align*}
        \lim\limits_{n\to \infty}a_n = \pm \infty
    \end{align*}
    signifie qu'une suite est \textcolor{myr}{\textbf{divergente}} ne tend vers 
    \textbf{aucune valeur particulière} lorsque :
    \[%
        \resizebox{1\linewidth}{!}{$
        \boxed{\forall M > 0, \exists N(M) > 0 :n > N(M) \implies a_n > M
    }$}
\]%
        
    \section{Vulgarisation}
    Soit un nombre positif \textbf{aussi grand que l'on souhaite}, \(M > 0\). Il est toujours possible 
    de trouver un entier \(N(M)\) tel que, pour tous les termes de la suite dont 
    l'indice \(n\) est supérieur à \(N\), c'est-à-dire \(n > N(M)\), l'image \(a_n\) 
    sera plus grande que \(M\). Autrement dit, à partir d'un certain rang, 
    les termes de la suite deviennent arbitrairement grands, sans jamais 
    revenir vers des valeurs plus petites.

    \section{Corollaire de divergence}
    Si une série diverge, \textbf{alors} son inverse \( 1/a_n \) 
    \textbf{converge à zéro} :
    \begin{align*}
        \left[\lim\limits_{n\to+\infty }a_n  = \infty\right] \implies
     \lim\limits_{n\to+\infty }{\dfrac{1}{a_n}}  = 0
    \end{align*}
    \textcolor{myr}{\textbf{Attention}} 
        \begin{align*}
            \left[\lim\limits_{n\to\infty  }\frac{1}{a_n} = 0 \right]
            \textcolor{myr}{\boldsymbol{\notimplies}}
            \lim\limits_{n\to+\infty }a_n  = \infty
        \end{align*}  

    Mais lorsqu'une série \( 1/a_n \) converge à zéro cela 
    \textbf{n'implique pas nécessairement} que son inverse
    \( a_n \) diverge à l'infini.  

    \section{Association d'une fonction à une suite}
        Soit \( f \colon x \rightarrow \mathbb{R} \) une fonction admettant une limite $L$ à 
        $+\infty$, Alors, la suite 
        $\{a_n\}_{n=1}^{\infty} = f\left(n\right)$ admet \textbf{la même limite} : 
        \begin{align*}
            \resizebox{1\linewidth}{!}{$ 
            \left[f(n) = a_n, \lim\limits_{x\to\infty  }f(x) = L \right]   \implies \lim\limits_{n\to\infty  }a_n = L$}
        \end{align*}
        De la même façon :
        \begin{align*}
            \resizebox{1\linewidth}{!}{$ 
                \left[f(n) = a_n, \lim\limits_{x\to\infty  }f(x) = \infty \right] 
                \implies 
                \lim\limits_{n\to\infty  }a_n = \infty
        $}
        \end{align*}


        \begin{center}
        \begin{tikzpicture}[scale=0.5]
        \begin{axis}[
            domain=0:5,
            samples=100,
            axis lines=middle,
            grid=none,
            xtick={0,1,2,3,4,5},
            ytick={0,1},
            xmin=0, xmax=5,
            ymin=0, ymax=1.45,
            width=12cm,
            height=6cm,
            xlabel style={anchor=east},
            ylabel style={anchor=south},
            legend pos=north east,
            legend style={draw=none},
        ]
            % Tracé de la fonction continue y = f(x)
            \addplot[smooth, thick, color=myr] 
            plot (\x, {1 - exp(-\x) * cos(3*\x)});
            \addlegendentry{$y = f(x)$}
            \addlegendentry{$a_n = f(n)$}

            % Tracé de la suite sous forme de points en fonction de la même formule
            \addplot[only marks, mark=*, color=myr] 
            plot coordinates {
                (1, {1 - exp(-1) * cos(3*1)})
                (2, {1 - exp(-2) * cos(3*2)})
                (3, {1 - exp(-3) * cos(3*3)})
                (4, {1 - exp(-4) * cos(3*4)})
                (5, {1 - exp(-5) * cos(3*5)})
            };
            
            % Ajout de l'asymptote horizontale y = L
            \addplot[dashed, color=myb] coordinates {(0, 1) (5, 1)};
            \node at (axis cs:0.3,1.05) {\textcolor{myb}{$L$}};
            
        \end{axis}
    \end{tikzpicture}       
        \end{center}
        Par ailleurs, si $f\left(x\right)$ est \textbf{une fonction continue en}   
        $L$ et si la suite $\{a_n\}$
        converge vers $L$, alors la limite suivante \textcolor{myb}{\textbf{converge}} 
        vers $f\left(L\right)$ :
        \begin{align*}
            \boxed{
                    \lim\limits_{n\to+\infty  }f(a_n)  = 
                    f(\lim\limits_{n \to \infty} a_n)  
                    = f(L)  
            }
        \end{align*}
        \underline{\textbf{Exemple}}
        \vspace{1em}\\


Soit la fonction $f(x) = \sin(x)$, qui est continue sur $\mathbb{R}$.
Considérons la suite $\{a_n\}$ définie par :
\[
a_n = \dfrac{\pi}{2} - \dfrac{1}{n}
\]
Cette suite converge vers $L = \dfrac{\pi}{2}$ lorsque $n \to +\infty$.

Selon le théorème de continuité des fonctions en limite de suite, nous avons :
\[
\lim_{n\to+\infty} f(a_n) = f\left( \lim_{n\to+\infty} a_n \right) = f\left( \dfrac{\pi}{2} \right)
\]
Ainsi, nous avons :
\[
f\left( \dfrac{\pi}{2} \right) = \sin\left( \dfrac{\pi}{2} \right) = 1
\]
Donc :
\[
\lim_{n\to+\infty} \sin\left( \dfrac{\pi}{2} - \dfrac{1}{n} \right) = 1
\]


    \section{Théorème des gendarmes}{}
        Soient $\{a_n\}$, $\{b_n\}$ et $\{c_n\}$ des suites et $n_0 \in \mathbb{N}$ tels 
        que
        \begin{itemize}
            \item[$\blacktriangleright$ ]  $\lim\limits_{n\to+\infty }a_n  = 
                \lim\limits_{n\to+\infty }c_n  = L 
                \in \mathbb{R} \cup \{\infty \}$; 
            \item[$\blacktriangleright$ ] $\forall n \geq n_0, \; a_n \leq b_n \leq c_n$ 
        \end{itemize}
        \textbf{Alors},
        \begin{align*}
            \lim\limits_{n\to+\infty}b_n  = L                   
        \end{align*}
        \begin{tikzpicture}[scale=0.85]
        \begin{axis}[
            domain=1:20, 
            xlabel={$n$}, 
            ylabel={}, 
            axis lines=middle,
            grid=none,
            xmin=0, xmax=21,
            ymin=0, ymax=2,
            width=8cm,
            height=6cm,
            enlargelimits,
            xlabel style={anchor=west},
            ylabel style={anchor=south},
            xtick={0,5,10,15,20},
            ytick={0,0.5,1,1.5,2},
            legend style={draw=none,at={(0.5,1.15)},anchor=south},
            every axis plot/.append style={thick}
        ]
        
            % Tracé de la suite a_n en noir
            \addplot[only marks, mark=*, color=myp, mark size=.75pt] 
            plot (\x, {1+ 1/\x}); % a_n = 1/n
        \node[anchor=south west] at (axis cs:20,1.1) {\textcolor{myp}{$a_n$} };
            
            % Tracé de la suite b_n en rouge (myr)
            \addplot[only marks, mark=*, color=myr, mark size=.75pt] 
            plot (\x, {0.9 + 0.1*\x/20}); % b_n = 0.9 + (0.1*n/20)
            \node[anchor=north west] at (axis cs:20,1.08) {\textcolor{myr}{$b_n$}};
            
            \addplot[only marks, mark=*, color=black, mark size=.75pt] 
                plot (\x, {1 - 1/\x}); % Réflexion de la fonction par rapport à y
            \node[anchor=south west] at (axis cs:20,0.65) {$c_n$};

            
        \end{axis}
    \end{tikzpicture}


    \chapter{Séries numériques}

    \section{Définition d'une série numérique}
    \vspace{-2em}
        $$s = a_1 + a_2 + \cdots + a_n + \cdots = \sum_{n=1}^{\infty }a_n$$
        est une série numérique de \textbf{somme} $s$. Lorsqu'on additionne
        une quantité \textit{finie} de termes d'une série, on obtient une \textbf{somme partielle} :
        $$s_n = a_1 + a_2 + \cdots + a_n = \sum_{k=1}^{n }a_k $$ 

    \section{Propriétés des séries convergentes}
        On ne change pas \textcolor{myb}{\textbf{la nature}}  
        d'une série en lui enlevant ou en lui ajoutant un 
        nombre \textbf{fini} de termes.  
        \\\\
        Soit $k \geq 0 \in \mathbb{N}$, les sommes suivantes ont le même comportement ; l'une converge 
        \textbf{uniquement si l'autre converge également}  :
        \[%  
            \sum_{n=1}^{\infty }a_n = L \; (\textbf{\textcolor{myb}{conv.}})  
            \Leftrightarrow 
            \sum_{n=1}^{\infty }a_{n+k} \; \textbf{\textcolor{myb}{conv.}}  
        \]%
        \[%  
            \sum_{n=1}^{\infty }a_n  = \pm \infty \;  (\textbf{\textcolor{myr}{div.}})  
            \Leftrightarrow 
            \sum_{n=1}^{\infty }a_{n+k} \; \textbf{\textcolor{myr}{div.}}  
        \]%



    \section{Addition et multiplication scalaire}
    \begin{itemize}
        \item[$\rhd$] Si \(\sum_{n=1}^{\infty } a_n = s\) (\textcolor{myb}{\textbf{conv}}.) 
            et \(\lambda\) est un scalaire, 
            alors la série de terme général 
            \(\sum_{n=1}^{\infty }\lambda a_n\) converge et a pour somme :
        \[
            \boxed{ 
        \sum_{n=1}^{\infty } \lambda a_n = \lambda \sum_{n=1}^{\infty } a_n = \lambda s
        }
        \]
        
        \item[$\rhd$] Si \(\sum_{n=1}^{\infty} a_n = s\) (\textcolor{myb}{\textbf{conv}}.) et 
            \(\sum_{n=1}^{\infty} b_n = t\) (\textcolor{myb}{\textbf{conv}}.), 
            alors la série de terme général \(a_n + b_n\) 
            converge et a pour somme : 
        \[
        \sum_{n=1}^{\infty} (a_n + b_n) = \sum_{n=1}^{\infty } a_n + \sum_{n=1}^{\infty } b_n = s + t
        \]
        \textcolor{myr}{\textbf{Attention}} 
        \item[$\blacktriangleright$] 
            $\sum_{n=1}^{\infty }a_nb_n \textcolor{red}{\neq} 
            \left(\sum_{n=1}^{\infty }a_n\right) \left(\sum_{n=1}^{\infty}b_n\right)$
        \item[$\blacktriangleright$]
            $\sum_{n=1}^{\infty }\dfrac{a_n}{b_n} 
            \textcolor{red}{\neq} 
            \dfrac{\sum_{n=1}^{\infty }a_n}{\sum_{n=1}^{\infty }b_n}$
    \end{itemize}

    \section{Théorème de convergence du terme général}
        Le terme général d'une série convergente \textbf{tend vers 0} : 
        $$% 
        \boxed{\sum_{n=1}^{\infty }a_n = s \; (\textcolor{myb}{\textbf{conv.}}) 
        \;\implies \lim\limits_{n\to \infty} a_n = 0}
        $$%
        $\blacktriangleright$ \textbf{Implique} qu'il n'est pas 
        nécessaire d'évaluer la série si on sait que 
        $a_n$ \textcolor{myr}{\textbf{div.}} 

    \section{Corollaire de divergence du terme général}
    \vspace{-1em}%
        Une suite qui ne converge pas à zéro engendre une série qui \textbf{diverge}.        
        \[% 
        \boxed{
        \lim\limits_{n\to \infty} a_n \; \textcolor{myr}{\boldsymbol{\neq}} \; 0 \implies   
        \sum_{n=1}^{\infty }a_n = \infty \; (\textcolor{myr}{\textbf{div.}})}
        \]%
    \vspace{-1em}
    \begin{note}{}{}
        La \textbf{réciproque } est fausse. Par exemple la série 
        \textbf{harmonique} de terme général $a_n$  : 

        \begin{align*}
            \lim\limits_{n \to+\infty }a_n = 
            \lim\limits_{n \to+\infty } \dfrac{1}{n} 
            \longrightarrow 0
        \end{align*}
        Mais on a également  : 
        \begin{align*}
            \sum_{n=1}^{\infty }\dfrac{1}{n} \longrightarrow 
            + \infty (\textcolor{myr}{\textbf{div}}.  )
        \end{align*}
    \end{note}

    \section{Théorème du reste}

        Soient la série de terme général $a_n$, la quantité $s_n$ et le rest $R_n$, :
        \[%
        \sum_{n=1}^{\infty }a_n, \;\;
        \sum_{k=1}^{n }a_k = s_n, \; \sum_{k= n+1}^{\infty }a_k = R_n
        \]%
        On a l'implication suivante :
        \begin{align*}
            \lim\limits_{n\to\infty }R_n = 0 
            &\implies \sum_{n=1}^{\infty }a_n = s \; (\textcolor{myb}{\textbf{conv.}})  \\
            &\implies R_n = s - s_n 
        \end{align*}
        Cela signifie que, pour des valeurs suffisamment grandes de \(n\), 
        le reste \(R_n\) devient 
        arbitrairement petit et la somme totale \(s\) est bien approchée 
        par la somme partielle \(s_n\).    


\chapter{Série à termes positifs}


    \section{Définition d'une série à termes positifs}
    Il s'agit d'une série dont tous les termes $a_{n}$ sont \textbf{positifs} :
    \begin{align*}
        \left[ \forall n \geq 1 \colon a_n \geq 0 \right] \implies \sum_{n=1}^{\infty }a_n 
     \textbf{ à termes positifs}.
    \end{align*}
    
                

    \section{Théorème de convergence des séries à termes positfs}
    Une condition \textbf{nécessaire et suffisante} pour que la série à termes positifs 
        $\sum_{n=1}^{\infty }a_n$ converge est que la suite des sommes partielles
        $s_n = \sum_{k=1}^{n }a_k$ soit majorée. 
        \\\\ 

        \textbf{Sommes partielles} : La suite des sommes partielles \(s_n\) est la somme des 
        premiers termes de la série, jusqu'au \(n\)-ième terme. Si cette somme partielle 
        devient \textbf{majorée}, cela signifie qu'\textit{il existe une limite supérieure que les 
        sommes ne pourront jamais dépasser}, peu importe la valeur de \(n\). En d'autres 
        termes, les termes de la série s'accumulent, mais ils le font de manière contrôlée.
        \\\\
        \textbf{Séries à termes positifs} : Puisque les \(a_n\) sont positifs, chaque nouveau 
        terme \(a_n\) ajouté à la somme partielle rend la somme \(s_n\) de plus en plus 
        grande. Si cette accumulation ne finit jamais par dépasser un certain seuil 
        (c'est-à-dire, si elle est majorée), cela signifie que les termes \(a_n\) doivent 
        devenir de plus en plus petits et que leur contribution totale ne fait qu'approcher 
        un certain nombre, sans jamais devenir infinie. C'est le signe que la série 
        \textbf{converge}.


    \vspace{-1em}
    \section{Test de comparaison}
        Soient $\sum a_n, \sum b_n$ des séries à \textbf{termes positifs} 
        et $n_0 \in \mathbb{R}$, une série \textbf{plus petite} qu'une série convergente 
        converge nécessairement ; et une série \textbf{plus grande} qu'une série 
        divergente diverge nécessairement :
        \begin{align*}
        \resizebox{1\linewidth}{!}{$ 
         \sum_{n=1}^{\infty }b_n  \; \textcolor{myb}{\textbf{conv.}}, \; 
                   a_n \leq b_n \forall n \geq n_0\implies 
                   \sum_{n=1}^{\infty }a_n \; \textcolor{myb}{\textbf{conv.}}  
        $}
        \end{align*}
        \begin{align*}
        \resizebox{1\linewidth}{!}{$ 
         \sum_{n=1}^{\infty }b_n  \; \textcolor{myr}{\textbf{div.}}, 
                   a_n \geq b_n \forall n \geq n_0 \implies 
                   \sum_{n=1}^{\infty }a_n \;\; \textcolor{myr}{\textbf{div}}.  
        $}
        \end{align*}                
\section{Forme limite du test de comparaison}

En évaluant le rapport entre deux suites de termes positifs, on peut déterminer
si leurs séries correspondantes convergent ou divergent toutes deux.

Soient \( (a_n) \) et \( (b_n) \) deux suites de nombres réels positifs, avec
\( a_n > 0 \) et \( b_n > 0 \) pour tout \( n \) suffisamment grand. \textbf{Si}  :

\begin{align*}
\lim_{n \to+\infty} \dfrac{b_n}{a_n} = L \quad \text{avec} \quad 0 < L < \infty,
\end{align*}

\textbf{alors}  :

    \begin{align*}
    \sum_{n=1}^{\infty} a_n \; \textcolor{myb}{\textbf{conv}.}  \quad 
    \Longleftrightarrow 
    \quad% 
    \sum_{n=1}^{\infty} b_n \; 
    \textcolor{myb}{\textbf{conv}.} 
    \end{align*}
    \vspace{-1em}
    \section{Critère de Riemann}
    Soit $\sum_{n=1}^{\infty} a_n$ une \textbf{série à termes positifs}. Supposons 
    qu'il existe un réel $p > 0$ tel que 
    \[
    \lim_{n \to \infty} n^p a_n = l.
    \]
    \textbf{Dans ce cas } :

    \noindent
      $\rhd$ Si $p > 1$ et $l$ est fini, \textbf{alors} $\sum_{n=1}^{\infty} a_n$ 
    \textcolor{myb}{\textbf{conv}} 

    \vspace{0.75em}
    \noindent
      $\rhd$ Si $p \leq 1$ et $l \neq 0$ (ou $l = +\infty$), \textbf{alors }   
      $\sum_{n=1}^{\infty} a_n$ \textcolor{myr}{\textbf{div.}}
 

    \vspace{1em}
    \noindent
    \underline{\textbf{Exemple}} :

    \textbf{Analysons}  $a_n = \dfrac{n-1}{n^4+1}$. 

    \vspace{1em}
Tout d'abord, nous cherchons $p > 0$ tel que la limite
$\lim_{n \to \infty} n^p a_n$ existe et est finie.

Approximons $a_n$ pour $n$ grand :
\[
a_n = \frac{n - 1}{n^4 + 1} \approx \frac{n}{n^4} = \frac{1}{n^3}.
\]

Ainsi, pour $n$ grand :
\[
n^p a_n \approx n^p \cdot \frac{1}{n^3} = n^{p - 3}.
\]

Pour que la limite soit finie et non nulle, il faut que $p - 3 = 0$, donc $p = 3$.


Calculons la limite avec $p = 3$ :
\[
\lim_{n \to \infty} n^3 a_n = \lim_{n \to \infty}
n^3 \cdot \frac{n - 1}{n^4 + 1} = \lim_{n \to \infty}
\frac{n^3 (n - 1)}{n^4 + 1}.
\]

Simplifions le numérateur et le dénominateur :
\[
\frac{n^3 (n - 1)}{n^4 + 1} = \frac{n^4 - n^3}{n^4 + 1}.
\]

Divisons le numérateur et le dénominateur par $n^4$ :
\[
\frac{n^4 - n^3}{n^4 + 1} = \frac{1 - \frac{1}{n}}{1 + \frac{1}{n^4}}.
\]

Lorsque $n \to \infty$, $\frac{1}{n} \to 0$ et $\frac{1}{n^4} \to 0$, donc :
\[
\lim_{n \to \infty} \frac{1 - \frac{1}{n}}{1 + \frac{1}{n^4}} = 1.
\]

Ainsi :
\[
\lim_{n \to \infty} n^3 a_n = 1.
\]


Puisque $p = 3 > 1$ et $l = 1$ est fini, le critère de Riemann nous indique que
la série $\sum_{n=1}^\infty a_n$ converge.


La série $\displaystyle \sum_{n=1}^\infty \frac{n - 1}{n^4 + 1}$ converge
d’après le critère de Riemann.



    \section{Astuce pour le critère de Riemann}
    Soit $k$ et $l$, les valeurs de l'exposant de plus petit degré et du plus grand 
    degré du polynôme, il suffit de choisir $p$ tel que $p + k = l$.  
    Dans l'\textcolor{myb}{\textbf{exemple}} précédent, nous avions                 
    $k = 1$, $l = 4$ et $p = l - k  = 3$ et $n^p = n^3$.  

    \section{Série de Riemann et série puissance} 
    \vspace{-1em}
    \begin{align*}
        \sum_{n=1}^{\infty }\dfrac{1}{n^p} 
        \begin{cases} 
        \text{ converge si } \quad\quad\quad p > 1 \\
        \text{ diverge si } \quad\quad\quad\;\;\, p \leq 1
        \end{cases}
        \\
        \sum_{n=1}^{\infty} n^p 
        \begin{cases} 
        \text{ converge si } \quad\quad\quad\; p < 1 \\
        \text{ diverge si } \quad\quad\quad\;\;\;\, p \geq 1
        \end{cases}
    \end{align*}
    La première est un \textbf{série de Riemann}; la seconde 
    est une \textbf{série puissance}. 

    \section{Test de l'intégrale}
        Soit $f : [1, \infty [ \rightarrow  \mathbb{R}$ 
        \textbf{continue}, \textbf{positive} et \textbf{décroissante} 
        et $a_n : f(n) = a_n$, la somme suivante converge ou diverge avec 
        son intégrale correspondante :
            \begin{align*}
                &\sum_{n=1}^{\infty }a_n \;\; \textcolor{myb}{\textbf{conv.}} 
                \\
                & \Updownarrow 
                \\
                &\lim\limits_{a \to +\infty }\int_{x=1}^{x = a}f(x)dx = S \;\; 
                (\textcolor{myb}{\textbf{conv}}.)    
            \end{align*}
    $\blacktriangleright$ Utile lorsqu'on connait \textbf{une intégrale convergente} qui est 
    analogue à la somme qu'on veut calculer. 


    \section{Approximation de la somme}
        \textbf{Si} $f:$ \textbf{positive}, \textbf{continue} et \textbf{décroissante} 
        sur son domaine, soit un seuil $n \in \mathbb{N}^*, \text{et  soient }  
        a_n = f(n), \sum_{n=1}^{\infty }a_n = s \in \mathbb{R}, 
        R_n = s - s_n$, \textbf{alors} le reste $R_n$ est borné et peut être 
         estimé :

     \begin{align*}
        \resizebox{1\linewidth}{!}{$ 
            \boxed{
                \int_{n+1}^{\infty }f(x)dx \leq R_n 
                = 
                \sum_{k=n+1}^{\infty} a_k 
                \leq \int_{n}^{\infty }f(x)dx
            }
        $}
        \\ 
        s_n + \int_{n+1}^{\infty }f(x)dx \leq s  
        \leq \int_{n}^{\infty }f(x)dx + s_n
     \end{align*}
     Si $f:$ \textbf{positive}, \textbf{continue},  \textbf{décroissante} 
     après un certain range $N \geq 0 : [N, +\infty]$, l'inégalité tient 
     (même si $f$ n'est pas positive décroissante et continue avant $N$) 
     et on a 
    
     \begin{align*}
         \sum_{n=N}^{\infty} a_n = s \; 
         (\textbf{\textcolor{myb}{ conv}}.) 
         \Leftrightarrow    
         \int_{n}^{\infty }f(x)dx  \; \textbf{conv}.   
     \end{align*}               


    \begin{center}
    \begin{tikzpicture}[xscale=0.7, yscale=0.4]

            \node at (2, 5.5) {\textcolor{myr}{$y =f(x)$}};

        % Axes
        \draw[->] (0,0) -- (6,0) node[right] {$x$};
        \draw[->] (0,0) -- (0,5) node[above] {$y$};

        % Courbe
        \draw[thick, myr, domain=1:5.5] plot (\x, {5/\x}) node[above] {};

        % Remplissage de l'aire sous la courbe à partir de n=1
        \fill[myb!50] (2,0) -- plot[domain=2:5] (\x, {5/\x}) -- (5,0) -- cycle;

        % Ligne verticale légèrement plus foncée que myb
        \draw[very thick, myb!70] (2,0) -- (2,2.5);

        % Point de départ de l'aire
        \node[below] at (2,0) {$n+1$};
        \node[] at (3,3.75) {\textcolor{myb}{$\int_{n + 1 }^{\infty }f(x)dx$}};


    \end{tikzpicture}
    \end{center}

        \begin{center}
        \begin{tikzpicture}[xscale=0.7, yscale=0.4]
            % Axes
            \draw[->] (0,0) -- (6,0) node[right] {$x$};
            \draw[->] (0,0) -- (0,5) node[above] {$y$};

            % Blocs avec bordures noires
            \filldraw[fill=orange!30, draw=black] (1,0) rectangle (2,5/2);
            \filldraw[fill=orange!30, draw=black] (2,0) rectangle (3,5/3);
            \filldraw[fill=orange!30, draw=black] (3,0) rectangle (4,5/4);
            \filldraw[fill=orange!30, draw=black] (4,0) rectangle (5,5/5);

            % Courbe par-dessus

            \draw[] (1,0) -- (1,5);
            \draw[thick, myr, domain=0.9:5.5] plot (\x, {5/\x}) node[above] {};

            \node at (2, 5.5) {\textcolor{myr}{$y =f(x)$}};

            \node at (3, 3.75) {\textcolor{orange!70}{$R_n = \sum_{k = n+1}^{\infty }a_k$}};
            % Étiquettes
            \node at (1.5, 2.) {$a_{n+1}$};
            \node at (2.5, 1.25) {$a_{n+2}$};
            \node at (3.5, 0.75) {$a_{n+3}$};
            \node at (4.5, 0.5) {$a_{n+4}$};

            % Points de référence
            \node[below] at (1,0) {$n$};
        \end{tikzpicture}
        \end{center}


    \begin{center}
    \begin{tikzpicture}[xscale=0.7, yscale=0.4]

        % Axes
        \draw[->] (0,0) -- (6,0) node[right] {$x$};
        \draw[->] (0,0) -- (0,5) node[above] {$y$};

        % Courbe
        \draw[thick, myr, domain=1:5.5] plot (\x, {5/\x}) node[above] {};
            \node at (2, 5.5) {\textcolor{myr}{$y =f(x)$}};

        % Remplissage de l'aire sous la courbe à partir de n=1
        \fill[myb!50] (1,0) -- plot[domain=1:5] (\x, {5/\x}) -- (5,0) -- cycle;

        % Ligne verticale légèrement plus foncée que myb
        \draw[very thick, myb!70] (1,0) -- (1,5);

        % Point de départ de l'aire
        \node[below] at (1,0) {$n$};
        \node[below] at (3,3.95) {\textcolor{myb}{$\int_{n}^{\infty }f(x)dx$}};


    \end{tikzpicture}
    \end{center}


     \chapter{Séries alternées}

     \section{Définition d'une série alternée}
     \vspace{-2em}
     \begin{align*}
         s=a_1-a_2+\cdots&+(-1)^{n-1}a_n +\cdots 
         \\
         &=\sum_{n=1}^\infty(-1)^{n-1}a_n
         \\
         s=-a_1+a_2 +\cdots&+(-1)^na_n+\cdots 
         \\
         &=\sum_{n=1}^\infty(-1)^na_n                
     \end{align*}               




     \section{Test sur séries alternées}
     Soit un \textbf{rang} $N \in \mathbb{N}$ et 
     soit une \textbf{série alternée} 
     $\sum_{n=1}^{\infty } (-1)^nb_n$ telle que :
     \begin{itemize}
       \item [$\rhd$ ]  $b_n$ \textbf{décroissante}  et \textbf{positive} \\
           $\Bigl(f(n) = b_n, f^{\prime}(x) \leq 0, \; \forall \;\; x > N\Bigr)$
       \item [$\rhd$ ] $\lim\limits_{n\to\infty }b_n = 0$ 
     \end{itemize}
     
     \textbf{Alors}, 

    \noindent
    $\blacktriangleright$
    $\sum_{n=1}^{\infty }  (-1)n \; \textcolor{myb}{\textbf{conv}}$ vers $s \in
    \mathbb{R} \;\; \forall m \geq N $ 

     \begin{itemize}
    \item[$\blacktriangleright$]
    $0 \leq s \leq b_1$ 
        \item[$\blacktriangleright$] 
    $R_n = |s - s_n | \leq b_{n+1}$
     \end{itemize}
\vspace{-3.5em}
\chapter{Convergence absolue}
    \section{Définition de convergence absolue}
    \vspace{-2em}
     \[% 
        \sum_{n=1}^{\infty }|a_n| \; \textcolor{myb}{\textbf{conv}}.  \implies 
        \sum_{n=1}^{\infty }a_n \; \textcolor{myb}{\textbf{conv}.} 
        \textcolor{myb}{\textbf{ absolument}}  
     \]%
    \\
    \resizebox{1\linewidth}{!}{
    $\rhd$ 
    \textbf{Semi-conv}. : 
    $\sum_{n=1}^{\infty }a_n$ \; \textcolor{myb}{\textbf{conv}}.  \; 
    \textbf{et} $\sum_{n=1}^{\infty }|a_n| \; \textcolor{myr}{\textbf{div.}}$} 
     
    \vspace{1em}
    \noindent 
    \underline{\textbf{Exemple}} 
    \vspace{1em}\\
    La \textbf{série harmonique} \textcolor{myb}{\textbf{alternée}} converge 
    vers $\ln(2)$, mais la \textbf{série harmonique} 
    \textcolor{myr}{\textbf{simple}} est \textcolor{myr}{\textbf{divergente}} :  


    \noindent 
    \begin{align*}
    \resizebox{1\linewidth}{!}{$
    \sum_{n=1}^{\infty }\dfrac{(-1)^n}{n} \; 
    \textcolor{myb}{\textbf{conv}}. \;\;  \textbf{mais} 
          \sum_{n=1}^{\infty }\left|\dfrac{(-1)^n}{n} \right| = 
          \sum_{n=1}^{\infty }\dfrac{1}{n} \;
          \textcolor{myr}{\textbf{div.}}  
    $}          
    \end{align*}
    \section{Théorème de convergence absolue}
    \textbf{Si} une série $\sum_{n=1}^{\infty }a_n$ converge \textit{absolument}, 
       \textbf{alors} elle converge simplement.  


    \section{Test du rapport (d'Alembert)}
    Soit $\lim\limits_{n \to+\infty } \Big|\frac{a_{n+1}}{a_n}  \Big| = L$ 
    \textbf{alors} si :   
    \begin{itemize}
    \item [$\rhd$ ] $L = 1 \implies$ \textit{ inconclusif}  
    \item [$\blacktriangleright$ ] $L > 1 \implies \sum_{n=1}^{\infty } a_n$ 
      \textcolor{myr}{\textbf{div}} .   
    \item [$\blacktriangleright$ ] $L < 1 \implies \sum_{n=1}^{\infty } a_n$ 
    \textcolor{myb}{\textbf{conv}}. 
    \end{itemize}


    \section{Test de Cauchy}
    \mbox{}\vspace{0.2em}
    Soit $\lim\limits_{n \to+\infty } \sqrt[n]{\big| a_n \big|} = L$ 
    \textbf{alors} si :   
    \begin{itemize}
    \item [$\rhd$ ] $L = 1 \implies$ \textit{inconclusif}  
    \item [$\blacktriangleright$ ] $L > 1 \implies \sum_{n=1}^{\infty }a_n$ 
        \textcolor{myr}{\textbf{div.}} 
    \item [$\blacktriangleright$ ] $L < 1 \implies \sum_{n=1}^{\infty }a_n$ 
        \textcolor{myb}{\textbf{conv.}} 
    \end{itemize}


                                                



     





    

       








        





  





 
    \chapter{Astuces pour Stewart 1.1 - 1.5}
    \section{Propriétés des limites}
        Si $\{a_n\}$ et $\{b_n\}$ sont des suites convergentes et 
        si $c$ est une constante, \textbf{alors} \\\\ 
        $\lim\limits_{n\to\infty  }\left(a_n \text{+} b_n \right) = 
        \lim\limits_{n\to\infty  }a_n \text{+} 
        \lim\limits_{n\to\infty  }b_n$
        \\\\
        $\lim\limits_{n\to\infty  }\left(a_n - b_n \right) = 
        \lim\limits_{n\to\infty  }a_n - \lim\limits_{n\to\infty  }b_n$ 
        \\\\
        $\lim\limits_{n\to\infty  }ca_n = c \lim\limits_{n \to \infty  }a_n$ 
        \\\\
        $\lim\limits_{n\to\infty  }\left(a_nb_n \right) = 
        \lim\limits_{n\to\infty  }a_n \cdot \lim\limits_{n\to\infty  }b_n$
        \\\\
        $\lim\limits_{n\to\infty  }\left(\frac{a_n}{b_n} \right) = 
        \frac{\lim\limits_{n\to\infty  }a_n}{\lim\limits_{n\to\infty  }b_n}
        \;
        \text{si} \lim\limits_{x\to\infty  }b_n \neq 0$
        \\\\
        $\lim\limits_{n\to\infty  }a_n^{p} = 
        \left[\lim\limits_{n\to\infty  }a_n \right]^p \text{si } 
        p > 0 \; \text{et} \; a_n > 0$



    \section{Limite d'une suite polynomiale} 
        Soit deux polynomes,
        $\lim\limits_{n\to \infty } \dfrac{p(n)}{q(n)}$, 
        et 
        $k = \min\bigl(\deg(p), \deg(q)\bigr)$
        \textbf{Alors},   
        \[ \lim\limits_{n\to \infty } \dfrac{p(n)}{q(n)} =
        \lim\limits_{n\to+\infty}\dfrac{p(n)/{n^k}}{q(n)/n^{k}} \]



    \section{Règle de l'Hôpital}
        Soit une \textbf{constante} $c \in \mathbb{R} \cup \{+\infty\}$ et 
        supposon que : 

        \noindent 
        $\rhd $ $\lim\limits_{x\to c}\dfrac{|f(x)|}{|g(x)|}$ 
            est de la forme $\dfrac{0}{0}$ \; ou \; 
            $\dfrac{\infty }{\infty }$
        

        \noindent
        $\rhd$ $\lim\limits_{x\to c}\dfrac{f^{\prime}(x)}{|g^{\prime}(x)|}$
            \textbf{existe} et 
            $\textcolor{myr}{\boldsymbol{g^{\prime}(x) \neq 0 \;\; \forall x \approx c}}$
        \textbf{Alors}, 
        \begin{align*}
            \boxed{ 
            \lim\limits_{x\to c}\dfrac{f(x)}{g(x)} = 
            \lim\limits_{x\to c}\dfrac{f^{\prime}(x)}{g^{\prime}(x)}
            }
        \end{align*}

                        

    \section{Comparaison des suites}
        Si $a > 1$ et $k > 0$, on a 
        \begin{align*}
            \ln(n) \ll n^K \ll a^n \ll n! \ll n^n
        \end{align*}
        $c_n \ll d_n \implies 
        \lim\limits_{n\to+\infty }\dfrac{c_n}{d_n} = 0$ 



    \chapter{Les séries de fonctions}

    \section{Définition d'une série de fonction}
        Soit $u_n(x)$ une suite de fonction définie sur $E \subset \mathbb{R}$ 
        \textbf{Alors}, la série  
    

        \begin{align*}
        \resizebox{1\linewidth}{!}{$ 
            \sum_{n=1}^{\infty }u_n(x) = u_1(x) + u_2(x) + \cdots + u_n(x) + \cdots
        $}
        \end{align*}                    

    est une série de fonction de \textbf{terme général}  
    $u_n(x)$ et on a la somme partielle de range $n$ correspondante : 


    \begin{align*}
    \resizebox{1\linewidth}{!}{$ 
        \sum\limits_{k=1}^{n}u_k(x) =  
        u_1(x) + u_2(x) + \cdots + u_n(x)
    $}
    \end{align*}
    
    \section{Définition du domaine de convergence}
    L'ensemble des $x \in E$ qui sont tels que la suite $s_n(x)$ 
    convergent porte le nom de  
    \textbf{\textcolor{myb}{domaine de convergence}} $D \subset E$.    

    La somme $s_n(x)$ est donc la somme qu'on obtient lorsqu'on 
    ajoute une infinité de termes $u_n(x)$ :
 
    \begin{align*}
    \boxed{
        \sum\limits_{n=1}^{\infty }u_n(x) = 
        \lim\limits_{n \to+\infty } 
        \sum\limits_{k=1}^{n}u_k(x) =  
        \lim\limits_{n \to +\infty } s_n(x) 
    }
    \end{align*}

    \vspace{-2em}
    \section{Définition d'une série entière}

    Soit des coefficients $c_n$, 
    une \textbf{série entière} est une \textit{série de fonction}     
    de la forme 
    \begin{align*}
    \resizebox{1\linewidth}{!}{$ 
    \boxed{
        c_0 + c_1x + c_2x^2 + \cdots + c_nx^n + \cdots = 
        \sum\limits_{n=1}^{\infty }  c_nx^n  
    }
$}
    \end{align*} 

    Lorsqu'elle est \textbf{centrée en} $a$, une série entière prend la forme : 

    \begin{align*}
        \resizebox{\linewidth}{!}{$
         c_0 + c_1(x - a) + \cdots + c_n(x - a)^n + \cdots = 
        \sum\limits_{k=1}^{\infty }  c_n(x - a)^n  
    $}
    \end{align*}    


    \vspace{-1.5em}
    \section{Convergence de série entière}
    \vspace{-0.5em}
        Considérons la série entière 
        \begin{align*}
           S = \sum_{n=0}^{\infty }c_n(x - a)^n 
        \end{align*}
        \textbf{Alors}, on a l'une des trois possibilités : 

            \noindent$\rhd$
                $R = 0 \;\;\; \implies S$ \; 
                \textcolor{myb}{\textbf{conv.}}  à $x = a$ 
            \\
            $\rhd$  $R = +\infty \implies S \;$ 
            \textcolor{myb}{\textbf{conv.}} 
            \; $\forall x \in \mathbb{R}$  
            \\ \\
            $\blacktriangleright$ 
            $\exists R > 0 : |x - a| < R \implies  S \; \textbf{\textcolor{myb}{conv.}}$

            \noindent 
            $\blacktriangleright$ $\exists R > 0 : |x - a| > R \; \implies  S \;
            \textbf{\textcolor{myr}{ div.}}$   
            \\\\ 
            Mais pour les points $x = a$ et $x = -a$, il faut déterminer  
            la convergence cas par cas. 


    \vspace{-1em}
    \section{Dérivation et intégration de série entière}
    \vspace{-2em}
        Considérons la \textbf{série entière} 
        \begin{align*}
            \sum\limits_{n = 0}^{\infty}c_n(x - a)^n \text{ avec } R > 0
        \end{align*}                
        \textbf{Alors}, la fonction $f(x) =  \sum_{n=0}^{\infty }c_n(x - a)^n$ 
        est \textbf{\textcolor{myb}{dérivable}} 
        sur l'intervalle ouvert $]a, -R, a + R[$ et on peut 
        \textcolor{myb}{\textbf{intégrer}}    \textit{termes à termes} :  
        \begin{align*}
            f^{\prime}(x) = \sum_{n=0}^{\infty }nc_n(x - a)^{n + 1}
        \end{align*}
        \vspace{-1em} 
        \begin{align*}
            \int f(x)dx  = C + \sum_{n=0}^{\infty }c_n
            \dfrac{(x - a)^{n+1}}{n+1} 
        \end{align*}    



    \section{Théorème des série de fonctions}
        Si une fonction \( f(x) \) peut se mettre sous la forme d'une série entière 
        au voisinage de \( a \) :
        \begin{align*}
        \resizebox{\linewidth}{!}{$
            c_0 + c1(x - a) +  \cdots + c_n (x - a)^n + \cdots (*)
        $}
        \end{align*}
        \textbf{Alors}, le \textcolor{myb}{\textbf{coefficient }} 
        \( c_n \) est donné par : 
        \begin{align*}
            c_n = \dfrac{1}{n!}f^{n}(a) 
        \end{align*}
        et la fonction \( f(x) \) peut s'écrire sous la forme d'une 
        \textcolor{myb}{\textbf{série de Taylor }}: 
        \begin{align*}
            \resizebox{\linewidth}{!}{ $
            f(x) = \sum_{n= 0}^{\infty }\dfrac{1}{n!}f^{n}(a)(x - a)^n, \; |x - a| < R 
        $}
        \end{align*}

    \section{Polynôme de Taylor}
    Si \( f(x) \) peut se mettre sous la forme d'une \textbf{série entière} au voisinage de 
    \( a \) et que \( f^{n}(x) \) existe au voisinage de \( a  \; \forall n \geq 0\) on 
    a la somme partielle \( T_n(x) \) portant le nom de polynôme de Taylor : 
    \begin{align*}
        \resizebox{\linewidth}{!}{$
            f(x) = \sum_{k=0}^{\infty }\dfrac{1}{k!}f^{k}(a)(x - a)^k, \; |x - a| < R
        $}
    \end{align*}            

    $$\int_{n=1}^{\infty }f(x)dx$$

     \section{Théorème de l'inégalité de Taylor}

     S'il existe une quantité $M$ telle que 
     $\left| f^{n+1}(x) \leq M(x) \right|$ pour 
     $|x - a| < R$, c.-à-d. \textbf{si}  
     $f^{n+1}(x)$ est bornée sur l'intervalle 
     $]a - R, a + R[$, \textbf{alors}, on a : 

     \begin{align*}
         \resizebox{1\linewidth}{!}{$ 
         |R_n(x)| \leq \dfrac{M(x)}{(n + 1)}|x - a|^{n+1}, 
         \forall \; x \in ]a - R, a + R
    $}
     \end{align*}               

     \chapter{Fonctions plusieurs variables}


     \section{Kamel p.4}
     $\mathbb{R}^2$ est équivalent à l'ensemble des réels qui comprends 
     les couples $(x, y)$


     Les équations de la forme : 
     \begin{align*}
            x^2 + y^2 = z^2
     \end{align*}

     Sont représenté graphiquement par un cercle de rayon $z$.  
     Lorsqu'on cherche le \textbf{domaine}, il est possible de  
     simplifier.


     \begin{EExample}{}{}
         $\sqrt{x^2 + y^2 - 9}$ a un domaine dans $\mathbb{R}^2$ pour 
         toutes valeur à l'extérieur du cercle de rayon $r = 3$ 
     \end{EExample}


    %
    % \begin{theorem}{Dérivation et intégration des séries entières}{}
    %    Considérons la série entière 
    %    \[ S = \sum_{n=0}^{\infty }c_n(x - a)^2 \] 
    %    avec $R > 0$. \textbf{Alors}, la fonction $f(x) = S$ est 
    %    \textcolor{myb}{\textbf{dérivable}} sur l'intervalle ouvert 
    %    $]a -R, a + R[$ et on peut dériver et intégrer terme à terme :
    %
    %    \begin{align*}
    %        f^{\prime}(x) = \sum_{n=0}^{\infty }nc_n(x - a)^{n - 1} \\ 
    %    \end{align*}
    %     \vspace{-3em}
    %    \begin{align*}
    %        \int f(x)dx = C + \sum_{n=0}^{\infty }c_n 
    %        \dfrac{(x - a)^{n+ 1}}{n + 1} 
    %    \end{align*}
    %
    %
    % \end{theorem}       
        





    % \section{Critère de divergence}
    %     Si la série converge, la suite correspondante 
    %     \textbf{converge vers 0},
    %     et si la suite ne converge pas vers zéro, la série 
    %         est divergente
    %     \begin{itemize} 
    %         \item $\sum_ {n}^{ \infty }a_n = s \; (\textbf{conv.}) 
    %             \implies 
    %         \lim\limits_{n\to+\infty }a_n  = 0$
    %         \item
    %         $\lim\limits_{n\to+\infty }a_n  \neq 0  
    %         \implies 
    %         \sum_{ n}^{\infty }a_n$ \textbf{ div.}  
    %     \end{itemize}


  \end{multicols*}


\chapter{Preuve de Théorèmes et formules}

\section{Terme général d'une suite \textcolor{myb}{\textbf{arithmétique}}}
    \begin{Preuve}{Terme général $a_n$ d'une suite arithmétique}{}
       Montrons que le terme général $a_n$d'une suite arithmétique  
       $a_{n = 1}^{\infty}$ est donné par 
       \begin{align*}
            a_n = a_1 + (n - 1)r
       \end{align*}

       Nous savons que les $n$ premiers termes de la suite sont donné par : 
       \begin{align*}
           a_n = \quad
           \underbrace{a_1}_{n = 1; a_1}, \;\;
           \underbrace{(a_1 + r)}_{n = 1, a_1 + r}, \;\;
           \underbrace{\Bigl((a_1 + r) +r \Bigr)}_{n = 3; a_1 + 2r}, \;\;
           \underbrace{\Bigl(\Bigl( (a_1 + r) + r \Bigr) +r \Bigr)}_{n = 4, a_1 + 3r}, \;\;
           \underbrace{\Bigl(a_1 + rn \Bigr)}_{n = n; a_1 + rn}
       \end{align*}

       \begin{align*}
        a_n = \quad 
        a_1, \;l\; a_1 + (2 -1)r, \;\; + a_1 + (3 - 1)r,\;\;  \cdots \;\; a_1 + (n -1)r
       \end{align*} 


       On constate pour tout $n \geq 2$, chaque terme de la suite prend la forme 
       $a_1 + (n -1)r$. Nous avons donc : 

       \begin{align*}
            \boxed{ 
                a_n = a_1 + (n - 1)r, \;\; \forall n \geq 2
            }   
       \end{align*}         
       \qed
    \end{Preuve}


    \section{Somme des $n$ premiers terme d'une suite \textcolor{myb}{\textbf{arithmétique}}}
    \begin{Preuve}{\hypertarget{Somme des n premiers termes arithmétique}{Somme partielle d'une suite arithmétique}}{}
        Montrons que la formule générale pour la somme $s_n$ des $n$ premiers 
        termes d'une \textbf{suite arithmétique} est donnée par : 

        \begin{align*}
            S_n = \dfrac{n}{2}(a_1 + a_n) 
        \end{align*}

        Nous savons déjà que le $n$-ième terme de la suite est : 

        \begin{align*}
            a_n = a_1 + (n - 1)r
        \end{align*}                    

        Nous pouvons développer la série des $n$ premiers termes comme suit : 
        \begin{align*}
        \sum\limits_{k=1}^{n }a_k = S_n = a_1 + a_2 + \cdots a_n 
        \end{align*}
        En \textbf{multipliant} la somme partielle par $2$, nous obtenons et 
        et en additionnant termes à termes.
        \begin{align*}
            &2S_n =
            \bigl(a_1 + a_2 + \cdots  + a_n \bigr) +
            \bigl( a_n + a_{n -1} + \cdots  + a_1 \bigr)
            \\
            &2S_n = (a_1 + a_n) + (a_2 + a_{n - 1}) + (a_3 + a_{n - 2}) + \cdots 
            + (a_n + a_1)
        \end{align*}

        On constate alors que chaque terme entre parenthèse de l'équation  
        se simplifie à $2a_n + (n - 1)r$. \textbf{Par exemple}, 
        \begin{align*}
            (a_1 + a_n) = a_1 + \Bigl(a_1 + (n - 1)r \Bigr) = 
            a_1 + a_1 + (n -1)r = \textcolor{myb}{\textbf{$2a_1 + (n -1)r$}}  
        \end{align*}

        \textbf{Similairement}, on a :
        \begin{align*}
            (a_2 + a_{n-1}) = a_1 + (2 - 1)r + \Bigl(a_1 + (n - 1 - 1)r \Bigr) = 
            a_1 +r + a_1 +rn -2r = 2a_1 + -r +rn =
            \textcolor{myb}{\textbf{$ 2a_1 + (n - 1)r$}} 
        \end{align*}

        \textbf{Par ailleurs}, nous savons que $2a_1 + (n - 1)r$ est simplement 
        équivalent à $a_1 + a_n$ : 

        \begin{align*}
            2a_1 + (n -1)r = a_1 + \Bigl( a_1 (n - 1)r \Bigr) = a_1 + a_n
        \end{align*}

        \textbf{Ainsi}, puisque nous chaque terme entre parenthèse se simplifie 
        à $2a_n + (n -1)r$ ou $a_1 + a_n$ 
        et qu'il y a $n$ termes entre parenthèse, nous avons 

        \begin{align*}
            2S_n = n\Bigl(2a_n + (n -1)r \Bigr) \\ 
            2S_n = n(a_1 + a_n) \\\\ 
            \boxed{
            S_n = \dfrac{n}{2} (a_1 + a_n)}
        \end{align*}
        \qed
    \end{Preuve}    

    \section{Terme général d'une suite \textcolor{myb}{\textbf{géométrique}}}
    \begin{Preuve}{Terme général $a_n$ d'une suite géométrique}{}
        Montrons que le $n$-ième terme $a_n$ d'une suite géométrique 
        $a_{n = 1}^{\infty}$ est donné par la formule : 
        \begin{align*}
            a_n = a_1r^{n-1}
        \end{align*}

        Nous savons, par la définition d'une suite géométrique, que pour tout terme 
        $n \geq 2$, le terme $n$ est obtenu en multipliant son prédécesseur par la 
        raison $r$. Une suite géométrique a donc les termes suivants : 
        
        \begin{align*}
        a_n = \quad 
        \underbrace{a_1}_{n = 1}, 
        \underbrace{\;\; a_1 \cdot r}_{n = 2}, \;\; 
        \underbrace{a_1 \cdot r \cdot r}_{n = 3}, \;\; 
        \cdots, \;\; 
        \underbrace{a_1 \cdot r^{n-1}}_{n = n}
        \end{align*}

        Ainsi, pour chaque terme $n \geq 2$, l'image $a_n$ est donnée par 
        $a_1 \cdot r^{n -1}$ : 

        \begin{align*}
            \boxed{
            a_n = a_1 \cdot r^{n - 1}, \;\; \forall n \geq 2
        }
        \end{align*} 
        \qed
    \end{Preuve}


    \section{Somme des $n$ premiers termes d'une suite \textcolor{myb}{\textbf{géométrique}} }
    \begin{Preuve}{\hypertarget{Somme des n premiers termes géométrique}{Somme partielle d'une suite géométrique}}{}
            Montrons que la somme des $n$ premiers termes d'une suite géométrique est 
            donnée par :

            \[
            S_n = a_0 + a_0r + \dots + a_n  =  \sum\limits_{k=0}^{n} a_0r^k = 
            \dfrac{r^{n+1} - 1}{r - 1}
            \]

            \textbf{Étape 1 : Écriture générale de la somme} \\
            La somme des $n$ premiers termes de la suite géométrique est :

            \[
            S_n = a_0 + a_0 r + a_0 r^2 + \dots + a_0 r^n
            \]

            Factorisons par $a_0$ :

            \[
            S_n = a_0 \left( 1 + r + r^2 + \dots + r^n \right)
            \]

            Nous devons maintenant trouver la formule de la somme des puissances de $r$.

            \textbf{Étape 2 : Multiplier par $r$} \\
            Multiplions la somme par $r$ :

            \[
            r \cdot \left( 1 + r + r^2 + \dots + r^n \right) = r + r^2 + \dots + r^{n+1}
            \]

            \textbf{Étape 3 : Soustraction des deux équations} \\
            Soustrayons la somme obtenue de la somme initiale :

            \[
            \left( 1 + r + r^2 + \dots + r^n \right) - \left( r + r^2 + \dots + r^{n+1} 
            \right) = 1 - r^{n+1}
            \]

            Simplification :

            \[
            \left( 1 + r + r^2 + \dots + r^n \right) (1 - r) = 1 - r^{n+1}
            \]

            \textbf{Étape 4 : Isoler la somme} \\
            Isolons la somme :

            \[
            1 + r + r^2 + \dots + r^n = \dfrac{1 - r^{n+1}}{1 - r}, \quad \text{pour } 
            r \neq 1
            \]

            \textbf{Étape 5 : Formule finale} \\
            Remplaçons cette expression dans la somme $S_n$ :

            \[
            S_n = a_0 \cdot \dfrac{1 - r^{n+1}}{1 - r}
            \]

            Ce qui nous donne la formule finale :

            \[
                \boxed{S_n = a_0 \dfrac{1 - r^{n+1}}{1 -r}}
            \]
            \qed
        \end{Preuve}


    \section{Convergence d'une suite géométrique}

    \begin{Preuve}{}{}
        Montrons qu'une suite géométrique converge vers $0$ lorsque 
        $-1 < r < 1$ et vers $1$ si $r = 1$. Montrons aussi qu'elle diverge pour tout autre valeur. 
        \vspace{1em}\\ 
        \textbf{Cas 1 :} Si $-1 < r < 1$, alors $|r| < 1$. Ainsi,
        \[
            \lim\limits_{n\to+\infty} r^n = \lim\limits_{n \to+\infty } 
            \left[ \left(\dfrac{p}{q}\right)^r, p < q \right] \longrightarrow 0
        \]
        car les puissances successives de $r$ tendent vers 0. 
        Cela est dû au fait que tout rationnel $r \in \mathbb{R}$
        peut s'écrire sous la forme d'un quotien de naturels 
        $p, q \in \mathbb{N}$ et que le dénominateur, 
        étant plus grand que le numérateur, 
        \textbf{croît plus rapidement que le numérateur}. Ainsi, le rapport 
        $p^n/q^n$ tend vers 0 lorsque $n$ tend vers l'infini. 

        \textbf{Cas 2 :} Si $r = 1$, alors pour tout $n$,
        \[
        a_n = 1^n = 1
        \]
        donc,
        \[
        \lim\limits_{n\to+\infty} a_n = 1
        \]

        \textbf{Cas 3 :} Si $r = -1$, alors
        \[
        a_n = (-1)^n
        \]
        La suite alterne entre $-1$ et $1$ et n'a pas de limite. Donc, elle ne converge pas.

        \textbf{Cas 4 :} Si $|r| > 1$, alors $|r^n|$ tend vers l'infini. Donc,
        \[
        \lim\limits_{n\to+\infty} r^n \text{ n'existe pas (la suite diverge).}
        \]

        \textbf{Conclusion :} La suite converge si et seulement si $-1 < r \leq 1$ et $r \neq -1$. Plus spécifiquement, 
        elle converge vers $0$ si $ -1 < r < 1$ et elle converge vers $1$ si $r = 0$. Pour tout autre valeur, 
        hors de l'intervalle $]-1, 1]$, la série diverge. 
        \[
        \boxed{
            -1 < r \leq 1 \text{ avec } r \neq -1
    } \]           
    \end{Preuve}
    

    \section{Convergence d'un série géométrique}
    \begin{Preuve}{}{}
        \hypertarget{Convergence série géométrique}{Considérons} une suite géométrique de terme initial $a$ et de raison $r$. Les
        termes de cette suite sont donnés par $u_n = ar^n$. Nous devons prouver que si 
        $|r| < 1$, la somme de cette suite converge vers $\dfrac{a}{1 - r}$, et qu'elle
        diverge lorsque $|r| \geq 1$.

        \textbf{Cas 1 : $\boldsymbol{|r| < 1}$}\\

        Si $|r| < 1$, la somme partielle $S_n$ est donnée par :
        \[
        S_n = a + ar + ar^2 + \dots + ar^n = a \sum_{k=0}^{n} r^k
        \]

        Nous savons que la somme d'une suite géométrique est :
        \[
        S_n = a \frac{1 - r^{n+1}}{1 - r}
        \]
        Lorsque $|r| < 1$, $r^{n+1} \to 0$ lorsque $n \to \infty$. Donc, la somme 
        infinie converge vers :
        \[
        \boxed{S = \frac{a}{1 - r}}
        \]
        \textbf{Cas 2 : $\boldsymbol{|r| \geq 1}$}\\

        Si $|r| \geq 1$, alors $r^{n+1}$ ne tend pas vers $0$. Ainsi, la série ne 
        converge pas, et la somme diverge. 
        \[
        \boxed{\text{La suite diverge pour } |r| \geq 1}
    \]       
    \end{Preuve}







\begin{Preuve}{Coefficient de Taylor}{}
    Soit le coefficient de Taylor 
    \begin{align*}
        c_n = \dfrac{1}{n!}f^{n}(a)
    \end{align*}
     Ce résultat est obenu par la dérivation séquentielle d'une fonction 
     $f(x)$ et l'évaluation de la dévirvée à la valeur $x = a$. 
     Soit une fonction $f(x)$ qui peut s'écrire sous la forme d'une série 
     entière au voisinage de $a$ : 

     \begin{align*}
        f(x) = c_o  + c_1(x - a) + c_2(x - a)^2 + \cdots + c_n(x - a)^n 
        + \cdots (*)
     \end{align*}

     Les dérivées sont données par : 
     \begin{align*}
         f(a) = c_0 + c_1(a - a) + c_2(a -a)^2 + \cdots + c_n(a - a)^{n - 1} + \cdots + (*)  
         = \textcolor{myb}{\textbf{\textit{$c_0$}  }  }  \\
        f^{\prime}(x) = 
        \cancelto{0}{c_0} + c_1(1) + (2)(1)c_2(x - a) + 
        c_3(3)(2)(1)(x - a)^2 + \cdots + n!(x - a)^{n-1} + \cdots (*) \\ 
        f^{\prime}(a) =  c_1 + 2!c_2(a - a) + 3!(a - a)^2 + \cdots + n!(a - a)^{n - 1} + 
        \cdots (*) = \textbf{\textit{\textcolor{myb}{$c_1$}}}  \\ 
         f^{\prime\prime}(x) = 
         \cancelto{0}{c_2} + 2!c_2(1) + 
        c_33!\cdot(x - a) + \cdots + (n-1)n!(x - a)^{n-2} + \cdots (*) \\ 
         f^{\prime\prime}(a) = 
          2!c_2(1) + 
        c_33!\cdot(a - a) + \cdots + (n-1)n!(a - a)^{n-2} + \cdots (*)   
        = \textcolor{myb}{\textbf{\textit{$2!c_2$}}}  
     \end{align*}

     De façon générale, on a : 

     \begin{align*}
         \boxed{f^{n}(a) = n!c_n \Leftrightarrow \dfrac{1}{n!}f^{n}(a) = c_n}
     \end{align*}






    



\end{Preuve}




\end{document}
