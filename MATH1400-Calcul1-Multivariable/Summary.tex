\documentclass{report}
%\usepackage[utopia]{mathdesign}
%\usepackage{amsmath, amsthm}


\usepackage{amsmath,amsfonts,amsthm,amssymb,mathtools}
%\usepackage[varbb]{newpxmath}
%\usepackage[osf,largesc,theoremfont]{newpxtext}
%\usepackage{coelacanth}
%\usepackage{beraserif} % Bitstream Vera Serif font
%\usepackage{berasans} % Bitstream Vera Sans font
%\usepackage{beramono} % Bitstream Vera Sans Mono font
%\usepackage{berasans}
%\usepackage{libertine}
%\usepackage{mathpazo}
%\usepackage{palatino}
%\usepackage{crimson}


%% Choose one of the following (if not choosing the  
%% default, viz., Computer Modern, font family):
%\usepackage{lmodern}
\usepackage{bold-extra}
%%
%\usepackage{mathpazo}
% \usepackage{newpxmath}
%\usepackage{kpfonts} % Very good
%%
%\usepackage{mathptmx} %Very good
%\usepackage{stix} 
%\usepackage{txfonts} %Very good
\usepackage{newtxtext,newtxmath} %Very good
%%
%\usepackage{libertine} \usepackage[libertine]{newtxmath}
%\usepackage{libertine,libertinust1math} % added 2019/11/28
%%
%\usepackage{newpxtext} \usepackage[euler-digits]{eulervm}
%\usepackage{textcomp}
%\usepackage{bm}
\usepackage{contour}
\usepackage{adjustbox}





\input{/home/cryptopsy/Semesters/LaTeXTemplates/UniversalTeXTemplate/preamble.tex}
%From M275 "Topology" at SJSU
\newcommand{\id}{\mathrm{id}} % Identité
\newcommand{\taking}[1]{\xrightarrow{#1}} % Flèche avec annotation
\newcommand{\inv}{^{-1}} % Inverse

%From M170 "Introduction to Graph Theory" at SJSU
\DeclareMathOperator{\diam}{diam} % Diamètre
\DeclareMathOperator{\ord}{ord} % Ordre
\newcommand{\defeq}{\overset{\mathrm{def}}{=}} % Défini comme égal

%From the USAMO .tex files
\newcommand{\ts}{\textsuperscript} % Exposant
\newcommand{\dg}{^\circ} % Degré
\newcommand{\ii}{\item} % Item

% % From Math 55 and Math 145 at Harvard
% \newenvironment{subproof}[1][Proof]{%
% \begin{proof}[#1] \renewcommand{\qedsymbol}{$\blacksquare$}}%
% {\end{proof}}

\newcommand{\liff}{\leftrightarrow} % Si et seulement si
\newcommand{\lthen}{\rightarrow} % Implique
\newcommand{\opname}{\operatorname} % Opérateur générique
\newcommand{\surjto}{\twoheadrightarrow} % Flèche surjective
\newcommand{\injto}{\hookrightarrow} % Flèche injective
\newcommand{\On}{\mathrm{On}} % Ordinaux
\DeclareMathOperator{\img}{im} % Image
\DeclareMathOperator{\Img}{Im} % Image
\DeclareMathOperator{\coker}{coker} % Cokernel
\DeclareMathOperator{\Coker}{Coker} % Cokernel
\DeclareMathOperator{\Ker}{Ker} % Noyau
\DeclareMathOperator{\rank}{rank} % Rang
\DeclareMathOperator{\Spec}{Spec} % Spectre
\DeclareMathOperator{\Tr}{Tr} % Trace
\DeclareMathOperator{\pr}{pr} % Projection
\DeclareMathOperator{\ext}{ext} % Extension
\DeclareMathOperator{\pred}{pred} % Prédécesseur
\DeclareMathOperator{\dom}{dom} % Domaine
\DeclareMathOperator{\ran}{ran} % Image (range)
\DeclareMathOperator{\Hom}{Hom} % Homomorphisme
\DeclareMathOperator{\Mor}{Mor} % Morphismes
\DeclareMathOperator{\End}{End} % Endomorphisme

\newcommand{\eps}{\epsilon} % Épsilon
\newcommand{\veps}{\varepsilon} % Variance d'épsilon
\newcommand{\ol}{\overline} % Ligne au-dessus
\newcommand{\ul}{\underline} % Ligne en-dessous
\newcommand{\wt}{\widetilde} % Tilde large
\newcommand{\wh}{\widehat} % Chapeau large
\newcommand{\vocab}[1]{\textbf{\color{blue} #1}} % Texte en gras et bleu
\providecommand{\half}{\frac{1}{2}} % Fraction 1/2
\newcommand{\dang}{\measuredangle} % Angle dirigé
\newcommand{\ray}[1]{\overrightarrow{#1}} % Ray
\newcommand{\seg}[1]{\overline{#1}} % Segment
\newcommand{\arc}[1]{\wideparen{#1}} % Arc
\DeclareMathOperator{\cis}{cis} % cis
\DeclareMathOperator*{\lcm}{lcm} % Plus petit commun multiple
\DeclareMathOperator*{\argmin}{arg min} % Argument du minimum
\DeclareMathOperator*{\argmax}{arg max} % Argument du maximum
\newcommand{\cycsum}{\sum_{\mathrm{cyc}}} % Somme cyclique
\newcommand{\symsum}{\sum_{\mathrm{sym}}} % Somme symétrique
\newcommand{\cycprod}{\prod_{\mathrm{cyc}}} % Produit cyclique
\newcommand{\symprod}{\prod_{\mathrm{sym}}} % Produit symétrique
\newcommand{\Qed}{\begin{flushright}\qed\end{flushright}} % QED aligné à droite
\newcommand{\parinn}{\setlength{\parindent}{1cm}} % Indentation de paragraphe à 1 cm
\newcommand{\parinf}{\setlength{\parindent}{0cm}} % Pas d'indentation de paragraphe
% \newcommand{\norm}{\|\cdot\|} % Norme
\newcommand{\inorm}{\norm_{\infty}} % Norme infinie
\newcommand{\opensets}{\{V_{\alpha}\}_{\alpha\in I}} % Ensemble ouvert
\newcommand{\oset}{V_{\alpha}} % Ensemble ouvert V
\newcommand{\opset}[1]{V_{\alpha_{#1}}} % Ensemble ouvert V avec indice
\newcommand{\lub}{\text{lub}} % Plus petite borne supérieure
\newcommand{\del}[2]{\frac{\partial #1}{\partial #2}} % Dérivée partielle
\newcommand{\Del}[3]{\frac{\partial^{#1} #2}{\partial^{#1} #3}} % Dérivée partielle d'ordre élevé
\newcommand{\deld}[2]{\dfrac{\partial #1}{\partial #2}} % Dérivée partielle avec dfrac
\newcommand{\Deld}[3]{\dfrac{\partial^{#1} #2}{\partial^{#1} #3}} % Dérivée partielle d'ordre élevé avec dfrac
\newcommand{\lm}{\lambda} % Lambda
\newcommand{\uin}{\mathbin{\rotatebox[origin=c]{90}{$\in$}}} % Appartient, tourné de 90 degrés
\newcommand{\usubset}{\mathbin{\rotatebox[origin=c]{90}{$\subset$}}} % Sous-ensemble, tourné de 90 degrés
\newcommand{\lt}{\left} % Gauche
\newcommand{\rt}{\right} % Droite
\newcommand{\bs}[1]{\boldsymbol{#1}} % Symbole en gras
\newcommand{\exs}{\exists} % Il existe
\newcommand{\st}{\strut} % Strut
\newcommand{\dps}[1]{\displaystyle{#1}} % Disposition en ligne

\newcommand{\sol}{\setlength{\parindent}{0cm}\textbf{\textit{Solution:}}\setlength{\parindent}{1cm} } % Solution sans indentation initiale puis rétablie
\newcommand{\solve}[1]{\setlength{\parindent}{0cm}\textbf{\textit{Solution: }}\setlength{\parindent}{1cm}#1 \Qed}

\newcommand{\entoure}[1]{\fcolorbox{black}{gray!30}{\texttt{#1}}}

\renewcommand{\ttdefault}{cmtt}
\newcommand{\textttbf}[1]{\contour{yellow!45}{\texttt{#1}}}
\newcommand{\varitem}[3][black]{%
    \item [%
        \colorbox{#2}{\textcolor{#1}{\makebox(5.5,7){#3}}}%
    ]
}
% Allow you to do the non implication (implication barred)
\newcommand{\notimplies}{%
  \mathrel{{\ooalign{\hidewidth$\not\phantom{=}$\hidewidth\cr$\implies$}}}}


\newcommand*{\authorimg}[1]%
    { \raisebox{-1\baselineskip}{\includegraphics[width=\imagesize]{#1}}}
\newlength\imagesize 

\input{/home/cryptopsy/Semesters/LaTeXTemplates/UniversalTeXTemplate/letterfonts.tex}
% lstlistingsEnvs.tex

\usepackage{minted}


\lstset{
  basicstyle=\ttfamily, % Set
  columns=fullflexible,
  keepspaces=true,
  language=Python % You can specify the language if you want syntax highlighting
}

%%%%%%%%%%%%%%%%%%%%%%%%%%%%%%%%%%%%%%%%%%%%%%%%%%%%%%%%%%%%%%%%%%%%%%%%%%%%%%%%%%%%%%%%%%%%%%%%%
%                                 Custom lstlisting Environments
%%%%%%%%%%%%%%%%%%%%%%%%%%%%%%%%%%%%%%%%%%%%%%%%%%%%%%%%%%%%%%%%%%%%%%%%%%%%%%%%%%%%%%%%%%%%%%%%%
% Gruvbox style for Python
\definecolor{Pgruvbox-bg}{HTML}{282828}
\definecolor{Pgruvbox-fg}{HTML}{ebdbb2}
\definecolor{Pgruvbox-red}{HTML}{fb4934}
\definecolor{Pgruvbox-green}{HTML}{b8bb26}
\definecolor{Pgruvbox-yellow}{HTML}{fabd2f}
\definecolor{Pgruvbox-blue}{HTML}{83a598}
\definecolor{Pgruvbox-purple}{HTML}{d3869b}
\definecolor{Pgruvbox-aqua}{HTML}{8ec07c}
\definecolor{BBBlack}{rgb}{0.05, 0.06, 0.09}



% JAVA LSTLISTING STYLE IN Gruvbox Colorscheme
\definecolor{gruvbox-bg}{rgb}{0.282, 0.247, 0.204}
\definecolor{gruvbox-fg1}{rgb}{0.949, 0.898, 0.776}
\definecolor{gruvbox-fg2}{rgb}{0.871, 0.804, 0.671}
\definecolor{gruvbox-red}{rgb}{0.788, 0.255, 0.259}
\definecolor{gruvbox-green}{rgb}{0.518, 0.604, 0.239}
\definecolor{gruvbox-yellow}{rgb}{0.914, 0.808, 0.427}
\definecolor{gruvbox-blue}{rgb}{0.353, 0.510, 0.784}
\definecolor{gruvbox-purple}{rgb}{0.576, 0.412, 0.659}
\definecolor{gruvbox-aqua}{rgb}{0.459, 0.631, 0.737}
\definecolor{gruvbox-gray}{rgb}{0.518, 0.494, 0.471}

\definecolor{lst-bg}{RGB}{45, 45, 45}
\definecolor{lst-fg}{RGB}{220, 220, 204}
\definecolor{lst-keyword}{RGB}{215, 186, 125}
\definecolor{lst-comment}{RGB}{117, 113, 94}
\definecolor{lst-string}{RGB}{163, 190, 140}
\definecolor{lst-number}{RGB}{181, 206, 168}
\definecolor{lst-type}{RGB}{218, 142, 130}

\lstdefinestyle{PythonGruvbox}{
    language=Python,
    identifierstyle=\color{lst-fg},
    basicstyle=\ttfamily\color{Pgruvbox-fg},
    keywordstyle=\color{Pgruvbox-yellow},
    keywordstyle=[2]\color{Pgruvbox-blue},
    stringstyle=\color{Pgruvbox-green},
    commentstyle=\color{Pgruvbox-aqua},
    backgroundcolor=\color{BBBlack},
    rulecolor=\color{BBBlack},
    showstringspaces=false,
    keepspaces=true,
    captionpos=b,
    breaklines=true,
    tabsize=4,
    showspaces=false,
    numbers=left,
    numbersep=5pt,
    numberstyle=\tiny\color{gray},
    showtabs=false,
    columns=fullflexible,
    morekeywords={True,False,None},
    morekeywords=[2]{and,as,assert,break,class,continue,def,del,elif,else,except,exec,
    finally,for,from,global,if,import,in,is,lambda,nonlocal,not,or,pass,print,raise,
    return,try,while,with,yield},
    morecomment=[s]{"""}{"""},
    morecomment=[s]{'''}{'''},
    morecomment=[l]{\#},
    morestring=[b]",
    morestring=[b]',
    literate=
    {0}{{\textcolor{Pgruvbox-purple}{0}}}{1}
    {1}{{\textcolor{Pgruvbox-purple}{1}}}{1}
    {2}{{\textcolor{Pgruvbox-purple}{2}}}{1}
    {3}{{\textcolor{Pgruvbox-purple}{3}}}{1}
    {4}{{\textcolor{Pgruvbox-purple}{4}}}{1}
    {5}{{\textcolor{Pgruvbox-purple}{5}}}{1}
    {6}{{\textcolor{Pgruvbox-purple}{6}}}{1}
    {7}{{\textcolor{Pgruvbox-purple}{7}}}{1}
    {8}{{\textcolor{Pgruvbox-purple}{8}}}{1}
    {9}{{\textcolor{Pgruvbox-purple}{9}}}{1}
}

% Gruvbox style for Java
\definecolor{gruvbox-bg}{rgb}{0.282, 0.247, 0.204}
\definecolor{gruvbox-fg1}{rgb}{0.949, 0.898, 0.776}
\definecolor{gruvbox-fg2}{rgb}{0.871, 0.804, 0.671}
\definecolor{gruvbox-red}{rgb}{0.788, 0.255, 0.259}
\definecolor{gruvbox-green}{rgb}{0.518, 0.604, 0.239}
\definecolor{gruvbox-yellow}{rgb}{0.914, 0.808, 0.427}
\definecolor{gruvbox-blue}{rgb}{0.353, 0.510, 0.784}
\definecolor{gruvbox-purple}{rgb}{0.576, 0.412, 0.659}
\definecolor{gruvbox-aqua}{rgb}{0.459, 0.631, 0.737}
\definecolor{gruvbox-gray}{rgb}{0.518, 0.494, 0.471}

\lstdefinestyle{JavaGruvbox}{
    language=Java,
    basicstyle=\ttfamily\color{Pgruvbox-fg},
    keywordstyle=\color{Pgruvbox-yellow},
    keywordstyle=[2]\color{lst-type},
    commentstyle=\itshape\color{lst-comment},
    stringstyle=\color{lst-string},
    numberstyle=\color{lst-number},
    backgroundcolor=\color{BBBlack},
    rulecolor=\color{gruvbox-aqua},
    showstringspaces=false,
    keepspaces=true,
    captionpos=b,
    breaklines=true,
    tabsize=4,
    showspaces=false,
    showtabs=false,
    columns=fullflexible,
    morekeywords={var},
    morekeywords=[2]{boolean, byte, char, double, float, int, long, short, void},
    morecomment=[s]{/}{/},
    morecomment=[l]{//},
    morestring=[b]",
    morestring=[b]',
    numbers=left,
    numbersep=5pt,
    numberstyle=\tiny\color{gray},
}

% Dracula style for Java
\definecolor{draculawhite-background}{RGB}{237, 239, 252}
\definecolor{draculawhite-comment}{RGB}{98, 114, 164}
\definecolor{draculawhite-keyword}{RGB}{189, 147, 249}
\definecolor{draculawhite-string}{RGB}{152, 195, 121}
\definecolor{draculawhite-number}{RGB}{249, 189, 89}
\definecolor{draculawhite-operator}{RGB}{248, 248, 242}

\lstdefinestyle{JavaDraculaWhite}{
    language=Java,
    backgroundcolor=\color{draculawhite-background},
    commentstyle=\itshape\color{draculawhite-comment},
    keywordstyle=\color{draculawhite-keyword},
    stringstyle=\color{draculawhite-string},
    basicstyle=\ttfamily\footnotesize\color{black},
    identifierstyle=\color{black},
    keywordstyle=\color{draculawhite-keyword}\bfseries,
    morecomment=[s][\color{draculawhite-comment}]{/**}{*/},
    showstringspaces=false,
    showspaces=false,
    breaklines=true,
    %frame=single,
    rulecolor=\color{draculawhite-operator},
    tabsize=2,  
    numbers=left,
    numbersep=4pt,
    numberstyle=\ttfamily\tiny\color{gray}
}

% Dracula style for Python
\definecolor{draculawhite-bg}{HTML}{FAFAFA}
\definecolor{draculawhite-fg}{HTML}{282A36}
\definecolor{pdraculawhite-keyword}{HTML}{BD93F9}
\definecolor{pdraculawhite-comment}{HTML}{6272A4}
\definecolor{draculawhite-number}{HTML}{FF79C6}

\lstdefinestyle{PythonDraculaWhite}{
    language=Python,
    basicstyle=\ttfamily\small\color{draculawhite-fg},
    backgroundcolor=\color{draculawhite-background},
    keywordstyle=\color{orange}\bfseries,
    stringstyle=\color{draculawhite-string},
    commentstyle=\color{pdraculawhite-comment}\itshape,
    numberstyle=\color{draculawhite-number},
    showstringspaces=false,
    showspaces=false,
    breaklines=true,
    frame=single,
    rulecolor=\color{draculawhite-operator}, 
    tabsize=4,
    morekeywords={as,with,1,2,3,4, 5,6,7,8,9,True,False},
    numbers=left,
    numbersep=5pt,
    numberstyle=\small\bfseries\ttfamily\color{htmlcomment},
}

% Dracula Dark style for HTML
\definecolor{htmltag}{HTML}{ff79c6}
\definecolor{htmlattr}{HTML}{f1fa8c}
\definecolor{htmlvalue}{HTML}{bd93f9}
\definecolor{htmlcomment}{HTML}{6272a4}
\definecolor{htmltext}{HTML}{401E31}
\definecolor{htmlbackground}{HTML}{282a36}
\definecolor{comphtmlbackground}{HTML}{8093FF}

\lstdefinestyle{HTMLDraculaDark}{
    basicstyle=\normalsize\bfseries\ttfamily\color{htmltext},
    commentstyle=\itshape\color{htmlcomment},
    keywordstyle=\bfseries\color{htmltag},
    stringstyle=\color{htmlvalue},
    emph={DOCTYPE,html,head,body,div,span,a,script},
    emphstyle={\color{htmltag}\bfseries},
    sensitive=true,
    showstringspaces=false,
    backgroundcolor=\color{white},
    inputencoding=utf8,
    extendedchars=true,
    language=HTML,
    tabsize=4,
    breaklines=true,
    breakatwhitespace=true,
    numbers=left,
    numbersep=10pt,
    numberstyle=\small\bfseries\ttfamily\color{htmlcomment},
    escapeinside={<@}{@>},
    rulecolor=\color{htmlbackground},
    xleftmargin=10pt,
    frame=none, 
    breaklines=true,
    postbreak=\mbox{\textcolor{gray}{$\hookrightarrow$}\space},
    showlines=false,
    moredelim=[s][\itshape\color{htmlcomment}]{<!--}{-->},
    morekeywords={id,class,type,name,value,placeholder,checked,src,href,alt},
    literate={é}{{\'e}}1 {è}{{\`e}}1 {ê}{{\^e}}1 {ë}{{\"e}}1 {à}{{\`a}}1 {ù}{{\`u}}1 {û}{{\^u}}1 {ç}{{\c{c}}}1 {â}{{\^a}}1 {î}{{\^i}}1 {ï}{{\"i}}1
}


\lstdefinestyle{Haskell}{
  frame=none,
  xleftmargin=2pt,
  stepnumber=1,
  numbers=left,
  numbersep=5pt,
  numberstyle=\ttfamily\tiny\color[gray]{0.3},
  belowcaptionskip=\bigskipamount,
  captionpos=b,
  escapeinside={*'}{'*},
  language=haskell,
  tabsize=2,
  emphstyle={\bf},
  %commentstyle=\it,
  stringstyle=\mdseries\ttfamily,
  showspaces=false,
  keywordstyle=\bfseries\ttfamily,
  columns=flexible,
  basicstyle=\small\ttfamily,
  showstringspaces=false,
  morecomment=[l]\%,
}



\lstdefinestyle{CSSDraculaLight}{
    basicstyle=\bfseries\scriptsize\ttfamily\color{htmltext},
    commentstyle=\color{htmlcomment},
    keywordstyle=\bfseries\color{htmlvalue},
    stringstyle=\color{htmlvalue},
    emph={DOCTYPE,html,head,body,div,span,a,script},
    emphstyle={\color{htmltag}\bfseries},
    sensitive=true,
    showstringspaces=false,
    backgroundcolor=\color{white},
    inputencoding=utf8,
    extendedchars=true, % Support extended characters
    frame=none, 
    %frame=tb,
    tabsize=4,
    breaklines=true,
    breakatwhitespace=true,
    numbers=left,
    numbersep=10pt,
    numberstyle=\small\bfseries\ttfamily\color{htmlcomment},
    escapeinside={<@}{@>},
    rulecolor=\color{htmlbackground},
    xleftmargin=20pt,
    % Add a vertical line for opening and closing tags
    %frame=lines,
    framesep=2pt,
    framexleftmargin=4pt,
    % Add a vertical line for closing tags that go through multiple lines
    breaklines=true,
    postbreak=\mbox{\textcolor{gray}{$\hookrightarrow$}\space},
    showlines=true,
    % Add a rule to apply commentstyle to keywords inside comments
    moredelim=[s][\color{htmlcomment}]{/*}{*/},
    literate={é}{{\'e}}1
             {è}{{\`e}}1
             {ê}{{\^e}}1
             {ë}{{\"e}}1
             {à}{{\`a}}1
             {ù}{{\`u}}1
             {û}{{\^u}}1
             {ç}{{\c{c}}}1
             {â}{{\^a}}1
             {î}{{\^i}}1
             {ï}{{\"i}}1,
    morekeywords={color, background, background-color, font-size, font-weight, border, border-radius, padding, margin, display, position, top, right, bottom, left, flex, grid, width, height, min-width, max-width, min-height, max-height, transition, transform, animation, keyframes, content, z-index,id,class,type,name,value,placeholder,checked,src,href,alt},
    morestring=[s][\color{htmltag}]{:}{;},
}


\renewcommand{\ttdefault}{cmtt}
\newcommand{\textttbf}[1]{\contour{yellow!45}{\texttt{#1}}}
\newcommand{\varitem}[3][black]{%
    \item[%
        \colorbox{#2}{\textcolor{#1}{\makebox(5.5,7){#3}}}%
    ]
}

% Allow you to do the non implication (implication barred)
\newcommand{\notimplies}{%
  \mathrel{{\ooalign{\hidewidth$\not\phantom{=}$\hidewidth\cr$\implies$}}}}






\title{\huge{MATH1400}\\\Huge{Calcul à plusieurs variables}\\\vspace{2em}Ensemble de théorèmes, lemmes et définitions }
\author{\huge{Franz Girardin}}
\date{\today}


   

\begin{document}

\maketitle
\newpage% or \cleardoublepage
% \pdfbookmark[<level>]{<title>}{<dest>}
\pdfbookmark[section]{\contentsname}{toc}
\tableofcontents
\pagebreak


\titleformat*{\section}{%
    \normalsize\bfseries%
}

\titleformat{\section}[block]{\normalsize\bfseries}{}{0pt}{}


\begin{multicols*}{3}
    \footnotesize

    \section{Définition d'une suite}
        \textbf{Fonction} $\mathbb{N}^* 
        \rightarrow \mathbb{R}$ qui accepte 
        $n \in \mathbb{N}^*$ et engendre une \textbf{séquence ordonnées} de $a_n \in \mathbb{R}$. 


    \section{Définition d'une suite arithmétique}    
        \begin{align*}
                &a_n \Coloneqq 
                \begin{cases}
                    a_1 = r  &\omit\quad\quad\quad Raison\\  
                    a_{n} = a_{n-1} + r &\omit\hfill \quad\quad\quad Récurrence
                \end{cases}
                \\\\
                &r = a_n - a_{n-1} \;\;| n \geq 2  
                \quad\quad\quad\quad\;\;\text{Trouver $r$} \\
                &a_n = a_1 \text{+} \left(n - 1\right)\cdot{n}  \;\;|n \geq 1  
                \quad\quad\text{Trouver $n$\up{e} terme}
        \end{align*}
    \section{Définition d'une suite géométrique}
        \begin{align*}
                &a_n \Coloneqq 
                \begin{cases}
                    a_1 = r  &\omit\quad\quad Raison\\  
                    a_{n} = a_{n-1} \cdot r &\omit\hfill 
                    \quad\quad Récurrence
                \end{cases}
                \\\\ 
                &r = \frac{a_n}{a_{n-1}} \; | \; n \geq 2
                \quad\quad\quad\quad\;\;\text{Trouver $r$} \\ 
                &a_n = a_1r^{n - 1} \; | \; n \geq 1
                \text{\quad\quad\quad\; Trouver $n$\up{e} terme}
        \end{align*}






        \section{Convergence d'une suite géométrique}
        $\forall r \in \mathbb{R}$, la suite $\{r^n\}$ converge        
        \textit{ssi} $-1 < r \leq 1$ : 
        \begin{align*}
            \lim\limits_{n\to+\infty }\{r^n\} = 
                    \begin{cases}
                        0 \quad \text{si $-1 < r < 1$} \\
                        1 \quad \text{si $r = 1$}
                    \end{cases}
        \end{align*}


        

    \section{Définition formelle de convergence d'une suite}
      \begin{align*}
        \lim\limits_{n \to\infty  }{a_n} = L \\
      \end{align*}
      \textbf{si et seulement si}, 
      \begin{align*}
        \forall\varepsilon > 0, \exists N\left( \varepsilon \right) > 0 : 
        n > N\left( \varepsilon \right) \implies 
        |a_n -L| < \varepsilon
      \end{align*}



    \section{Définition formelle de divergence d'une suite}
      \begin{align*}
        \lim\limits_{n  \to\infty  }{a_n} = \infty
      \end{align*}
      \textbf{si et seulement si},  
        \begin{align*}
            \forall M \in \mathbb{R}, \exists N \in \mathbb{N}^{*} :
            n > N \implies |a_n| > M 
        \end{align*}

    \section{Corollaire}
    \textbf{Si} $\lim\limits_{n\to+\infty }a_n  = \infty$,  
    \textbf{alors}, 
    $\lim\limits_{n\to+\infty }{\dfrac{1}{a_n}}  = 0$




    \section{Attention}
        \begin{align*}
            \lim\limits_{n\to\infty  }\frac{1}{a_n} = 0 
            \textcolor{red}{\notimplies}
            \lim\limits_{n\to+\infty }a_n  = \infty
        \end{align*}    

    \section{Lemme de convergence des suite éventuellement signées}



        \begin{enumerate}
            \item 
                Si $\{ a_n \}$ est une suite 
                \textbf{éventuellement positive}, 
                alors,  
                    $\lim\limits_{n\to+\infty }\frac{1}{a_n}  = 0 
                    \implies 
                    \lim\limits_{n\to+\infty }a_n  = \infty$
            \item                                 
                Si $\{ a_n \}$ est une suite 
                \textbf{éventuellement négative}, 
                alors,   
                    $\lim\limits_{n\to+\infty }\frac{1}{a_n}  = 0 
                    \implies 
                    \lim\limits_{n\to+\infty }a_n  = -\infty$
        \end{enumerate} 


 

    \section{Propriétés des limites}
        Si $\{a_n\}$ et $\{b_n\}$ sont des suites convergentes et 
        si $c$ est une constante, \textbf{alors} \\\\ 
        $\lim\limits_{n\to\infty  }\left(a_n \text{+} b_n \right) = 
        \lim\limits_{n\to\infty  }a_n \text{+} 
        \lim\limits_{n\to\infty  }b_n$
        \\\\
        $\lim\limits_{n\to\infty  }\left(a_n - b_n \right) = 
        \lim\limits_{n\to\infty  }a_n - \lim\limits_{n\to\infty  }b_n$ 
        \\\\
        $\lim\limits_{n\to\infty  }ca_n = c \lim\limits_{n \to \infty  }a_n$ 
        \\\\
        $\lim\limits_{n\to\infty  }\left(a_nb_n \right) = 
        \lim\limits_{n\to\infty  }a_n \cdot \lim\limits_{n\to\infty  }b_n$
        \\\\
        $\lim\limits_{n\to\infty  }\left(\frac{a_n}{b_n} \right) = 
        \frac{\lim\limits_{n\to\infty  }a_n}{\lim\limits_{n\to\infty  }b_n}
        \;
        \text{si} \lim\limits_{x\to\infty  }b_n \neq 0$
        \\\\
        $\lim\limits_{n\to\infty  }a_n^{p} = 
        \left[\lim\limits_{n\to\infty  }a_n \right]^p \text{si } 
        p > 0 \; \text{et} \; a_n > 0$



    \section{Limite d'une suite polynomiale} 
        Soit deux polynomes,
        $\lim\limits_{n\to \infty } \dfrac{p(n)}{q(n)}$, 
        et 
        $k = \min\bigl(deg(p), deg(q)\bigr)$
        \textbf{Alors},   
        \[ \lim\limits_{n\to \infty } \dfrac{p(n)}{q(n)} =
        \lim\limits_{n\to+\infty}\dfrac{p(n)/{n^k}}{q(n)/n^{k}} \]



    \section{Règle de l'Hôpital}
        Soit une \textbf{constante} $c \in \mathbb{R} \cup \{+\infty\}$ et supposon que : 
        \begin{itemize}
        \item $\lim\limits_{x\to c}\dfrac{|f(x)|}{|g(x)|}$ 
            est de la forme $\dfrac{0}{0}$ ou 
            $\dfrac{\infty }{\infty }$
        \item $\lim\limits_{x\to c}\dfrac{f^{\prime}(x)}{|g^{\prime}(x)|}$
            \textbf{existe} et 
            \textcolor{red}{$g^{\prime}(x) \neq 0 \;\; \forall x \approx c$ }
        \end{itemize}
        \textbf{Alors}, 
        \begin{align*}
            \lim\limits_{x\to c}\dfrac{f(x)}{g(x)} = 
            \lim\limits_{x\to c}\dfrac{f^{\prime}(x)}{g^{\prime}(x)}
        \end{align*}
        


    \section{Monotonicité} 
        Soit une suite $\{a_n \}$, on dit que la suite est :
        \begin{itemize}
            \item \textbf{Strictement croissant} si $\forall n \geq 1, 
                a_{n+1} > a_n$
            \item \textbf{Croissante} si $\forall n \geq 1, 
                a_{n+1} \geq a_n$ 
            \item \textbf{Strictement décroissante} si $\forall n \geq 1, 
                a_{n+1} < a_n$ 
            \item \textbf{Décroissante} si $\forall n \geq 1, 
                a_{n+1} < a_n$ 
            \item \textbf{Stationnaire} ou \textbf{constante} si 
                $\forall n \geq 1, 
                a_{n+1} < a_n$ 
            \item \textbf{Monotone}  
        \end{itemize} 



    \section{Définitions de bornes d'une suite}
       \begin{align*}
           \textbf{Minonant } m \Coloneqq 
           \exists m \in \mathbb{R} : \forall n \in \{ a_n \}, 
           a_n \geq m
       \end{align*}
       \begin{align*}
           \textbf{Majorant} M \Coloneqq 
           \exists M \in \mathbb{R} : \forall n \in \{ a_n \}, 
           a_n \leq M
       \end{align*}
       \begin{align*}
           \textbf{Bornée} \Coloneqq 
            \textit{Majorée} \land \textit{minorée}
       \end{align*}
       


    \section{Théorème des suites monotones}
        Toute suite monotone et bornée est \textbf{convergente}  


    \section{Lemmes des suites monotones}
        \begin{itemize}
            \item Toute suite éventuellement croissante et majorée 
        est également \textbf{convergente}  
            \item Tout suite éventuellement décroissante et 
            minorée est également \textbf{convergente}  
        \end{itemize}
    


    \section{Association d'une fonction à une suite}
        Soit $f\left(x\right)$ une fonction admettant une limite $L$ à 
        $\text{+}\infty$, Alors, la suite 
        $\{a_n\} = f\left(n\right)$ admet la même limite : 
        \begin{align*}
          \lim\limits_{x\to\infty  }f(x) = L \implies \lim\limits_{n\to\infty  }a_n = L
        \end{align*}
        De la même façon :
        \begin{align*}
          \lim\limits_{x\to\infty  }f(x) = \infty \implies \lim\limits_{n\to\infty  }a_n = \infty
        \end{align*}
        Par ailleurs, si $f\left(x\right)$ est une fonction continue en 
        $L$ et si la suite $\{a_n\}$
        converge vers $L$, alors la limite suivante converge vers $f\left(L\right)$ :
        \begin{align*}
                    \lim\limits_{n\to+\infty  }f(a_n)  = f(L) \\ 
                    \lim\limits_{n\to+\infty  }f(\lim\limits_{n \to \infty} a_n)  
                    = f(L)  
        \end{align*}
        $\rhd$ $ \textbf{Exemple}  \lim\limits_{n \to+\infty }\sin(\pi/2)  = 
        \sin(\lim\limits_{n \to+\infty })  \pi/2  = 0$




    \section{Comparaison des suites}
        Si $a > 1$ et $k > 0$, on a 
        \begin{align*}
            \ln(n) \ll n^K \ll a^n \ll n! \ll n^n
        \end{align*}
        $c_n \ll d_n \implies 
        \lim\limits_{n\to+\infty }\dfrac{c_n}{d_n} = 0$ 


    \section{Théorème des gendarmes}{}
        Soient $\{a_n\}$, $\{b_n\}$ et $\{c_n\}$ des suites et $n_0 \in \mathbb{N}$ tels 
        que
        \begin{itemize}
            \item $\lim\limits_{n\to+\infty }a_n  = 
                \lim\limits_{n\to+\infty }c_n  = L 
                \in \mathbb{R} \cup \{\infty \}$; 
            \item $\forall n \geq n_0, \; a_n \leq b_n \leq c_n$ 
        \end{itemize}
        \textbf{Alors},
        \begin{align*}
            \lim\limits_{n\to+\infty}b_n  = L                   
        \end{align*}



    \section{Corollaire}
         \textbf{Si} $\lim\limits_{n\to+\infty }|a_n|  = 0$, 
         \textbf{alors}      
         $\lim\limits_{n\to+\infty }a_n  = 0$

    \section{Définition d'une série numérique}
    Somme infinie des termes d'une suite numérique 
    correspondante $a_n$ : 
    $\sum_{n=1}^{\infty}a_n$. 
    \begin{itemize}
        \item \textbf{Premiers termes} $s_n = \sum_{i=1}^{n } a_i$   
        \item \textbf{Convergence}   
        $\lim\limits_{n\to+\infty }s_n  = s \implies 
        \sum_{n}^{\infty}a_n \textbf{ conv.}$
    \end{itemize}



    \section{Critère de divergence}
        Si la série converge, la suite correspondante 
        \textbf{converge vers 0},
        et si la suite ne converge pas vers zéro, la série 
            est divergente
        \begin{itemize} 
            \item $\sum_ {n}^{ \infty }a_n = s \; (\textbf{conv.}) 
                \implies 
            \lim\limits_{n\to+\infty }a_n  = 0$
            \item
            $\lim\limits_{n\to+\infty }a_n  \neq 0  
            \implies 
            \sum_{ n}^{\infty }a_n$ \textbf{ div.}  
        \end{itemize}

    \section{Attention} 
    $\lim\limits_{n\to+\infty }a_n = 0 
    \textcolor{red}{\notimplies} 
    \sum_{n=1}^{\infty }a_n$ \textbf{conv}.   



    \section{Convergence d'une série géométrique}
    \begin{itemize}
        \item[$\rhd$] 
            $|r| < 1 \implies  
            \sum_{n=0}^{\infty }ar^{n} = 
            \dfrac{a}{1 - r} $ \;\;(\textbf{conv.})  
        \item [$\rhd$] 
            $(|r| \geq 1) \land (a \neq 0) \implies 
            \sum_{n=0}^{\infty }ar^{n} = \infty$ 
            (\textbf{div.})
        
    \end{itemize}



    \section{Propriétés des séries}
    \begin{itemize}
        \item[$\blacktriangleright$]  \textbf{+} et \textbf{-} de deux séries convergentes 
            ainsi que $c \cdot \sum_{n}^{\infty }a_n$ engendrent une série \textbf{conv}.
        \item[$\blacktriangleright$] 
            $\sum_{n=1}^{\infty}a_nb_n \textcolor{red}{\neq} 
            (\sum_{n=1}^{\infty}a_n)(\sum_{n=1}^{\infty}b_n)$
        \item[$\blacktriangleright$]
            $\sum_{n=1}^{\infty }\dfrac{a_n}{b_n} 
            \textcolor{red}{\neq} 
            \dfrac{\sum_{n=1}^{\infty}a_n}{\sum_{n=1}^{\infty}b_n}$
    \end{itemize}
    
    
    \section{Test de l'intégrale}
    Soit $f : [1, \infty [ \rightarrow \mathbb{R}$ 
    \textbf{continue}, \textbf{positive} et \textbf{décroissante} 
    et $a_n : f(n) = a_n$,
    \begin{align*}
        \sum_{n=1}^{\infty }a_n \;\; \textbf{conv.} \;\; \leftrightarrow
        \lim\limits_{a \to+\infty }\int_{x=1}^{x = a}f(x)dx = s \;\; (\textbf{conv.})    
    \end{align*}


    \section{Série de Riemann et série puissance} 
    \begin{align*}
        \sum_{n=1}^{\infty }\dfrac{1}{n^p} 
        \text{ converge si } p > 1 \text{, diverge si } p \leq 1
        \\
        \sum_{n=1}^{\infty}n^p \text{ converge  si }
         p > 1 
    \end{align*}
    La première est un \textbf{série de Riemann}; la seconde 
    est une \textbf{série puissance}.  


    
    \section{Estimation du reste par TI}
    \textbf{Si} $f: \lambda, + , \downarrow [m, \infty [$ et soit $m \in \mathbb{N}^*, 
     a_n = f(n), \sum_{n=1}^{\infty}a_n = s \in \mathbb{R}, 
     R_m = s - s_m$, \textbf{alors} le reste $R_m$ est borné et peut être 
     estimé :
     \begin{align*}
        \int_{m+1}^{\infty }f(x)dx \leq \left(R_m = \sum_{k=1}^{m} a_k\right) 
        \leq \int_{m}^{\infty }f(x)dx
     \end{align*}


    
     \section{Test de comparaison}
     Soient $\sum a_n, \sum b_n$ des séries à \textbf{termes positifs} 
     et $n_0 \in \mathbb{R}$ :
     \begin{itemize}
         \item[$\rhd$ ] $\left(\sum   b_n  \;\; \textbf{conv.}\right) \land 
             (a_n \leq b_n \forall n \geq n_0) 
             \implies a_n \; \textbf{conv.}$ 
         \item[$\rhd$ ] $\left(\sum   b_n  \;\; \textbf{div.}\right) \land 
             (a_n \geq b_n \forall n \geq n_0) 
             \implies a_n \;\; \textbf{div.}$ 
         \item[$\rhd$ ] $\lim\limits_{n\to\infty}\dfrac{a_n}{b_n}
             \in \mathbb{R} \textbf{ existe } \textbf{ et } L > 0 
             \implies 
        \sum a_n \;\; \textbf{conv.} \;\; \leftrightarrow \;\; 
        \sum b_n \;\; \textbf{conv.}$ 
    \item[$\blacktriangleright$] Principalement pour Riemann \& géométriques
     \end{itemize}

     

     \section{Test sur séries alternées}
     Soit un \textbf{rang} $n_0 \in \mathbb{N}$ et 
     soit une \textbf{série alternée} 
     $\sum_{n=1}^{\infty } (-1)^nb_n$ telle que 
     \begin{itemize}
       \item [$\rhd$ ]  $0 \leq b_{n+1} \leq b_n \;\;(\downarrow \textbf{ et } +)$ 
       \item [$\rhd$ ] $\lim\limits_{n\to\infty }b_n = 0$ 
     \end{itemize}
     \textbf{Alors}, 
     \begin{itemize}
         \item[$\blacktriangleright$]
     $\sum  (-1)^nb_n \textbf{ conv. }$ vers $s \in
     \mathbb{R} \;\; \forall 
     m \geq n_0 $ \textbf{et}  
        \item[$\blacktriangleright$] 
    $|s - s_m| \leq b_{m+1}$
     \end{itemize}

  \section{Définition de convergence absolue}
     \mbox{}\\
     \[ \sum_{n=1}^{\infty }|a_n| \textbf{conv.} \implies \sum_{n=1}^{\infty }a_n  
     \textbf{ conv. } \textit{absoluement} \]
      \\
      $\rhd$  \textbf{Semi-conv.} : $\sum_{n=1}^{\infty }a_n$ conv. 
         \texttt{\&\&} $\neg(\sum_{n=1}^{\infty }|a_n|)$ 
      $\blacktriangleright$  \textbf{Exemple} $\sum_{n=1}^{\infty }\frac{(-1)^n}{n}$ conv 
          \textbf{mais} $\neg(\sum_{n=1}^{\infty }\frac{1}{n})$ conv.  

  \section{Test du rapport (d'Alembert)}
  \mbox{}\vspace{0.2em}\\
  Soit $\lim\limits_{n \to+\infty } \Big|\frac{a_{n+1}}{a_n}  \Big| = L$ 
  \textbf{alors} si :   
  \begin{itemize}
    \item [$\rhd$ ] $L = 1 \implies$ \textit{ inconclusif}  
    \item [$\blacktriangleright$ ] $L > 1 \implies \sum_{n=1}^{\infty } a_n$ 
      \textbf{div}.   
    \item [$\blacktriangleright$ ] $L < 1 \implies \sum_{n=1}^{\infty } a_n$ \textbf{conv}.   
  \end{itemize}



  \section{Test de Cauchy}
  \mbox{}\vspace{0.2em}
  Soit $\lim\limits_{n \to+\infty } \sqrt[n]{\big| a_n \big|} = L$ 
  \textbf{alors} si :   
  \begin{itemize}
    \item [$\rhd$ ] $L = 1 \implies$ \textit{inconclusif}  
    \item [$\blacktriangleright$ ] $L > 1 \implies \sum_{n=1}^{\infty }a_n$ 
      \textbf{div}.   
    \item [$\blacktriangleright$ ] $L < 1 \implies \sum_{n=1}^{\infty }a_n$ 
      \textbf{conv}.
  \end{itemize}

  \section{Défininition d'une série entière} Soit une variable 
  $x$, les contantes $c_0, c_1, \cdots , c_n$ 
  et $n \in \mathbb{N}$, une série 
  est dite centrée en $a \in \mathbb{R}$ si on a :  
  \[ \sum_{n=0}^{\infty } c_n(x-a)^n = c_0 + c_1(x-a) + c_2(x-a)^2 + \cdots \]

  \section{Famille de séries paramétrées par $x$} 
  Soit l'ensemble $D$ des $x$ pour lesquels la série converge, 
  la fonction $f : D \rightarrow  \mathbb{R}$ engendre une 
  somme $f(x) \in \mathbb{D}$ :
  \[ f(x) = \sum_{n=0}^{\infty } c_n(x - a)^n \]
  \vspace{-1.5em} %vspace

  \begin{table}[H]
    \begin{center}
      \renewcommand{\arraystretch}{2}
      \fontfamily{flr}\selectfont
      \footnotesize
      \begin{tabular}{l|p{3.75cm}}
          \arrayrulecolor{blue}\hline
          \rowcolor{lightBlue}
          \textcolor{myb}{Fonction} & \textcolor{myb}{Somme}
          \\
          \hline
          \arrayrulecolor{black}
          $\sin x$ & $x - \frac{x^3}{3!} +  \frac{x^5}{5!} 
          -  \frac{x^7}{7!} + \dots $ 
          \\
          \hline
          $\cos x$ & $1 - \frac{x^2}{2!} +  \frac{x^4}{4!} 
          -  \frac{x^6}{6!} + \dots $ 
          \\
          \hline
          $\dfrac{1}{1-x}$ & $\forall \; |x| <1 \textcolor{red}{\colon}  
          1 + x + x^2 + x^3 + \dots $  
          \\
          \hline 
          $e^x$ & $1 + x + \frac{x^2}{2!} + \frac{x^3}{3!} \cdots$ 
          \\
          \hline
          $\ln(1 + x)$ & $\forall \; |x| < 1 \textcolor{red}{\colon} 
          x - \frac{x^2}{2} +  \frac{x^3}{3} 
          -  \frac{x^4}{4} + \dots $ 
          \\
          \hline
          $\arctan (x)$ & $\forall \; |x| < 1 \textcolor{red}{\colon}  
          x - \frac{x^3}{3} +  \frac{x^5}{5!} 
          -  \frac{x^7}{7!} + \dots $ 
          \\
          \hline
          $(1 + x)^k$ &           
              $\forall \; |x| < 1 \textcolor{red}{\colon}  
              1+ kx + \frac{k(k-1)}{2!}x^2 
                    + \frac{k(k-1)(k-2)}{3!}x^3 
              \cdots$
          \\
          \hline

          \end{tabular}
  \end{center}
  \end{table}

  \section{Rayon de convergence}
  Soit la série $S = \sum_{n=0}^{\infty }c_n(x - a)^n$, il a 
  \textit{trois possibilités} :   
  \begin{itemize}
    \item [$\rhd$ ] $S$ \textbf{conv.} \textcolor{red}{à}   
           $x = a \implies  R = 0$  
    \item [$\rhd$ ] $\forall \; x, \; S \textbf{ conv.} \implies  
                 R = \infty$       
    \item [$\rhd$ ] $\exists R > 0 \colon $ 
        \begin{itemize}
          \item [$\blacktriangleright$ ] $|x - a| < R \implies 
            S \textbf{ conv.}  $ 
          \item [$\blacktriangleright$ ] $|x - a| > R \implies 
            S \textbf{ div.}$ 
        \end{itemize}
  \end{itemize}

  \section{Intervale de convergence}
  Soit $R$, le rayon de convergence d'une série 
  $\sum_{n=1}^{\infty }c_n(x - a)^n$,  
  l'\textbf{intervale de convergence} $I$ est donné par : 
  \begin{itemize}
    \item [$\rhd$ ]\textbf{Si}  $R = 0 \implies  I = a = [a, a]$
    \item [$\rhd$ ] Si $R = \infty \implies  I = \mathbb{R} $
    \item [$\rhd$ ] \textbf{Si} $R > 0$ \textbf{et} $R \in \mathbb{R}$    
      \begin{itemize}
        \item [$\blacktriangleright$ ] $I = ]a - R, a + R[$
        \item [$\blacktriangleright$ ] $I = ]a - R, a + R]$
        \item [$\blacktriangleright$ ] $I = [ a - R, a + R[$
        \item [$\blacktriangleright$ ] $I = [ a - R, a + R] $
      \end{itemize}
  \end{itemize}

  \section{Dérivation et intégration termes à termes}
  Soit la série $S = \sum_{n=0}^{\infty }c_n(x - a)^n$ 
  ayant un rayon de convergence $R > 0$, et $f(x) = S$ est 
  \textbf{dérivable} sur $(a-R, a + R)$, \textbf{alors},     
  \begin{itemize}
    \item [$\rhd$ ] $f^{\prime}(x) = \sum_{n=1}^{\infty }na_n(x-a)^{n-1}$ 
    \item [$\rhd$ ] $\int f(x) dx = 
           C + \sum_{n=0}^{\infty }\frac{(x-a)^{n+1}}{n+1}$
  \end{itemize}

  \section{Expression d'une fonction en série géométrique}
  Chaque fonction $f : (a - R, a + R) \rightarrow \mathbb{R}$ peut être 
  approximé par une série géométrique :
  \begin{align*}
      f(x) &= \sum_{n=0}^{\infty }c_n(x - a)^n \\ 
           &= c_0 + c_1(x - a) + c_2(x - a)^2 + \dots
  \end{align*}    
  Pour trouver les \textbf{coefficients} $c_0, c_1, \dots c_n$, on a :

  \tiny{
  
  \begin{align*}
    &f(a)  = \textcolor{red}{c_0}   + 
      \cancel{c_1(a - a)} + \cancel{c_2(a - a)^2} + \cdots 
            \cancel{c_n(a - a)^n } \\ 
    &f^{\prime}(a) = 0 + \textcolor{red}{c_1} + 
          \cancel{2c_2(a - a)} + \cancel{3c_3(a - a)^2} + \cdots 
          + \cancel{nc_n(a - a)^n} \\
    &f^{\prime\prime}(a) = 0 + 0 + \textcolor{red}{2c_2} + 
          \cancel{6c_2(a - a)} + \cancel{12c_4(a - a)^2} + \cdots 
          + \cancel{nc_n(a - a)^n} \\
    &f^{\prime\prime\prime}(a) = \textcolor{red}{6(c_3)} 
                = 3!c_3 \\
    &f^{4}(a) = \textcolor{red}{24c_4} = 4!c_4  
  \end{align*}  
  }
  \noindent
  On peut donc exprimer les coefficients en généralisant et on 
  obtient 
  alors la formule de Taylor :
  \begin{align*}
        c_n =  \frac{f^{n}(a)}{n!} 
  \end{align*}

  Ainsi, nous avons l'expression générale : 
  
  \begin{align*}
      &\sum_{n=0}^{\infty } \frac{f^{n}(a)}{n!}(x - a)^n 
       \textbf{ Taylor}   
      \\ 
      &\sum_{n=0}^{\infty } \frac{f^{n}(0)}{n!}(x)^n 
       \textbf{ McLaurin}   
      \\
      &f(a) + \frac{f^{\prime}(a)}{1!}(x - a) + 
             \frac{f^{\prime\prime}(a)}{2!}(x - a)^2 + 
             \cdots \textbf{Expression générale}  
  \end{align*}    

  \section{Polynôme de Taylor}
  Si une $f(x)$ est \textit{infiniement dérivable}, 
  et est représentée par une série entière 
  \textbf{alors} cette série est la 
  \textit{série de Taylor} de $f$.   
  \vspace{-1em}

  \begin{align*}
    T_n(x) = f(a) &+ \frac{f^{\prime}(a)}{1!}(x - a) + 
             \frac{f^{\prime\prime}(a)}{2!}(x - a)^2  
             \\
       &+      \cdots
      \frac{f^{n}(a)}{n!}(x - a)^n 
      \textbf{ Polynôme de Taylor}             
  \end{align*}

  \section{Reste du polynôme de Taylor}
  Soit le reste $R_n(x) =  f(x) - T_n(x)$, si 
  $\lim\limits_{n \to+\infty }R_n(x) = 0 
  \;\forall \;\; |x - a| < R $, \textbf{alors}  :
  \[ 
        f(x) = \sum_{n=0}^{\infty } 
        \frac{f^{n}(a)}{n!}(x - a)^n   
    \]

    Le \textbf{reste du polynôme de Taylor} ou 
    erreur sur l'approximation de Taylor 
    $R_n(x)$ est la différence entre la valeur 
    réelle de la fonction $f(x)$ 
    et l'approximation donnée par le polynôme de 
    Taylor de degré $n$ noté $T_n(x)$
    Si ce reste tend vers zéro pour $n$ tendant 
    vers l'infini, pour tout $x$ dans un 
    intervalle autour de $a$ de rayon $R$, 
    cela signifie que le polynôme de Taylor 
    converge vers la fonction $f(x)$. dans cet 
    intervalle. Cela indique que l'on peut 
    représenter $f(x)$ aussi précisément que 
    souhaité (dans cet intervalle) en augmentant 
    le degré $n$ du polynôme de Taylor.


  \section{Inégalité de Taylor}
  Soit $M, d \in \mathbb{R}$ deux 
  \textbf{constantes positives}  
  et $|f^{n+1}(x)| \leq M, \;\; \forall \;\; |x - a| < d$, 
  \textbf{alors}
  \[ 
    |R_n(x)| \leq \frac{M}{(n+1)!}|x - a|^{n+1}, \;\; 
        \forall \;\; |x - a | < d 
  \]

  L'\textbf{inégalité de Taylor} fournit une 
  estimation de l'erreur (le reste $R_n(x)$)
  commise en utilisant le polynôme de Taylor de 
  degré $n$ pour approximer $f(x)$  
  Si on connaît une borne supérieure 
  $M$ pour la $(n+1)-ième $ dérivée de 
  $f$ dans un intervalle autour de 
  $a$, alors on peut utiliser cette borne pour 
  estimer l'erreur maximale de l'approximation 
  sur cet intervalle.

  \section{Corollaire de l'inégalité de Taylor}
  Soient $N, c, d \in \mathbb{R}$ des 
  \textbf{constantes positives}   
  et soit $a \in \mathbb{R}$. Si 
  $\forall \;\; n > N$ et $|x - a | < d$ 
  \textbf{et}  si on a  
  \[ |f^{n}(x)| \leq c^n \textbf{  ou } 
  |f^{n}(x)| \leq c \text{ (version faible) } \]
  Cela implique (dans la cas le plus fort) que 

  \begin{align*}
      f^{n+1}(x) \leq c^{n+1}
  \end{align*}
  \textbf{Et par conséquent}, $f(x)$ est 
  représentée par sa série de 
  Taylor sur $$(a - d, a + d)$$. 

  et on a alors 

  \begin{align*}
    |R_n(x)| \leq \frac{c^{n+1}}{(n+1)!} 
          \left|x - a \right|^{n+1}
  \end{align*}

  Le \textbf{ corollaire de l'inégalité de Taylor}
  étend l'inégalité de Taylor en considérant 
  des conditions sur les dérivées successives de 
  $f$. Il indique que si les dérivées de 
  $f$ au-delà d'un certain ordre $N$ croissent 
  d'une manière contrôlée 
  (soit proportionnellement à $c^n$
  ou restent sous une certaine constante 
  $c$), alors $f(x)$ peut être approximée par 
  sa série de Taylor dans un intervalle autour de 
  $a$




  











     











 \end{multicols*}
\end{document}
