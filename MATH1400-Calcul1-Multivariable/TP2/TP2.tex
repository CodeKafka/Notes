\documentclass{report}
%\usepackage[utopia]{mathdesign}
%\usepackage{amsmath, amsthm}
\usepackage{pgfplots}


\usepackage{amsmath,amsfonts,amsthm,amssymb,mathtools}
%\usepackage[varbb]{newpxmath}
%\usepackage[osf,largesc,theoremfont]{newpxtext}
%\usepackage{coelacanth}
%\usepackage{beraserif} % Bitstream Vera Serif font
%\usepackage{berasans} % Bitstream Vera Sans font
%\usepackage{beramono} % Bitstream Vera Sans Mono font
%\usepackage{berasans}
%\usepackage{libertine}
%\usepackage{mathpazo}
%\usepackage{palatino}
%\usepackage{crimson}


%% Choose one of the following (if not choosing the  
%% default, viz., Computer Modern, font family):
%\usepackage{lmodern}
\usepackage{bold-extra}
%%
%\usepackage{mathpazo}
% \usepackage{newpxmath}
%\usepackage{kpfonts} % Very good
%%
%\usepackage{mathptmx} %Very good
%\usepackage{stix} 
%\usepackage{txfonts} %Very good
%\usepackage{newtxtext,newtxmath} %Very good
%%
%\usepackage{libertine} \usepackage[libertine]{newtxmath}
%\usepackage{libertine,libertinust1math} % added 2019/11/28
%%
%\usepackage{newpxtext} 
%\usepackage{breqn} 
\usepackage[euler-digits]{eulervm}
%\usepackage{textcomp}
%\usepackage{bm}
\usepackage{contour}
\usepackage{adjustbox}






%%%%%%%%%%%%%%%%%%%%%%%%%%%%%%%%%%%%%%%%%%%%%%%%%%%%%%%%%%%%%%%%%%%%%%%%%%%%%%%%%%%%%%%%%%%%%%%%%
%                                 Additional Packages (commented)
%%%%%%%%%%%%%%%%%%%%%%%%%%%%%%%%%%%%%%%%%%%%%%%%%%%%%%%%%%%%%%%%%%%%%%%%%%%%%%%%%%%%%%%%%%%%%%%%%
%%%%%%%%%%%%%%%%%%%%%%%%%%%%%%%%%%%%%%%%%%%%%%%%%%%%%%%%%%%%%%%%%%%%%%%%%%%%%%%%%%%%%%%%%%%%%%%%%
%                                 Language and Encoding
%%%%%%%%%%%%%%%%%%%%%%%%%%%%%%%%%%%%%%%%%%%%%%%%%%%%%%%%%%%%%%%%%%%%%%%%%%%%%%%%%%%%%%%%%%%%%%%%%
\usepackage[french]{babel} % Utilisation du français pour la typographie et les hyphenations
\usepackage[T1]{fontenc} % Utilisation de l'encodage de police T1 pour une meilleure gestion des caractères accentués
\usepackage{titlesec}
\usepackage[utf8]{inputenc} % Permet l'utilisation de l'encodage UTF-8 dans le fichier source
\usepackage{csquotes} % Gestion avancée des guillemets
\usepackage{microtype}
\usepackage{listings}

%%%%%%%%%%%%%%%%%%%%%%%%%%%%%%%%%
% PACKAGE IMPORTS
%%%%%%%%%%%%%%%%%%%%%%%%%%%%%%%%%


\usepackage[tmargin=2cm,rmargin=1in,lmargin=1in,margin=0.85in,bmargin=2cm,footskip=.2in]{geometry}
\usepackage{xfrac}
\usepackage[makeroom]{cancel}
\usepackage{bookmark}
\usepackage{enumitem}
\usepackage{hyperref,theoremref}
\hypersetup{
	pdftitle={Assignment},
	colorlinks=true, linkcolor=black,
	bookmarksnumbered=true,
	bookmarksopen=true
}
\hypersetup{
    colorlinks=true,       % false: boxed links; true: colored links
    linkcolor=black,       % color of internal links
    citecolor=black,       % color of citation links
    filecolor=black,       % color of file links
    urlcolor=black         % color of external links (like \url and \href)
}
\usepackage[most,many,breakable]{tcolorbox}
\usepackage{xcolor}
\usepackage{varwidth}
\usepackage{varwidth}
\usepackage{etoolbox}
%\usepackage{authblk}
\usepackage{nameref}
\usepackage{multicol,array}
\usepackage{colortbl}
\usepackage{tikz-cd}
\usepackage[ruled,vlined,linesnumbered]{algorithm2e}
\usepackage{comment} % enables the use of multi-line comments (\ifx \fi) 
\usepackage{import}
\usepackage{xifthen}
\usepackage{pdfpages}
\usepackage{transparent}

\newcommand\mycommfont[1]{\footnotesize\ttfamily\textcolor{blue}{#1}}
\SetCommentSty{mycommfont}
\newcommand{\incfig}[1]{%
    \def\svgwidth{\columnwidth}
    \import{./figures/}{#1.pdf_tex}
}

\usepackage{tikzsymbols}
%\renewcommand\qedsymbol{$\Laughey$}


%\usepackage{import}
%\usepackage{xifthen}
%\usepackage{pdfpages}
%\usepackage{transparent}


%%%%%%%%%%%%%%%%%%%%%%%%%%%%%%
% SELF MADE COLORS
%%%%%%%%%%%%%%%%%%%%%%%%%%%%%%



\definecolor{lightBlue}{rgb}{0.88,1,1}
\definecolor{myg}{RGB}{56, 140, 70}
\definecolor{myb}{RGB}{45, 111, 177}
\definecolor{myr}{RGB}{199, 68, 64}
\definecolor{mytheorembg}{HTML}{F2F2F9}
\definecolor{mytheoremfr}{HTML}{00007B}
\definecolor{mylenmabg}{HTML}{FFFAF8}
\definecolor{mylenmafr}{HTML}{983b0f}
\definecolor{mypropbg}{HTML}{f2fbfc}
\definecolor{mypropfr}{HTML}{191971}
\definecolor{myexamplebg}{HTML}{F2FBF8}
\definecolor{myexamplefr}{HTML}{88D6D1}
\definecolor{myexampleti}{HTML}{2A7F7F}
\definecolor{mydefinitbg}{HTML}{E5E5FF}
\definecolor{mydefinitfr}{HTML}{3F3FA3}
\definecolor{notesgreen}{RGB}{0,162,0}
\definecolor{myp}{RGB}{197, 92, 212}
\definecolor{mygr}{HTML}{2C3338}
\definecolor{myred}{RGB}{127,0,0}
\definecolor{myyellow}{RGB}{169,121,69}
\definecolor{myexercisebg}{HTML}{F2FBF8}
\definecolor{myexercisefg}{HTML}{88D6D1}
\definecolor{myblue}{RGB}{0,82,155}

\usepackage{forest}
\usepackage{adjustbox}



%%%%%%%%%%%%%%%%%%%%%%%%%%%%
% chapter format
%%%%%%%%%%%%%%%%%%%%%%%%%%%%

\titleformat{\chapter}[display]
  {\normalfont\bfseries\color{doc!60}}
  {\filleft%
    \begin{tikzpicture}
    \node[
      outer sep=-4pt,
      text width=0.75cm,
      minimum height=1cm,
      fill=doc!60,
      font=\color{white}\fontsize{20}{25}\selectfont\fontfamily{lmss},
      align=center
      ] (num) {\thechapter};
    \node[
      rotate=90,
      anchor=south,
      font=\color{black}\normalsize\fontfamily{lmss}
      ] at ([xshift=-5pt]num.west) {\textls[180]{\textsc{}}};  
    \end{tikzpicture}%
  }
  {10pt}
  {\titlerule[2.5pt]\vskip1.5pt\titlerule\vskip4pt\large\bfseries\sc}
\titlespacing*{\chapter}{0pt}{*0}{*0}

\makeatletter
\patchcmd{\chapter}{\if@openright\cleardoublepage\else\clearpage\fi}{}{}{}
\makeatother




%%%%%%%%%%%%%%%%%%%%%%%%%%%%
% TCOLORBOX SETUPS
%%%%%%%%%%%%%%%%%%%%%%%%%%%%

\setlength{\parindent}{1cm}
%================================
% THEOREM BOX
%================================

\tcbuselibrary{theorems,skins,hooks}
\newtcbtheorem[number within=section]{Theorem}{Theorem}
{%
	enhanced,
	breakable,
	colback = mytheorembg,
	frame hidden,
	boxrule = 0sp,
	borderline west = {2pt}{0pt}{mytheoremfr},
	sharp corners,
	detach title,
	before upper = \tcbtitle\par\smallskip,
	coltitle = mytheoremfr,
	fonttitle = \bfseries\sffamily,
	description font = \mdseries,
	separator sign none,
	segmentation style={solid, mytheoremfr},
}
{th}

\tcbuselibrary{theorems,skins,hooks}
\newtcbtheorem[number within=chapter]{theorem}{Theorem}
{%
	enhanced,
	breakable,
	colback = mytheorembg,
	frame hidden,
	boxrule = 0sp,
	borderline west = {2pt}{0pt}{mytheoremfr},
	sharp corners,
	detach title,
	before upper = \tcbtitle\par\smallskip,
	coltitle = mytheoremfr,
	fonttitle = \bfseries\sffamily,
	description font = \mdseries,
	separator sign none,
	segmentation style={solid, mytheoremfr},
}
{th}


\tcbuselibrary{theorems,skins,hooks}
\newtcolorbox{Theoremcon}
{%
	enhanced
	,breakable
	,colback = mytheorembg
	,frame hidden
	,boxrule = 0sp
	,borderline west = {2pt}{0pt}{mytheoremfr}
	,sharp corners
	,description font = \mdseries
	,separator sign none
}

%================================
% Preuve
%================================

% Crée un environnement "Preuve" numéroté en fonction du document
\tcbuselibrary{theorems,skins,hooks}
\newtcbtheorem[number within=chapter]{Preuve}{Preuve}
{
	enhanced,
	breakable,
	colback=white,
	frame hidden,
	boxrule = 0sp,
	borderline west = {2pt}{0pt}{mytheoremfr},
	sharp corners,
	detach title,
	before upper = \tcbtitle\par\smallskip,
	coltitle = mytheoremfr,
	description font=\fontfamily{lmss}\selectfont,
	fonttitle=\fontfamily{lmss}\selectfont\bfseries,
	separator sign none,
	segmentation style={solid, mytheoremfr},
}
{th}



%================================
% Corollery
%================================
\tcbuselibrary{theorems,skins,hooks}
\newtcbtheorem[number within=section]{Corollary}{Corollaire}
{%
	enhanced
	,breakable
	,colback = myp!10
	,frame hidden
	,boxrule = 0sp
	,borderline west = {2pt}{0pt}{myp!85!black}
	,sharp corners
	,detach title
	,before upper = \tcbtitle\par\smallskip
	,coltitle = myp!85!black
	,fonttitle = \bfseries\sffamily
	,description font = \mdseries
	,separator sign none
	,segmentation style={solid, myp!85!black}
}
{th}
\tcbuselibrary{theorems,skins,hooks}
\newtcbtheorem[number within=chapter]{corollary}{Corollaire}
{%
	enhanced
	,breakable
	,colback = myp!10
	,frame hidden
	,boxrule = 0sp
	,borderline west = {2pt}{0pt}{myp!85!black}
	,sharp corners
	,detach title
	,before upper = \tcbtitle\par\smallskip
	,coltitle = myp!85!black
	,fonttitle = \bfseries\sffamily
	,description font = \mdseries
	,separator sign none
	,segmentation style={solid, myp!85!black}
}
{th}


%================================
% LENMA
%================================

\tcbuselibrary{theorems,skins,hooks}
\newtcbtheorem[number within=section]{Lenma}{Lemme}
{%
	enhanced,
	breakable,
	colback = mylenmabg,
	frame hidden,
	boxrule = 0sp,
	borderline west = {2pt}{0pt}{mylenmafr},
	sharp corners,
	detach title,
	before upper = \tcbtitle\par\smallskip,
	coltitle = mylenmafr,
	fonttitle = \bfseries\sffamily,
	description font = \mdseries,
	separator sign none,
	segmentation style={solid, mylenmafr},
}
{th}

\tcbuselibrary{theorems,skins,hooks}
\newtcbtheorem[number within=chapter]{lenma}{Lemme}
{%
	enhanced,
	breakable,
	colback = mylenmabg,
	frame hidden,
	boxrule = 0sp,
	borderline west = {2pt}{0pt}{mylenmafr},
	sharp corners,
	detach title,
	before upper = \tcbtitle\par\smallskip,
	coltitle = mylenmafr,
	fonttitle = \bfseries\sffamily,
	description font = \mdseries,
	separator sign none,
	segmentation style={solid, mylenmafr},
}
{th}


%================================
% PROPOSITION
%================================

\tcbuselibrary{theorems,skins,hooks}
\newtcbtheorem[number within=section]{Prop}{Proposition}
{%
	enhanced,
	breakable,
	colback = mypropbg,
	frame hidden,
	boxrule = 0sp,
	borderline west = {2pt}{0pt}{mypropfr},
	sharp corners,
	detach title,
	before upper = \tcbtitle\par\smallskip,
	coltitle = mypropfr,
	fonttitle = \bfseries\sffamily,
	description font = \mdseries,
	separator sign none,
	segmentation style={solid, mypropfr},
}
{th}

\tcbuselibrary{theorems,skins,hooks}
\newtcbtheorem[number within=chapter]{prop}{Proposition}
{%
	enhanced,
	breakable,
	colback = mypropbg,
	frame hidden,
	boxrule = 0sp,
	borderline west = {2pt}{0pt}{mypropfr},
	sharp corners,
	detach title,
	before upper = \tcbtitle\par\smallskip,
	coltitle = mypropfr,
	fonttitle = \bfseries\sffamily,
	description font = \mdseries,
	separator sign none,
	segmentation style={solid, mypropfr},
}
{th}


%================================
% CLAIM
%================================

\tcbuselibrary{theorems,skins,hooks}
\newtcbtheorem[number within=section]{claim}{Affirmation}
{%
	enhanced
	,breakable
	,colback = myg!10
	,frame hidden
	,boxrule = 0sp
	,borderline west = {2pt}{0pt}{myg}
	,sharp corners
	,detach title
	,before upper = \tcbtitle\par\smallskip
	,coltitle = myg!85!black
	,fonttitle = \bfseries\sffamily
	,description font = \mdseries
	,separator sign none
	,segmentation style={solid, myg!85!black}
}
{th}



%================================
% Exercise
%================================

\tcbuselibrary{theorems,skins,hooks}
\newtcbtheorem[number within=section]{Exercise}{Exercice}
{%
	enhanced,
	breakable,
	colback = myexercisebg,
	frame hidden,
	boxrule = 0sp,
	borderline west = {2pt}{0pt}{myexercisefg},
	sharp corners,
	detach title,
	before upper = \tcbtitle\par\smallskip,
	coltitle = myexercisefg,
	fonttitle = \bfseries\sffamily,
	description font = \mdseries,
	separator sign none,
	segmentation style={solid, myexercisefg},
}
{th}

\tcbuselibrary{theorems,skins,hooks}
\newtcbtheorem[number within=chapter]{exercise}{Exercice}
{%
	enhanced,
	breakable,
	colback = myexercisebg,
	frame hidden,
	boxrule = 0sp,
	borderline west = {2pt}{0pt}{myexercisefg},
	sharp corners,
	detach title,
	before upper = \tcbtitle\par\smallskip,
	coltitle = myexercisefg,
	fonttitle = \bfseries\sffamily,
	description font = \mdseries,
	separator sign none,
	segmentation style={solid, myexercisefg},
}
{th}

%================================
% EXAMPLE BOX
%================================

\newtcbtheorem[number within=section]{Example}{Example}
{%
	colback = myexamplebg
	,breakable
	,colframe = myexamplefr
	,coltitle = myexampleti
	,boxrule = 1pt
	,sharp corners
	,detach title
	,before upper=\tcbtitle\par\smallskip
	,fonttitle = \bfseries
	,description font = \mdseries
	,separator sign none
	,description delimiters parenthesis
}
{ex}

\newtcbtheorem[number within=chapter]{example}{Example}
{%
	colback = myexamplebg
	,breakable
	,colframe = myexamplefr
	,coltitle = myexampleti
	,boxrule = 1pt
	,sharp corners
	,detach title
	,before upper=\tcbtitle\par\smallskip
	,fonttitle = \bfseries
	,description font = \mdseries
	,separator sign none
	,description delimiters parenthesis
}
{ex}

%================================
% DEFINITION BOX
%================================

\newtcbtheorem[number within=chapter]{Definition}{Définition}{enhanced,
	before skip=2mm,after skip=2mm, colback=red!5,colframe=red!80!black,boxrule=0.5mm,
	attach boxed title to top left={xshift=1cm,yshift*=1mm-\tcboxedtitleheight}, varwidth boxed title*=-3cm,
	boxed title style={frame code={
			\path[fill=tcbcolback!10!red]
			([yshift=-1mm,xshift=-1mm]frame.north west)
			arc[start angle=0,end angle=180,radius=1mm]
			([yshift=-1mm,xshift=1mm]frame.north east)
			arc[start angle=180,end angle=0,radius=1mm];
			\path[left color=tcbcolback!10!myred,right color=tcbcolback!10!myred,
			middle color=tcbcolback!60!myred]
			([xshift=-2mm]frame.north west) -- ([xshift=2mm]frame.north east)
			[rounded corners=1mm]-- ([xshift=1mm,yshift=-1mm]frame.north east)
			-- (frame.south east) -- (frame.south west)
			-- ([xshift=-1mm,yshift=-1mm]frame.north west)
			[sharp corners]-- cycle;
		},interior engine=empty,
	},
	fonttitle=\bfseries,
	title={#2},#1}{def}
\newtcbtheorem[number within=chapter]{definition}{Definition}{enhanced,
	before skip=2mm,after skip=2mm, colback=red!5,colframe=red!80!black,boxrule=0.5mm,
	attach boxed title to top left={xshift=1cm,yshift*=1mm-\tcboxedtitleheight}, varwidth boxed title*=-3cm,
	boxed title style={frame code={
					\path[fill=tcbcolback]
					([yshift=-1mm,xshift=-1mm]frame.north west)
					arc[start angle=0,end angle=180,radius=1mm]
					([yshift=-1mm,xshift=1mm]frame.north east)
					arc[start angle=180,end angle=0,radius=1mm];
					\path[left color=tcbcolback!60!black,right color=tcbcolback!60!black,
						middle color=tcbcolback!80!black]
					([xshift=-2mm]frame.north west) -- ([xshift=2mm]frame.north east)
					[rounded corners=1mm]-- ([xshift=1mm,yshift=-1mm]frame.north east)
					-- (frame.south east) -- (frame.south west)
					-- ([xshift=-1mm,yshift=-1mm]frame.north west)
					[sharp corners]-- cycle;
				},interior engine=empty,
		},
	fonttitle=\bfseries,
	title={#2},#1}{def}


    \newtcbtheorem{Definitionx}{Définition}
    {
    enhanced,
    breakable,
    colback=red!5,
      before upper=\tcbtitle\par\Hugeskip,
    frame hidden,
    boxrule = 0sp,
    borderline west = {2pt}{0pt}{red!80!black},
    sharp corners,
    detach title,
    before upper = \tcbtitle\par\smallskip,
    coltitle = red!80!black,
    description font=\mdseries\fontfamily{lmss}\selectfont,
    fonttitle=\fontfamily{lmss}\selectfont\bfseries,
    fontlower=\fontfamily{cmr}\selectfont,
      separator sign none,
    segmentation style={solid, mytheoremfr},
    }
    {th}



%================================
% Solution BOX
%================================

\makeatletter
\newtcbtheorem{question}{Question}{enhanced,
	breakable,
	colback=white,
	colframe=myb!80!black,
	attach boxed title to top left={yshift*=-\tcboxedtitleheight},
	fonttitle=\bfseries,
	title={#2},
	boxed title size=title,
	boxed title style={%
			sharp corners,
			rounded corners=northwest,
			colback=tcbcolframe,
			boxrule=0pt,
		},
	underlay boxed title={%
			\path[fill=tcbcolframe] (title.south west)--(title.south east)
			to[out=0, in=180] ([xshift=5mm]title.east)--
			(title.center-|frame.east)
			[rounded corners=\kvtcb@arc] |-
			(frame.north) -| cycle;
		},
	#1
}{def}
\makeatother

%================================
% SOLUTION BOX
%================================

\makeatletter
\newtcolorbox{solution}{enhanced,
	breakable,
	colback=white,
	colframe=myg!80!black,
	attach boxed title to top left={yshift*=-\tcboxedtitleheight},
	title=Solution,
	boxed title size=title,
	boxed title style={%
			sharp corners,
			rounded corners=northwest,
			colback=tcbcolframe,
			boxrule=0pt,
		},
	underlay boxed title={%
			\path[fill=tcbcolframe] (title.south west)--(title.south east)
			to[out=0, in=180] ([xshift=5mm]title.east)--
			(title.center-|frame.east)
			[rounded corners=\kvtcb@arc] |-
			(frame.north) -| cycle;
		},
}
\makeatother

%================================
% Question BOX
%================================

\makeatletter
\newtcbtheorem{qstion}{Question}{enhanced,
	breakable,
	colback=white,
	colframe=mygr,
	attach boxed title to top left={yshift*=-\tcboxedtitleheight},
	fonttitle=\bfseries,
	title={#2},
	boxed title size=title,
	boxed title style={%
			sharp corners,
			rounded corners=northwest,
			colback=tcbcolframe,
			boxrule=0pt,
		},
	underlay boxed title={%
			\path[fill=tcbcolframe] (title.south west)--(title.south east)
			to[out=0, in=180] ([xshift=5mm]title.east)--
			(title.center-|frame.east)
			[rounded corners=\kvtcb@arc] |-
			(frame.north) -| cycle;
		},
	#1
}{def}
\makeatother

\newtcbtheorem[number within=chapter]{wconc}{Wrong Concept}{
	breakable,
	enhanced,
	colback=white,
	colframe=myr,
	arc=0pt,
	outer arc=0pt,
	fonttitle=\bfseries\sffamily\large,
	colbacktitle=myr,
	attach boxed title to top left={},
	boxed title style={
			enhanced,
			skin=enhancedfirst jigsaw,
			arc=3pt,
			bottom=0pt,
			interior style={fill=myr}
		},
	#1
}{def}


%%%%%%%%%%%%%%%%%%%%%%%%%%%%%%%%%%%%%%%%%%%%%%%%%%%%%%%%%%%%%%%%%%%%%%%%%%%%%%%%%%%%%%%%%%%%%%%%%
%                                Environnement Explication
%%%%%%%%%%%%%%%%%%%%%%%%%%%%%%%%%%%%%%%%%%%%%%%%%%%%%%%%%%%%%%%%%%%%%%%%%%%%%%%%%%%%%%%%%%%%%%%%%
\newtcbtheorem{Explication}{Explication}
{
	enhanced,
	breakable,
	colback=white,
	frame hidden,
	boxrule = 0sp,
	borderline west = {2pt}{0pt}{mytheoremfr},
	sharp corners,
	detach title,
	before upper = \tcbtitle\par\smallskip,
	coltitle = mytheoremfr,
	description font=\fontfamily{lmss}\selectfont,
	fonttitle=\fontfamily{lmss}\selectfont\bfseries,
	separator sign none,
	segmentation style={solid, mytheoremfr},
}
{th}



%%%%%%%%%%%%%%%%%%%%%%%%%%%%%%%%%%%%%%%%%%%%%%%%%%%%%%%%%%%%%%%%%%%%%%%%%%%%%%%%%%%%%%%%%%%%%%%%%
%                                Environnement EExample
%%%%%%%%%%%%%%%%%%%%%%%%%%%%%%%%%%%%%%%%%%%%%%%%%%%%%%%%%%%%%%%%%%%%%%%%%%%%%%%%%%%%%%%%%%%%%%%%%
% Crée un environnement "EExample" numéroté en fonction du document
\newtcbtheorem{EExample}{Exemple}
{
	enhanced,
	breakable,
	colback=white,
	frame hidden,
	boxrule = 0sp,
	borderline west = {2pt}{0pt}{myb},
	sharp corners,
	detach title,
	before upper = \tcbtitle\par\smallskip,
	coltitle = myb,
	description font=\mdseries\fontfamily{lmss}\selectfont,
	fonttitle=\fontfamily{lmss}\selectfont\bfseries,
	separator sign none,
	segmentation style={solid, mytheoremfr},
}
{th}

%%%%%%%%%%%%%%%%%%%%%%%%%%%%%%%%%%%%%%%%%%%%%%%%%%%%%%%%%%%%%%%%%%%%%%%%%%%%%%%%%%%%%%%%%%%%%%%%%
%                                Environnement Liste
%%%%%%%%%%%%%%%%%%%%%%%%%%%%%%%%%%%%%%%%%%%%%%%%%%%%%%%%%%%%%%%%%%%%%%%%%%%%%%%%%%%%%%%%%%%%%%%%%

% Pour créer un environnement "Liste" 

\tcbuselibrary{theorems,skins,hooks}
\newtcbtheorem[number within=chapter]{Liste}{Liste}
{%
	enhanced
	,breakable
	,colback = myp!10
	,frame hidden
	,boxrule = 0sp
	,borderline west = {2pt}{0pt}{myp!85!black}
	,sharp corners
	,detach title
	,before upper = \tcbtitle\par\smallskip
	,coltitle = myp!85!black
	,fonttitle = \bfseries\sffamily
	,description font = \mdseries
	,separator sign none
	,segmentation style={solid, myp!85!black}

}
{th}


%%%%%%%%%%%%%%%%%%%%%%%%%%%%%%%%%%%%%%%%%%%%%%%%%%%%%%%%%%%%%%%%%%%%%%%%%%%%%%%%%%%%%%%%%%%%%%%%%
%                                Environnement Syntaxe
%%%%%%%%%%%%%%%%%%%%%%%%%%%%%%%%%%%%%%%%%%%%%%%%%%%%%%%%%%%%%%%%%%%%%%%%%%%%%%%%%%%%%%%%%%%%%%%%%
\tcbuselibrary{theorems,skins,hooks}
\newtcbtheorem{Syntaxe}{Syntaxe.}
{%
	enhanced
	,breakable
	,colback = myp!10
	,frame hidden
	,boxrule = 0sp
	,borderline west = {2pt}{0pt}{myp!85!black}
	,sharp corners
	,detach title
	,before upper = \tcbtitle\par\smallskip
	,coltitle = myp!85!black
	,fonttitle = \bfseries\fontfamily{lmss}\selectfont 
	,description font = \mdseries\fontfamily{lmss}\selectfont 
	,separator sign none
	,segmentation style={solid, myp!85!black}
}
{th}



%%%%%%%%%%%%%%%%%%%%%%%%%%%%%%%%%%%%%%%%%%%%%%%%%%%%%%%%%%%%%%%%%%%%%%%%%%%%%%%%%%%%%%%%%%%%%%%%%
%                                Environnement Concept
%%%%%%%%%%%%%%%%%%%%%%%%%%%%%%%%%%%%%%%%%%%%%%%%%%%%%%%%%%%%%%%%%%%%%%%%%%%%%%%%%%%%%%%%%%%%%%%%%
% Crée un environnement "Concept" numéroté en fonction du document
\tcbuselibrary{theorems,skins,hooks}
\newtcbtheorem{Concept}{Concept.}
{
	enhanced,
	breakable,
	colback=mylenmabg,
	frame hidden,
	boxrule = 0sp,
	borderline west = {2pt}{0pt}{mylenmafr},
	sharp corners,
	detach title,
	before upper = \tcbtitle\par\smallskip,
	coltitle = mylenmafr,
	description font=\mdseries\fontfamily{lmss}\selectfont,
	fonttitle=\fontfamily{lmss}\selectfont\bfseries,
	separator sign none,
	segmentation style={solid, mytheoremfr},
}
{th}



%%%%%%%%%%%%%%%%%%%%%%%%%%%%%%%%%%%%%%%%%%%%%%%%%%%%%%%%%%%%%%%%%%%%%%%%%%%%%%%%%%%%%%%%%%%%%%%%%
%                                Environnement codeEx
%%%%%%%%%%%%%%%%%%%%%%%%%%%%%%%%%%%%%%%%%%%%%%%%%%%%%%%%%%%%%%%%%%%%%%%%%%%%%%%%%%%%%%%%%%%%%%%%%
% Crée un environnement "codeEx" numéroté en fonction du document
\tcbuselibrary{theorems,skins,hooks}
\newtcbtheorem{codeEx}{Exemple}
{
	enhanced,
	breakable,
	colback=white,
	frame hidden,
	boxrule = 0sp,
	borderline west = {2pt}{0pt}{gruvbox-bg},
	sharp corners,
	detach title,
	before upper = \tcbtitle\par\smallskip,
	coltitle = gruvbox-bg,
	description font=\md:wqseries\fontfamily{lmss}\selectfont,
	fonttitle=\fontfamily{lmss}\selectfont\bfseries,
	separator sign none,
	segmentation style={solid, mytheoremfr},
}
{th}



%%%%%%%%%%%%%%%%%%%%%%%%%%%%%%%%%%%%%%%%%%%%%%%%%%%%%%%%%%%%%%%%%%%%%%%%%%%%%%%%%%%%%%%%%%%%%%%%%
%                                Environnement Remarque
%%%%%%%%%%%%%%%%%%%%%%%%%%%%%%%%%%%%%%%%%%%%%%%%%%%%%%%%%%%%%%%%%%%%%%%%%%%%%%%%%%%%%%%%%%%%%%%%%
% Crée un environnement "Remarque" numéroté en fonction du document
\tcbuselibrary{theorems,skins,hooks}
\newtcbtheorem{codeRem}{Remarque}
{
	enhanced,
	breakable,
	colback=white,
	frame hidden,
	boxrule = 0sp,
	borderline west = {2pt}{0pt}{gruvbox-bg},
	sharp corners,
	detach title,
	before upper = \tcbtitle\par\smallskip,
	coltitle = gruvbox-bg,
	description font=\mdseries\fontfamily{lmss}\selectfont,
	fonttitle=\fontfamily{lmss}\selectfont\bfseries,
	separator sign none,
	segmentation style={solid, mytheoremfr},
}
{th}


%%%%%%%%%%%%%%%%%%%%%%%%%%%%%%%%%%%%%%%%%%%%%%%%%%%%%%%%%%%%%%%%%%%%%%%%%%%%%%%%%%%%%%%%%%%%%%%%%
%                                Environnement Identité
%%%%%%%%%%%%%%%%%%%%%%%%%%%%%%%%%%%%%%%%%%%%%%%%%%%%%%%%%%%%%%%%%%%%%%%%%%%%%%%%%%%%%%%%%%%%%%%%%
\tcbuselibrary{theorems,skins,hooks}
\newtcbtheorem{Identite}{Identité}
{
	enhanced,
	breakable,
	colback=white,
  before upper=\tcbtitle\par\Hugeskip,
	frame hidden,
	boxrule = 0sp,
	borderline west = {2pt}{0pt}{gruvbox-bg},
	sharp corners,
	detach title,
	before upper = \tcbtitle\par\smallskip,
	coltitle = gruvbox-bg,
	description font=\mdseries\fontfamily{lmss}\selectfont,
	fonttitle=\fontfamily{lmss}\selectfont\bfseries,
	fontlower=\fontfamily{cmr}\selectfont,
  separator sign none,
	segmentation style={solid, mytheoremfr},
}
{th}



%%%%%%%%%%%%%%%%%%%%%%%%%%%%%%%%%%%%%%%%%%%%%%%%%%%%%%%%%%%%%%%%%%%%%%%%%%%%%%%%%%%%%%%%%%%%%%%%%
%                                Environnement Exercice 
%%%%%%%%%%%%%%%%%%%%%%%%%%%%%%%%%%%%%%%%%%%%%%%%%%%%%%%%%%%%%%%%%%%%%%%%%%%%%%%%%%%%%%%%%%%%%%%%%
\tcbuselibrary{theorems,skins,hooks}
\newtcbtheorem{Exercice}{Exercice}
{
	enhanced,
	breakable,
	colback=white,
  before upper=\tcbtitle\par\Hugeskip,
	frame hidden,
	boxrule = 0sp,
	borderline west = {2pt}{0pt}{gruvbox-green},
	sharp corners,
	detach title,
	before upper = \tcbtitle\par\smallskip,
	coltitle = gruvbox-green,
	description font=\mdseries\fontfamily{lmss}\selectfont,
	fonttitle=\fontfamily{lmss}\selectfont\bfseries,
	fontlower=\fontfamily{cmr}\selectfont,
  separator sign none,
	segmentation style={solid, mytheoremfr},
}
{th}


%%%%%%%%%%%%%%%%%%%%%%%%%%%%%%%%%%%%%%%%%%%%%%%%%%%%%%%%%%%%%%%%%%%%%%%%%%%%%%%%%%%%%%%%%%%%%%%%%
%                                Environnement Réponse
%%%%%%%%%%%%%%%%%%%%%%%%%%%%%%%%%%%%%%%%%%%%%%%%%%%%%%%%%%%%%%%%%%%%%%%%%%%%%%%%%%%%%%%%%%%%%%%%%
% Crée un environnement "Réponse" numéroté en fonction du document
\tcbuselibrary{theorems,skins,hooks}
\newtcbtheorem{Reponse}{Réponse}
{
	enhanced,
	breakable,
	colback=white,
	frame hidden,
	boxrule = 0sp,
	borderline west = {2pt}{0pt}{mytheoremfr},
	sharp corners,
	detach title,
	before upper = \tcbtitle\par\smallskip,
	coltitle = mytheoremfr,
	description font=\fontfamily{lmss}\selectfont,
	fonttitle=\fontfamily{lmss}\selectfont\bfseries,
	separator sign none,
	segmentation style={solid, mytheoremfr},
}
{th}

\newtcbtheorem{Remarque}{Remarque.}
{
	enhanced,
	breakable,
	colback=white,
	frame hidden,
	boxrule = 0sp,
	borderline west = {2pt}{0pt}{myb},
	sharp corners,
	detach title,
	before upper = \tcbtitle\par\smallskip,
	coltitle = myb,
	description font=\mdseries\fontfamily{lmss}\selectfont,
	fonttitle=\fontfamily{lmss}\selectfont\bfseries,
	separator sign none,
	segmentation style={solid, mytheoremfr},
}
{th}





%================================
% NOTE BOX
%================================

\usetikzlibrary{arrows,calc,shadows.blur}
\usetikzlibrary {arrows.meta,backgrounds,fit,positioning,petri}
\tcbuselibrary{skins}
\newtcolorbox{note}[1][]{%
	enhanced jigsaw,
	colback=gray!20!white,%
	colframe=gray!80!black,
	size=small,
	boxrule=1pt,
	title=\textbf{Note:-},
	halign title=flush center,
	coltitle=black,
	breakable,
	drop shadow=black!50!white,
	attach boxed title to top left={xshift=1cm,yshift=-\tcboxedtitleheight/2,yshifttext=-\tcboxedtitleheight/2},
	minipage boxed title=1.5cm,
	boxed title style={%
			colback=white,
			size=fbox,
			boxrule=1pt,
			boxsep=2pt,
			underlay={%
					\coordinate (dotA) at ($(interior.west) + (-0.5pt,0)$);
					\coordinate (dotB) at ($(interior.east) + (0.5pt,0)$);
					\begin{scope}
						\clip (interior.north west) rectangle ([xshift=3ex]interior.east);
						\filldraw [white, blur shadow={shadow opacity=60, shadow yshift=-.75ex}, rounded corners=2pt] (interior.north west) rectangle (interior.south east);
					\end{scope}
					\begin{scope}[gray!80!black]
						\fill (dotA) circle (2pt);
						\fill (dotB) circle (2pt);
					\end{scope}
				},
		},
	#1,
}

%%%%%%%%%%%%%%%%%%%%%%%%%%%%%%
% SELF MADE COMMANDS
%%%%%%%%%%%%%%%%%%%%%%%%%%%%%%


\newcommand{\thm}[2]{\begin{Theorem}{#1}{}#2\end{Theorem}}
\newcommand{\cor}[2]{\begin{Corollary}{#1}{}#2\end{Corollary}}
\newcommand{\mlenma}[2]{\begin{Lenma}{#1}{}#2\end{Lenma}}
\newcommand{\mprop}[2]{\begin{Prop}{#1}{}#2\end{Prop}}
\newcommand{\clm}[3]{\begin{claim}{#1}{#2}#3\end{claim}}
\newcommand{\wc}[2]{\begin{wconc}{#1}{}\setlength{\parindent}{1cm}#2\end{wconc}}
\newcommand{\thmcon}[1]{\begin{Theoremcon}{#1}\end{Theoremcon}}
\newcommand{\ex}[2]{\begin{Example}{#1}{}#2\end{Example}}
\newcommand{\dfn}[2]{\begin{Definition}[colbacktitle=red!75!black]{#1}{}#2\end{Definition}}
\newcommand{\dfnc}[2]{\begin{definition}[colbacktitle=red!75!black]{#1}{}#2\end{definition}}
\newcommand{\qs}[2]{\begin{question}{#1}{}#2\end{question}}
\newcommand{\pf}[2]{\begin{myproof}[#1]#2\end{myproof}}
\newcommand{\nt}[1]{\begin{note}#1\end{note}}

\newcommand*\circled[1]{\tikz[baseline=(char.base)]{
		\node[shape=circle,draw,inner sep=1pt] (char) {#1};}}
\newcommand\getcurrentref[1]{%
	\ifnumequal{\value{#1}}{0}
	{??}
	{\the\value{#1}}%
}
\newcommand{\getCurrentSectionNumber}{\getcurrentref{section}}
\newenvironment{myproof}[1][\proofname]{%
	\proof[\bfseries #1: ]%
}{\endproof}

\newcommand{\mclm}[2]{\begin{myclaim}[#1]#2\end{myclaim}}
\newenvironment{myclaim}[1][\claimname]{\proof[\bfseries #1: ]}{}

\newcounter{mylabelcounter}

\makeatletter
\newcommand{\setword}[2]{%
	\phantomsection
	#1\def\@currentlabel{\unexpanded{#1}}\label{#2}%
}
\makeatother




\tikzset{
	symbol/.style={
			draw=none,
			every to/.append style={
					edge node={node [sloped, allow upside down, auto=false]{$#1$}}}
		}
}


% deliminators
% Manually define paired delimiters
\newcommand{\abs}[1]{\lvert#1\rvert}
\newcommand{\norm}[1]{\lVert#1\rVert}
\newcommand{\ceil}[1]{\lceil#1\rceil}
\newcommand{\floor}[1]{\lfloor#1\rfloor}
\newcommand{\round}[1]{\lfloor#1\rceil}

%
\newsavebox\diffdbox
\newcommand{\slantedromand}{{\mathpalette\makesl{d}}}
\newcommand{\makesl}[2]{%
\begingroup
\sbox{\diffdbox}{$\mathsurround=0pt#1\mathrm{#2}$}%
\pdfsave
\pdfsetmatrix{1 0 0.2 1}%
\rlap{\usebox{\diffdbox}}%
\pdfrestore
\hskip\wd\diffdbox
\endgroup
}
\newcommand{\dd}[1][]{\ensuremath{\mathop{}\!\ifstrempty{#1}{%
\slantedromand\@ifnextchar^{\hspace{0.2ex}}{\hspace{0.1ex}}}%
{\slantedromand\hspace{0.2ex}^{#1}}}}
\ProvideDocumentCommand\dv{o m g}{%
  \ensuremath{%
    \IfValueTF{#3}{%
      \IfNoValueTF{#1}{%
        \frac{\dd #2}{\dd #3}%
      }{%
        \frac{\dd^{#1} #2}{\dd #3^{#1}}%
      }%
    }{%
      \IfNoValueTF{#1}{%
        \frac{\dd}{\dd #2}%
      }{%
        \frac{\dd^{#1}}{\dd #2^{#1}}%
      }%
    }%
  }%
}
\providecommand*{\pdv}[3][]{\frac{\partial^{#1}#2}{\partial#3^{#1}}}
%  - others
\DeclareMathOperator{\Lap}{\mathcal{L}}
\DeclareMathOperator{\Var}{Var} % varience
\DeclareMathOperator{\Cov}{Cov} % covarience
\DeclareMathOperator{\E}{E} % expected

% Since the amsthm package isn't loaded

% I prefer the slanted \leq
\let\oldleq\leq % save them in case they're every wanted
\let\oldgeq\geq
\renewcommand{\leq}{\leqslant}
\renewcommand{\geq}{\geqslant}

% % redefine matrix env to allow for alignment, use r as default
% \renewcommand*\env@matrix[1][r]{\hskip -\arraycolsep
%     \let\@ifnextchar\new@ifnextchar
%     \array{*\c@MaxMatrixCols #1}}


%\usepackage{framed}
%\usepackage{titletoc}
%\usepackage{etoolbox}
%\usepackage{lmodern}


%\patchcmd{\tableofcontents}{\contentsname}{\sffamily\contentsname}{}{}

%\renewenvironment{leftbar}
%{\def\FrameCommand{\hspace{6em}%
%		{\color{myyellow}\vrule width 2pt depth 6pt}\hspace{1em}}%
%	\MakeFramed{\parshape 1 0cm \dimexpr\textwidth-6em\relax\FrameRestore}\vskip2pt%
%}
%{\endMakeFramed}

%\titlecontents{chapter}
%[0em]{\vspace*{2\baselineskip}}
%{\parbox{4.5em}{%
%		\hfill\Huge\sffamily\bfseries\color{myred}\thecontentspage}%
%	\vspace*{-2.3\baselineskip}\leftbar\textsc{\small\chaptername~\thecontentslabel}\\\sffamily}
%{}{\endleftbar}
%\titlecontents{section}
%[8.4em]
%{\sffamily\contentslabel{3em}}{}{}
%{\hspace{0.5em}\nobreak\itshape\color{myred}\contentspage}
%\titlecontents{subsection}
%[8.4em]
%{\sffamily\contentslabel{3em}}{}{}  
%{\hspace{0.5em}\nobreak\itshape\color{myred}\contentspage}



%%%%%%%%%%%%%%%%%%%%%%%%%%%%%%%%%%%%%%%%%%%
% TABLE OF CONTENTS
%%%%%%%%%%%%%%%%%%%%%%%%%%%%%%%%%%%%%%%%%%%

\usepackage{tikz}
\usetikzlibrary{automata, positioning}
\definecolor{doc}{RGB}{0,60,110}
\usepackage{titletoc}
\contentsmargin{0cm}
\titlecontents{chapter}[4.9pc]
{\addvspace{40pt}%
	\begin{tikzpicture}[remember picture, overlay]%
		\draw[fill=doc!60,draw=doc!60] (-7,-.1) rectangle (-0.9,.5);%
		\pgftext[left,x=-3.5cm,y=0.2cm]{\color{white}\Large\sc\bfseries Section \ \thecontentslabel};%
	\end{tikzpicture}\color{doc!60}\large\sc\bfseries}%
{}
{}
{\;\titlerule\;\large\sc\bfseries Page \thecontentspage
	\begin{tikzpicture}[remember picture, overlay]
		\draw[fill=doc!60,draw=doc!60] (2pt,0) rectangle (4,0.1pt);
	\end{tikzpicture}}%
\titlecontents{section}[3.7pc]
{\addvspace{2pt}}
{\contentslabel[\thecontentslabel]{2pc}}
{}
{\hfill\small \thecontentspage}
[]
\titlecontents*{subsection}[3.7pc]
{\addvspace{-1pt}\small}
{}
{}
{\ --- \small\thecontentspage}
[ \textbullet\ ][]

\makeatletter
\renewcommand{\tableofcontents}{%
	\chapter*{%
	  \vspace*{-20\p@}%
	  \begin{tikzpicture}[remember picture, overlay]%
		  \pgftext[right,x=15cm,y=0.2cm]{\color{doc!60}\Huge\sc\bfseries \contentsname};%
		  \draw[fill=doc!60,draw=doc!60] (13,-.75) rectangle (20,2);%
		  \clip (13,-.75) rectangle (20,1);
		  \pgftext[right,x=15cm,y=0.2cm]{\color{white}\Huge\sc\bfseries \contentsname};%
	  \end{tikzpicture}}%
	\@starttoc{toc}}
\makeatother






%From M275 "Topology" at SJSU
\newcommand{\id}{\mathrm{id}} % Identité
\newcommand{\taking}[1]{\xrightarrow{#1}} % Flèche avec annotation
\newcommand{\inv}{^{-1}} % Inverse

%From M170 "Introduction to Graph Theory" at SJSU
\DeclareMathOperator{\diam}{diam} % Diamètre
\DeclareMathOperator{\ord}{ord} % Ordre
\newcommand{\defeq}{\overset{\mathrm{def}}{=}} % Défini comme égal

%From the USAMO .tex files
\newcommand{\ts}{\textsuperscript} % Exposant
\newcommand{\dg}{^\circ} % Degré
\newcommand{\ii}{\item} % Item

% % From Math 55 and Math 145 at Harvard
% \newenvironment{subproof}[1][Proof]{%
% \begin{proof}[#1] \renewcommand{\qedsymbol}{$\blacksquare$}}%
% {\end{proof}}

\newcommand{\liff}{\leftrightarrow} % Si et seulement si
\newcommand{\lthen}{\rightarrow} % Implique
\newcommand{\opname}{\operatorname} % Opérateur générique
\newcommand{\surjto}{\twoheadrightarrow} % Flèche surjective
\newcommand{\injto}{\hookrightarrow} % Flèche injective
\newcommand{\On}{\mathrm{On}} % Ordinaux
\DeclareMathOperator{\img}{im} % Image
\DeclareMathOperator{\Img}{Im} % Image
\DeclareMathOperator{\coker}{coker} % Cokernel
\DeclareMathOperator{\Coker}{Coker} % Cokernel
\DeclareMathOperator{\Ker}{Ker} % Noyau
\DeclareMathOperator{\rank}{rank} % Rang
\DeclareMathOperator{\Spec}{Spec} % Spectre
\DeclareMathOperator{\Tr}{Tr} % Trace
\DeclareMathOperator{\pr}{pr} % Projection
\DeclareMathOperator{\ext}{ext} % Extension
\DeclareMathOperator{\pred}{pred} % Prédécesseur
\DeclareMathOperator{\dom}{dom} % Domaine
\DeclareMathOperator{\ran}{ran} % Image (range)
\DeclareMathOperator{\Hom}{Hom} % Homomorphisme
\DeclareMathOperator{\Mor}{Mor} % Morphismes
\DeclareMathOperator{\End}{End} % Endomorphisme

\newcommand{\eps}{\epsilon} % Épsilon
\newcommand{\veps}{\varepsilon} % Variance d'épsilon
\newcommand{\ol}{\overline} % Ligne au-dessus
\newcommand{\ul}{\underline} % Ligne en-dessous
\newcommand{\wt}{\widetilde} % Tilde large
\newcommand{\wh}{\widehat} % Chapeau large
\newcommand{\vocab}[1]{\textbf{\color{blue} #1}} % Texte en gras et bleu
\providecommand{\half}{\frac{1}{2}} % Fraction 1/2
\newcommand{\dang}{\measuredangle} % Angle dirigé
\newcommand{\ray}[1]{\overrightarrow{#1}} % Ray
\newcommand{\seg}[1]{\overline{#1}} % Segment
\newcommand{\arc}[1]{\wideparen{#1}} % Arc
\DeclareMathOperator{\cis}{cis} % cis
\DeclareMathOperator*{\lcm}{lcm} % Plus petit commun multiple
\DeclareMathOperator*{\argmin}{arg min} % Argument du minimum
\DeclareMathOperator*{\argmax}{arg max} % Argument du maximum
\newcommand{\cycsum}{\sum_{\mathrm{cyc}}} % Somme cyclique
\newcommand{\symsum}{\sum_{\mathrm{sym}}} % Somme symétrique
\newcommand{\cycprod}{\prod_{\mathrm{cyc}}} % Produit cyclique
\newcommand{\symprod}{\prod_{\mathrm{sym}}} % Produit symétrique
\newcommand{\Qed}{\begin{flushright}\qed\end{flushright}} % QED aligné à droite
\newcommand{\parinn}{\setlength{\parindent}{1cm}} % Indentation de paragraphe à 1 cm
\newcommand{\parinf}{\setlength{\parindent}{0cm}} % Pas d'indentation de paragraphe
% \newcommand{\norm}{\|\cdot\|} % Norme
\newcommand{\inorm}{\norm_{\infty}} % Norme infinie
\newcommand{\opensets}{\{V_{\alpha}\}_{\alpha\in I}} % Ensemble ouvert
\newcommand{\oset}{V_{\alpha}} % Ensemble ouvert V
\newcommand{\opset}[1]{V_{\alpha_{#1}}} % Ensemble ouvert V avec indice
\newcommand{\lub}{\text{lub}} % Plus petite borne supérieure
\newcommand{\del}[2]{\frac{\partial #1}{\partial #2}} % Dérivée partielle
\newcommand{\Del}[3]{\frac{\partial^{#1} #2}{\partial^{#1} #3}} % Dérivée partielle d'ordre élevé
\newcommand{\deld}[2]{\dfrac{\partial #1}{\partial #2}} % Dérivée partielle avec dfrac
\newcommand{\Deld}[3]{\dfrac{\partial^{#1} #2}{\partial^{#1} #3}} % Dérivée partielle d'ordre élevé avec dfrac
\newcommand{\lm}{\lambda} % Lambda
\newcommand{\uin}{\mathbin{\rotatebox[origin=c]{90}{$\in$}}} % Appartient, tourné de 90 degrés
\newcommand{\usubset}{\mathbin{\rotatebox[origin=c]{90}{$\subset$}}} % Sous-ensemble, tourné de 90 degrés
\newcommand{\lt}{\left} % Gauche
\newcommand{\rt}{\right} % Droite
\newcommand{\bs}[1]{\boldsymbol{#1}} % Symbole en gras
\newcommand{\exs}{\exists} % Il existe
\newcommand{\st}{\strut} % Strut
\newcommand{\dps}[1]{\displaystyle{#1}} % Disposition en ligne

\newcommand{\sol}{\setlength{\parindent}{0cm}\textbf{\textit{Solution:}}\setlength{\parindent}{1cm} } % Solution sans indentation initiale puis rétablie
\newcommand{\solve}[1]{\setlength{\parindent}{0cm}\textbf{\textit{Solution: }}\setlength{\parindent}{1cm}#1 \Qed}

\newcommand{\entoure}[1]{\fcolorbox{black}{gray!30}{\texttt{#1}}}

\renewcommand{\ttdefault}{cmtt}
\newcommand{\textttbf}[1]{\contour{yellow!45}{\texttt{#1}}}
\newcommand{\varitem}[3][black]{%
    \item [%
        \colorbox{#2}{\textcolor{#1}{\makebox(5.5,7){#3}}}%
    ]
}
% Allow you to do the non implication (implication barred)
\newcommand{\notimplies}{%
  \mathrel{{\ooalign{\hidewidth$\not\phantom{=}$\hidewidth\cr$\implies$}}}}


\newcommand*{\authorimg}[1]%
    { \raisebox{-1\baselineskip}{\includegraphics[width=\imagesize]{#1}}}
\newlength\imagesize 

% Things Lie
\newcommand{\kb}{\mathfrak b}
\newcommand{\kg}{\mathfrak g}
\newcommand{\kh}{\mathfrak h}
\newcommand{\kn}{\mathfrak n}
\newcommand{\ku}{\mathfrak u}
\newcommand{\kz}{\mathfrak z}
\DeclareMathOperator{\Ext}{Ext} % Ext functor
\DeclareMathOperator{\Tor}{Tor} % Tor functor
\newcommand{\gl}{\opname{\mathfrak{gl}}} % frak gl group
\renewcommand{\sl}{\opname{\mathfrak{sl}}} % frak sl group chktex 6

% More script letters etc.
\newcommand{\SA}{\mathcal A}
\newcommand{\SB}{\mathcal B}
\newcommand{\SC}{\mathcal C}
\newcommand{\SF}{\mathcal F}
\newcommand{\SG}{\mathcal G}
\newcommand{\SH}{\mathcal H}
\newcommand{\OO}{\mathcal O}

\newcommand{\SCA}{\mathscr A}
\newcommand{\SCB}{\mathscr B}
\newcommand{\SCC}{\mathscr C}
\newcommand{\SCD}{\mathscr D}
\newcommand{\SCE}{\mathscr E}
\newcommand{\SCF}{\mathscr F}
\newcommand{\SCG}{\mathscr G}
\newcommand{\SCH}{\mathscr H}

% Mathfrak primes
\newcommand{\km}{\mathfrak m}
\newcommand{\kp}{\mathfrak p}
\newcommand{\kq}{\mathfrak q}

% number sets
\newcommand{\RR}[1][]{\ensuremath{\ifstrempty{#1}{\mathbb{R}}{\mathbb{R}^{#1}}}}
\newcommand{\NN}[1][]{\ensuremath{\ifstrempty{#1}{\mathbb{N}}{\mathbb{N}^{#1}}}}
\newcommand{\ZZ}[1][]{\ensuremath{\ifstrempty{#1}{\mathbb{Z}}{\mathbb{Z}^{#1}}}}
\newcommand{\QQ}[1][]{\ensuremath{\ifstrempty{#1}{\mathbb{Q}}{\mathbb{Q}^{#1}}}}
\newcommand{\CC}[1][]{\ensuremath{\ifstrempty{#1}{\mathbb{C}}{\mathbb{C}^{#1}}}}
\newcommand{\PP}[1][]{\ensuremath{\ifstrempty{#1}{\mathbb{P}}{\mathbb{P}^{#1}}}}
\newcommand{\HH}[1][]{\ensuremath{\ifstrempty{#1}{\mathbb{H}}{\mathbb{H}^{#1}}}}
\newcommand{\FF}[1][]{\ensuremath{\ifstrempty{#1}{\mathbb{F}}{\mathbb{F}^{#1}}}}
% expected value
\newcommand{\EE}{\ensuremath{\mathbb{E}}}
\newcommand{\charin}{\text{ char }}
\DeclareMathOperator{\sign}{sign}
\DeclareMathOperator{\Aut}{Aut}
\DeclareMathOperator{\Inn}{Inn}
\DeclareMathOperator{\Syl}{Syl}
\DeclareMathOperator{\Gal}{Gal}
\DeclareMathOperator{\GL}{GL} % General linear group
\DeclareMathOperator{\SL}{SL} % Special linear group

%---------------------------------------
% BlackBoard Math Fonts :-
%---------------------------------------

%Captital Letters
\newcommand{\bbA}{\mathbb{A}}	\newcommand{\bbB}{\mathbb{B}}
\newcommand{\bbC}{\mathbb{C}}	\newcommand{\bbD}{\mathbb{D}}
\newcommand{\bbE}{\mathbb{E}}	\newcommand{\bbF}{\mathbb{F}}
\newcommand{\bbG}{\mathbb{G}}	\newcommand{\bbH}{\mathbb{H}}
\newcommand{\bbI}{\mathbb{I}}	\newcommand{\bbJ}{\mathbb{J}}
\newcommand{\bbK}{\mathbb{K}}	\newcommand{\bbL}{\mathbb{L}}
\newcommand{\bbM}{\mathbb{M}}	\newcommand{\bbN}{\mathbb{N}}
\newcommand{\bbO}{\mathbb{O}}	\newcommand{\bbP}{\mathbb{P}}
\newcommand{\bbQ}{\mathbb{Q}}	\newcommand{\bbR}{\mathbb{R}}
\newcommand{\bbS}{\mathbb{S}}	\newcommand{\bbT}{\mathbb{T}}
\newcommand{\bbU}{\mathbb{U}}	\newcommand{\bbV}{\mathbb{V}}
\newcommand{\bbW}{\mathbb{W}}	\newcommand{\bbX}{\mathbb{X}}
\newcommand{\bbY}{\mathbb{Y}}	\newcommand{\bbZ}{\mathbb{Z}}

%---------------------------------------
% MathCal Fonts :-
%---------------------------------------

%Captital Letters
\newcommand{\mcA}{\mathcal{A}}	\newcommand{\mcB}{\mathcal{B}}
\newcommand{\mcC}{\mathcal{C}}	\newcommand{\mcD}{\mathcal{D}}
\newcommand{\mcE}{\mathcal{E}}	\newcommand{\mcF}{\mathcal{F}}
\newcommand{\mcG}{\mathcal{G}}	\newcommand{\mcH}{\mathcal{H}}
\newcommand{\mcI}{\mathcal{I}}	\newcommand{\mcJ}{\mathcal{J}}
\newcommand{\mcK}{\mathcal{K}}	\newcommand{\mcL}{\mathcal{L}}
\newcommand{\mcM}{\mathcal{M}}	\newcommand{\mcN}{\mathcal{N}}
\newcommand{\mcO}{\mathcal{O}}	\newcommand{\mcP}{\mathcal{P}}
\newcommand{\mcQ}{\mathcal{Q}}	\newcommand{\mcR}{\mathcal{R}}
\newcommand{\mcS}{\mathcal{S}}	\newcommand{\mcT}{\mathcal{T}}
\newcommand{\mcU}{\mathcal{U}}	\newcommand{\mcV}{\mathcal{V}}
\newcommand{\mcW}{\mathcal{W}}	\newcommand{\mcX}{\mathcal{X}}
\newcommand{\mcY}{\mathcal{Y}}	\newcommand{\mcZ}{\mathcal{Z}}


%---------------------------------------
% Bold Math Fonts :-
%---------------------------------------

%Captital Letters
\newcommand{\bmA}{\boldsymbol{A}}	\newcommand{\bmB}{\boldsymbol{B}}
\newcommand{\bmC}{\boldsymbol{C}}	\newcommand{\bmD}{\boldsymbol{D}}
\newcommand{\bmE}{\boldsymbol{E}}	\newcommand{\bmF}{\boldsymbol{F}}
\newcommand{\bmG}{\boldsymbol{G}}	\newcommand{\bmH}{\boldsymbol{H}}
\newcommand{\bmI}{\boldsymbol{I}}	\newcommand{\bmJ}{\boldsymbol{J}}
\newcommand{\bmK}{\boldsymbol{K}}	\newcommand{\bmL}{\boldsymbol{L}}
\newcommand{\bmM}{\boldsymbol{M}}	\newcommand{\bmN}{\boldsymbol{N}}
\newcommand{\bmO}{\boldsymbol{O}}	\newcommand{\bmP}{\boldsymbol{P}}
\newcommand{\bmQ}{\boldsymbol{Q}}	\newcommand{\bmR}{\boldsymbol{R}}
\newcommand{\bmS}{\boldsymbol{S}}	\newcommand{\bmT}{\boldsymbol{T}}
\newcommand{\bmU}{\boldsymbol{U}}	\newcommand{\bmV}{\boldsymbol{V}}
\newcommand{\bmW}{\boldsymbol{W}}	\newcommand{\bmX}{\boldsymbol{X}}
\newcommand{\bmY}{\boldsymbol{Y}}	\newcommand{\bmZ}{\boldsymbol{Z}}
%Small Letters
\newcommand{\bma}{\boldsymbol{a}}	\newcommand{\bmb}{\boldsymbol{b}}
\newcommand{\bmc}{\boldsymbol{c}}	\newcommand{\bmd}{\boldsymbol{d}}
\newcommand{\bme}{\boldsymbol{e}}	\newcommand{\bmf}{\boldsymbol{f}}
\newcommand{\bmg}{\boldsymbol{g}}	\newcommand{\bmh}{\boldsymbol{h}}
\newcommand{\bmi}{\boldsymbol{i}}	\newcommand{\bmj}{\boldsymbol{j}}
\newcommand{\bmk}{\boldsymbol{k}}	\newcommand{\bml}{\boldsymbol{l}}
\newcommand{\bmm}{\boldsymbol{m}}	\newcommand{\bmn}{\boldsymbol{n}}
\newcommand{\bmo}{\boldsymbol{o}}	\newcommand{\bmp}{\boldsymbol{p}}
\newcommand{\bmq}{\boldsymbol{q}}	\newcommand{\bmr}{\boldsymbol{r}}
\newcommand{\bms}{\boldsymbol{s}}	\newcommand{\bmt}{\boldsymbol{t}}
\newcommand{\bmu}{\boldsymbol{u}}	\newcommand{\bmv}{\boldsymbol{v}}
\newcommand{\bmw}{\boldsymbol{w}}	\newcommand{\bmx}{\boldsymbol{x}}
\newcommand{\bmy}{\boldsymbol{y}}	\newcommand{\bmz}{\boldsymbol{z}}

%---------------------------------------
% Scr Math Fonts :-
%---------------------------------------

\newcommand{\sA}{{\mathscr{A}}}   \newcommand{\sB}{{\mathscr{B}}}
\newcommand{\sC}{{\mathscr{C}}}   \newcommand{\sD}{{\mathscr{D}}}
\newcommand{\sE}{{\mathscr{E}}}   \newcommand{\sF}{{\mathscr{F}}}
\newcommand{\sG}{{\mathscr{G}}}   \newcommand{\sH}{{\mathscr{H}}}
\newcommand{\sI}{{\mathscr{I}}}   \newcommand{\sJ}{{\mathscr{J}}}
\newcommand{\sK}{{\mathscr{K}}}   \newcommand{\sL}{{\mathscr{L}}}
\newcommand{\sM}{{\mathscr{M}}}   \newcommand{\sN}{{\mathscr{N}}}
\newcommand{\sO}{{\mathscr{O}}}   \newcommand{\sP}{{\mathscr{P}}}
\newcommand{\sQ}{{\mathscr{Q}}}   \newcommand{\sR}{{\mathscr{R}}}
\newcommand{\sS}{{\mathscr{S}}}   \newcommand{\sT}{{\mathscr{T}}}
\newcommand{\sU}{{\mathscr{U}}}   \newcommand{\sV}{{\mathscr{V}}}
\newcommand{\sW}{{\mathscr{W}}}   \newcommand{\sX}{{\mathscr{X}}}
\newcommand{\sY}{{\mathscr{Y}}}   \newcommand{\sZ}{{\mathscr{Z}}}


%---------------------------------------
% Math Fraktur Font
%---------------------------------------

%Captital Letters
\newcommand{\mfA}{\mathfrak{A}}	\newcommand{\mfB}{\mathfrak{B}}
\newcommand{\mfC}{\mathfrak{C}}	\newcommand{\mfD}{\mathfrak{D}}
\newcommand{\mfE}{\mathfrak{E}}	\newcommand{\mfF}{\mathfrak{F}}
\newcommand{\mfG}{\mathfrak{G}}	\newcommand{\mfH}{\mathfrak{H}}
\newcommand{\mfI}{\mathfrak{I}}	\newcommand{\mfJ}{\mathfrak{J}}
\newcommand{\mfK}{\mathfrak{K}}	\newcommand{\mfL}{\mathfrak{L}}
\newcommand{\mfM}{\mathfrak{M}}	\newcommand{\mfN}{\mathfrak{N}}
\newcommand{\mfO}{\mathfrak{O}}	\newcommand{\mfP}{\mathfrak{P}}
\newcommand{\mfQ}{\mathfrak{Q}}	\newcommand{\mfR}{\mathfrak{R}}
\newcommand{\mfS}{\mathfrak{S}}	\newcommand{\mfT}{\mathfrak{T}}
\newcommand{\mfU}{\mathfrak{U}}	\newcommand{\mfV}{\mathfrak{V}}
\newcommand{\mfW}{\mathfrak{W}}	\newcommand{\mfX}{\mathfrak{X}}
\newcommand{\mfY}{\mathfrak{Y}}	\newcommand{\mfZ}{\mathfrak{Z}}
%Small Letters
\newcommand{\mfa}{\mathfrak{a}}	\newcommand{\mfb}{\mathfrak{b}}
\newcommand{\mfc}{\mathfrak{c}}	\newcommand{\mfd}{\mathfrak{d}}
\newcommand{\mfe}{\mathfrak{e}}	\newcommand{\mff}{\mathfrak{f}}
\newcommand{\mfg}{\mathfrak{g}}	\newcommand{\mfh}{\mathfrak{h}}
\newcommand{\mfi}{\mathfrak{i}}	\newcommand{\mfj}{\mathfrak{j}}
\newcommand{\mfk}{\mathfrak{k}}	\newcommand{\mfl}{\mathfrak{l}}
\newcommand{\mfm}{\mathfrak{m}}	\newcommand{\mfn}{\mathfrak{n}}
\newcommand{\mfo}{\mathfrak{o}}	\newcommand{\mfp}{\mathfrak{p}}
\newcommand{\mfq}{\mathfrak{q}}	\newcommand{\mfr}{\mathfrak{r}}
\newcommand{\mfs}{\mathfrak{s}}	\newcommand{\mft}{\mathfrak{t}}
\newcommand{\mfu}{\mathfrak{u}}	\newcommand{\mfv}{\mathfrak{v}}
\newcommand{\mfw}{\mathfrak{w}}	\newcommand{\mfx}{\mathfrak{x}}
\newcommand{\mfy}{\mathfrak{y}}	\newcommand{\mfz}{\mathfrak{z}}

% lstlistingsEnvs.tex

\usepackage{minted}


\lstset{
  basicstyle=\ttfamily, % Set
  columns=fullflexible,
  keepspaces=true,
  language=Python % You can specify the language if you want syntax highlighting
}

%%%%%%%%%%%%%%%%%%%%%%%%%%%%%%%%%%%%%%%%%%%%%%%%%%%%%%%%%%%%%%%%%%%%%%%%%%%%%%%%%%%%%%%%%%%%%%%%%
%                                 Custom lstlisting Environments
%%%%%%%%%%%%%%%%%%%%%%%%%%%%%%%%%%%%%%%%%%%%%%%%%%%%%%%%%%%%%%%%%%%%%%%%%%%%%%%%%%%%%%%%%%%%%%%%%
% Gruvbox style for Python
\definecolor{Pgruvbox-bg}{HTML}{282828}
\definecolor{Pgruvbox-fg}{HTML}{ebdbb2}
\definecolor{Pgruvbox-red}{HTML}{fb4934}
\definecolor{Pgruvbox-green}{HTML}{b8bb26}
\definecolor{Pgruvbox-yellow}{HTML}{fabd2f}
\definecolor{Pgruvbox-blue}{HTML}{83a598}
\definecolor{Pgruvbox-purple}{HTML}{d3869b}
\definecolor{Pgruvbox-aqua}{HTML}{8ec07c}
\definecolor{BBBlack}{rgb}{0.05, 0.06, 0.09}



% JAVA LSTLISTING STYLE IN Gruvbox Colorscheme
\definecolor{gruvbox-bg}{rgb}{0.282, 0.247, 0.204}
\definecolor{gruvbox-fg1}{rgb}{0.949, 0.898, 0.776}
\definecolor{gruvbox-fg2}{rgb}{0.871, 0.804, 0.671}
\definecolor{gruvbox-red}{rgb}{0.788, 0.255, 0.259}
\definecolor{gruvbox-green}{rgb}{0.518, 0.604, 0.239}
\definecolor{gruvbox-yellow}{rgb}{0.914, 0.808, 0.427}
\definecolor{gruvbox-blue}{rgb}{0.353, 0.510, 0.784}
\definecolor{gruvbox-purple}{rgb}{0.576, 0.412, 0.659}
\definecolor{gruvbox-aqua}{rgb}{0.459, 0.631, 0.737}
\definecolor{gruvbox-gray}{rgb}{0.518, 0.494, 0.471}

\definecolor{lst-bg}{RGB}{45, 45, 45}
\definecolor{lst-fg}{RGB}{220, 220, 204}
\definecolor{lst-keyword}{RGB}{215, 186, 125}
\definecolor{lst-comment}{RGB}{117, 113, 94}
\definecolor{lst-string}{RGB}{163, 190, 140}
\definecolor{lst-number}{RGB}{181, 206, 168}
\definecolor{lst-type}{RGB}{218, 142, 130}

\lstdefinestyle{PythonGruvbox}{
    language=Python,
    identifierstyle=\color{lst-fg},
    basicstyle=\ttfamily\color{Pgruvbox-fg},
    keywordstyle=\color{Pgruvbox-yellow},
    keywordstyle=[2]\color{Pgruvbox-blue},
    stringstyle=\color{Pgruvbox-green},
    commentstyle=\color{Pgruvbox-aqua},
    backgroundcolor=\color{BBBlack},
    rulecolor=\color{BBBlack},
    showstringspaces=false,
    keepspaces=true,
    captionpos=b,
    breaklines=true,
    tabsize=4,
    showspaces=false,
    numbers=left,
    numbersep=5pt,
    numberstyle=\tiny\color{gray},
    showtabs=false,
    columns=fullflexible,
    morekeywords={True,False,None},
    morekeywords=[2]{and,as,assert,break,class,continue,def,del,elif,else,except,exec,
    finally,for,from,global,if,import,in,is,lambda,nonlocal,not,or,pass,print,raise,
    return,try,while,with,yield},
    morecomment=[s]{"""}{"""},
    morecomment=[s]{'''}{'''},
    morecomment=[l]{\#},
    morestring=[b]",
    morestring=[b]',
    literate=
    {0}{{\textcolor{Pgruvbox-purple}{0}}}{1}
    {1}{{\textcolor{Pgruvbox-purple}{1}}}{1}
    {2}{{\textcolor{Pgruvbox-purple}{2}}}{1}
    {3}{{\textcolor{Pgruvbox-purple}{3}}}{1}
    {4}{{\textcolor{Pgruvbox-purple}{4}}}{1}
    {5}{{\textcolor{Pgruvbox-purple}{5}}}{1}
    {6}{{\textcolor{Pgruvbox-purple}{6}}}{1}
    {7}{{\textcolor{Pgruvbox-purple}{7}}}{1}
    {8}{{\textcolor{Pgruvbox-purple}{8}}}{1}
    {9}{{\textcolor{Pgruvbox-purple}{9}}}{1}
}

% Gruvbox style for Java
\definecolor{gruvbox-bg}{rgb}{0.282, 0.247, 0.204}
\definecolor{gruvbox-fg1}{rgb}{0.949, 0.898, 0.776}
\definecolor{gruvbox-fg2}{rgb}{0.871, 0.804, 0.671}
\definecolor{gruvbox-red}{rgb}{0.788, 0.255, 0.259}
\definecolor{gruvbox-green}{rgb}{0.518, 0.604, 0.239}
\definecolor{gruvbox-yellow}{rgb}{0.914, 0.808, 0.427}
\definecolor{gruvbox-blue}{rgb}{0.353, 0.510, 0.784}
\definecolor{gruvbox-purple}{rgb}{0.576, 0.412, 0.659}
\definecolor{gruvbox-aqua}{rgb}{0.459, 0.631, 0.737}
\definecolor{gruvbox-gray}{rgb}{0.518, 0.494, 0.471}

\lstdefinestyle{JavaGruvbox}{
    language=Java,
    basicstyle=\ttfamily\color{Pgruvbox-fg},
    keywordstyle=\color{Pgruvbox-yellow},
    keywordstyle=[2]\color{lst-type},
    commentstyle=\itshape\color{lst-comment},
    stringstyle=\color{lst-string},
    numberstyle=\color{lst-number},
    backgroundcolor=\color{BBBlack},
    rulecolor=\color{gruvbox-aqua},
    showstringspaces=false,
    keepspaces=true,
    captionpos=b,
    breaklines=true,
    tabsize=4,
    showspaces=false,
    showtabs=false,
    columns=fullflexible,
    morekeywords={var},
    morekeywords=[2]{boolean, byte, char, double, float, int, long, short, void},
    morecomment=[s]{/}{/},
    morecomment=[l]{//},
    morestring=[b]",
    morestring=[b]',
    numbers=left,
    numbersep=5pt,
    numberstyle=\tiny\color{gray},
}

% Dracula style for Java
\definecolor{draculawhite-background}{RGB}{237, 239, 252}
\definecolor{draculawhite-comment}{RGB}{98, 114, 164}
\definecolor{draculawhite-keyword}{RGB}{189, 147, 249}
\definecolor{draculawhite-string}{RGB}{152, 195, 121}
\definecolor{draculawhite-number}{RGB}{249, 189, 89}
\definecolor{draculawhite-operator}{RGB}{248, 248, 242}

\lstdefinestyle{JavaDraculaWhite}{
    language=Java,
    backgroundcolor=\color{draculawhite-background},
    commentstyle=\itshape\color{draculawhite-comment},
    keywordstyle=\color{draculawhite-keyword},
    stringstyle=\color{draculawhite-string},
    basicstyle=\ttfamily\footnotesize\color{black},
    identifierstyle=\color{black},
    keywordstyle=\color{draculawhite-keyword}\bfseries,
    morecomment=[s][\color{draculawhite-comment}]{/**}{*/},
    showstringspaces=false,
    showspaces=false,
    breaklines=true,
    %frame=single,
    rulecolor=\color{draculawhite-operator},
    tabsize=2,  
    numbers=left,
    numbersep=4pt,
    numberstyle=\ttfamily\tiny\color{gray}
}

% Dracula style for Python
\definecolor{draculawhite-bg}{HTML}{FAFAFA}
\definecolor{draculawhite-fg}{HTML}{282A36}
\definecolor{pdraculawhite-keyword}{HTML}{BD93F9}
\definecolor{pdraculawhite-comment}{HTML}{6272A4}
\definecolor{draculawhite-number}{HTML}{FF79C6}

\lstdefinestyle{PythonDraculaWhite}{
    language=Python,
    basicstyle=\ttfamily\small\color{draculawhite-fg},
    backgroundcolor=\color{draculawhite-background},
    keywordstyle=\color{orange}\bfseries,
    stringstyle=\color{draculawhite-string},
    commentstyle=\color{pdraculawhite-comment}\itshape,
    numberstyle=\color{draculawhite-number},
    showstringspaces=false,
    showspaces=false,
    breaklines=true,
    frame=single,
    rulecolor=\color{draculawhite-operator}, 
    tabsize=4,
    morekeywords={as,with,1,2,3,4, 5,6,7,8,9,True,False},
    numbers=left,
    numbersep=5pt,
    numberstyle=\small\bfseries\ttfamily\color{htmlcomment},
}

% Dracula Dark style for HTML
\definecolor{htmltag}{HTML}{ff79c6}
\definecolor{htmlattr}{HTML}{f1fa8c}
\definecolor{htmlvalue}{HTML}{bd93f9}
\definecolor{htmlcomment}{HTML}{6272a4}
\definecolor{htmltext}{HTML}{401E31}
\definecolor{htmlbackground}{HTML}{282a36}
\definecolor{comphtmlbackground}{HTML}{8093FF}

\lstdefinestyle{HTMLDraculaDark}{
    basicstyle=\normalsize\bfseries\ttfamily\color{htmltext},
    commentstyle=\itshape\color{htmlcomment},
    keywordstyle=\bfseries\color{htmltag},
    stringstyle=\color{htmlvalue},
    emph={DOCTYPE,html,head,body,div,span,a,script},
    emphstyle={\color{htmltag}\bfseries},
    sensitive=true,
    showstringspaces=false,
    backgroundcolor=\color{white},
    inputencoding=utf8,
    extendedchars=true,
    language=HTML,
    tabsize=4,
    breaklines=true,
    breakatwhitespace=true,
    numbers=left,
    numbersep=10pt,
    numberstyle=\small\bfseries\ttfamily\color{htmlcomment},
    escapeinside={<@}{@>},
    rulecolor=\color{htmlbackground},
    xleftmargin=10pt,
    frame=none, 
    breaklines=true,
    postbreak=\mbox{\textcolor{gray}{$\hookrightarrow$}\space},
    showlines=false,
    moredelim=[s][\itshape\color{htmlcomment}]{<!--}{-->},
    morekeywords={id,class,type,name,value,placeholder,checked,src,href,alt},
    literate={é}{{\'e}}1 {è}{{\`e}}1 {ê}{{\^e}}1 {ë}{{\"e}}1 {à}{{\`a}}1 {ù}{{\`u}}1 {û}{{\^u}}1 {ç}{{\c{c}}}1 {â}{{\^a}}1 {î}{{\^i}}1 {ï}{{\"i}}1
}


\lstdefinestyle{Haskell}{
  frame=none,
  xleftmargin=2pt,
  stepnumber=1,
  numbers=left,
  numbersep=5pt,
  numberstyle=\ttfamily\tiny\color[gray]{0.3},
  belowcaptionskip=\bigskipamount,
  captionpos=b,
  escapeinside={*'}{'*},
  language=haskell,
  tabsize=2,
  emphstyle={\bf},
  %commentstyle=\it,
  stringstyle=\mdseries\ttfamily,
  showspaces=false,
  keywordstyle=\bfseries\ttfamily,
  columns=flexible,
  basicstyle=\small\ttfamily,
  showstringspaces=false,
  morecomment=[l]\%,
}



\lstdefinestyle{CSSDraculaLight}{
    basicstyle=\bfseries\scriptsize\ttfamily\color{htmltext},
    commentstyle=\color{htmlcomment},
    keywordstyle=\bfseries\color{htmlvalue},
    stringstyle=\color{htmlvalue},
    emph={DOCTYPE,html,head,body,div,span,a,script},
    emphstyle={\color{htmltag}\bfseries},
    sensitive=true,
    showstringspaces=false,
    backgroundcolor=\color{white},
    inputencoding=utf8,
    extendedchars=true, % Support extended characters
    frame=none, 
    %frame=tb,
    tabsize=4,
    breaklines=true,
    breakatwhitespace=true,
    numbers=left,
    numbersep=10pt,
    numberstyle=\small\bfseries\ttfamily\color{htmlcomment},
    escapeinside={<@}{@>},
    rulecolor=\color{htmlbackground},
    xleftmargin=20pt,
    % Add a vertical line for opening and closing tags
    %frame=lines,
    framesep=2pt,
    framexleftmargin=4pt,
    % Add a vertical line for closing tags that go through multiple lines
    breaklines=true,
    postbreak=\mbox{\textcolor{gray}{$\hookrightarrow$}\space},
    showlines=true,
    % Add a rule to apply commentstyle to keywords inside comments
    moredelim=[s][\color{htmlcomment}]{/*}{*/},
    literate={é}{{\'e}}1
             {è}{{\`e}}1
             {ê}{{\^e}}1
             {ë}{{\"e}}1
             {à}{{\`a}}1
             {ù}{{\`u}}1
             {û}{{\^u}}1
             {ç}{{\c{c}}}1
             {â}{{\^a}}1
             {î}{{\^i}}1
             {ï}{{\"i}}1,
    morekeywords={color, background, background-color, font-size, font-weight, border, border-radius, padding, margin, display, position, top, right, bottom, left, flex, grid, width, height, min-width, max-width, min-height, max-height, transition, transform, animation, keyframes, content, z-index,id,class,type,name,value,placeholder,checked,src,href,alt},
    morestring=[s][\color{htmltag}]{:}{;},
}












\title{\huge{MATH1400}\\\Huge{Calcul à plusieurs variables}\\\vspace{2em}Travail Pratique 2 }
\author{\huge{Franz Girardin}}
\date{\today}


   

\begin{document}

\maketitle
\newpage% or \cleardoublepage
% \pdfbookmark[<level>]{<title>}{<dest>}
\pdfbookmark[section]{\contentsname}{toc}
\tableofcontents
\pagebreak


\titleformat*{\section}{%
    \normalsize\bfseries%
}

\titleformat{\section}[block]{\normalsize\bfseries}{}{0pt}{}


    \chapter*{Exercices sur la convergence de suite et séries}
    \section{Définitions}
    
    \begin{Exercice}{(Stewart 1.2.2)}{}
       Expliquez ce que signifie 
       $\sum_{n=1}^{\infty }a_n = 5$ 
    \end{Exercice}

    Cette expression signifie que la somme ayant le terme général $a_n$ 
    converge vers la veleur $L = 5$. Autrement dit, lorsqu'on additionne les 
    termes de la somme $a_n$ de façon \textbf{indéfinie}, on obtient la 
    somme $5$.


    \begin{Exercice}{(Stewart 1.2.4)}{}
        Calculez la somme de la série $\sum_{n=1}^{\infty }a_n$ dont les 
        les sommes partielles sont données :
        \begin{align*}
                S_n = \dfrac{n^2 - 1}{4^n +1} 
        \end{align*}
    \end{Exercice}

    Soit la somme parielle $S_n$, nous pouvons calculer la série comme suit :


    \begin{align*}
        S = 
        \lim\limits_{n \to+\infty } S_n  = 
        \lim\limits_{n \to+\infty }\dfrac{n^2 - 1}{4^n +1} 
        = 
        \lim\limits_{n \to+\infty }  
        \dfrac{1 -  \cancelto{0}{1}}{\frac{4^n}{n^2} + \cancelto{0}{1}} 
        = 
        \lim\limits_{n \to+\infty } 
        \dfrac{n^2}{4^n} 
    \end{align*}
    Puisque la quantité $4^n$ croît plus rapidement que $n^2$, nous avons 

    \begin{align*}
        \left[ n^2 \ll 4^n \right] \implies 
        \lim\limits_{ \to+\infty }S_n = \dfrac{n^2}{4^n} \longrightarrow 0 = S 
    \end{align*}

    Ainsi, nous avons 

    \begin{align*}
        \boxed{
            \lim\limits_{n \to+\infty }S_n = S = \sum_{n=1}^{\infty }a_n = 
            \textcolor{red}{0} 
        }
    \end{align*}


    \begin{Exercice}{(Stewart 1.2.16)}{}
        Expliquez la différence entre
    \end{Exercice}

    \noindent
    \textbf{a)} 
    \begin{align*}    
        \sum_{i=1}^{n}a_i \quad 
        \text{et} \quad%
        \sum_{j=1}^{n}a_j
    \end{align*}

    Les deux sommes représentent la même somme. La différente entre elles 
    est uniquement la variable de sommation. Or, puisque les variables de 
    sommation $i$ et $j$ sont considéré comme des variables muettes, le nom 
    de la variable n'affecte pas le résultat de la somme. 



    \vspace{1em}
    \textbf{b)}
    \begin{align*}
        \sum_{i=1}^{\infty }a_i 
        \quad%
        \text{et}
        \quad%
        \sum_{i=1}^{\infty }a_j
    \end{align*}

    Les sommes sont \textbf{différentes}. La première somme implique l'addition de termes 
    $a_i$ sur un intervalle de sommation de $i = 1$ à l'infini. Or, pour la seconde 
    somme la variable de sommation $j$ n'affecte pas les termes $a_j$ de la somme. 
    \textbf{Ainsi}, la somme constante par rapport à $i$.   


    \section{Convergence de série géométrique}
    \begin{Exercice}{(Stewart 1.2.20)}{}
        Déterminez si la série géométrique converge ou diverge. Si elle converge, 
        trouvez sa somme. 
    \end{Exercice}

    
    \noindent
    \textbf{20.} 
    \begin{align*}
            \sum_{n=1}^{\infty }a_n = 2 + 0.5 + 0.125 + 0.03125 + \cdots 
    \end{align*}

    La série géométrique suit la règle $a_1 = 2$ et $a_n = a_1r^{n-1} \forall n \geq 1$.
    \textbf{Donc}, nous avons :
    \begin{align*}
        \left[ a_2 = a_1r^{n-1}  \right] \implies 
        \left[ 0.5 = 2 + r^{1} \right] \implies 
        \textcolor{myb}{r =  -\frac{3}{2}} 
    \end{align*}

    \textbf{Ainsi}, nous avons :
    
    \begin{align*}
        \sum_{n=1}^{\infty }a_n = \sum_{n=1}^{\infty } 2r^{-3/2}
    \end{align*}

    Puisque la raison $r = -3/2$ de la somme est hors de l'intervalle de convergence 
    d'une suite géométrique, nous pouvons conclure que la somme \textcolor{myr}{\textbf{div}}.

    \begin{align*}
        \boxed{
        \left[ r = -3/2 \notin ]-1, 1] \right] \implies \sum_{n=1}^{\infty }a_n 
        \;\; \textcolor{myr}{\textbf{div}} 
    }
    \end{align*}

    \section{Convergence de série}
    \begin{Exercice}{(Stewart 17-26)}{}
        Déterminez si la série géométrique converge ou diverge.
        Si elle converge, trouvez sa somme
    \end{Exercice}


    \noindent \textbf{24.}                    

    \begin{align*}
         \sum\limits_{n=1}^{\infty } \dfrac{3^{n+1}}{(-2)^n}
    \end{align*}

    Consirérons la minupulation suivante :

    \begin{align*}
         \sum\limits_{n=1}^{\infty } \dfrac{3^{n+1}}{(-2)^n} = 
         (-1)^n \dfrac{3^{n+1}}{2^n}
    \end{align*}

    Selon le \textbf{théorème de convergence du terme général}, une 
    une condition nécessaire à la convergence d'une série est que 
    la limite du terme général $a_n$ de celle-ci tende vers $0$. 
    \textbf{Ainsi}, nous avons : 

    \begin{align*}
        \lim\limits_{n \to+\infty }a_n = 
        \lim\limits_{n\to+\infty } (-1)^n \dfrac{3^{n+1}}{2^n} = 
        \lim\limits_{n\to+\infty } 3(-1)^n \dfrac{3^n}{2^n} =
        \lim\limits_{n\to+\infty } 3(-1)^n \left(\dfrac{3}{2} \right)^n
        \textcolor{red}{\neq} \; 0  
    \end{align*}

    Ainsi, puisque la limite du terme général $a_n$ ne tend pas vers $0$, 
    nous pouvons conclure que la suite est \textcolor{myr}{\textbf{div}}. 


    \vspace{1em}

    \noindent\textbf{25}.  
    \begin{align*}
        \sum_{n=1}^{\infty }\dfrac{e^{2n}}{6^{n-1}}
    \end{align*}

    Considérons la limite du terme général $a_n$ :

    \begin{align*}
        \lim\limits_{n \to+\infty }a_n = 
        \lim\limits_{n \to+\infty } \dfrac{e^{2n}}{6^{n-1}} = 
        \lim\limits_{n \to+\infty } \dfrac{6e^{2n}}{6^{n}}  = 
        6 \left[\lim\limits_{n \to+\infty } 
        \left(\dfrac{e^{2}}{6}\right)^n \right]
        \longrightarrow + \infty \textcolor{red}{\neq} \; 0  
    \end{align*}

    Puisque la limite du terme général \textbf{tend vers l'infini}, 
    nous pouvons conclure que la suite 
    $\sum_{n=1}^{\infty }a_n$ \textcolor{myr}{\textbf{div}}.    



    \noindent\textbf{26}.  


    \begin{align*}
        \sum_{n=1}^{\infty }\dfrac{6 - 2^{2n -1}}{3^n} = 
        \sum_{n=1}^{\infty }
        \left[% 
        \frac{6}{3^n}  
        - 
        \dfrac{2^{2n -1}}{3^n}
        \right]%
    \end{align*}

    Évaluons le terme général $a_n$ de la suite, sachant que 
    $a_n = b_n  + c_n$ où $b_n = \sum_{n=1}^{\infty }\dfrac{6}{3^n}$ 
    et $c_n = \sum_{n=1}^{\infty }\dfrac{2^{2n -1}}{3^n}$ 

    \begin{align*}
        \lim\limits_{n \to+\infty }a_n = 
        \lim\limits_{n \to+\infty }b_n  
        + 
        \lim\limits_{n \to+\infty }c_n 
        = 
        \cancelto{0}{
        \left[ 
            \lim\limits_{n\to+\infty }\sum_{n=1}^{\infty }\frac{6}{3^n} 
        \right]}
        +
        \dfrac{1}{2}
        \lim\limits_{n \to+\infty} \dfrac{2^{2n}}{3^n}
        = \textcolor{red}{0} +   
        \frac{1}{2} 
        \lim\limits_{n \to+\infty} \left(\dfrac{4}{3}\right)^n 
    \end{align*}                    


    La seconde suite est une \textbf{suite géométrique} de raison 
    $r  = \frac{4}{3} \geq 1$. Ce suite divergente. \textbf{Ainsi}, puisque 
    la somme $\sum_{n=1}^{\infty }a_n$ se décompose en une suite 
    d'une somme $\sum_{n=1}^{\infty }b_n + c_n$ et la limite 
    du terme général 
    $b_n + c_n = a_n$ tend vers l'infini, on peut conclure que la 
    sommme $\sum_{n=1}^{\infty }a_n$ \textcolor{myr}{\textbf{div}}. 

    \begin{Exercice}{(Stewart 27 - 42)}{}
        Déterminez si la série converge ou diverge. Si elle converge, trouvez 
        la somme.
    \end{Exercice}

    \vspace{1em}
    \noindent
    \textbf{27.}  

    \begin{align*}
        \dfrac{1}{3} + \frac{1}{6} + \frac{1}{9} + \frac{1}{12} + \frac{1}{15} + \cdots    
    \end{align*}

    Nous avons la somme : 
    

    \begin{align*}
        \dfrac{1}{3} + \frac{1}{6} + \frac{1}{9} + \frac{1}{12} + \frac{1}{15} + \cdots    
        = 
        \sum_{n=1}^{\infty }\dfrac{1}{3^n} = \sum_{n=1}^{\infty }3^{-n}
    \end{align*}

    Il s'agit d'une suite géométrique de raison $r =  -1$, par la critère de convergence 
    d'une suite géométrique, on peut conclure que la somme \textcolor{myr}{\textbf{div}} :

    \begin{align*}
        \boxed{
        r = -1 \notin ]-1, 1] \implies \sum_{n=1}^{\infty }ar^n \textcolor{myr}{\textbf{div}} 
    }
    \end{align*}                                                                                                                                                                                                                                                                                                                                                                                                                        


    \begin{Exercice}{(Stewart 1.2.27 - 1.2.42)}{}
        Déterminez si la série converge ou diverge. SI elle converge, trouvez sa somme
    \end{Exercice}
    
    
    \noindent
    \textbf{28.} 
    $\sum_{n=1}^{\infty }a_n = \dfrac{1}{3} + \dfrac{2}{9} + \dfrac{1}{27} + 
    \dfrac{2}{81} + \dfrac{1}{243} + \dfrac{2}{729} + \cdots$

    \vspace{1em}
    Nous pouvons déduire le terme général $a_n$ de la somme : 

    \begin{align*}
        \sum_{n=1}^{\infty }a_n = 
        \sum_{n=1}^{\infty }\dfrac{3 - (-1)^n}{2} \cdot \frac{1}{3^n}  
    \end{align*}            

    Il s'agit donc d'une \textbf{suite géométrique} de raison \boxed{$r = 1/3 < 1$}.
    \textbf{Ainsi}, on peut conclure que la série \textcolor{myb}{\textbf{conv}}. 



    \section{Comparaison de séries}

    \begin{Exercice}{(Stewart 1.3.2)}{}
        Supposez que $f$ est une fonction continue, positive et décroissante 
        pour $x \geq 1$ et que $a_n = f(n)$. À l'aide d'un figure, classez les trois 
        quantités suivantes dans l'ordre croissant.

        \begin{align*}
            \int_{1}^{6}f(x)dx \quad 
            \sum_{i=1}^{5 }a_i \quad 
            \sum_{i=2}^{6 }a_i
        \end{align*}
    \end{Exercice}

    Selon le \textbf{théorème du reste}, nous avons :   

    \begin{align*}
        \int_{n+1}^{\infty }f(x)dx  \leq 
        \sum_{n=1}^{\infty }a_n = R_n \leq 
        \int_{n=1}^{\infty }f(x)dx 
        \\
    \end{align*}

     Supposons que $f$ est une fonction continue, positive et décroissante pour 
    $x \geq 1$, et que $a_n = f(n)$. Nous devons classer les trois quantités 
    suivantes dans l'ordre croissant :

    \[
    \int_{1}^{6} f(x) dx, \quad \sum_{i=1}^{5} a_i, \quad \sum_{i=2}^{6} a_i
    \]

    La somme $\sum_{i=1}^{5} a_i$ additionne les valeurs de $f(x)$ pour 
    $i = 1$ à $i = 5$. Comme $f$ est \textbf{décroissante}, chaque terme $a_i$ est plus 
    grand que $a_{i+1}$.
    \vspace{1em}

    La somme $\sum_{i=2}^{6} a_i$ additionne les valeurs de $f(x)$ pour 
    $i = 2$ à $i = 6$, \textbf{en excluant} $a_1$, qui est le plus grand terme. Donc, 
    $\sum_{i=1}^{5} a_i > \sum_{i=2}^{6} a_i$.
    \vspace{1em}

    L'intégrale $\int_{1}^{6} f(x) dx$ représente \textbf{l'aire sous la courbe }   
    de $f(x)$ entre $x = 1$ et $x = 6$. Cette aire se situe entre les deux 
    sommes discrètes, car l'intégrale correspond à la somme d'une infinité 
    de petites contributions situées entre les rectangles formés par les 
    sommes discrètes.

    \noindent
    Ainsi, nous avons :

    \[
        \boxed{\sum_{i=2}^{6} a_i < \int_{1}^{6} f(x) dx < \sum_{i=1}^{5} a_i}
    \]

        \begin{center}
        \begin{tikzpicture}[xscale=1, yscale=0.6]
            % Axes
            \draw[->] (0,0) -- (8,0) node[right] {$x$};
            \draw[->] (0,0) -- (0,6) node[above] {$y$};



            \filldraw[fill=orange!30, draw=black] (1,0) rectangle (2,5/1);
            \filldraw[fill=orange!30, draw=black] (2,0) rectangle (3,5/2);
            \filldraw[fill=orange!30, draw=black] (3,0) rectangle (4,5/3);
            \filldraw[fill=orange!30, draw=black] (4,0) rectangle (5,5/4);
            \filldraw[fill=orange!30, draw=black] (5,0) rectangle (6,5/5);



            % Blocs avec bordures noires
            \filldraw[fill=red!30, draw=black] (1,0) rectangle (2,5/2);
            \filldraw[fill=red!30, draw=black] (2,0) rectangle (3,5/3);
            \filldraw[fill=red!30, draw=black] (3,0) rectangle (4,5/4);
            \filldraw[fill=red!30, draw=black] (4,0) rectangle (5,5/5);
            \filldraw[fill=red!30, draw=black] (5,0) rectangle (6,5/6);


            % Courbe par-dessus

            \draw[] (1,0) -- (1,5);
            \draw[thick, myr, domain=0.7:8.2] plot (\x, {5/\x}) node[above] {};

            \node at (2, 5.5) {\textcolor{myr}{$y =f(x)$}};

            \node at (3.5, 3.75) {\textcolor{red!70}{$S_1 = 
                \sum_{i = 1}^{6 }a_i$}};
            % Étiquettes
            \node at (4.5, 2.75) {\textcolor{orange!70}{$S_2 = 
                \sum_{i = 1}^{5 }a_i$}};


            \node at (1.5, 2.) {\textcolor{red}{$i_{1}$}};
            \node at (2.5, 1.25) {\textcolor{red}{$i_{3}$}};
            \node at (3.5, 0.75) {\textcolor{red}{$i_{3}$}};
            \node at (4.5, 0.5) {\textcolor{red}{$i_{4}$}};
            \node at (5.5, 0.3) {\textcolor{red}{$i_{5}$}};




            % Points de référence
            \node[below] at (1,0) {$1$};
            \node[below] at (2,0) {$2$};
            \node[below] at (6,0) {$6$};
        \end{tikzpicture}
        \end{center}


    \begin{Exercice}{(Stewart 1.3.8)}{}
       Utilisez le test de l'intégrale pour déterminer si la série converge 
       ou diverge. 

       \begin{align*}
            \sum_{n=1}^{\infty }n^2e^{-n^3}
       \end{align*}
   \end{Exercice}


    \noindent
    Le \textbf{test de l'intégrale} consiste à évaluer l'intégrale correspondante pour 
    une fonction continue positive et décroissante associée au terme général 
    de la série. Considérons la fonction :
    \[
    f(x) = x^2 e^{-x^3}
    \]

    Nous devons évaluer l'intégrale impropre suivante :

    \[
    \int_{1}^{\infty} x^2 e^{-x^3} \, dx
    \]

    Pour résoudre cette intégrale, nous effectuons le \textbf{changement de variable }   
    \( u = x^3 \), ce qui donne \( du = 3x^2 dx \), ou encore :

    \[
    dx = \frac{du}{3x^2}
    \]

    Ainsi, l'intégrale devient :

    \[
    \int_{1}^{\infty} x^2 e^{-x^3} \, dx = \int_{1}^{\infty} \frac{e^{-u}}{3} \, du
    \]

    Cette dernière intégrale est une intégrale exponentielle classique :

    \[
    \frac{1}{3} \int_{1}^{\infty} e^{-u} \, du
    \]

    La primitive de \( e^{-u} \) est \( -e^{-u} \), donc nous avons :

    \[
    \frac{1}{3} \left[ -e^{-u} \right]_{1}^{\infty} 
    = \frac{1}{3} \left( 0 + e^{-1} \right) = \frac{e^{-1}}{3}
    \]

    L'intégrale converge donc, ce qui implique que la série \(\sum_{n=1}^{\infty} n^2 e^{-n^3}\) 
    converge par le test de l'intégrale. 


    \begin{Exercice}{(Stewart 1.3.9-1.3.26)}{}
        Déterminez si la série converge ou diverge
    \end{Exercice}

    \vspace{1em}
    \section{Utilisation du critère de Riemann}
    \noindent
    \textbf{14.} $\sum_{n=1}^{\infty }a_n = 
    1 + \dfrac{1}{2\sqrt{2}} + \dfrac{1}{3\sqrt{3}} + \dfrac{1}{4\sqrt{4}} + \dfrac{1}{5\sqrt{5}} + \cdots$ 

    \vspace{1em}
    Nous pouvons déduire le terme général $a_n$ de la suite et réécrire la somme comme suit :

    \begin{align*}
        \sum_{n=1}^{\infty }a_n = \dfrac{1}{n\sqrt{n}} = \sum_{n=1}^{\infty }\dfrac{1}{n^{3/2}} 
    \end{align*}

    Puisqu'il s'agit d'une \textbf{série à termes positifs}, nous pouvons appliquer le critère de Riemann. 
    \textbf{Ainsi}, nous avons :
    \begin{align*}
        \lim\limits_{n\to \infty}n^pa_n = \lim\limits_{n \to+\infty }n^{3/2}\dfrac{1}{n^{3/2}} = l = 1  
    \end{align*}

    Puisque la quantité $p$ est telle que $p > 1$, par \textbf{le critère de Riemann}, la série \textcolor{myb}{\textbf{conv}}. 

    \vspace{1em}
    \noindent 
    \textbf{20.} $\sum_{n=3}^{\infty }\dfrac{3n - 4}{n^2 -2n}$

    \vspace{1em}
    Puisqu'il s'agit d'une \textbf{série à termes positifs}, nous pouvons appliquer le 
    critère de Riemann. 
    \textbf{Considérons} la fraction polynômiale donnée 
    par le terme général :
    $$a_n = \dfrac{q}{r} = \frac{3n -4}{n^2 -2n} $$  
    \textbf{Considérons} le degré le plus faible de cette 
    fraction polynômiale, soit $\deg(q) = 1 = p$. Multiplions la fraction par 
    $n^p = n^1$.
    \textbf{Ainsi}, nous avons :
    \begin{align*}
        n a_n = n \cdot \frac{3n -4}{n^2 -2} = \frac{3n^2 -4n}{n^2 -2}  
    \end{align*}

    Nous pouvons maintenant évaluer la limite :
    \begin{align*}
        \lim\limits_{n \to+\infty }n^pa_n = 
        \lim\limits_{n \to+\infty }\frac{3n^2 -4n}{n^2 -2}  
        \equiv
        \lim\limits_{n \to+\infty } \dfrac{3 - \frac{4}{n} }{1 - \frac{2}{n}} 
        \longrightarrow 3
    \end{align*}

    Ainsi, nous avons $p = 1 \leq 1$ et $l = 3 \; \textcolor{red}{ \neq } \; 0 $.
    Par le \textbf{critère de Riemann}, nous pouvons conclure que la série 
    \textcolor{myr}{\textbf{div}}.


    \begin{Exercice}{(Stewart 1.3.28)}{}
        Expliquez pourquoi on ne peut pas utiliser le test de l'intégrale pour
        déterminer si la série converge. 
        \begin{align*}
            \sum_{n=1}^{\infty }\dfrac{\cos^2n}{1 + n^2}
        \end{align*}
    \end{Exercice}

    Le test de l'intégral peut s'appliquer sur une fonction 
    $f : [1, \infty] \rightarrow \mathbb{R}$ \textbf{positive} 
    \textbf{croissante} et \textbf{continue}. Or, si l'une de ces conditions n'est pas 
    respectée, on ne peut appliquer le test. La suite  
    $\sum_{n=1}^{\infty }a_n$ est associée à la fonction $f(x)$ correspondante :
    \begin{align*}
        \left[ a_n = \dfrac{\cos^2n}{1 + n^2}  \right]
        \implies f(x) = \dfrac{\cos^2x}{1 + x^2}
    \end{align*}            

    Or, le numérateur de la fonction est $2\pi$ périodique ; 
    la fonction n'est donc pas monotone. Ainsi, nous ne pouvonsa lui appliquer 
    le test de l'intégrale. 



    \section{Convergence de série alternées}
    \begin{Exercice}{(Stewart 1.4.7 et 1.4.10)}{}
        Déterminez si la série converge ou diverge.
    \end{Exercice}

    \noindent
    \textbf{7.} $\sum_{n=1}^{\infty }(-1)^n \dfrac{3n - 1}{2n + 1}$

    Appliquons le critère de convergence nécessaire pour le terme général. 
    Soit une série $\sum_{n=1}^{\infty }a_n$, une condition nécessaire pour que 
    cette série converge est que la limite du terme général tende vers 0. 
    Nous avons : 

    \begin{align*}
        \lim\limits_{n \to+\infty }a_n  =  
        \lim\limits_{n \to+\infty } \dfrac{3n - 1}{2n + 1} \equiv
        \lim\limits_{n \to+\infty }  
        \dfrac{3 - \cancelto{0}{\frac{1}{n}}}{2 + \cancelto{0}{\frac{1}{n}}}
        \longrightarrow \dfrac{3}{2} \; \textcolor{red}{\neq} \; 0 
    \end{align*}        

    \textbf{Ainsi}, par le \textbf{critère de divergence du terme général},
    nous pouvons conclure que la série \textcolor{myr}{\textbf{div}}.   
    
    \begin{align*}
        \boxed{
        \left[  \lim\limits_{n \to+\infty }a_n \neq 0  \right]
        \implies \sum_{n=1}^{\infty }a_n \longrightarrow \infty \; 
        (\textcolor{myr}{\textbf{div}} )
    }
    \end{align*}

    \vspace{1em}% 
    \noindent
    \textbf{10.} $\sum_{n=1}^{\infty }(-1)^n\dfrac{\sqrt{n}}{2n +3}$   

    La somme ressemble à une \textbf{série alternée} sur laquelle on peut
    appliquer le critère de convergence des série alternées. 
    Vérifions la \textbf{limite du terme général}  $a_n$ : 

    \begin{align*}
        \lim\limits_{n\to \infty}a_n =  
        \lim\limits_{n \to+\infty } (-1)^n\dfrac{\sqrt{n}}{2n +3} \equiv
        \lim\limits_{n \to +\infty }
        (-1)^n \cdot \frac{1}{2n^{1/2} + 
        \cancelto{0}{\frac{3}{n^{1/2}} }}
        \longrightarrow  \textcolor{myb}{\textbf{0}}  
    \end{align*}

    Ainsi, nous savons que la limite du terme général tend vers 0. 
    Considérons la la fonction $f : [N, \infty] \rightarrow \mathbb{R}$
    telle que $f(n) = b_n$. Calculons la dérivée :

    \begin{align*}
        \dfrac{d}{dx}f(x) = \dfrac{d}{dx}
        \left[ \dfrac{(n)^{1/2}}{2n +3}\right]
        =
        \dfrac{\left( n^{1/2}\right)^{\prime}(2n +3) 
        - (2n + 3)^{\prime}n^{1/2}}{(2n + 3)^2} 
        = 
        \dfrac{(2n+3)\cdot 1/2n^{-1/2} -1/2(n^{1/2})}{(2n +3)^2} 
        = 
        \dfrac{\dfrac{(2 + 3)}{n^{1/2}} - \dfrac{n^{1/2}}{2}}{(2n +3)^2} 
    \end{align*}


    Puisque la différence du numérateur est négative 
    $\forall n \in \mathbb{N}$, la fraction polynômiale engendre 
    une quantité négative. Ainsi, la dérivée est négative, ce 
    qui implique que la fonction est \textbf{décroissante}.   
    \textbf{Ainsi}, la suite $a_{n+1}$ est décroissante et majorée 
    par $a_n$ et puisque le terme général $a_n$ tend vers 
    0 lorsque $n \longrightarrow \infty$, par 
    \textbf{le critère des série alternées}, nous pouvons conclure 
    que la série \textcolor{myb}{\textbf{conv}}. 

    \begin{align*}
        \boxed{ 
        \left[
        \lim\limits_{n \to+\infty }b_n = 0 
        \; 
        \textbf{et}
        \; 
        f(n) = b_n, f^{\prime}(x) \leq 0, \forall \;\; n \geq N 
    \right] 
    \implies \sum_{n=1}^{\infty }a_n \; \textcolor{myb}{\textbf{conv}} 
    }
    \end{align*}


    \section{Convergence absolue}

    \begin{Exercice}{(Stewart 1.5.4)}{}
       Déterminez si la série est \textbf{absoluement convergente}, 
       simplement convergente ou divergente. 
    \end{Exercice}

    \vspace{1em} 
    \noindent % 
    \textbf{4.} $\sum_{n=1}^{\infty }\dfrac{(-1)^n}{n^3 + 1}$ 

    Considérons la valeur absolue du terme général $|a_n|$ :

    \begin{align*}
        \sum_{n=1}^{\infty }|a_n| = 
        \sum_{n=1}^{\infty }\left|\dfrac{(-1)^n}{n^3 +1}\right|
        = 
        \dfrac{1}{n^3 + 1}, \; \forall \;\; n \geq 0 
    \end{align*}


    \begin{align*}
        \sum_{n=1}^{\infty }| a_n | \approx 
        \sum_{n=1}^{\infty }\dfrac{1}{n^3} =
        \sum_{n=1}^{\infty }n^p, p = 3
    \end{align*}

    Il s'agit d'une \textbf{série de Riemann} avec $p = 3 > 1$. 
    Par la \textbf{définition d'une série de Riemann}, nous pouvons 
    conclure que la somme converge \textbf{absoluement}. puisque cette 
    somme est plus grande que la somme originale, par le test 
    de comparaison, nous pouvons conclure que la somme originale 
    diverge. 

    \begin{align*}
        \boxed{
        \sum_{n=1}^{\infty }|a_n| \approx 
        \left[
        \sum_{n=1}^{\infty } \dfrac{1}{n^3} \;\;
        (\textcolor{myb}{\textbf{conv}}., \; \textbf{Riemann}) 
        \right]
        \geq 
        \textcolor{myb}{
            \sum_{n=1}^{\infty } \dfrac{1}{n^3 + 1}
        }  
    }
    \end{align*}            


    \vspace{1em}
    \noindent %
    \textbf{10.} $\sum_{n=1}^{\infty }\dfrac{(-3)^n}{(2n + 1)!}$ 

    Évaluons la somme par le test de Cauchy :

    \begin{align*}
        \lim\limits_{n \to+\infty }\sqrt[n]{|a_n|} =  
        \lim\limits_{n \to+\infty } 
        \sqrt[n]{\left|\dfrac{(-3)^n}{(2n + 1)!}\right|} 
        = 
        \lim\limits_{n \to+\infty } 
        \frac{3}{\sqrt[n]{(2n+1)}(2n)!} 
        \longrightarrow \textcolor{myb}{\textbf{0}} 
    \end{align*} 

    Par le critère de \textbf{Cauchy}, nous pouvons conclure que 
    la somme \textcolor{myb}{\textbf{conv}}. 

    \begin{align*}
    \boxed{
        \left[ 
        \lim\limits_{n \to+\infty } \sqrt[n]{|a_n|} = 0 < 1 
    \right] 
    \implies 
    \sum_{n=1}^{\infty }a_n \;\; \textcolor{myb}{\textbf{conv}}.
    }
    \end{align*}


    \vspace{1em}
    \noindent 
    \textbf{26.} $\sum_{n=1}^{\infty }\dfrac{(-2)^n}{n^n}$  


    Évaluons la somme par le test de Cauchy :

    \begin{align*}
        \lim\limits_{n \to+\infty }\sqrt[n]{|a_n|} 
        = 
        \lim\limits_{n \to+\infty } 
        \sqrt[n]{ \left| \dfrac{(-2)^n}{n^n} \right| } = 
        \lim\limits_{n \to+\infty } 
        \dfrac{2}{n} 
        \longrightarrow \textcolor{myb}{\textbf{0}} < 1  
    \end{align*}

    Par le critère de \textbf{Cauchy}, nous pouvons conclure que 
    la somme \textcolor{myb}{\textbf{conv}}. 

    \begin{Exercice}{(Stewart 1.5.10)}{}
        Utilisez le terst du rapport pour déterminer si la série est convergente ou 
        divergente.
    \end{Exercice}

    \vspace{1em} 
    \noindent
    \textbf{10.} $\sum_{n=1}^{\infty }\dfrac{(-3)^n}{(2n + 1)!}$

    Évaluons la limite du rapport :

    \begin{align*}
        \lim\limits_{n \to+\infty } \left|\dfrac{a_{n+1}}{a_n}\right|  =  
        \lim\limits_{n \to+\infty } 
        \left|
        \dfrac{(-3)^{n+1}}{(2n + 2)(2n +1)!} \cdot \frac{(2n +1)!}{(-3)^n} 
        \right|
        = 
        \lim\limits_{n \to+\infty }  
        \left|
        \frac{(-3)^1}{(2n + 2)} 
        \right|
        \longrightarrow 0 < 1 
    \end{align*}

    Ainsi, par le test du rapport, nous pouvons conclure que la série \textbf{converge}. 

    \begin{align*}
        \boxed{
        \lim\limits_{n \to+\infty }\frac{a_{n+1}}{a_n}  = 0 < 1 
        \implies \sum_{n=1}^{\infty }\dfrac{(-3)^n}{(2n + 1)!} \;\; \textcolor{myb}{\textbf{conv}}. 
    }
    \end{align*}    
    
    \begin{Exercice}{(Stewart 1.5.25 - 1.5.30)}{}
        Utilisez le critère de convergence de Cauchy pour déterminer si la  
        série est convergente ou divergente.
    \end{Exercice}

    \vspace{1em}%
    
    \noindent
    \textbf{26.} $\sum_{n=1}^{\infty }\dfrac{(-2)^n}{n^n}$  

    Selon \textbf{Cauchy}, nous avons : 

    \begin{align*}
        \lim\limits_{n \to+\infty } \sqrt[n]{|a_n|} = 
        \lim\limits_{n \to+\infty } \sqrt[n]{\left|\dfrac{(-2)^n}{n^n}\right|} =
        \lim\limits_{n \to+\infty } \frac{2}{n} \longrightarrow 0 < 1  
    \end{align*}

    Ainsi, par le test de Cauchy, nous pouvons conclure que la série est \textcolor{myb}{\textbf{conv}}. 

    \begin{align*}
        \boxed{
        \left[\lim\limits_{n \to+\infty } \sqrt[n]{\left| \dfrac{(-2)^n}{n^n} \right|} = 0 < 1 \right] 
        \implies 
        \sum_{n=1}^{\infty } \sqrt[n]{\dfrac{(-2)^n}{n^n}} \;\; \textcolor{myb}{\textbf{conv}}. 
    }
    \end{align*}

    \vspace{1em}
    \noindent
    \textbf{28.}  $\sum_{n=1}^{\infty }\left(  \dfrac{-2n}{n+1} \right)^{5n}$ 

    \vspace{1em}
    Selon \textbf{Cauchy}, nous avons : 

    \begin{align*}
        \lim\limits_{n \to+\infty } \sqrt[n]{\left| a_n \right|} = 
        \lim\limits_{n \to+\infty }  \sqrt[n]{\left| \left(  \dfrac{-2n}{n+1} \right)^{5n} \right|} = 
        2^5 \lim\limits_{n \to+\infty } \dfrac{n^5}{(n + 1)^5} \longrightarrow 2^5 \cdot 1 > 1
    \end{align*}

    Ainsi, par le test de Cauchy, nous pouvons conclure que la série \textcolor{myr}{\textbf{div}}. 


    \begin{align*}
        \boxed{
        \left[ \lim\limits_{n \to+\infty } \sqrt[n]{\left| \left(  \dfrac{-2n}{n+1} \right)^{5n} \right|}  
        \longrightarrow 2^5 > 1 \right]
        \implies \sum_{n=1}^{\infty }\left(  \dfrac{-2n}{n+1} \right)^{5n} \;\; \textcolor{myr}{\textbf{div}}.
    }
    \end{align*}

 \begin{Exercice}{(Stewart 1.5.31 - 1.5.38)}{}
        Utilisez le test approprié pour déterminer si la  
        série est convergente ou divergente.
    \end{Exercice}


    \vspace{1em}
    \noindent 
    \textbf{31.} $\sum_{n=2}^{\infty }\dfrac{(-1)^n}{\ln n}$   

    On a un série alternée de la forme $a_n = (-1)^n b_n  = (-1)^n \dfrac{1}{\ln n}$.
    Ainsi, nous pouvons appliquer le test des \textbf{séries alternées} :

    \begin{align*}
        \lim\limits_{n \to+\infty } b_n = \lim\limits_{n \to+\infty } \dfrac{1}{\ln n} \longrightarrow 0 
    \end{align*}

    Ainsi, par le test des séries alternées, nous pouvons conclure que la série \textcolor{myb}{\textbf{conv}}.    


    \begin{align*}
        \boxed{
        \left[ a_n = (-1)^n b_n, \;\; \lim\limits_{n \to+\infty } b_n = 0 \right] 
        \implies \sum_{n=2}^{\infty }a_n = 
        \sum_{n=1}^{\infty }  \frac{(-1)^n}{\ln n } \;\; \textcolor{myb}{\textbf{conv}}.  
    }
    \end{align*}



    \vspace{1em}
    \noindent 
    \textbf{37.} $\sum_{n=1}^{\infty }\left( 1 + \dfrac{1}{n}  \right)^{n^2}$

    Appliquons le critère de divergence en évaluant le limite du terme général de la série :

    \begin{align*}
        \lim\limits_{n \to+\infty } a_n = 
        \lim\limits_{n \to+\infty } \left( 1 + \cancelto{0}{\frac{1}{n}} \right)^{n^2}   
        \longrightarrow 1^{\infty} = 1
    \end{align*}
    Puisque le terme général $a_n$ de la série approche la valeur $L = 1 \neq 0$, nous pouvons 
    conclure que la somme \textcolor{myr}{\textbf{div}}. 

    \begin{align*}
        \boxed{
        \left[ \lim\limits_{n \to+\infty } a_n = 1 \neq 0  \right] 
        \implies
        \sum_{n=1}^{\infty } \left( 1 + \dfrac{1}{n}  \right)^{n^2} \;\; \textcolor{myr}{\textbf{div}}. 
    }
    \end{align*}                


    \begin{Exercice}{(Stewart 1.5.40)}{}
        On définit les termes d'une série par les équatons de récurrence  
        \begin{align*}
            a_1 = 2 \quad a_{n+1} = \dfrac{5n + 1}{4n + 3}a_n.
        \end{align*}
        Déterminez si $\sum_{n=1}^{\infty }a_n$ converge ou diverge.
    \end{Exercice}
    Nous souhaitons déterminer si la série 
    \[
    \sum_{n=1}^{\infty} a_n
    \] 
    converge ou diverge.
    À partir de la relation de récurrence, nous obtenons que
    \[
    a_{n+1} = \frac{5n + 1}{4n + 3} a_n.
    \]
    Lorsque $n \to \infty$, le terme de ratio 
    \[
    \frac{5n + 1}{4n + 3}
    \]
    tend vers 
    \[
    \frac{5}{4} > 1.
    \]
    Ainsi, chaque terme $a_{n+1}$ est environ $\frac{5}{4}$ fois plus grand que $a_n$ 
    lorsque $n$ devient grand.
    Le comportement de $a_n$ est donné par :
    \[
    a_n = a_1 \prod_{k=1}^{n-1} \frac{5k + 1}{4k + 3}.
    \]
    Comme $\frac{5k + 1}{4k + 3} \to \frac{5}{4}$ lorsque $k \to \infty$, le produit 
    tend à croître de manière exponentielle. Plus précisément, $a_n$ croît comme une 
    suite géométrique de raison $\frac{5}{4}$.
    Puisque $a_n$ ne tend pas vers 0 mais diverge vers l'infini, la série 
    \[
    \sum_{n=1}^{\infty} a_n
    \]
    diverge selon le critère de divergence (ou critère de nœud).

    \[
    \boxed{\text{La série diverge.}}
\]
    


        






\end{document}
