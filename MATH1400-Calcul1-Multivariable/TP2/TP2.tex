\documentclass{report}
%\usepackage[utopia]{mathdesign}
%\usepackage{amsmath, amsthm}
\usepackage{pgfplots}


\usepackage{amsmath,amsfonts,amsthm,amssymb,mathtools}
%\usepackage[varbb]{newpxmath}
%\usepackage[osf,largesc,theoremfont]{newpxtext}
%\usepackage{coelacanth}
%\usepackage{beraserif} % Bitstream Vera Serif font
%\usepackage{berasans} % Bitstream Vera Sans font
%\usepackage{beramono} % Bitstream Vera Sans Mono font
%\usepackage{berasans}
%\usepackage{libertine}
%\usepackage{mathpazo}
%\usepackage{palatino}
%\usepackage{crimson}


%% Choose one of the following (if not choosing the  
%% default, viz., Computer Modern, font family):
%\usepackage{lmodern}
\usepackage{bold-extra}
%%
%\usepackage{mathpazo}
% \usepackage{newpxmath}
%\usepackage{kpfonts} % Very good
%%
%\usepackage{mathptmx} %Very good
%\usepackage{stix} 
%\usepackage{txfonts} %Very good
%\usepackage{newtxtext,newtxmath} %Very good
%%
%\usepackage{libertine} \usepackage[libertine]{newtxmath}
%\usepackage{libertine,libertinust1math} % added 2019/11/28
%%
%\usepackage{newpxtext} 
%\usepackage{breqn} 
\usepackage[euler-digits]{eulervm}
%\usepackage{textcomp}
%\usepackage{bm}
\usepackage{contour}
\usepackage{adjustbox}






\input{/home/cryptopsy/Semesters/LaTeXTemplates/UniversalTeXTemplate/preamble.tex}
%From M275 "Topology" at SJSU
\newcommand{\id}{\mathrm{id}} % Identité
\newcommand{\taking}[1]{\xrightarrow{#1}} % Flèche avec annotation
\newcommand{\inv}{^{-1}} % Inverse

%From M170 "Introduction to Graph Theory" at SJSU
\DeclareMathOperator{\diam}{diam} % Diamètre
\DeclareMathOperator{\ord}{ord} % Ordre
\newcommand{\defeq}{\overset{\mathrm{def}}{=}} % Défini comme égal

%From the USAMO .tex files
\newcommand{\ts}{\textsuperscript} % Exposant
\newcommand{\dg}{^\circ} % Degré
\newcommand{\ii}{\item} % Item

% % From Math 55 and Math 145 at Harvard
% \newenvironment{subproof}[1][Proof]{%
% \begin{proof}[#1] \renewcommand{\qedsymbol}{$\blacksquare$}}%
% {\end{proof}}

\newcommand{\liff}{\leftrightarrow} % Si et seulement si
\newcommand{\lthen}{\rightarrow} % Implique
\newcommand{\opname}{\operatorname} % Opérateur générique
\newcommand{\surjto}{\twoheadrightarrow} % Flèche surjective
\newcommand{\injto}{\hookrightarrow} % Flèche injective
\newcommand{\On}{\mathrm{On}} % Ordinaux
\DeclareMathOperator{\img}{im} % Image
\DeclareMathOperator{\Img}{Im} % Image
\DeclareMathOperator{\coker}{coker} % Cokernel
\DeclareMathOperator{\Coker}{Coker} % Cokernel
\DeclareMathOperator{\Ker}{Ker} % Noyau
\DeclareMathOperator{\rank}{rank} % Rang
\DeclareMathOperator{\Spec}{Spec} % Spectre
\DeclareMathOperator{\Tr}{Tr} % Trace
\DeclareMathOperator{\pr}{pr} % Projection
\DeclareMathOperator{\ext}{ext} % Extension
\DeclareMathOperator{\pred}{pred} % Prédécesseur
\DeclareMathOperator{\dom}{dom} % Domaine
\DeclareMathOperator{\ran}{ran} % Image (range)
\DeclareMathOperator{\Hom}{Hom} % Homomorphisme
\DeclareMathOperator{\Mor}{Mor} % Morphismes
\DeclareMathOperator{\End}{End} % Endomorphisme

\newcommand{\eps}{\epsilon} % Épsilon
\newcommand{\veps}{\varepsilon} % Variance d'épsilon
\newcommand{\ol}{\overline} % Ligne au-dessus
\newcommand{\ul}{\underline} % Ligne en-dessous
\newcommand{\wt}{\widetilde} % Tilde large
\newcommand{\wh}{\widehat} % Chapeau large
\newcommand{\vocab}[1]{\textbf{\color{blue} #1}} % Texte en gras et bleu
\providecommand{\half}{\frac{1}{2}} % Fraction 1/2
\newcommand{\dang}{\measuredangle} % Angle dirigé
\newcommand{\ray}[1]{\overrightarrow{#1}} % Ray
\newcommand{\seg}[1]{\overline{#1}} % Segment
\newcommand{\arc}[1]{\wideparen{#1}} % Arc
\DeclareMathOperator{\cis}{cis} % cis
\DeclareMathOperator*{\lcm}{lcm} % Plus petit commun multiple
\DeclareMathOperator*{\argmin}{arg min} % Argument du minimum
\DeclareMathOperator*{\argmax}{arg max} % Argument du maximum
\newcommand{\cycsum}{\sum_{\mathrm{cyc}}} % Somme cyclique
\newcommand{\symsum}{\sum_{\mathrm{sym}}} % Somme symétrique
\newcommand{\cycprod}{\prod_{\mathrm{cyc}}} % Produit cyclique
\newcommand{\symprod}{\prod_{\mathrm{sym}}} % Produit symétrique
\newcommand{\Qed}{\begin{flushright}\qed\end{flushright}} % QED aligné à droite
\newcommand{\parinn}{\setlength{\parindent}{1cm}} % Indentation de paragraphe à 1 cm
\newcommand{\parinf}{\setlength{\parindent}{0cm}} % Pas d'indentation de paragraphe
% \newcommand{\norm}{\|\cdot\|} % Norme
\newcommand{\inorm}{\norm_{\infty}} % Norme infinie
\newcommand{\opensets}{\{V_{\alpha}\}_{\alpha\in I}} % Ensemble ouvert
\newcommand{\oset}{V_{\alpha}} % Ensemble ouvert V
\newcommand{\opset}[1]{V_{\alpha_{#1}}} % Ensemble ouvert V avec indice
\newcommand{\lub}{\text{lub}} % Plus petite borne supérieure
\newcommand{\del}[2]{\frac{\partial #1}{\partial #2}} % Dérivée partielle
\newcommand{\Del}[3]{\frac{\partial^{#1} #2}{\partial^{#1} #3}} % Dérivée partielle d'ordre élevé
\newcommand{\deld}[2]{\dfrac{\partial #1}{\partial #2}} % Dérivée partielle avec dfrac
\newcommand{\Deld}[3]{\dfrac{\partial^{#1} #2}{\partial^{#1} #3}} % Dérivée partielle d'ordre élevé avec dfrac
\newcommand{\lm}{\lambda} % Lambda
\newcommand{\uin}{\mathbin{\rotatebox[origin=c]{90}{$\in$}}} % Appartient, tourné de 90 degrés
\newcommand{\usubset}{\mathbin{\rotatebox[origin=c]{90}{$\subset$}}} % Sous-ensemble, tourné de 90 degrés
\newcommand{\lt}{\left} % Gauche
\newcommand{\rt}{\right} % Droite
\newcommand{\bs}[1]{\boldsymbol{#1}} % Symbole en gras
\newcommand{\exs}{\exists} % Il existe
\newcommand{\st}{\strut} % Strut
\newcommand{\dps}[1]{\displaystyle{#1}} % Disposition en ligne

\newcommand{\sol}{\setlength{\parindent}{0cm}\textbf{\textit{Solution:}}\setlength{\parindent}{1cm} } % Solution sans indentation initiale puis rétablie
\newcommand{\solve}[1]{\setlength{\parindent}{0cm}\textbf{\textit{Solution: }}\setlength{\parindent}{1cm}#1 \Qed}

\newcommand{\entoure}[1]{\fcolorbox{black}{gray!30}{\texttt{#1}}}

\renewcommand{\ttdefault}{cmtt}
\newcommand{\textttbf}[1]{\contour{yellow!45}{\texttt{#1}}}
\newcommand{\varitem}[3][black]{%
    \item [%
        \colorbox{#2}{\textcolor{#1}{\makebox(5.5,7){#3}}}%
    ]
}
% Allow you to do the non implication (implication barred)
\newcommand{\notimplies}{%
  \mathrel{{\ooalign{\hidewidth$\not\phantom{=}$\hidewidth\cr$\implies$}}}}


\newcommand*{\authorimg}[1]%
    { \raisebox{-1\baselineskip}{\includegraphics[width=\imagesize]{#1}}}
\newlength\imagesize 

\input{/home/cryptopsy/Semesters/LaTeXTemplates/UniversalTeXTemplate/letterfonts.tex}
% lstlistingsEnvs.tex

\usepackage{minted}


\lstset{
  basicstyle=\ttfamily, % Set
  columns=fullflexible,
  keepspaces=true,
  language=Python % You can specify the language if you want syntax highlighting
}

%%%%%%%%%%%%%%%%%%%%%%%%%%%%%%%%%%%%%%%%%%%%%%%%%%%%%%%%%%%%%%%%%%%%%%%%%%%%%%%%%%%%%%%%%%%%%%%%%
%                                 Custom lstlisting Environments
%%%%%%%%%%%%%%%%%%%%%%%%%%%%%%%%%%%%%%%%%%%%%%%%%%%%%%%%%%%%%%%%%%%%%%%%%%%%%%%%%%%%%%%%%%%%%%%%%
% Gruvbox style for Python
\definecolor{Pgruvbox-bg}{HTML}{282828}
\definecolor{Pgruvbox-fg}{HTML}{ebdbb2}
\definecolor{Pgruvbox-red}{HTML}{fb4934}
\definecolor{Pgruvbox-green}{HTML}{b8bb26}
\definecolor{Pgruvbox-yellow}{HTML}{fabd2f}
\definecolor{Pgruvbox-blue}{HTML}{83a598}
\definecolor{Pgruvbox-purple}{HTML}{d3869b}
\definecolor{Pgruvbox-aqua}{HTML}{8ec07c}
\definecolor{BBBlack}{rgb}{0.05, 0.06, 0.09}



% JAVA LSTLISTING STYLE IN Gruvbox Colorscheme
\definecolor{gruvbox-bg}{rgb}{0.282, 0.247, 0.204}
\definecolor{gruvbox-fg1}{rgb}{0.949, 0.898, 0.776}
\definecolor{gruvbox-fg2}{rgb}{0.871, 0.804, 0.671}
\definecolor{gruvbox-red}{rgb}{0.788, 0.255, 0.259}
\definecolor{gruvbox-green}{rgb}{0.518, 0.604, 0.239}
\definecolor{gruvbox-yellow}{rgb}{0.914, 0.808, 0.427}
\definecolor{gruvbox-blue}{rgb}{0.353, 0.510, 0.784}
\definecolor{gruvbox-purple}{rgb}{0.576, 0.412, 0.659}
\definecolor{gruvbox-aqua}{rgb}{0.459, 0.631, 0.737}
\definecolor{gruvbox-gray}{rgb}{0.518, 0.494, 0.471}

\definecolor{lst-bg}{RGB}{45, 45, 45}
\definecolor{lst-fg}{RGB}{220, 220, 204}
\definecolor{lst-keyword}{RGB}{215, 186, 125}
\definecolor{lst-comment}{RGB}{117, 113, 94}
\definecolor{lst-string}{RGB}{163, 190, 140}
\definecolor{lst-number}{RGB}{181, 206, 168}
\definecolor{lst-type}{RGB}{218, 142, 130}

\lstdefinestyle{PythonGruvbox}{
    language=Python,
    identifierstyle=\color{lst-fg},
    basicstyle=\ttfamily\color{Pgruvbox-fg},
    keywordstyle=\color{Pgruvbox-yellow},
    keywordstyle=[2]\color{Pgruvbox-blue},
    stringstyle=\color{Pgruvbox-green},
    commentstyle=\color{Pgruvbox-aqua},
    backgroundcolor=\color{BBBlack},
    rulecolor=\color{BBBlack},
    showstringspaces=false,
    keepspaces=true,
    captionpos=b,
    breaklines=true,
    tabsize=4,
    showspaces=false,
    numbers=left,
    numbersep=5pt,
    numberstyle=\tiny\color{gray},
    showtabs=false,
    columns=fullflexible,
    morekeywords={True,False,None},
    morekeywords=[2]{and,as,assert,break,class,continue,def,del,elif,else,except,exec,
    finally,for,from,global,if,import,in,is,lambda,nonlocal,not,or,pass,print,raise,
    return,try,while,with,yield},
    morecomment=[s]{"""}{"""},
    morecomment=[s]{'''}{'''},
    morecomment=[l]{\#},
    morestring=[b]",
    morestring=[b]',
    literate=
    {0}{{\textcolor{Pgruvbox-purple}{0}}}{1}
    {1}{{\textcolor{Pgruvbox-purple}{1}}}{1}
    {2}{{\textcolor{Pgruvbox-purple}{2}}}{1}
    {3}{{\textcolor{Pgruvbox-purple}{3}}}{1}
    {4}{{\textcolor{Pgruvbox-purple}{4}}}{1}
    {5}{{\textcolor{Pgruvbox-purple}{5}}}{1}
    {6}{{\textcolor{Pgruvbox-purple}{6}}}{1}
    {7}{{\textcolor{Pgruvbox-purple}{7}}}{1}
    {8}{{\textcolor{Pgruvbox-purple}{8}}}{1}
    {9}{{\textcolor{Pgruvbox-purple}{9}}}{1}
}

% Gruvbox style for Java
\definecolor{gruvbox-bg}{rgb}{0.282, 0.247, 0.204}
\definecolor{gruvbox-fg1}{rgb}{0.949, 0.898, 0.776}
\definecolor{gruvbox-fg2}{rgb}{0.871, 0.804, 0.671}
\definecolor{gruvbox-red}{rgb}{0.788, 0.255, 0.259}
\definecolor{gruvbox-green}{rgb}{0.518, 0.604, 0.239}
\definecolor{gruvbox-yellow}{rgb}{0.914, 0.808, 0.427}
\definecolor{gruvbox-blue}{rgb}{0.353, 0.510, 0.784}
\definecolor{gruvbox-purple}{rgb}{0.576, 0.412, 0.659}
\definecolor{gruvbox-aqua}{rgb}{0.459, 0.631, 0.737}
\definecolor{gruvbox-gray}{rgb}{0.518, 0.494, 0.471}

\lstdefinestyle{JavaGruvbox}{
    language=Java,
    basicstyle=\ttfamily\color{Pgruvbox-fg},
    keywordstyle=\color{Pgruvbox-yellow},
    keywordstyle=[2]\color{lst-type},
    commentstyle=\itshape\color{lst-comment},
    stringstyle=\color{lst-string},
    numberstyle=\color{lst-number},
    backgroundcolor=\color{BBBlack},
    rulecolor=\color{gruvbox-aqua},
    showstringspaces=false,
    keepspaces=true,
    captionpos=b,
    breaklines=true,
    tabsize=4,
    showspaces=false,
    showtabs=false,
    columns=fullflexible,
    morekeywords={var},
    morekeywords=[2]{boolean, byte, char, double, float, int, long, short, void},
    morecomment=[s]{/}{/},
    morecomment=[l]{//},
    morestring=[b]",
    morestring=[b]',
    numbers=left,
    numbersep=5pt,
    numberstyle=\tiny\color{gray},
}

% Dracula style for Java
\definecolor{draculawhite-background}{RGB}{237, 239, 252}
\definecolor{draculawhite-comment}{RGB}{98, 114, 164}
\definecolor{draculawhite-keyword}{RGB}{189, 147, 249}
\definecolor{draculawhite-string}{RGB}{152, 195, 121}
\definecolor{draculawhite-number}{RGB}{249, 189, 89}
\definecolor{draculawhite-operator}{RGB}{248, 248, 242}

\lstdefinestyle{JavaDraculaWhite}{
    language=Java,
    backgroundcolor=\color{draculawhite-background},
    commentstyle=\itshape\color{draculawhite-comment},
    keywordstyle=\color{draculawhite-keyword},
    stringstyle=\color{draculawhite-string},
    basicstyle=\ttfamily\footnotesize\color{black},
    identifierstyle=\color{black},
    keywordstyle=\color{draculawhite-keyword}\bfseries,
    morecomment=[s][\color{draculawhite-comment}]{/**}{*/},
    showstringspaces=false,
    showspaces=false,
    breaklines=true,
    %frame=single,
    rulecolor=\color{draculawhite-operator},
    tabsize=2,  
    numbers=left,
    numbersep=4pt,
    numberstyle=\ttfamily\tiny\color{gray}
}

% Dracula style for Python
\definecolor{draculawhite-bg}{HTML}{FAFAFA}
\definecolor{draculawhite-fg}{HTML}{282A36}
\definecolor{pdraculawhite-keyword}{HTML}{BD93F9}
\definecolor{pdraculawhite-comment}{HTML}{6272A4}
\definecolor{draculawhite-number}{HTML}{FF79C6}

\lstdefinestyle{PythonDraculaWhite}{
    language=Python,
    basicstyle=\ttfamily\small\color{draculawhite-fg},
    backgroundcolor=\color{draculawhite-background},
    keywordstyle=\color{orange}\bfseries,
    stringstyle=\color{draculawhite-string},
    commentstyle=\color{pdraculawhite-comment}\itshape,
    numberstyle=\color{draculawhite-number},
    showstringspaces=false,
    showspaces=false,
    breaklines=true,
    frame=single,
    rulecolor=\color{draculawhite-operator}, 
    tabsize=4,
    morekeywords={as,with,1,2,3,4, 5,6,7,8,9,True,False},
    numbers=left,
    numbersep=5pt,
    numberstyle=\small\bfseries\ttfamily\color{htmlcomment},
}

% Dracula Dark style for HTML
\definecolor{htmltag}{HTML}{ff79c6}
\definecolor{htmlattr}{HTML}{f1fa8c}
\definecolor{htmlvalue}{HTML}{bd93f9}
\definecolor{htmlcomment}{HTML}{6272a4}
\definecolor{htmltext}{HTML}{401E31}
\definecolor{htmlbackground}{HTML}{282a36}
\definecolor{comphtmlbackground}{HTML}{8093FF}

\lstdefinestyle{HTMLDraculaDark}{
    basicstyle=\normalsize\bfseries\ttfamily\color{htmltext},
    commentstyle=\itshape\color{htmlcomment},
    keywordstyle=\bfseries\color{htmltag},
    stringstyle=\color{htmlvalue},
    emph={DOCTYPE,html,head,body,div,span,a,script},
    emphstyle={\color{htmltag}\bfseries},
    sensitive=true,
    showstringspaces=false,
    backgroundcolor=\color{white},
    inputencoding=utf8,
    extendedchars=true,
    language=HTML,
    tabsize=4,
    breaklines=true,
    breakatwhitespace=true,
    numbers=left,
    numbersep=10pt,
    numberstyle=\small\bfseries\ttfamily\color{htmlcomment},
    escapeinside={<@}{@>},
    rulecolor=\color{htmlbackground},
    xleftmargin=10pt,
    frame=none, 
    breaklines=true,
    postbreak=\mbox{\textcolor{gray}{$\hookrightarrow$}\space},
    showlines=false,
    moredelim=[s][\itshape\color{htmlcomment}]{<!--}{-->},
    morekeywords={id,class,type,name,value,placeholder,checked,src,href,alt},
    literate={é}{{\'e}}1 {è}{{\`e}}1 {ê}{{\^e}}1 {ë}{{\"e}}1 {à}{{\`a}}1 {ù}{{\`u}}1 {û}{{\^u}}1 {ç}{{\c{c}}}1 {â}{{\^a}}1 {î}{{\^i}}1 {ï}{{\"i}}1
}


\lstdefinestyle{Haskell}{
  frame=none,
  xleftmargin=2pt,
  stepnumber=1,
  numbers=left,
  numbersep=5pt,
  numberstyle=\ttfamily\tiny\color[gray]{0.3},
  belowcaptionskip=\bigskipamount,
  captionpos=b,
  escapeinside={*'}{'*},
  language=haskell,
  tabsize=2,
  emphstyle={\bf},
  %commentstyle=\it,
  stringstyle=\mdseries\ttfamily,
  showspaces=false,
  keywordstyle=\bfseries\ttfamily,
  columns=flexible,
  basicstyle=\small\ttfamily,
  showstringspaces=false,
  morecomment=[l]\%,
}



\lstdefinestyle{CSSDraculaLight}{
    basicstyle=\bfseries\scriptsize\ttfamily\color{htmltext},
    commentstyle=\color{htmlcomment},
    keywordstyle=\bfseries\color{htmlvalue},
    stringstyle=\color{htmlvalue},
    emph={DOCTYPE,html,head,body,div,span,a,script},
    emphstyle={\color{htmltag}\bfseries},
    sensitive=true,
    showstringspaces=false,
    backgroundcolor=\color{white},
    inputencoding=utf8,
    extendedchars=true, % Support extended characters
    frame=none, 
    %frame=tb,
    tabsize=4,
    breaklines=true,
    breakatwhitespace=true,
    numbers=left,
    numbersep=10pt,
    numberstyle=\small\bfseries\ttfamily\color{htmlcomment},
    escapeinside={<@}{@>},
    rulecolor=\color{htmlbackground},
    xleftmargin=20pt,
    % Add a vertical line for opening and closing tags
    %frame=lines,
    framesep=2pt,
    framexleftmargin=4pt,
    % Add a vertical line for closing tags that go through multiple lines
    breaklines=true,
    postbreak=\mbox{\textcolor{gray}{$\hookrightarrow$}\space},
    showlines=true,
    % Add a rule to apply commentstyle to keywords inside comments
    moredelim=[s][\color{htmlcomment}]{/*}{*/},
    literate={é}{{\'e}}1
             {è}{{\`e}}1
             {ê}{{\^e}}1
             {ë}{{\"e}}1
             {à}{{\`a}}1
             {ù}{{\`u}}1
             {û}{{\^u}}1
             {ç}{{\c{c}}}1
             {â}{{\^a}}1
             {î}{{\^i}}1
             {ï}{{\"i}}1,
    morekeywords={color, background, background-color, font-size, font-weight, border, border-radius, padding, margin, display, position, top, right, bottom, left, flex, grid, width, height, min-width, max-width, min-height, max-height, transition, transform, animation, keyframes, content, z-index,id,class,type,name,value,placeholder,checked,src,href,alt},
    morestring=[s][\color{htmltag}]{:}{;},
}












\title{\huge{MATH1400}\\\Huge{Calcul à plusieurs variables}\\\vspace{2em}Travail Pratique 2 }
\author{\huge{Franz Girardin}}
\date{\today}


   

\begin{document}

\maketitle
\newpage% or \cleardoublepage
% \pdfbookmark[<level>]{<title>}{<dest>}
\pdfbookmark[section]{\contentsname}{toc}
\tableofcontents
\pagebreak


\titleformat*{\section}{%
    \normalsize\bfseries%
}

\titleformat{\section}[block]{\normalsize\bfseries}{}{0pt}{}


    \chapter*{Exercices sur la convergence de suite et séries}
    \section{Définitions}
    
    \begin{Exercice}{(Stewart 1.2.2)}{}
       Expliquez ce que signifie 
       $\sum_{n=1}^{\infty }a_n = 5$ 
    \end{Exercice}

    Cette expression signifie que la somme ayant le terme général $a_n$ 
    converge vers la veleur $L = 5$. Autrement dit, lorsqu'on additionne les 
    termes de la somme $a_n$ de façon \textbf{indéfinie}, on obtient la 
    somme $5$.


    \begin{Exercice}{(Stewart 1.2.4)}{}
        Calculez la somme de la série $\sum_{n=1}^{\infty }a_n$ dont les 
        les sommes partielles sont données :
        \begin{align*}
                S_n = \dfrac{n^2 - 1}{4^n +1} 
        \end{align*}
    \end{Exercice}

    Soit la somme parielle $S_n$, nous pouvons calculer la série comme suit :


    \begin{align*}
        S = 
        \lim\limits_{n \to+\infty } S_n  = 
        \lim\limits_{n \to+\infty }\dfrac{n^2 - 1}{4^n +1} 
        = 
        \lim\limits_{n \to+\infty }  
        \dfrac{1 -  \cancelto{0}{1}}{\frac{4^n}{n^2} + \cancelto{0}{1}} 
        = 
        \lim\limits_{n \to+\infty } 
        \dfrac{n^2}{4^n} 
    \end{align*}
    Puisque la quantité $4^n$ croît plus rapidement que $n^2$, nous avons 

    \begin{align*}
        \left[ n^2 \ll 4^n \right] \implies 
        \lim\limits_{ \to+\infty }S_n = \dfrac{n^2}{4^n} \longrightarrow 0 = S 
    \end{align*}

    Ainsi, nous avons 

    \begin{align*}
        \boxed{
            \lim\limits_{n \to+\infty }S_n = S = \sum_{n=1}^{\infty }a_n = 
            \textcolor{red}{0} 
        }
    \end{align*}


    \begin{Exercice}{(Stewart 1.2.16)}{}
        Expliquez la différence entre
    \end{Exercice}

    \noindent
    \textbf{a)} 
    \begin{align*}    
        \sum_{i=1}^{n}a_i \quad 
        \text{et} \quad%
        \sum_{j=1}^{n}a_j
    \end{align*}

    Les deux sommes représentent la même somme. La différente entre elles 
    est uniquement la variable de sommation. Or, puisque les variables de 
    sommation $i$ et $j$ sont considéré comme des variables muettes, le nom 
    de la variable n'affecte pas le résultat de la somme. 



    \vspace{1em}
    \textbf{b)}
    \begin{align*}
        \sum_{i=1}^{\infty }a_i 
        \quad%
        \text{et}
        \quad%
        \sum_{i=1}^{\infty }a_j
    \end{align*}

    Les sommes sont \textbf{différentes}. La première somme implique l'addition de termes 
    $a_i$ sur un intervalle de sommation de $i = 1$ à l'infini. Or, pour la seconde 
    somme la variable de sommation $j$ n'affecte pas les termes $a_j$ de la somme. 
    \textbf{Ainsi}, la somme constante par rapport à $i$.   


    \section{Convergence de série géométrique}
    \begin{Exercice}{(Stewart 1.2.20)}{}
        Déterminez si la série géométrique converge ou diverge. Si elle converge, 
        trouvez sa somme. 
    \end{Exercice}

    
    \noindent
    \textbf{20.} 
    \begin{align*}
            \sum_{n=1}^{\infty }a_n = 2 + 0.5 + 0.125 + 0.03125 + \cdots 
    \end{align*}

    La série géométrique suit la règle $a_1 = 2$ et $a_n = a_1r^{n-1} \forall n \geq 1$.
    \textbf{Donc}, nous avons :
    \begin{align*}
        \left[ a_2 = a_1r^{n-1}  \right] \implies 
        \left[ 0.5 = 2 + r^{1} \right] \implies 
        \textcolor{myb}{r =  -\frac{3}{2}} 
    \end{align*}

    \textbf{Ainsi}, nous avons :
    
    \begin{align*}
        \sum_{n=1}^{\infty }a_n = \sum_{n=1}^{\infty } 2r^{-3/2}
    \end{align*}

    Puisque la raison $r = -3/2$ de la somme est hors de l'intervalle de convergence 
    d'une suite géométrique, nous pouvons conclure que la somme \textcolor{myr}{\textbf{div}}.

    \begin{align*}
        \boxed{
        \left[ r = -3/2 \notin ]-1, 1] \right] \implies \sum_{n=1}^{\infty }a_n 
        \;\; \textcolor{myr}{\textbf{div}} 
    }
    \end{align*}

    \section{Convergence de série}
    \begin{Exercice}{(Stewart 17-26)}{}
        Déterminez si la série géométrique converge ou diverge.
        Si elle converge, trouvez sa somme
    \end{Exercice}


    \noindent \textbf{24.}                    

    \begin{align*}
         \sum\limits_{n=1}^{\infty } \dfrac{3^{n+1}}{(-2)^n}
    \end{align*}

    Consirérons la minupulation suivante :

    \begin{align*}
         \sum\limits_{n=1}^{\infty } \dfrac{3^{n+1}}{(-2)^n} = 
         (-1)^n \dfrac{3^{n+1}}{2^n}
    \end{align*}

    Selon le \textbf{théorème de convergence du terme général}, une 
    une condition nécessaire à la convergence d'une série est que 
    la limite du terme général $a_n$ de celle-ci tende vers $0$. 
    \textbf{Ainsi}, nous avons : 

    \begin{align*}
        \lim\limits_{n \to+\infty }a_n = 
        \lim\limits_{n\to+\infty } (-1)^n \dfrac{3^{n+1}}{2^n} = 
        \lim\limits_{n\to+\infty } 3(-1)^n \dfrac{3^n}{2^n} =
        \lim\limits_{n\to+\infty } 3(-1)^n \left(\dfrac{3}{2} \right)^n
        \textcolor{red}{\neq} \; 0  
    \end{align*}

    Ainsi, puisque la limite du terme général $a_n$ ne tend pas vers $0$, 
    nous pouvons conclure que la suite est \textcolor{myr}{\textbf{div}}. 


    \vspace{1em}

    \noindent\textbf{25}.  
    \begin{align*}
        \sum_{n=1}^{\infty }\dfrac{e^{2n}}{6^{n-1}}
    \end{align*}

    Considérons la limite du terme général $a_n$ :

    \begin{align*}
        \lim\limits_{n \to+\infty }a_n = 
        \lim\limits_{n \to+\infty } \dfrac{e^{2n}}{6^{n-1}} = 
        \lim\limits_{n \to+\infty } \dfrac{6e^{2n}}{6^{n}}  = 
        6 \left[\lim\limits_{n \to+\infty } 
        \left(\dfrac{e^{2}}{6}\right)^n \right]
        \longrightarrow + \infty \textcolor{red}{\neq} \; 0  
    \end{align*}

    Puisque la limite du terme général \textbf{tend vers l'infini}, 
    nous pouvons conclure que la suite 
    $\sum_{n=1}^{\infty }a_n$ \textcolor{myr}{\textbf{div}}.    



    \noindent\textbf{26}.  


    \begin{align*}
        \sum_{n=1}^{\infty }\dfrac{6 - 2^{2n -1}}{3^n} = 
        \sum_{n=1}^{\infty }
        \left[% 
        \frac{6}{3^n}  
        - 
        \dfrac{2^{2n -1}}{3^n}
        \right]%
    \end{align*}

    Évaluons le terme général $a_n$ de la suite, sachant que 
    $a_n = b_n  + c_n$ où $b_n = \sum_{n=1}^{\infty }\dfrac{6}{3^n}$ 
    et $c_n = \sum_{n=1}^{\infty }\dfrac{2^{2n -1}}{3^n}$ 

    \begin{align*}
        \lim\limits_{n \to+\infty }a_n = 
        \lim\limits_{n \to+\infty }b_n  
        + 
        \lim\limits_{n \to+\infty }c_n 
        = 
        \cancelto{0}{
        \left[ 
            \lim\limits_{n\to+\infty }\sum_{n=1}^{\infty }\frac{6}{3^n} 
        \right]}
        +
        \dfrac{1}{2}
        \lim\limits_{n \to+\infty} \dfrac{2^{2n}}{3^n}
        = \textcolor{red}{0} +   
        \frac{1}{2} 
        \lim\limits_{n \to+\infty} \left(\dfrac{4}{3}\right)^n 
    \end{align*}                    


    La seconde suite est une \textbf{suite géométrique} de raison 
    $r  = \frac{4}{3} \geq 1$. Ce suite divergente. \textbf{Ainsi}, puisque 
    la somme $\sum_{n=1}^{\infty }a_n$ se décompose en une suite 
    d'une somme $\sum_{n=1}^{\infty }b_n + c_n$ et la limite 
    du terme général 
    $b_n + c_n = a_n$ tend vers l'infini, on peut conclure que la 
    sommme $\sum_{n=1}^{\infty }a_n$ \textcolor{myr}{\textbf{div}}. 

    \begin{Exercice}{(Stewart 27 - 42)}{}
        Déterminez si la série converge ou diverge. Si elle converge, trouvez 
        la somme.
    \end{Exercice}

    \vspace{1em}
    \noindent
    \textbf{27.}  

    \begin{align*}
        \dfrac{1}{3} + \frac{1}{6} + \frac{1}{9} + \frac{1}{12} + \frac{1}{15} + \cdots    
    \end{align*}

    Nous avons la somme : 
    

    \begin{align*}
        \dfrac{1}{3} + \frac{1}{6} + \frac{1}{9} + \frac{1}{12} + \frac{1}{15} + \cdots    
        = 
        \sum_{n=1}^{\infty }\dfrac{1}{3^n} = \sum_{n=1}^{\infty }3^{-n}
    \end{align*}

    Il s'agit d'une suite géométrique de raison $r =  -1$, par la critère de convergence 
    d'une suite géométrique, on peut conclure que la somme \textcolor{myr}{\textbf{div}} :

    \begin{align*}
        \boxed{
        r = -1 \notin ]-1, 1] \implies \sum_{n=1}^{\infty }ar^n \textcolor{myr}{\textbf{div}} 
    }
    \end{align*}                                                                                                                                                                                                                                                                                                                                                                                                                        


    \begin{Exercice}{(Stewart 1.2.27 - 1.2.42)}{}
        Déterminez si la série converge ou diverge. SI elle converge, trouvez sa somme
    \end{Exercice}
    
    
    \noindent
    \textbf{28.} 
    $\sum_{n=1}^{\infty }a_n = \dfrac{1}{3} + \dfrac{2}{9} + \dfrac{1}{27} + 
    \dfrac{2}{81} + \dfrac{1}{243} + \dfrac{2}{729} + \cdots$

    \vspace{1em}
    Nous pouvons déduire le terme général $a_n$ de la somme : 

    \begin{align*}
        \sum_{n=1}^{\infty }a_n = 
        \sum_{n=1}^{\infty }\dfrac{3 - (-1)^n}{2} \cdot \frac{1}{3^n}  
    \end{align*}            

    Il s'agit donc d'une \textbf{suite géométrique} de raison \boxed{$r = 1/3 < 1$}.
    \textbf{Ainsi}, on peut conclure que la série \textcolor{myb}{\textbf{conv}}. 



    \section{Comparaison de séries}

    \begin{Exercice}{(Stewart 1.3.2)}{}
        Supposez que $f$ est une fonction continue, positive et décroissante 
        pour $x \geq 1$ et que $a_n = f(n)$. À l'aide d'un figure, classez les trois 
        quantités suivantes dans l'ordre croissant.

        \begin{align*}
            \int_{1}^{6}f(x)dx \quad 
            \sum_{i=1}^{5 }a_i \quad 
            \sum_{i=2}^{6 }a_i
        \end{align*}
    \end{Exercice}

    Selon le \textbf{théorème du reste}, nous avons :   

    \begin{align*}
        \int_{n+1}^{\infty }f(x)dx  \leq 
        \sum_{n=1}^{\infty }a_n = R_n \leq 
        \int_{n=1}^{\infty }f(x)dx 
        \\
    \end{align*}

     Supposons que $f$ est une fonction continue, positive et décroissante pour 
    $x \geq 1$, et que $a_n = f(n)$. Nous devons classer les trois quantités 
    suivantes dans l'ordre croissant :

    \[
    \int_{1}^{6} f(x) dx, \quad \sum_{i=1}^{5} a_i, \quad \sum_{i=2}^{6} a_i
    \]

    La somme $\sum_{i=1}^{5} a_i$ additionne les valeurs de $f(x)$ pour 
    $i = 1$ à $i = 5$. Comme $f$ est \textbf{décroissante}, chaque terme $a_i$ est plus 
    grand que $a_{i+1}$.
    \vspace{1em}

    La somme $\sum_{i=2}^{6} a_i$ additionne les valeurs de $f(x)$ pour 
    $i = 2$ à $i = 6$, \textbf{en excluant} $a_1$, qui est le plus grand terme. Donc, 
    $\sum_{i=1}^{5} a_i > \sum_{i=2}^{6} a_i$.
    \vspace{1em}

    L'intégrale $\int_{1}^{6} f(x) dx$ représente \textbf{l'aire sous la courbe }   
    de $f(x)$ entre $x = 1$ et $x = 6$. Cette aire se situe entre les deux 
    sommes discrètes, car l'intégrale correspond à la somme d'une infinité 
    de petites contributions situées entre les rectangles formés par les 
    sommes discrètes.

    \noindent
    Ainsi, nous avons :

    \[
        \boxed{\sum_{i=2}^{6} a_i < \int_{1}^{6} f(x) dx < \sum_{i=1}^{5} a_i}
    \]

        \begin{center}
        \begin{tikzpicture}[xscale=1, yscale=0.6]
            % Axes
            \draw[->] (0,0) -- (8,0) node[right] {$x$};
            \draw[->] (0,0) -- (0,6) node[above] {$y$};



            \filldraw[fill=orange!30, draw=black] (1,0) rectangle (2,5/1);
            \filldraw[fill=orange!30, draw=black] (2,0) rectangle (3,5/2);
            \filldraw[fill=orange!30, draw=black] (3,0) rectangle (4,5/3);
            \filldraw[fill=orange!30, draw=black] (4,0) rectangle (5,5/4);
            \filldraw[fill=orange!30, draw=black] (5,0) rectangle (6,5/5);



            % Blocs avec bordures noires
            \filldraw[fill=red!30, draw=black] (1,0) rectangle (2,5/2);
            \filldraw[fill=red!30, draw=black] (2,0) rectangle (3,5/3);
            \filldraw[fill=red!30, draw=black] (3,0) rectangle (4,5/4);
            \filldraw[fill=red!30, draw=black] (4,0) rectangle (5,5/5);
            \filldraw[fill=red!30, draw=black] (5,0) rectangle (6,5/6);


            % Courbe par-dessus

            \draw[] (1,0) -- (1,5);
            \draw[thick, myr, domain=0.7:8.2] plot (\x, {5/\x}) node[above] {};

            \node at (2, 5.5) {\textcolor{myr}{$y =f(x)$}};

            \node at (3.5, 3.75) {\textcolor{red!70}{$S_1 = 
                \sum_{i = 1}^{6 }a_i$}};
            % Étiquettes
            \node at (4.5, 2.75) {\textcolor{orange!70}{$S_2 = 
                \sum_{i = 1}^{5 }a_i$}};


            \node at (1.5, 2.) {\textcolor{red}{$i_{1}$}};
            \node at (2.5, 1.25) {\textcolor{red}{$i_{3}$}};
            \node at (3.5, 0.75) {\textcolor{red}{$i_{3}$}};
            \node at (4.5, 0.5) {\textcolor{red}{$i_{4}$}};
            \node at (5.5, 0.3) {\textcolor{red}{$i_{5}$}};




            % Points de référence
            \node[below] at (1,0) {$1$};
            \node[below] at (2,0) {$2$};
            \node[below] at (6,0) {$6$};
        \end{tikzpicture}
        \end{center}


    \begin{Exercice}{(Stewart 1.3.8)}{}
       Utilisez le test de l'intégrale pour déterminer si la série converge 
       ou diverge. 

       \begin{align*}
            \sum_{n=1}^{\infty }n^2e^{-n^3}
       \end{align*}
   \end{Exercice}


    \noindent
    Le \textbf{test de l'intégrale} consiste à évaluer l'intégrale correspondante pour 
    une fonction continue positive et décroissante associée au terme général 
    de la série. Considérons la fonction :
    \[
    f(x) = x^2 e^{-x^3}
    \]

    Nous devons évaluer l'intégrale impropre suivante :

    \[
    \int_{1}^{\infty} x^2 e^{-x^3} \, dx
    \]

    Pour résoudre cette intégrale, nous effectuons le \textbf{changement de variable }   
    \( u = x^3 \), ce qui donne \( du = 3x^2 dx \), ou encore :

    \[
    dx = \frac{du}{3x^2}
    \]

    Ainsi, l'intégrale devient :

    \[
    \int_{1}^{\infty} x^2 e^{-x^3} \, dx = \int_{1}^{\infty} \frac{e^{-u}}{3} \, du
    \]

    Cette dernière intégrale est une intégrale exponentielle classique :

    \[
    \frac{1}{3} \int_{1}^{\infty} e^{-u} \, du
    \]

    La primitive de \( e^{-u} \) est \( -e^{-u} \), donc nous avons :

    \[
    \frac{1}{3} \left[ -e^{-u} \right]_{1}^{\infty} 
    = \frac{1}{3} \left( 0 + e^{-1} \right) = \frac{e^{-1}}{3}
    \]

    L'intégrale converge donc, ce qui implique que la série \(\sum_{n=1}^{\infty} n^2 e^{-n^3}\) 
    converge par le test de l'intégrale. 


    \begin{Exercice}{(Stewart 1.3.9-1.3.26)}{}
        Déterminez si la série converge ou diverge
    \end{Exercice}

    \vspace{1em}
    \section{Utilisation du critère de Riemann}
    \noindent
    \textbf{14.} $\sum_{n=1}^{\infty }a_n = 
    1 + \dfrac{1}{2\sqrt{2}} + \dfrac{1}{3\sqrt{3}} + \dfrac{1}{4\sqrt{4}} + \dfrac{1}{5\sqrt{5}} + \cdots$ 

    \vspace{1em}
    Nous pouvons déduire le terme général $a_n$ de la suite et réécrire la somme comme suit :

    \begin{align*}
        \sum_{n=1}^{\infty }a_n = \dfrac{1}{n\sqrt{n}} = \sum_{n=1}^{\infty }\dfrac{1}{n^{3/2}} 
    \end{align*}

    Puisqu'il s'agit d'une \textbf{série à termes positifs}, nous pouvons appliquer le critère de Riemann. 
    \textbf{Ainsi}, nous avons :
    \begin{align*}
        \lim\limits_{n\to \infty}n^pa_n = \lim\limits_{n \to+\infty }n^{3/2}\dfrac{1}{n^{3/2}} = l = 1  
    \end{align*}

    Puisque la quantité $p$ est telle que $p > 1$, par \textbf{le critère de Riemann}, la série \textcolor{myb}{\textbf{conv}}. 

    \vspace{1em}
    \noindent 
    \textbf{20.} $\sum_{n=3}^{\infty }\dfrac{3n - 4}{n^2 -2n}$

    \vspace{1em}
    Puisqu'il s'agit d'une \textbf{série à termes positifs}, nous pouvons appliquer le 
    critère de Riemann. 
    \textbf{Considérons} la fraction polynômiale donnée 
    par le terme général :
    $$a_n = \dfrac{q}{r} = \frac{3n -4}{n^2 -2n} $$  
    \textbf{Considérons} le degré le plus faible de cette 
    fraction polynômiale, soit $\deg(q) = 1 = p$. Multiplions la fraction par 
    $n^p = n^1$.
    \textbf{Ainsi}, nous avons :
    \begin{align*}
        n a_n = n \cdot \frac{3n -4}{n^2 -2} = \frac{3n^2 -4n}{n^2 -2}  
    \end{align*}

    Nous pouvons maintenant évaluer la limite :
    \begin{align*}
        \lim\limits_{n \to+\infty }n^pa_n = 
        \lim\limits_{n \to+\infty }\frac{3n^2 -4n}{n^2 -2}  
        \equiv
        \lim\limits_{n \to+\infty } \dfrac{3 - \frac{4}{n} }{1 - \frac{2}{n}} 
        \longrightarrow 3
    \end{align*}

    Ainsi, nous avons $p = 1 \leq 1$ et $l = 3 \; \textcolor{red}{ \neq } \; 0 $.
    Par le \textbf{critère de Riemann}, nous pouvons conclure que la série 
    \textcolor{myr}{\textbf{div}}.


    \begin{Exercice}{(Stewart 1.3.28)}{}
        Expliquez pourquoi on ne peut pas utiliser le test de l'intégrale pour
        déterminer si la série converge. 
        \begin{align*}
            \sum_{n=1}^{\infty }\dfrac{\cos^2n}{1 + n^2}
        \end{align*}
    \end{Exercice}

    Le test de l'intégral peut s'appliquer sur une fonction 
    $f : [1, \infty] \rightarrow \mathbb{R}$ \textbf{positive} 
    \textbf{croissante} et \textbf{continue}. Or, si l'une de ces conditions n'est pas 
    respectée, on ne peut appliquer le test. La suite  
    $\sum_{n=1}^{\infty }a_n$ est associée à la fonction $f(x)$ correspondante :
    \begin{align*}
        \left[ a_n = \dfrac{\cos^2n}{1 + n^2}  \right]
        \implies f(x) = \dfrac{\cos^2x}{1 + x^2}
    \end{align*}            

    Or, le numérateur de la fonction est $2\pi$ périodique ; 
    la fonction n'est donc pas monotone. Ainsi, nous ne pouvonsa lui appliquer 
    le test de l'intégrale. 



    \section{Convergence de série alternées}
    \begin{Exercice}{(Stewart 1.4.7)}{}
        Déterminez si la série converge ou diverge.
    \end{Exercice}

    \noindent
    \textbf{7.} $\sum_{n=1}^{\infty }(-1)^n \dfrac{3n - 1}{2n + 1}$

    Appliquons le critère de convergence nécessaire pour le terme général. 
    Soit une série $\sum_{n=1}^{\infty }a_n$, une condition nécessaire pour que 
    cette série converge est que la limite du terme général tende vers 0. 
    Nous avons : 

    \begin{align*}
        \lim\limits_{n \to+\infty }a_n  =  
        \lim\limits_{n \to+\infty } \dfrac{3n - 1}{2n + 1} \equiv
        \lim\limits_{n \to+\infty }  
        \dfrac{3 - \cancelto{0}{\frac{1}{n}}}{2 + \cancelto{0}{\frac{1}{n}}}
        \longrightarrow \dfrac{3}{2} \; \textcolor{red}{\neq} \; 0 
    \end{align*}        

    \textbf{Ainsi}, par le \textbf{critère de divergence du terme général},
    nous pouvons conclure que la série \textcolor{myr}{\textbf{div}}.   
    
    \begin{align*}
        \boxed{
        \left[  \lim\limits_{n \to+\infty }a_n \neq 0  \right]
        \implies \sum_{n=1}^{\infty }a_n \longrightarrow \infty \; 
        (\textcolor{myr}{\textbf{div}} )
    }
    \end{align*}

    \vspace{1em}% 
    \noindent
    \textbf{10.} $\sum_{n=1}^{\infty }(-1)^n\dfrac{\sqrt{n}}{2n +3}$   

    La somme ressemble à une \textbf{série alternée} sur laquelle on peut
    appliquer le critère de convergence des série alternées. 
    Vérifions la \textbf{limite du terme général}  $a_n$ : 

    \begin{align*}
        \lim\limits_{n\to \infty}a_n =  
        \lim\limits_{n \to+\infty } (-1)^n\dfrac{\sqrt{n}}{2n +3} \equiv
        \lim\limits_{n \to +\infty }
        (-1)^n \cdot \frac{1}{2n^{1/2} + 
        \cancelto{0}{\frac{3}{n^{1/2}} }}
        \longrightarrow  \textcolor{myb}{\textbf{0}}  
    \end{align*}

    Ainsi, nous savons que la limite du terme général tend vers 0. 
    Considérons la la fonction $f : [N, \infty] \rightarrow \mathbb{R}$
    telle que $f(n) = b_n$. Calculons la dérivée :

    \begin{align*}
        \dfrac{d}{dx}f(x) = \dfrac{d}{dx}
        \left[ \dfrac{(n)^{1/2}}{2n +3}\right]
        =
        \dfrac{\left( n^{1/2}\right)^{\prime}(2n +3) 
        - (2n + 3)^{\prime}n^{1/2}}{(2n + 3)^2} 
        = 
        \dfrac{(2n+3)\cdot 1/2n^{-1/2} -1/2(n^{1/2})}{(2n +3)^2} 
        = 
        \dfrac{\dfrac{(2 + 3)}{n^{1/2}} - \dfrac{n^{1/2}}{2}}{(2n +3)^2} 
    \end{align*}


    Puisque la différence du numérateur est négative 
    $\forall n \in \mathbb{N}$, la fraction polynômiale engendre 
    une quantité négative. Ainsi, la dérivée est négative, ce 
    qui implique que la fonction est \textbf{décroissante}.   
    \textbf{Ainsi}, la suite $a_{n+1}$ est décroissante et majorée 
    par $a_n$ et puisque le terme général $a_n$ tend vers 
    0 lorsque $n \longrightarrow \infty$, par 
    \textbf{le critère des série alternées}, nous pouvons conclure 
    que la série \textcolor{myb}{\textbf{conv}}. 

    \begin{align*}
        \boxed{ 
        \left[
        \lim\limits_{n \to+\infty }b_n = 0 
        \; 
        \textbf{et}
        \; 
        f(n) = b_n, f^{\prime}(x) \leq 0, \forall \;\; n \geq N 
    \right] 
    \implies \sum_{n=1}^{\infty }a_n \; \textcolor{myb}{\textbf{conv}} 
    }
    \end{align*}


    \section{Convergence absolue}

    \begin{Exercice}{(Stewart 1.5.4)}{}
       Déterminez si la série est \textbf{absoluement convergente}, 
       simplement convergente ou divergente. 
    \end{Exercice}

    \vspace{1em} 
    \noindent % 
    \textbf{4.} $\sum_{n=1}^{\infty }\dfrac{(-1)^n}{n^3 + 1}$ 

    Considérons la valeur absolue du terme général $|a_n|$ :

    \begin{align*}
        \sum_{n=1}^{\infty }|a_n| = 
        \sum_{n=1}^{\infty }\left|\dfrac{(-1)^n}{n^3 +1}\right|
        = 
        \dfrac{1}{n^3 + 1}, \; \forall \;\; n \geq 0 
    \end{align*}


    \begin{align*}
        \sum_{n=1}^{\infty }| a_n | \approx 
        \sum_{n=1}^{\infty }\dfrac{1}{n^3} =
        \sum_{n=1}^{\infty }n^p, p = 3
    \end{align*}

    Il s'agit d'une \textbf{série de Riemann} avec $p = 3 > 1$. 
    Par la \textbf{définition d'une série de Riemann}, nous pouvons 
    conclure que la somme converge \textbf{absoluement}. puisque cette 
    somme est plus grande que la somme originale, par le test 
    de comparaison, nous pouvons conclure que la somme originale 
    diverge. 

    \begin{align*}
        \boxed{
        \sum_{n=1}^{\infty }|a_n| \approx 
        \left[
        \sum_{n=1}^{\infty } \dfrac{1}{n^3} \;\;
        (\textcolor{myb}{\textbf{conv}}, \; \textbf{Riemann}) 
        \right]
        \geq 
        \textcolor{myb}{
            \sum_{n=1}^{\infty } \dfrac{1}{n^3 + 1}
        }  
    }
    \end{align*}            


    \vspace{1em}
    \noindent %
    \textbf{10.} $\sum_{n=1}^{\infty }\dfrac{(-3)^n}{(2n + 1)!}$ 

    Évaluons la somme par le test de Cauchy :

    \begin{align*}
        \lim\limits_{n \to+\infty }\sqrt[n]{|a_n|} =  
        \lim\limits_{n \to+\infty } 
        \sqrt[n]{\left|\dfrac{(-3)^n}{(2n + 1)!}\right|} 
        = 
        \lim\limits_{n \to+\infty } 
        \frac{3}{\sqrt[n]{(2n+1)}(2n)!} 
        \longrightarrow \textcolor{myb}{\textbf{0}} 
    \end{align*} 

    Par le critère de \textbf{Cauchy}, nous pouvons conclure que 
    la somme \textcolor{myb}{\textbf{conv}}. 

    \begin{align*}
    \boxed{
        \left[ 
        \lim\limits_{n \to+\infty } \sqrt[n]{|a_n|} = 0 < 1 
    \right] 
    \implies 
    \sum_{n=1}^{\infty }a_n \;\; \textcolor{myb}{\textbf{conv}} 
    }
    \end{align*}


    \vspace{1em}
    \noindent 
    \textbf{26.} $\sum_{n=1}^{\infty }\dfrac{(-2)^n}{n^n}$  


    Évaluons la somme par le test de Cauchy :

    \begin{align*}
        \lim\limits_{n \to+\infty }\sqrt[n]{|a_n|} 
        = 
        \lim\limits_{n \to+\infty } 
        \sqrt[n]{ \left| \dfrac{(-2)^n}{n^n} \right| } = 
        \lim\limits_{n \to+\infty } 
        \dfrac{2}{n} 
        \longrightarrow \textcolor{myb}{\textbf{0}} < 1  
    \end{align*}

    Par le critère de \textbf{Cauchy}, nous pouvons conclure que 
    la somme \textcolor{myb}{\textbf{conv}}. 




\end{document}
