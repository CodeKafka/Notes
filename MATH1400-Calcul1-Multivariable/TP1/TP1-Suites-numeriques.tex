\documentclass{report}
%\usepackage[utopia]{mathdesign}
%\usepackage{amsmath, amsthm}
\usepackage{pgfplots}


\usepackage{amsmath,amsfonts,amsthm,amssymb,mathtools}
%\usepackage[varbb]{newpxmath}
%\usepackage[osf,largesc,theoremfont]{newpxtext}
%\usepackage{coelacanth}
%\usepackage{beraserif} % Bitstream Vera Serif font
%\usepackage{berasans} % Bitstream Vera Sans font
%\usepackage{beramono} % Bitstream Vera Sans Mono font
%\usepackage{berasans}
%\usepackage{libertine}
%\usepackage{mathpazo}
%\usepackage{palatino}
%\usepackage{crimson}


%% Choose one of the following (if not choosing the  
%% default, viz., Computer Modern, font family):
%\usepackage{lmodern}
\usepackage{bold-extra}
%%
%\usepackage{mathpazo}
% \usepackage{newpxmath}
%\usepackage{kpfonts} % Very good
%%
%\usepackage{mathptmx} %Very good
%\usepackage{stix} 
%\usepackage{txfonts} %Very good
\usepackage{newtxtext,newtxmath} %Very good
%%
%\usepackage{libertine} \usepackage[libertine]{newtxmath}
%\usepackage{libertine,libertinust1math} % added 2019/11/28
%%
%\usepackage{newpxtext} 
%\usepackage{breqn} 
\usepackage[euler-digits]{eulervm}
%\usepackage{textcomp}
%\usepackage{bm}
\usepackage{contour}
\usepackage{adjustbox}






%%%%%%%%%%%%%%%%%%%%%%%%%%%%%%%%%%%%%%%%%%%%%%%%%%%%%%%%%%%%%%%%%%%%%%%%%%%%%%%%%%%%%%%%%%%%%%%%%
%                                 Additional Packages (commented)
%%%%%%%%%%%%%%%%%%%%%%%%%%%%%%%%%%%%%%%%%%%%%%%%%%%%%%%%%%%%%%%%%%%%%%%%%%%%%%%%%%%%%%%%%%%%%%%%%
%%%%%%%%%%%%%%%%%%%%%%%%%%%%%%%%%%%%%%%%%%%%%%%%%%%%%%%%%%%%%%%%%%%%%%%%%%%%%%%%%%%%%%%%%%%%%%%%%
%                                 Language and Encoding
%%%%%%%%%%%%%%%%%%%%%%%%%%%%%%%%%%%%%%%%%%%%%%%%%%%%%%%%%%%%%%%%%%%%%%%%%%%%%%%%%%%%%%%%%%%%%%%%%
\usepackage[french]{babel} % Utilisation du français pour la typographie et les hyphenations
\usepackage[T1]{fontenc} % Utilisation de l'encodage de police T1 pour une meilleure gestion des caractères accentués
\usepackage{titlesec}
\usepackage[utf8]{inputenc} % Permet l'utilisation de l'encodage UTF-8 dans le fichier source
\usepackage{csquotes} % Gestion avancée des guillemets
\usepackage{microtype}
\usepackage{listings}

%%%%%%%%%%%%%%%%%%%%%%%%%%%%%%%%%
% PACKAGE IMPORTS
%%%%%%%%%%%%%%%%%%%%%%%%%%%%%%%%%


\usepackage[tmargin=2cm,rmargin=1in,lmargin=1in,margin=0.85in,bmargin=2cm,footskip=.2in]{geometry}
\usepackage{xfrac}
\usepackage[makeroom]{cancel}
\usepackage{bookmark}
\usepackage{enumitem}
\usepackage{hyperref,theoremref}
\hypersetup{
	pdftitle={Assignment},
	colorlinks=true, linkcolor=black,
	bookmarksnumbered=true,
	bookmarksopen=true
}
\hypersetup{
    colorlinks=true,       % false: boxed links; true: colored links
    linkcolor=black,       % color of internal links
    citecolor=black,       % color of citation links
    filecolor=black,       % color of file links
    urlcolor=black         % color of external links (like \url and \href)
}
\usepackage[most,many,breakable]{tcolorbox}
\usepackage{xcolor}
\usepackage{varwidth}
\usepackage{varwidth}
\usepackage{etoolbox}
%\usepackage{authblk}
\usepackage{nameref}
\usepackage{multicol,array}
\usepackage{colortbl}
\usepackage{tikz-cd}
\usepackage[ruled,vlined,linesnumbered]{algorithm2e}
\usepackage{comment} % enables the use of multi-line comments (\ifx \fi) 
\usepackage{import}
\usepackage{xifthen}
\usepackage{pdfpages}
\usepackage{transparent}

\newcommand\mycommfont[1]{\footnotesize\ttfamily\textcolor{blue}{#1}}
\SetCommentSty{mycommfont}
\newcommand{\incfig}[1]{%
    \def\svgwidth{\columnwidth}
    \import{./figures/}{#1.pdf_tex}
}

\usepackage{tikzsymbols}
%\renewcommand\qedsymbol{$\Laughey$}


%\usepackage{import}
%\usepackage{xifthen}
%\usepackage{pdfpages}
%\usepackage{transparent}


%%%%%%%%%%%%%%%%%%%%%%%%%%%%%%
% SELF MADE COLORS
%%%%%%%%%%%%%%%%%%%%%%%%%%%%%%



\definecolor{lightBlue}{rgb}{0.88,1,1}
\definecolor{myg}{RGB}{56, 140, 70}
\definecolor{myb}{RGB}{45, 111, 177}
\definecolor{myr}{RGB}{199, 68, 64}
\definecolor{mytheorembg}{HTML}{F2F2F9}
\definecolor{mytheoremfr}{HTML}{00007B}
\definecolor{mylenmabg}{HTML}{FFFAF8}
\definecolor{mylenmafr}{HTML}{983b0f}
\definecolor{mypropbg}{HTML}{f2fbfc}
\definecolor{mypropfr}{HTML}{191971}
\definecolor{myexamplebg}{HTML}{F2FBF8}
\definecolor{myexamplefr}{HTML}{88D6D1}
\definecolor{myexampleti}{HTML}{2A7F7F}
\definecolor{mydefinitbg}{HTML}{E5E5FF}
\definecolor{mydefinitfr}{HTML}{3F3FA3}
\definecolor{notesgreen}{RGB}{0,162,0}
\definecolor{myp}{RGB}{197, 92, 212}
\definecolor{mygr}{HTML}{2C3338}
\definecolor{myred}{RGB}{127,0,0}
\definecolor{myyellow}{RGB}{169,121,69}
\definecolor{myexercisebg}{HTML}{F2FBF8}
\definecolor{myexercisefg}{HTML}{88D6D1}
\definecolor{myblue}{RGB}{0,82,155}

\usepackage{forest}
\usepackage{adjustbox}



%%%%%%%%%%%%%%%%%%%%%%%%%%%%
% chapter format
%%%%%%%%%%%%%%%%%%%%%%%%%%%%

\titleformat{\chapter}[display]
  {\normalfont\bfseries\color{doc!60}}
  {\filleft%
    \begin{tikzpicture}
    \node[
      outer sep=-4pt,
      text width=0.75cm,
      minimum height=1cm,
      fill=doc!60,
      font=\color{white}\fontsize{20}{25}\selectfont\fontfamily{lmss},
      align=center
      ] (num) {\thechapter};
    \node[
      rotate=90,
      anchor=south,
      font=\color{black}\normalsize\fontfamily{lmss}
      ] at ([xshift=-5pt]num.west) {\textls[180]{\textsc{}}};  
    \end{tikzpicture}%
  }
  {10pt}
  {\titlerule[2.5pt]\vskip1.5pt\titlerule\vskip4pt\large\bfseries\sc}
\titlespacing*{\chapter}{0pt}{*0}{*0}

\makeatletter
\patchcmd{\chapter}{\if@openright\cleardoublepage\else\clearpage\fi}{}{}{}
\makeatother




%%%%%%%%%%%%%%%%%%%%%%%%%%%%
% TCOLORBOX SETUPS
%%%%%%%%%%%%%%%%%%%%%%%%%%%%

\setlength{\parindent}{1cm}
%================================
% THEOREM BOX
%================================

\tcbuselibrary{theorems,skins,hooks}
\newtcbtheorem[number within=section]{Theorem}{Theorem}
{%
	enhanced,
	breakable,
	colback = mytheorembg,
	frame hidden,
	boxrule = 0sp,
	borderline west = {2pt}{0pt}{mytheoremfr},
	sharp corners,
	detach title,
	before upper = \tcbtitle\par\smallskip,
	coltitle = mytheoremfr,
	fonttitle = \bfseries\sffamily,
	description font = \mdseries,
	separator sign none,
	segmentation style={solid, mytheoremfr},
}
{th}

\tcbuselibrary{theorems,skins,hooks}
\newtcbtheorem[number within=chapter]{theorem}{Theorem}
{%
	enhanced,
	breakable,
	colback = mytheorembg,
	frame hidden,
	boxrule = 0sp,
	borderline west = {2pt}{0pt}{mytheoremfr},
	sharp corners,
	detach title,
	before upper = \tcbtitle\par\smallskip,
	coltitle = mytheoremfr,
	fonttitle = \bfseries\sffamily,
	description font = \mdseries,
	separator sign none,
	segmentation style={solid, mytheoremfr},
}
{th}


\tcbuselibrary{theorems,skins,hooks}
\newtcolorbox{Theoremcon}
{%
	enhanced
	,breakable
	,colback = mytheorembg
	,frame hidden
	,boxrule = 0sp
	,borderline west = {2pt}{0pt}{mytheoremfr}
	,sharp corners
	,description font = \mdseries
	,separator sign none
}

%================================
% Preuve
%================================

% Crée un environnement "Preuve" numéroté en fonction du document
\tcbuselibrary{theorems,skins,hooks}
\newtcbtheorem[number within=chapter]{Preuve}{Preuve}
{
	enhanced,
	breakable,
	colback=white,
	frame hidden,
	boxrule = 0sp,
	borderline west = {2pt}{0pt}{mytheoremfr},
	sharp corners,
	detach title,
	before upper = \tcbtitle\par\smallskip,
	coltitle = mytheoremfr,
	description font=\fontfamily{lmss}\selectfont,
	fonttitle=\fontfamily{lmss}\selectfont\bfseries,
	separator sign none,
	segmentation style={solid, mytheoremfr},
}
{th}



%================================
% Corollery
%================================
\tcbuselibrary{theorems,skins,hooks}
\newtcbtheorem[number within=section]{Corollary}{Corollaire}
{%
	enhanced
	,breakable
	,colback = myp!10
	,frame hidden
	,boxrule = 0sp
	,borderline west = {2pt}{0pt}{myp!85!black}
	,sharp corners
	,detach title
	,before upper = \tcbtitle\par\smallskip
	,coltitle = myp!85!black
	,fonttitle = \bfseries\sffamily
	,description font = \mdseries
	,separator sign none
	,segmentation style={solid, myp!85!black}
}
{th}
\tcbuselibrary{theorems,skins,hooks}
\newtcbtheorem[number within=chapter]{corollary}{Corollaire}
{%
	enhanced
	,breakable
	,colback = myp!10
	,frame hidden
	,boxrule = 0sp
	,borderline west = {2pt}{0pt}{myp!85!black}
	,sharp corners
	,detach title
	,before upper = \tcbtitle\par\smallskip
	,coltitle = myp!85!black
	,fonttitle = \bfseries\sffamily
	,description font = \mdseries
	,separator sign none
	,segmentation style={solid, myp!85!black}
}
{th}


%================================
% LENMA
%================================

\tcbuselibrary{theorems,skins,hooks}
\newtcbtheorem[number within=section]{Lenma}{Lemme}
{%
	enhanced,
	breakable,
	colback = mylenmabg,
	frame hidden,
	boxrule = 0sp,
	borderline west = {2pt}{0pt}{mylenmafr},
	sharp corners,
	detach title,
	before upper = \tcbtitle\par\smallskip,
	coltitle = mylenmafr,
	fonttitle = \bfseries\sffamily,
	description font = \mdseries,
	separator sign none,
	segmentation style={solid, mylenmafr},
}
{th}

\tcbuselibrary{theorems,skins,hooks}
\newtcbtheorem[number within=chapter]{lenma}{Lemme}
{%
	enhanced,
	breakable,
	colback = mylenmabg,
	frame hidden,
	boxrule = 0sp,
	borderline west = {2pt}{0pt}{mylenmafr},
	sharp corners,
	detach title,
	before upper = \tcbtitle\par\smallskip,
	coltitle = mylenmafr,
	fonttitle = \bfseries\sffamily,
	description font = \mdseries,
	separator sign none,
	segmentation style={solid, mylenmafr},
}
{th}


%================================
% PROPOSITION
%================================

\tcbuselibrary{theorems,skins,hooks}
\newtcbtheorem[number within=section]{Prop}{Proposition}
{%
	enhanced,
	breakable,
	colback = mypropbg,
	frame hidden,
	boxrule = 0sp,
	borderline west = {2pt}{0pt}{mypropfr},
	sharp corners,
	detach title,
	before upper = \tcbtitle\par\smallskip,
	coltitle = mypropfr,
	fonttitle = \bfseries\sffamily,
	description font = \mdseries,
	separator sign none,
	segmentation style={solid, mypropfr},
}
{th}

\tcbuselibrary{theorems,skins,hooks}
\newtcbtheorem[number within=chapter]{prop}{Proposition}
{%
	enhanced,
	breakable,
	colback = mypropbg,
	frame hidden,
	boxrule = 0sp,
	borderline west = {2pt}{0pt}{mypropfr},
	sharp corners,
	detach title,
	before upper = \tcbtitle\par\smallskip,
	coltitle = mypropfr,
	fonttitle = \bfseries\sffamily,
	description font = \mdseries,
	separator sign none,
	segmentation style={solid, mypropfr},
}
{th}


%================================
% CLAIM
%================================

\tcbuselibrary{theorems,skins,hooks}
\newtcbtheorem[number within=section]{claim}{Affirmation}
{%
	enhanced
	,breakable
	,colback = myg!10
	,frame hidden
	,boxrule = 0sp
	,borderline west = {2pt}{0pt}{myg}
	,sharp corners
	,detach title
	,before upper = \tcbtitle\par\smallskip
	,coltitle = myg!85!black
	,fonttitle = \bfseries\sffamily
	,description font = \mdseries
	,separator sign none
	,segmentation style={solid, myg!85!black}
}
{th}



%================================
% Exercise
%================================

\tcbuselibrary{theorems,skins,hooks}
\newtcbtheorem[number within=section]{Exercise}{Exercice}
{%
	enhanced,
	breakable,
	colback = myexercisebg,
	frame hidden,
	boxrule = 0sp,
	borderline west = {2pt}{0pt}{myexercisefg},
	sharp corners,
	detach title,
	before upper = \tcbtitle\par\smallskip,
	coltitle = myexercisefg,
	fonttitle = \bfseries\sffamily,
	description font = \mdseries,
	separator sign none,
	segmentation style={solid, myexercisefg},
}
{th}

\tcbuselibrary{theorems,skins,hooks}
\newtcbtheorem[number within=chapter]{exercise}{Exercice}
{%
	enhanced,
	breakable,
	colback = myexercisebg,
	frame hidden,
	boxrule = 0sp,
	borderline west = {2pt}{0pt}{myexercisefg},
	sharp corners,
	detach title,
	before upper = \tcbtitle\par\smallskip,
	coltitle = myexercisefg,
	fonttitle = \bfseries\sffamily,
	description font = \mdseries,
	separator sign none,
	segmentation style={solid, myexercisefg},
}
{th}

%================================
% EXAMPLE BOX
%================================

\newtcbtheorem[number within=section]{Example}{Example}
{%
	colback = myexamplebg
	,breakable
	,colframe = myexamplefr
	,coltitle = myexampleti
	,boxrule = 1pt
	,sharp corners
	,detach title
	,before upper=\tcbtitle\par\smallskip
	,fonttitle = \bfseries
	,description font = \mdseries
	,separator sign none
	,description delimiters parenthesis
}
{ex}

\newtcbtheorem[number within=chapter]{example}{Example}
{%
	colback = myexamplebg
	,breakable
	,colframe = myexamplefr
	,coltitle = myexampleti
	,boxrule = 1pt
	,sharp corners
	,detach title
	,before upper=\tcbtitle\par\smallskip
	,fonttitle = \bfseries
	,description font = \mdseries
	,separator sign none
	,description delimiters parenthesis
}
{ex}

%================================
% DEFINITION BOX
%================================

\newtcbtheorem[number within=chapter]{Definition}{Définition}{enhanced,
	before skip=2mm,after skip=2mm, colback=red!5,colframe=red!80!black,boxrule=0.5mm,
	attach boxed title to top left={xshift=1cm,yshift*=1mm-\tcboxedtitleheight}, varwidth boxed title*=-3cm,
	boxed title style={frame code={
			\path[fill=tcbcolback!10!red]
			([yshift=-1mm,xshift=-1mm]frame.north west)
			arc[start angle=0,end angle=180,radius=1mm]
			([yshift=-1mm,xshift=1mm]frame.north east)
			arc[start angle=180,end angle=0,radius=1mm];
			\path[left color=tcbcolback!10!myred,right color=tcbcolback!10!myred,
			middle color=tcbcolback!60!myred]
			([xshift=-2mm]frame.north west) -- ([xshift=2mm]frame.north east)
			[rounded corners=1mm]-- ([xshift=1mm,yshift=-1mm]frame.north east)
			-- (frame.south east) -- (frame.south west)
			-- ([xshift=-1mm,yshift=-1mm]frame.north west)
			[sharp corners]-- cycle;
		},interior engine=empty,
	},
	fonttitle=\bfseries,
	title={#2},#1}{def}
\newtcbtheorem[number within=chapter]{definition}{Definition}{enhanced,
	before skip=2mm,after skip=2mm, colback=red!5,colframe=red!80!black,boxrule=0.5mm,
	attach boxed title to top left={xshift=1cm,yshift*=1mm-\tcboxedtitleheight}, varwidth boxed title*=-3cm,
	boxed title style={frame code={
					\path[fill=tcbcolback]
					([yshift=-1mm,xshift=-1mm]frame.north west)
					arc[start angle=0,end angle=180,radius=1mm]
					([yshift=-1mm,xshift=1mm]frame.north east)
					arc[start angle=180,end angle=0,radius=1mm];
					\path[left color=tcbcolback!60!black,right color=tcbcolback!60!black,
						middle color=tcbcolback!80!black]
					([xshift=-2mm]frame.north west) -- ([xshift=2mm]frame.north east)
					[rounded corners=1mm]-- ([xshift=1mm,yshift=-1mm]frame.north east)
					-- (frame.south east) -- (frame.south west)
					-- ([xshift=-1mm,yshift=-1mm]frame.north west)
					[sharp corners]-- cycle;
				},interior engine=empty,
		},
	fonttitle=\bfseries,
	title={#2},#1}{def}


    \newtcbtheorem{Definitionx}{Définition}
    {
    enhanced,
    breakable,
    colback=red!5,
      before upper=\tcbtitle\par\Hugeskip,
    frame hidden,
    boxrule = 0sp,
    borderline west = {2pt}{0pt}{red!80!black},
    sharp corners,
    detach title,
    before upper = \tcbtitle\par\smallskip,
    coltitle = red!80!black,
    description font=\mdseries\fontfamily{lmss}\selectfont,
    fonttitle=\fontfamily{lmss}\selectfont\bfseries,
    fontlower=\fontfamily{cmr}\selectfont,
      separator sign none,
    segmentation style={solid, mytheoremfr},
    }
    {th}



%================================
% Solution BOX
%================================

\makeatletter
\newtcbtheorem{question}{Question}{enhanced,
	breakable,
	colback=white,
	colframe=myb!80!black,
	attach boxed title to top left={yshift*=-\tcboxedtitleheight},
	fonttitle=\bfseries,
	title={#2},
	boxed title size=title,
	boxed title style={%
			sharp corners,
			rounded corners=northwest,
			colback=tcbcolframe,
			boxrule=0pt,
		},
	underlay boxed title={%
			\path[fill=tcbcolframe] (title.south west)--(title.south east)
			to[out=0, in=180] ([xshift=5mm]title.east)--
			(title.center-|frame.east)
			[rounded corners=\kvtcb@arc] |-
			(frame.north) -| cycle;
		},
	#1
}{def}
\makeatother

%================================
% SOLUTION BOX
%================================

\makeatletter
\newtcolorbox{solution}{enhanced,
	breakable,
	colback=white,
	colframe=myg!80!black,
	attach boxed title to top left={yshift*=-\tcboxedtitleheight},
	title=Solution,
	boxed title size=title,
	boxed title style={%
			sharp corners,
			rounded corners=northwest,
			colback=tcbcolframe,
			boxrule=0pt,
		},
	underlay boxed title={%
			\path[fill=tcbcolframe] (title.south west)--(title.south east)
			to[out=0, in=180] ([xshift=5mm]title.east)--
			(title.center-|frame.east)
			[rounded corners=\kvtcb@arc] |-
			(frame.north) -| cycle;
		},
}
\makeatother

%================================
% Question BOX
%================================

\makeatletter
\newtcbtheorem{qstion}{Question}{enhanced,
	breakable,
	colback=white,
	colframe=mygr,
	attach boxed title to top left={yshift*=-\tcboxedtitleheight},
	fonttitle=\bfseries,
	title={#2},
	boxed title size=title,
	boxed title style={%
			sharp corners,
			rounded corners=northwest,
			colback=tcbcolframe,
			boxrule=0pt,
		},
	underlay boxed title={%
			\path[fill=tcbcolframe] (title.south west)--(title.south east)
			to[out=0, in=180] ([xshift=5mm]title.east)--
			(title.center-|frame.east)
			[rounded corners=\kvtcb@arc] |-
			(frame.north) -| cycle;
		},
	#1
}{def}
\makeatother

\newtcbtheorem[number within=chapter]{wconc}{Wrong Concept}{
	breakable,
	enhanced,
	colback=white,
	colframe=myr,
	arc=0pt,
	outer arc=0pt,
	fonttitle=\bfseries\sffamily\large,
	colbacktitle=myr,
	attach boxed title to top left={},
	boxed title style={
			enhanced,
			skin=enhancedfirst jigsaw,
			arc=3pt,
			bottom=0pt,
			interior style={fill=myr}
		},
	#1
}{def}


%%%%%%%%%%%%%%%%%%%%%%%%%%%%%%%%%%%%%%%%%%%%%%%%%%%%%%%%%%%%%%%%%%%%%%%%%%%%%%%%%%%%%%%%%%%%%%%%%
%                                Environnement Explication
%%%%%%%%%%%%%%%%%%%%%%%%%%%%%%%%%%%%%%%%%%%%%%%%%%%%%%%%%%%%%%%%%%%%%%%%%%%%%%%%%%%%%%%%%%%%%%%%%
\newtcbtheorem{Explication}{Explication}
{
	enhanced,
	breakable,
	colback=white,
	frame hidden,
	boxrule = 0sp,
	borderline west = {2pt}{0pt}{mytheoremfr},
	sharp corners,
	detach title,
	before upper = \tcbtitle\par\smallskip,
	coltitle = mytheoremfr,
	description font=\fontfamily{lmss}\selectfont,
	fonttitle=\fontfamily{lmss}\selectfont\bfseries,
	separator sign none,
	segmentation style={solid, mytheoremfr},
}
{th}



%%%%%%%%%%%%%%%%%%%%%%%%%%%%%%%%%%%%%%%%%%%%%%%%%%%%%%%%%%%%%%%%%%%%%%%%%%%%%%%%%%%%%%%%%%%%%%%%%
%                                Environnement EExample
%%%%%%%%%%%%%%%%%%%%%%%%%%%%%%%%%%%%%%%%%%%%%%%%%%%%%%%%%%%%%%%%%%%%%%%%%%%%%%%%%%%%%%%%%%%%%%%%%
% Crée un environnement "EExample" numéroté en fonction du document
\newtcbtheorem{EExample}{Exemple}
{
	enhanced,
	breakable,
	colback=white,
	frame hidden,
	boxrule = 0sp,
	borderline west = {2pt}{0pt}{myb},
	sharp corners,
	detach title,
	before upper = \tcbtitle\par\smallskip,
	coltitle = myb,
	description font=\mdseries\fontfamily{lmss}\selectfont,
	fonttitle=\fontfamily{lmss}\selectfont\bfseries,
	separator sign none,
	segmentation style={solid, mytheoremfr},
}
{th}

%%%%%%%%%%%%%%%%%%%%%%%%%%%%%%%%%%%%%%%%%%%%%%%%%%%%%%%%%%%%%%%%%%%%%%%%%%%%%%%%%%%%%%%%%%%%%%%%%
%                                Environnement Liste
%%%%%%%%%%%%%%%%%%%%%%%%%%%%%%%%%%%%%%%%%%%%%%%%%%%%%%%%%%%%%%%%%%%%%%%%%%%%%%%%%%%%%%%%%%%%%%%%%

% Pour créer un environnement "Liste" 

\tcbuselibrary{theorems,skins,hooks}
\newtcbtheorem[number within=chapter]{Liste}{Liste}
{%
	enhanced
	,breakable
	,colback = myp!10
	,frame hidden
	,boxrule = 0sp
	,borderline west = {2pt}{0pt}{myp!85!black}
	,sharp corners
	,detach title
	,before upper = \tcbtitle\par\smallskip
	,coltitle = myp!85!black
	,fonttitle = \bfseries\sffamily
	,description font = \mdseries
	,separator sign none
	,segmentation style={solid, myp!85!black}

}
{th}


%%%%%%%%%%%%%%%%%%%%%%%%%%%%%%%%%%%%%%%%%%%%%%%%%%%%%%%%%%%%%%%%%%%%%%%%%%%%%%%%%%%%%%%%%%%%%%%%%
%                                Environnement Syntaxe
%%%%%%%%%%%%%%%%%%%%%%%%%%%%%%%%%%%%%%%%%%%%%%%%%%%%%%%%%%%%%%%%%%%%%%%%%%%%%%%%%%%%%%%%%%%%%%%%%
\tcbuselibrary{theorems,skins,hooks}
\newtcbtheorem{Syntaxe}{Syntaxe.}
{%
	enhanced
	,breakable
	,colback = myp!10
	,frame hidden
	,boxrule = 0sp
	,borderline west = {2pt}{0pt}{myp!85!black}
	,sharp corners
	,detach title
	,before upper = \tcbtitle\par\smallskip
	,coltitle = myp!85!black
	,fonttitle = \bfseries\fontfamily{lmss}\selectfont 
	,description font = \mdseries\fontfamily{lmss}\selectfont 
	,separator sign none
	,segmentation style={solid, myp!85!black}
}
{th}



%%%%%%%%%%%%%%%%%%%%%%%%%%%%%%%%%%%%%%%%%%%%%%%%%%%%%%%%%%%%%%%%%%%%%%%%%%%%%%%%%%%%%%%%%%%%%%%%%
%                                Environnement Concept
%%%%%%%%%%%%%%%%%%%%%%%%%%%%%%%%%%%%%%%%%%%%%%%%%%%%%%%%%%%%%%%%%%%%%%%%%%%%%%%%%%%%%%%%%%%%%%%%%
% Crée un environnement "Concept" numéroté en fonction du document
\tcbuselibrary{theorems,skins,hooks}
\newtcbtheorem{Concept}{Concept.}
{
	enhanced,
	breakable,
	colback=mylenmabg,
	frame hidden,
	boxrule = 0sp,
	borderline west = {2pt}{0pt}{mylenmafr},
	sharp corners,
	detach title,
	before upper = \tcbtitle\par\smallskip,
	coltitle = mylenmafr,
	description font=\mdseries\fontfamily{lmss}\selectfont,
	fonttitle=\fontfamily{lmss}\selectfont\bfseries,
	separator sign none,
	segmentation style={solid, mytheoremfr},
}
{th}



%%%%%%%%%%%%%%%%%%%%%%%%%%%%%%%%%%%%%%%%%%%%%%%%%%%%%%%%%%%%%%%%%%%%%%%%%%%%%%%%%%%%%%%%%%%%%%%%%
%                                Environnement codeEx
%%%%%%%%%%%%%%%%%%%%%%%%%%%%%%%%%%%%%%%%%%%%%%%%%%%%%%%%%%%%%%%%%%%%%%%%%%%%%%%%%%%%%%%%%%%%%%%%%
% Crée un environnement "codeEx" numéroté en fonction du document
\tcbuselibrary{theorems,skins,hooks}
\newtcbtheorem{codeEx}{Exemple}
{
	enhanced,
	breakable,
	colback=white,
	frame hidden,
	boxrule = 0sp,
	borderline west = {2pt}{0pt}{gruvbox-bg},
	sharp corners,
	detach title,
	before upper = \tcbtitle\par\smallskip,
	coltitle = gruvbox-bg,
	description font=\md:wqseries\fontfamily{lmss}\selectfont,
	fonttitle=\fontfamily{lmss}\selectfont\bfseries,
	separator sign none,
	segmentation style={solid, mytheoremfr},
}
{th}



%%%%%%%%%%%%%%%%%%%%%%%%%%%%%%%%%%%%%%%%%%%%%%%%%%%%%%%%%%%%%%%%%%%%%%%%%%%%%%%%%%%%%%%%%%%%%%%%%
%                                Environnement Remarque
%%%%%%%%%%%%%%%%%%%%%%%%%%%%%%%%%%%%%%%%%%%%%%%%%%%%%%%%%%%%%%%%%%%%%%%%%%%%%%%%%%%%%%%%%%%%%%%%%
% Crée un environnement "Remarque" numéroté en fonction du document
\tcbuselibrary{theorems,skins,hooks}
\newtcbtheorem{codeRem}{Remarque}
{
	enhanced,
	breakable,
	colback=white,
	frame hidden,
	boxrule = 0sp,
	borderline west = {2pt}{0pt}{gruvbox-bg},
	sharp corners,
	detach title,
	before upper = \tcbtitle\par\smallskip,
	coltitle = gruvbox-bg,
	description font=\mdseries\fontfamily{lmss}\selectfont,
	fonttitle=\fontfamily{lmss}\selectfont\bfseries,
	separator sign none,
	segmentation style={solid, mytheoremfr},
}
{th}


%%%%%%%%%%%%%%%%%%%%%%%%%%%%%%%%%%%%%%%%%%%%%%%%%%%%%%%%%%%%%%%%%%%%%%%%%%%%%%%%%%%%%%%%%%%%%%%%%
%                                Environnement Identité
%%%%%%%%%%%%%%%%%%%%%%%%%%%%%%%%%%%%%%%%%%%%%%%%%%%%%%%%%%%%%%%%%%%%%%%%%%%%%%%%%%%%%%%%%%%%%%%%%
\tcbuselibrary{theorems,skins,hooks}
\newtcbtheorem{Identite}{Identité}
{
	enhanced,
	breakable,
	colback=white,
  before upper=\tcbtitle\par\Hugeskip,
	frame hidden,
	boxrule = 0sp,
	borderline west = {2pt}{0pt}{gruvbox-bg},
	sharp corners,
	detach title,
	before upper = \tcbtitle\par\smallskip,
	coltitle = gruvbox-bg,
	description font=\mdseries\fontfamily{lmss}\selectfont,
	fonttitle=\fontfamily{lmss}\selectfont\bfseries,
	fontlower=\fontfamily{cmr}\selectfont,
  separator sign none,
	segmentation style={solid, mytheoremfr},
}
{th}



%%%%%%%%%%%%%%%%%%%%%%%%%%%%%%%%%%%%%%%%%%%%%%%%%%%%%%%%%%%%%%%%%%%%%%%%%%%%%%%%%%%%%%%%%%%%%%%%%
%                                Environnement Exercice 
%%%%%%%%%%%%%%%%%%%%%%%%%%%%%%%%%%%%%%%%%%%%%%%%%%%%%%%%%%%%%%%%%%%%%%%%%%%%%%%%%%%%%%%%%%%%%%%%%
\tcbuselibrary{theorems,skins,hooks}
\newtcbtheorem{Exercice}{Exercice}
{
	enhanced,
	breakable,
	colback=white,
  before upper=\tcbtitle\par\Hugeskip,
	frame hidden,
	boxrule = 0sp,
	borderline west = {2pt}{0pt}{gruvbox-green},
	sharp corners,
	detach title,
	before upper = \tcbtitle\par\smallskip,
	coltitle = gruvbox-green,
	description font=\mdseries\fontfamily{lmss}\selectfont,
	fonttitle=\fontfamily{lmss}\selectfont\bfseries,
	fontlower=\fontfamily{cmr}\selectfont,
  separator sign none,
	segmentation style={solid, mytheoremfr},
}
{th}


%%%%%%%%%%%%%%%%%%%%%%%%%%%%%%%%%%%%%%%%%%%%%%%%%%%%%%%%%%%%%%%%%%%%%%%%%%%%%%%%%%%%%%%%%%%%%%%%%
%                                Environnement Réponse
%%%%%%%%%%%%%%%%%%%%%%%%%%%%%%%%%%%%%%%%%%%%%%%%%%%%%%%%%%%%%%%%%%%%%%%%%%%%%%%%%%%%%%%%%%%%%%%%%
% Crée un environnement "Réponse" numéroté en fonction du document
\tcbuselibrary{theorems,skins,hooks}
\newtcbtheorem{Reponse}{Réponse}
{
	enhanced,
	breakable,
	colback=white,
	frame hidden,
	boxrule = 0sp,
	borderline west = {2pt}{0pt}{mytheoremfr},
	sharp corners,
	detach title,
	before upper = \tcbtitle\par\smallskip,
	coltitle = mytheoremfr,
	description font=\fontfamily{lmss}\selectfont,
	fonttitle=\fontfamily{lmss}\selectfont\bfseries,
	separator sign none,
	segmentation style={solid, mytheoremfr},
}
{th}

\newtcbtheorem{Remarque}{Remarque.}
{
	enhanced,
	breakable,
	colback=white,
	frame hidden,
	boxrule = 0sp,
	borderline west = {2pt}{0pt}{myb},
	sharp corners,
	detach title,
	before upper = \tcbtitle\par\smallskip,
	coltitle = myb,
	description font=\mdseries\fontfamily{lmss}\selectfont,
	fonttitle=\fontfamily{lmss}\selectfont\bfseries,
	separator sign none,
	segmentation style={solid, mytheoremfr},
}
{th}





%================================
% NOTE BOX
%================================

\usetikzlibrary{arrows,calc,shadows.blur}
\usetikzlibrary {arrows.meta,backgrounds,fit,positioning,petri}
\tcbuselibrary{skins}
\newtcolorbox{note}[1][]{%
	enhanced jigsaw,
	colback=gray!20!white,%
	colframe=gray!80!black,
	size=small,
	boxrule=1pt,
	title=\textbf{Note:-},
	halign title=flush center,
	coltitle=black,
	breakable,
	drop shadow=black!50!white,
	attach boxed title to top left={xshift=1cm,yshift=-\tcboxedtitleheight/2,yshifttext=-\tcboxedtitleheight/2},
	minipage boxed title=1.5cm,
	boxed title style={%
			colback=white,
			size=fbox,
			boxrule=1pt,
			boxsep=2pt,
			underlay={%
					\coordinate (dotA) at ($(interior.west) + (-0.5pt,0)$);
					\coordinate (dotB) at ($(interior.east) + (0.5pt,0)$);
					\begin{scope}
						\clip (interior.north west) rectangle ([xshift=3ex]interior.east);
						\filldraw [white, blur shadow={shadow opacity=60, shadow yshift=-.75ex}, rounded corners=2pt] (interior.north west) rectangle (interior.south east);
					\end{scope}
					\begin{scope}[gray!80!black]
						\fill (dotA) circle (2pt);
						\fill (dotB) circle (2pt);
					\end{scope}
				},
		},
	#1,
}

%%%%%%%%%%%%%%%%%%%%%%%%%%%%%%
% SELF MADE COMMANDS
%%%%%%%%%%%%%%%%%%%%%%%%%%%%%%


\newcommand{\thm}[2]{\begin{Theorem}{#1}{}#2\end{Theorem}}
\newcommand{\cor}[2]{\begin{Corollary}{#1}{}#2\end{Corollary}}
\newcommand{\mlenma}[2]{\begin{Lenma}{#1}{}#2\end{Lenma}}
\newcommand{\mprop}[2]{\begin{Prop}{#1}{}#2\end{Prop}}
\newcommand{\clm}[3]{\begin{claim}{#1}{#2}#3\end{claim}}
\newcommand{\wc}[2]{\begin{wconc}{#1}{}\setlength{\parindent}{1cm}#2\end{wconc}}
\newcommand{\thmcon}[1]{\begin{Theoremcon}{#1}\end{Theoremcon}}
\newcommand{\ex}[2]{\begin{Example}{#1}{}#2\end{Example}}
\newcommand{\dfn}[2]{\begin{Definition}[colbacktitle=red!75!black]{#1}{}#2\end{Definition}}
\newcommand{\dfnc}[2]{\begin{definition}[colbacktitle=red!75!black]{#1}{}#2\end{definition}}
\newcommand{\qs}[2]{\begin{question}{#1}{}#2\end{question}}
\newcommand{\pf}[2]{\begin{myproof}[#1]#2\end{myproof}}
\newcommand{\nt}[1]{\begin{note}#1\end{note}}

\newcommand*\circled[1]{\tikz[baseline=(char.base)]{
		\node[shape=circle,draw,inner sep=1pt] (char) {#1};}}
\newcommand\getcurrentref[1]{%
	\ifnumequal{\value{#1}}{0}
	{??}
	{\the\value{#1}}%
}
\newcommand{\getCurrentSectionNumber}{\getcurrentref{section}}
\newenvironment{myproof}[1][\proofname]{%
	\proof[\bfseries #1: ]%
}{\endproof}

\newcommand{\mclm}[2]{\begin{myclaim}[#1]#2\end{myclaim}}
\newenvironment{myclaim}[1][\claimname]{\proof[\bfseries #1: ]}{}

\newcounter{mylabelcounter}

\makeatletter
\newcommand{\setword}[2]{%
	\phantomsection
	#1\def\@currentlabel{\unexpanded{#1}}\label{#2}%
}
\makeatother




\tikzset{
	symbol/.style={
			draw=none,
			every to/.append style={
					edge node={node [sloped, allow upside down, auto=false]{$#1$}}}
		}
}


% deliminators
% Manually define paired delimiters
\newcommand{\abs}[1]{\lvert#1\rvert}
\newcommand{\norm}[1]{\lVert#1\rVert}
\newcommand{\ceil}[1]{\lceil#1\rceil}
\newcommand{\floor}[1]{\lfloor#1\rfloor}
\newcommand{\round}[1]{\lfloor#1\rceil}

%
\newsavebox\diffdbox
\newcommand{\slantedromand}{{\mathpalette\makesl{d}}}
\newcommand{\makesl}[2]{%
\begingroup
\sbox{\diffdbox}{$\mathsurround=0pt#1\mathrm{#2}$}%
\pdfsave
\pdfsetmatrix{1 0 0.2 1}%
\rlap{\usebox{\diffdbox}}%
\pdfrestore
\hskip\wd\diffdbox
\endgroup
}
\newcommand{\dd}[1][]{\ensuremath{\mathop{}\!\ifstrempty{#1}{%
\slantedromand\@ifnextchar^{\hspace{0.2ex}}{\hspace{0.1ex}}}%
{\slantedromand\hspace{0.2ex}^{#1}}}}
\ProvideDocumentCommand\dv{o m g}{%
  \ensuremath{%
    \IfValueTF{#3}{%
      \IfNoValueTF{#1}{%
        \frac{\dd #2}{\dd #3}%
      }{%
        \frac{\dd^{#1} #2}{\dd #3^{#1}}%
      }%
    }{%
      \IfNoValueTF{#1}{%
        \frac{\dd}{\dd #2}%
      }{%
        \frac{\dd^{#1}}{\dd #2^{#1}}%
      }%
    }%
  }%
}
\providecommand*{\pdv}[3][]{\frac{\partial^{#1}#2}{\partial#3^{#1}}}
%  - others
\DeclareMathOperator{\Lap}{\mathcal{L}}
\DeclareMathOperator{\Var}{Var} % varience
\DeclareMathOperator{\Cov}{Cov} % covarience
\DeclareMathOperator{\E}{E} % expected

% Since the amsthm package isn't loaded

% I prefer the slanted \leq
\let\oldleq\leq % save them in case they're every wanted
\let\oldgeq\geq
\renewcommand{\leq}{\leqslant}
\renewcommand{\geq}{\geqslant}

% % redefine matrix env to allow for alignment, use r as default
% \renewcommand*\env@matrix[1][r]{\hskip -\arraycolsep
%     \let\@ifnextchar\new@ifnextchar
%     \array{*\c@MaxMatrixCols #1}}


%\usepackage{framed}
%\usepackage{titletoc}
%\usepackage{etoolbox}
%\usepackage{lmodern}


%\patchcmd{\tableofcontents}{\contentsname}{\sffamily\contentsname}{}{}

%\renewenvironment{leftbar}
%{\def\FrameCommand{\hspace{6em}%
%		{\color{myyellow}\vrule width 2pt depth 6pt}\hspace{1em}}%
%	\MakeFramed{\parshape 1 0cm \dimexpr\textwidth-6em\relax\FrameRestore}\vskip2pt%
%}
%{\endMakeFramed}

%\titlecontents{chapter}
%[0em]{\vspace*{2\baselineskip}}
%{\parbox{4.5em}{%
%		\hfill\Huge\sffamily\bfseries\color{myred}\thecontentspage}%
%	\vspace*{-2.3\baselineskip}\leftbar\textsc{\small\chaptername~\thecontentslabel}\\\sffamily}
%{}{\endleftbar}
%\titlecontents{section}
%[8.4em]
%{\sffamily\contentslabel{3em}}{}{}
%{\hspace{0.5em}\nobreak\itshape\color{myred}\contentspage}
%\titlecontents{subsection}
%[8.4em]
%{\sffamily\contentslabel{3em}}{}{}  
%{\hspace{0.5em}\nobreak\itshape\color{myred}\contentspage}



%%%%%%%%%%%%%%%%%%%%%%%%%%%%%%%%%%%%%%%%%%%
% TABLE OF CONTENTS
%%%%%%%%%%%%%%%%%%%%%%%%%%%%%%%%%%%%%%%%%%%

\usepackage{tikz}
\usetikzlibrary{automata, positioning}
\definecolor{doc}{RGB}{0,60,110}
\usepackage{titletoc}
\contentsmargin{0cm}
\titlecontents{chapter}[4.9pc]
{\addvspace{40pt}%
	\begin{tikzpicture}[remember picture, overlay]%
		\draw[fill=doc!60,draw=doc!60] (-7,-.1) rectangle (-0.9,.5);%
		\pgftext[left,x=-3.5cm,y=0.2cm]{\color{white}\Large\sc\bfseries Section \ \thecontentslabel};%
	\end{tikzpicture}\color{doc!60}\large\sc\bfseries}%
{}
{}
{\;\titlerule\;\large\sc\bfseries Page \thecontentspage
	\begin{tikzpicture}[remember picture, overlay]
		\draw[fill=doc!60,draw=doc!60] (2pt,0) rectangle (4,0.1pt);
	\end{tikzpicture}}%
\titlecontents{section}[3.7pc]
{\addvspace{2pt}}
{\contentslabel[\thecontentslabel]{2pc}}
{}
{\hfill\small \thecontentspage}
[]
\titlecontents*{subsection}[3.7pc]
{\addvspace{-1pt}\small}
{}
{}
{\ --- \small\thecontentspage}
[ \textbullet\ ][]

\makeatletter
\renewcommand{\tableofcontents}{%
	\chapter*{%
	  \vspace*{-20\p@}%
	  \begin{tikzpicture}[remember picture, overlay]%
		  \pgftext[right,x=15cm,y=0.2cm]{\color{doc!60}\Huge\sc\bfseries \contentsname};%
		  \draw[fill=doc!60,draw=doc!60] (13,-.75) rectangle (20,2);%
		  \clip (13,-.75) rectangle (20,1);
		  \pgftext[right,x=15cm,y=0.2cm]{\color{white}\Huge\sc\bfseries \contentsname};%
	  \end{tikzpicture}}%
	\@starttoc{toc}}
\makeatother






%From M275 "Topology" at SJSU
\newcommand{\id}{\mathrm{id}} % Identité
\newcommand{\taking}[1]{\xrightarrow{#1}} % Flèche avec annotation
\newcommand{\inv}{^{-1}} % Inverse

%From M170 "Introduction to Graph Theory" at SJSU
\DeclareMathOperator{\diam}{diam} % Diamètre
\DeclareMathOperator{\ord}{ord} % Ordre
\newcommand{\defeq}{\overset{\mathrm{def}}{=}} % Défini comme égal

%From the USAMO .tex files
\newcommand{\ts}{\textsuperscript} % Exposant
\newcommand{\dg}{^\circ} % Degré
\newcommand{\ii}{\item} % Item

% % From Math 55 and Math 145 at Harvard
% \newenvironment{subproof}[1][Proof]{%
% \begin{proof}[#1] \renewcommand{\qedsymbol}{$\blacksquare$}}%
% {\end{proof}}

\newcommand{\liff}{\leftrightarrow} % Si et seulement si
\newcommand{\lthen}{\rightarrow} % Implique
\newcommand{\opname}{\operatorname} % Opérateur générique
\newcommand{\surjto}{\twoheadrightarrow} % Flèche surjective
\newcommand{\injto}{\hookrightarrow} % Flèche injective
\newcommand{\On}{\mathrm{On}} % Ordinaux
\DeclareMathOperator{\img}{im} % Image
\DeclareMathOperator{\Img}{Im} % Image
\DeclareMathOperator{\coker}{coker} % Cokernel
\DeclareMathOperator{\Coker}{Coker} % Cokernel
\DeclareMathOperator{\Ker}{Ker} % Noyau
\DeclareMathOperator{\rank}{rank} % Rang
\DeclareMathOperator{\Spec}{Spec} % Spectre
\DeclareMathOperator{\Tr}{Tr} % Trace
\DeclareMathOperator{\pr}{pr} % Projection
\DeclareMathOperator{\ext}{ext} % Extension
\DeclareMathOperator{\pred}{pred} % Prédécesseur
\DeclareMathOperator{\dom}{dom} % Domaine
\DeclareMathOperator{\ran}{ran} % Image (range)
\DeclareMathOperator{\Hom}{Hom} % Homomorphisme
\DeclareMathOperator{\Mor}{Mor} % Morphismes
\DeclareMathOperator{\End}{End} % Endomorphisme

\newcommand{\eps}{\epsilon} % Épsilon
\newcommand{\veps}{\varepsilon} % Variance d'épsilon
\newcommand{\ol}{\overline} % Ligne au-dessus
\newcommand{\ul}{\underline} % Ligne en-dessous
\newcommand{\wt}{\widetilde} % Tilde large
\newcommand{\wh}{\widehat} % Chapeau large
\newcommand{\vocab}[1]{\textbf{\color{blue} #1}} % Texte en gras et bleu
\providecommand{\half}{\frac{1}{2}} % Fraction 1/2
\newcommand{\dang}{\measuredangle} % Angle dirigé
\newcommand{\ray}[1]{\overrightarrow{#1}} % Ray
\newcommand{\seg}[1]{\overline{#1}} % Segment
\newcommand{\arc}[1]{\wideparen{#1}} % Arc
\DeclareMathOperator{\cis}{cis} % cis
\DeclareMathOperator*{\lcm}{lcm} % Plus petit commun multiple
\DeclareMathOperator*{\argmin}{arg min} % Argument du minimum
\DeclareMathOperator*{\argmax}{arg max} % Argument du maximum
\newcommand{\cycsum}{\sum_{\mathrm{cyc}}} % Somme cyclique
\newcommand{\symsum}{\sum_{\mathrm{sym}}} % Somme symétrique
\newcommand{\cycprod}{\prod_{\mathrm{cyc}}} % Produit cyclique
\newcommand{\symprod}{\prod_{\mathrm{sym}}} % Produit symétrique
\newcommand{\Qed}{\begin{flushright}\qed\end{flushright}} % QED aligné à droite
\newcommand{\parinn}{\setlength{\parindent}{1cm}} % Indentation de paragraphe à 1 cm
\newcommand{\parinf}{\setlength{\parindent}{0cm}} % Pas d'indentation de paragraphe
% \newcommand{\norm}{\|\cdot\|} % Norme
\newcommand{\inorm}{\norm_{\infty}} % Norme infinie
\newcommand{\opensets}{\{V_{\alpha}\}_{\alpha\in I}} % Ensemble ouvert
\newcommand{\oset}{V_{\alpha}} % Ensemble ouvert V
\newcommand{\opset}[1]{V_{\alpha_{#1}}} % Ensemble ouvert V avec indice
\newcommand{\lub}{\text{lub}} % Plus petite borne supérieure
\newcommand{\del}[2]{\frac{\partial #1}{\partial #2}} % Dérivée partielle
\newcommand{\Del}[3]{\frac{\partial^{#1} #2}{\partial^{#1} #3}} % Dérivée partielle d'ordre élevé
\newcommand{\deld}[2]{\dfrac{\partial #1}{\partial #2}} % Dérivée partielle avec dfrac
\newcommand{\Deld}[3]{\dfrac{\partial^{#1} #2}{\partial^{#1} #3}} % Dérivée partielle d'ordre élevé avec dfrac
\newcommand{\lm}{\lambda} % Lambda
\newcommand{\uin}{\mathbin{\rotatebox[origin=c]{90}{$\in$}}} % Appartient, tourné de 90 degrés
\newcommand{\usubset}{\mathbin{\rotatebox[origin=c]{90}{$\subset$}}} % Sous-ensemble, tourné de 90 degrés
\newcommand{\lt}{\left} % Gauche
\newcommand{\rt}{\right} % Droite
\newcommand{\bs}[1]{\boldsymbol{#1}} % Symbole en gras
\newcommand{\exs}{\exists} % Il existe
\newcommand{\st}{\strut} % Strut
\newcommand{\dps}[1]{\displaystyle{#1}} % Disposition en ligne

\newcommand{\sol}{\setlength{\parindent}{0cm}\textbf{\textit{Solution:}}\setlength{\parindent}{1cm} } % Solution sans indentation initiale puis rétablie
\newcommand{\solve}[1]{\setlength{\parindent}{0cm}\textbf{\textit{Solution: }}\setlength{\parindent}{1cm}#1 \Qed}

\newcommand{\entoure}[1]{\fcolorbox{black}{gray!30}{\texttt{#1}}}

\renewcommand{\ttdefault}{cmtt}
\newcommand{\textttbf}[1]{\contour{yellow!45}{\texttt{#1}}}
\newcommand{\varitem}[3][black]{%
    \item [%
        \colorbox{#2}{\textcolor{#1}{\makebox(5.5,7){#3}}}%
    ]
}
% Allow you to do the non implication (implication barred)
\newcommand{\notimplies}{%
  \mathrel{{\ooalign{\hidewidth$\not\phantom{=}$\hidewidth\cr$\implies$}}}}


\newcommand*{\authorimg}[1]%
    { \raisebox{-1\baselineskip}{\includegraphics[width=\imagesize]{#1}}}
\newlength\imagesize 

% Things Lie
\newcommand{\kb}{\mathfrak b}
\newcommand{\kg}{\mathfrak g}
\newcommand{\kh}{\mathfrak h}
\newcommand{\kn}{\mathfrak n}
\newcommand{\ku}{\mathfrak u}
\newcommand{\kz}{\mathfrak z}
\DeclareMathOperator{\Ext}{Ext} % Ext functor
\DeclareMathOperator{\Tor}{Tor} % Tor functor
\newcommand{\gl}{\opname{\mathfrak{gl}}} % frak gl group
\renewcommand{\sl}{\opname{\mathfrak{sl}}} % frak sl group chktex 6

% More script letters etc.
\newcommand{\SA}{\mathcal A}
\newcommand{\SB}{\mathcal B}
\newcommand{\SC}{\mathcal C}
\newcommand{\SF}{\mathcal F}
\newcommand{\SG}{\mathcal G}
\newcommand{\SH}{\mathcal H}
\newcommand{\OO}{\mathcal O}

\newcommand{\SCA}{\mathscr A}
\newcommand{\SCB}{\mathscr B}
\newcommand{\SCC}{\mathscr C}
\newcommand{\SCD}{\mathscr D}
\newcommand{\SCE}{\mathscr E}
\newcommand{\SCF}{\mathscr F}
\newcommand{\SCG}{\mathscr G}
\newcommand{\SCH}{\mathscr H}

% Mathfrak primes
\newcommand{\km}{\mathfrak m}
\newcommand{\kp}{\mathfrak p}
\newcommand{\kq}{\mathfrak q}

% number sets
\newcommand{\RR}[1][]{\ensuremath{\ifstrempty{#1}{\mathbb{R}}{\mathbb{R}^{#1}}}}
\newcommand{\NN}[1][]{\ensuremath{\ifstrempty{#1}{\mathbb{N}}{\mathbb{N}^{#1}}}}
\newcommand{\ZZ}[1][]{\ensuremath{\ifstrempty{#1}{\mathbb{Z}}{\mathbb{Z}^{#1}}}}
\newcommand{\QQ}[1][]{\ensuremath{\ifstrempty{#1}{\mathbb{Q}}{\mathbb{Q}^{#1}}}}
\newcommand{\CC}[1][]{\ensuremath{\ifstrempty{#1}{\mathbb{C}}{\mathbb{C}^{#1}}}}
\newcommand{\PP}[1][]{\ensuremath{\ifstrempty{#1}{\mathbb{P}}{\mathbb{P}^{#1}}}}
\newcommand{\HH}[1][]{\ensuremath{\ifstrempty{#1}{\mathbb{H}}{\mathbb{H}^{#1}}}}
\newcommand{\FF}[1][]{\ensuremath{\ifstrempty{#1}{\mathbb{F}}{\mathbb{F}^{#1}}}}
% expected value
\newcommand{\EE}{\ensuremath{\mathbb{E}}}
\newcommand{\charin}{\text{ char }}
\DeclareMathOperator{\sign}{sign}
\DeclareMathOperator{\Aut}{Aut}
\DeclareMathOperator{\Inn}{Inn}
\DeclareMathOperator{\Syl}{Syl}
\DeclareMathOperator{\Gal}{Gal}
\DeclareMathOperator{\GL}{GL} % General linear group
\DeclareMathOperator{\SL}{SL} % Special linear group

%---------------------------------------
% BlackBoard Math Fonts :-
%---------------------------------------

%Captital Letters
\newcommand{\bbA}{\mathbb{A}}	\newcommand{\bbB}{\mathbb{B}}
\newcommand{\bbC}{\mathbb{C}}	\newcommand{\bbD}{\mathbb{D}}
\newcommand{\bbE}{\mathbb{E}}	\newcommand{\bbF}{\mathbb{F}}
\newcommand{\bbG}{\mathbb{G}}	\newcommand{\bbH}{\mathbb{H}}
\newcommand{\bbI}{\mathbb{I}}	\newcommand{\bbJ}{\mathbb{J}}
\newcommand{\bbK}{\mathbb{K}}	\newcommand{\bbL}{\mathbb{L}}
\newcommand{\bbM}{\mathbb{M}}	\newcommand{\bbN}{\mathbb{N}}
\newcommand{\bbO}{\mathbb{O}}	\newcommand{\bbP}{\mathbb{P}}
\newcommand{\bbQ}{\mathbb{Q}}	\newcommand{\bbR}{\mathbb{R}}
\newcommand{\bbS}{\mathbb{S}}	\newcommand{\bbT}{\mathbb{T}}
\newcommand{\bbU}{\mathbb{U}}	\newcommand{\bbV}{\mathbb{V}}
\newcommand{\bbW}{\mathbb{W}}	\newcommand{\bbX}{\mathbb{X}}
\newcommand{\bbY}{\mathbb{Y}}	\newcommand{\bbZ}{\mathbb{Z}}

%---------------------------------------
% MathCal Fonts :-
%---------------------------------------

%Captital Letters
\newcommand{\mcA}{\mathcal{A}}	\newcommand{\mcB}{\mathcal{B}}
\newcommand{\mcC}{\mathcal{C}}	\newcommand{\mcD}{\mathcal{D}}
\newcommand{\mcE}{\mathcal{E}}	\newcommand{\mcF}{\mathcal{F}}
\newcommand{\mcG}{\mathcal{G}}	\newcommand{\mcH}{\mathcal{H}}
\newcommand{\mcI}{\mathcal{I}}	\newcommand{\mcJ}{\mathcal{J}}
\newcommand{\mcK}{\mathcal{K}}	\newcommand{\mcL}{\mathcal{L}}
\newcommand{\mcM}{\mathcal{M}}	\newcommand{\mcN}{\mathcal{N}}
\newcommand{\mcO}{\mathcal{O}}	\newcommand{\mcP}{\mathcal{P}}
\newcommand{\mcQ}{\mathcal{Q}}	\newcommand{\mcR}{\mathcal{R}}
\newcommand{\mcS}{\mathcal{S}}	\newcommand{\mcT}{\mathcal{T}}
\newcommand{\mcU}{\mathcal{U}}	\newcommand{\mcV}{\mathcal{V}}
\newcommand{\mcW}{\mathcal{W}}	\newcommand{\mcX}{\mathcal{X}}
\newcommand{\mcY}{\mathcal{Y}}	\newcommand{\mcZ}{\mathcal{Z}}


%---------------------------------------
% Bold Math Fonts :-
%---------------------------------------

%Captital Letters
\newcommand{\bmA}{\boldsymbol{A}}	\newcommand{\bmB}{\boldsymbol{B}}
\newcommand{\bmC}{\boldsymbol{C}}	\newcommand{\bmD}{\boldsymbol{D}}
\newcommand{\bmE}{\boldsymbol{E}}	\newcommand{\bmF}{\boldsymbol{F}}
\newcommand{\bmG}{\boldsymbol{G}}	\newcommand{\bmH}{\boldsymbol{H}}
\newcommand{\bmI}{\boldsymbol{I}}	\newcommand{\bmJ}{\boldsymbol{J}}
\newcommand{\bmK}{\boldsymbol{K}}	\newcommand{\bmL}{\boldsymbol{L}}
\newcommand{\bmM}{\boldsymbol{M}}	\newcommand{\bmN}{\boldsymbol{N}}
\newcommand{\bmO}{\boldsymbol{O}}	\newcommand{\bmP}{\boldsymbol{P}}
\newcommand{\bmQ}{\boldsymbol{Q}}	\newcommand{\bmR}{\boldsymbol{R}}
\newcommand{\bmS}{\boldsymbol{S}}	\newcommand{\bmT}{\boldsymbol{T}}
\newcommand{\bmU}{\boldsymbol{U}}	\newcommand{\bmV}{\boldsymbol{V}}
\newcommand{\bmW}{\boldsymbol{W}}	\newcommand{\bmX}{\boldsymbol{X}}
\newcommand{\bmY}{\boldsymbol{Y}}	\newcommand{\bmZ}{\boldsymbol{Z}}
%Small Letters
\newcommand{\bma}{\boldsymbol{a}}	\newcommand{\bmb}{\boldsymbol{b}}
\newcommand{\bmc}{\boldsymbol{c}}	\newcommand{\bmd}{\boldsymbol{d}}
\newcommand{\bme}{\boldsymbol{e}}	\newcommand{\bmf}{\boldsymbol{f}}
\newcommand{\bmg}{\boldsymbol{g}}	\newcommand{\bmh}{\boldsymbol{h}}
\newcommand{\bmi}{\boldsymbol{i}}	\newcommand{\bmj}{\boldsymbol{j}}
\newcommand{\bmk}{\boldsymbol{k}}	\newcommand{\bml}{\boldsymbol{l}}
\newcommand{\bmm}{\boldsymbol{m}}	\newcommand{\bmn}{\boldsymbol{n}}
\newcommand{\bmo}{\boldsymbol{o}}	\newcommand{\bmp}{\boldsymbol{p}}
\newcommand{\bmq}{\boldsymbol{q}}	\newcommand{\bmr}{\boldsymbol{r}}
\newcommand{\bms}{\boldsymbol{s}}	\newcommand{\bmt}{\boldsymbol{t}}
\newcommand{\bmu}{\boldsymbol{u}}	\newcommand{\bmv}{\boldsymbol{v}}
\newcommand{\bmw}{\boldsymbol{w}}	\newcommand{\bmx}{\boldsymbol{x}}
\newcommand{\bmy}{\boldsymbol{y}}	\newcommand{\bmz}{\boldsymbol{z}}

%---------------------------------------
% Scr Math Fonts :-
%---------------------------------------

\newcommand{\sA}{{\mathscr{A}}}   \newcommand{\sB}{{\mathscr{B}}}
\newcommand{\sC}{{\mathscr{C}}}   \newcommand{\sD}{{\mathscr{D}}}
\newcommand{\sE}{{\mathscr{E}}}   \newcommand{\sF}{{\mathscr{F}}}
\newcommand{\sG}{{\mathscr{G}}}   \newcommand{\sH}{{\mathscr{H}}}
\newcommand{\sI}{{\mathscr{I}}}   \newcommand{\sJ}{{\mathscr{J}}}
\newcommand{\sK}{{\mathscr{K}}}   \newcommand{\sL}{{\mathscr{L}}}
\newcommand{\sM}{{\mathscr{M}}}   \newcommand{\sN}{{\mathscr{N}}}
\newcommand{\sO}{{\mathscr{O}}}   \newcommand{\sP}{{\mathscr{P}}}
\newcommand{\sQ}{{\mathscr{Q}}}   \newcommand{\sR}{{\mathscr{R}}}
\newcommand{\sS}{{\mathscr{S}}}   \newcommand{\sT}{{\mathscr{T}}}
\newcommand{\sU}{{\mathscr{U}}}   \newcommand{\sV}{{\mathscr{V}}}
\newcommand{\sW}{{\mathscr{W}}}   \newcommand{\sX}{{\mathscr{X}}}
\newcommand{\sY}{{\mathscr{Y}}}   \newcommand{\sZ}{{\mathscr{Z}}}


%---------------------------------------
% Math Fraktur Font
%---------------------------------------

%Captital Letters
\newcommand{\mfA}{\mathfrak{A}}	\newcommand{\mfB}{\mathfrak{B}}
\newcommand{\mfC}{\mathfrak{C}}	\newcommand{\mfD}{\mathfrak{D}}
\newcommand{\mfE}{\mathfrak{E}}	\newcommand{\mfF}{\mathfrak{F}}
\newcommand{\mfG}{\mathfrak{G}}	\newcommand{\mfH}{\mathfrak{H}}
\newcommand{\mfI}{\mathfrak{I}}	\newcommand{\mfJ}{\mathfrak{J}}
\newcommand{\mfK}{\mathfrak{K}}	\newcommand{\mfL}{\mathfrak{L}}
\newcommand{\mfM}{\mathfrak{M}}	\newcommand{\mfN}{\mathfrak{N}}
\newcommand{\mfO}{\mathfrak{O}}	\newcommand{\mfP}{\mathfrak{P}}
\newcommand{\mfQ}{\mathfrak{Q}}	\newcommand{\mfR}{\mathfrak{R}}
\newcommand{\mfS}{\mathfrak{S}}	\newcommand{\mfT}{\mathfrak{T}}
\newcommand{\mfU}{\mathfrak{U}}	\newcommand{\mfV}{\mathfrak{V}}
\newcommand{\mfW}{\mathfrak{W}}	\newcommand{\mfX}{\mathfrak{X}}
\newcommand{\mfY}{\mathfrak{Y}}	\newcommand{\mfZ}{\mathfrak{Z}}
%Small Letters
\newcommand{\mfa}{\mathfrak{a}}	\newcommand{\mfb}{\mathfrak{b}}
\newcommand{\mfc}{\mathfrak{c}}	\newcommand{\mfd}{\mathfrak{d}}
\newcommand{\mfe}{\mathfrak{e}}	\newcommand{\mff}{\mathfrak{f}}
\newcommand{\mfg}{\mathfrak{g}}	\newcommand{\mfh}{\mathfrak{h}}
\newcommand{\mfi}{\mathfrak{i}}	\newcommand{\mfj}{\mathfrak{j}}
\newcommand{\mfk}{\mathfrak{k}}	\newcommand{\mfl}{\mathfrak{l}}
\newcommand{\mfm}{\mathfrak{m}}	\newcommand{\mfn}{\mathfrak{n}}
\newcommand{\mfo}{\mathfrak{o}}	\newcommand{\mfp}{\mathfrak{p}}
\newcommand{\mfq}{\mathfrak{q}}	\newcommand{\mfr}{\mathfrak{r}}
\newcommand{\mfs}{\mathfrak{s}}	\newcommand{\mft}{\mathfrak{t}}
\newcommand{\mfu}{\mathfrak{u}}	\newcommand{\mfv}{\mathfrak{v}}
\newcommand{\mfw}{\mathfrak{w}}	\newcommand{\mfx}{\mathfrak{x}}
\newcommand{\mfy}{\mathfrak{y}}	\newcommand{\mfz}{\mathfrak{z}}

% lstlistingsEnvs.tex

\usepackage{minted}


\lstset{
  basicstyle=\ttfamily, % Set
  columns=fullflexible,
  keepspaces=true,
  language=Python % You can specify the language if you want syntax highlighting
}

%%%%%%%%%%%%%%%%%%%%%%%%%%%%%%%%%%%%%%%%%%%%%%%%%%%%%%%%%%%%%%%%%%%%%%%%%%%%%%%%%%%%%%%%%%%%%%%%%
%                                 Custom lstlisting Environments
%%%%%%%%%%%%%%%%%%%%%%%%%%%%%%%%%%%%%%%%%%%%%%%%%%%%%%%%%%%%%%%%%%%%%%%%%%%%%%%%%%%%%%%%%%%%%%%%%
% Gruvbox style for Python
\definecolor{Pgruvbox-bg}{HTML}{282828}
\definecolor{Pgruvbox-fg}{HTML}{ebdbb2}
\definecolor{Pgruvbox-red}{HTML}{fb4934}
\definecolor{Pgruvbox-green}{HTML}{b8bb26}
\definecolor{Pgruvbox-yellow}{HTML}{fabd2f}
\definecolor{Pgruvbox-blue}{HTML}{83a598}
\definecolor{Pgruvbox-purple}{HTML}{d3869b}
\definecolor{Pgruvbox-aqua}{HTML}{8ec07c}
\definecolor{BBBlack}{rgb}{0.05, 0.06, 0.09}



% JAVA LSTLISTING STYLE IN Gruvbox Colorscheme
\definecolor{gruvbox-bg}{rgb}{0.282, 0.247, 0.204}
\definecolor{gruvbox-fg1}{rgb}{0.949, 0.898, 0.776}
\definecolor{gruvbox-fg2}{rgb}{0.871, 0.804, 0.671}
\definecolor{gruvbox-red}{rgb}{0.788, 0.255, 0.259}
\definecolor{gruvbox-green}{rgb}{0.518, 0.604, 0.239}
\definecolor{gruvbox-yellow}{rgb}{0.914, 0.808, 0.427}
\definecolor{gruvbox-blue}{rgb}{0.353, 0.510, 0.784}
\definecolor{gruvbox-purple}{rgb}{0.576, 0.412, 0.659}
\definecolor{gruvbox-aqua}{rgb}{0.459, 0.631, 0.737}
\definecolor{gruvbox-gray}{rgb}{0.518, 0.494, 0.471}

\definecolor{lst-bg}{RGB}{45, 45, 45}
\definecolor{lst-fg}{RGB}{220, 220, 204}
\definecolor{lst-keyword}{RGB}{215, 186, 125}
\definecolor{lst-comment}{RGB}{117, 113, 94}
\definecolor{lst-string}{RGB}{163, 190, 140}
\definecolor{lst-number}{RGB}{181, 206, 168}
\definecolor{lst-type}{RGB}{218, 142, 130}

\lstdefinestyle{PythonGruvbox}{
    language=Python,
    identifierstyle=\color{lst-fg},
    basicstyle=\ttfamily\color{Pgruvbox-fg},
    keywordstyle=\color{Pgruvbox-yellow},
    keywordstyle=[2]\color{Pgruvbox-blue},
    stringstyle=\color{Pgruvbox-green},
    commentstyle=\color{Pgruvbox-aqua},
    backgroundcolor=\color{BBBlack},
    rulecolor=\color{BBBlack},
    showstringspaces=false,
    keepspaces=true,
    captionpos=b,
    breaklines=true,
    tabsize=4,
    showspaces=false,
    numbers=left,
    numbersep=5pt,
    numberstyle=\tiny\color{gray},
    showtabs=false,
    columns=fullflexible,
    morekeywords={True,False,None},
    morekeywords=[2]{and,as,assert,break,class,continue,def,del,elif,else,except,exec,
    finally,for,from,global,if,import,in,is,lambda,nonlocal,not,or,pass,print,raise,
    return,try,while,with,yield},
    morecomment=[s]{"""}{"""},
    morecomment=[s]{'''}{'''},
    morecomment=[l]{\#},
    morestring=[b]",
    morestring=[b]',
    literate=
    {0}{{\textcolor{Pgruvbox-purple}{0}}}{1}
    {1}{{\textcolor{Pgruvbox-purple}{1}}}{1}
    {2}{{\textcolor{Pgruvbox-purple}{2}}}{1}
    {3}{{\textcolor{Pgruvbox-purple}{3}}}{1}
    {4}{{\textcolor{Pgruvbox-purple}{4}}}{1}
    {5}{{\textcolor{Pgruvbox-purple}{5}}}{1}
    {6}{{\textcolor{Pgruvbox-purple}{6}}}{1}
    {7}{{\textcolor{Pgruvbox-purple}{7}}}{1}
    {8}{{\textcolor{Pgruvbox-purple}{8}}}{1}
    {9}{{\textcolor{Pgruvbox-purple}{9}}}{1}
}

% Gruvbox style for Java
\definecolor{gruvbox-bg}{rgb}{0.282, 0.247, 0.204}
\definecolor{gruvbox-fg1}{rgb}{0.949, 0.898, 0.776}
\definecolor{gruvbox-fg2}{rgb}{0.871, 0.804, 0.671}
\definecolor{gruvbox-red}{rgb}{0.788, 0.255, 0.259}
\definecolor{gruvbox-green}{rgb}{0.518, 0.604, 0.239}
\definecolor{gruvbox-yellow}{rgb}{0.914, 0.808, 0.427}
\definecolor{gruvbox-blue}{rgb}{0.353, 0.510, 0.784}
\definecolor{gruvbox-purple}{rgb}{0.576, 0.412, 0.659}
\definecolor{gruvbox-aqua}{rgb}{0.459, 0.631, 0.737}
\definecolor{gruvbox-gray}{rgb}{0.518, 0.494, 0.471}

\lstdefinestyle{JavaGruvbox}{
    language=Java,
    basicstyle=\ttfamily\color{Pgruvbox-fg},
    keywordstyle=\color{Pgruvbox-yellow},
    keywordstyle=[2]\color{lst-type},
    commentstyle=\itshape\color{lst-comment},
    stringstyle=\color{lst-string},
    numberstyle=\color{lst-number},
    backgroundcolor=\color{BBBlack},
    rulecolor=\color{gruvbox-aqua},
    showstringspaces=false,
    keepspaces=true,
    captionpos=b,
    breaklines=true,
    tabsize=4,
    showspaces=false,
    showtabs=false,
    columns=fullflexible,
    morekeywords={var},
    morekeywords=[2]{boolean, byte, char, double, float, int, long, short, void},
    morecomment=[s]{/}{/},
    morecomment=[l]{//},
    morestring=[b]",
    morestring=[b]',
    numbers=left,
    numbersep=5pt,
    numberstyle=\tiny\color{gray},
}

% Dracula style for Java
\definecolor{draculawhite-background}{RGB}{237, 239, 252}
\definecolor{draculawhite-comment}{RGB}{98, 114, 164}
\definecolor{draculawhite-keyword}{RGB}{189, 147, 249}
\definecolor{draculawhite-string}{RGB}{152, 195, 121}
\definecolor{draculawhite-number}{RGB}{249, 189, 89}
\definecolor{draculawhite-operator}{RGB}{248, 248, 242}

\lstdefinestyle{JavaDraculaWhite}{
    language=Java,
    backgroundcolor=\color{draculawhite-background},
    commentstyle=\itshape\color{draculawhite-comment},
    keywordstyle=\color{draculawhite-keyword},
    stringstyle=\color{draculawhite-string},
    basicstyle=\ttfamily\footnotesize\color{black},
    identifierstyle=\color{black},
    keywordstyle=\color{draculawhite-keyword}\bfseries,
    morecomment=[s][\color{draculawhite-comment}]{/**}{*/},
    showstringspaces=false,
    showspaces=false,
    breaklines=true,
    %frame=single,
    rulecolor=\color{draculawhite-operator},
    tabsize=2,  
    numbers=left,
    numbersep=4pt,
    numberstyle=\ttfamily\tiny\color{gray}
}

% Dracula style for Python
\definecolor{draculawhite-bg}{HTML}{FAFAFA}
\definecolor{draculawhite-fg}{HTML}{282A36}
\definecolor{pdraculawhite-keyword}{HTML}{BD93F9}
\definecolor{pdraculawhite-comment}{HTML}{6272A4}
\definecolor{draculawhite-number}{HTML}{FF79C6}

\lstdefinestyle{PythonDraculaWhite}{
    language=Python,
    basicstyle=\ttfamily\small\color{draculawhite-fg},
    backgroundcolor=\color{draculawhite-background},
    keywordstyle=\color{orange}\bfseries,
    stringstyle=\color{draculawhite-string},
    commentstyle=\color{pdraculawhite-comment}\itshape,
    numberstyle=\color{draculawhite-number},
    showstringspaces=false,
    showspaces=false,
    breaklines=true,
    frame=single,
    rulecolor=\color{draculawhite-operator}, 
    tabsize=4,
    morekeywords={as,with,1,2,3,4, 5,6,7,8,9,True,False},
    numbers=left,
    numbersep=5pt,
    numberstyle=\small\bfseries\ttfamily\color{htmlcomment},
}

% Dracula Dark style for HTML
\definecolor{htmltag}{HTML}{ff79c6}
\definecolor{htmlattr}{HTML}{f1fa8c}
\definecolor{htmlvalue}{HTML}{bd93f9}
\definecolor{htmlcomment}{HTML}{6272a4}
\definecolor{htmltext}{HTML}{401E31}
\definecolor{htmlbackground}{HTML}{282a36}
\definecolor{comphtmlbackground}{HTML}{8093FF}

\lstdefinestyle{HTMLDraculaDark}{
    basicstyle=\normalsize\bfseries\ttfamily\color{htmltext},
    commentstyle=\itshape\color{htmlcomment},
    keywordstyle=\bfseries\color{htmltag},
    stringstyle=\color{htmlvalue},
    emph={DOCTYPE,html,head,body,div,span,a,script},
    emphstyle={\color{htmltag}\bfseries},
    sensitive=true,
    showstringspaces=false,
    backgroundcolor=\color{white},
    inputencoding=utf8,
    extendedchars=true,
    language=HTML,
    tabsize=4,
    breaklines=true,
    breakatwhitespace=true,
    numbers=left,
    numbersep=10pt,
    numberstyle=\small\bfseries\ttfamily\color{htmlcomment},
    escapeinside={<@}{@>},
    rulecolor=\color{htmlbackground},
    xleftmargin=10pt,
    frame=none, 
    breaklines=true,
    postbreak=\mbox{\textcolor{gray}{$\hookrightarrow$}\space},
    showlines=false,
    moredelim=[s][\itshape\color{htmlcomment}]{<!--}{-->},
    morekeywords={id,class,type,name,value,placeholder,checked,src,href,alt},
    literate={é}{{\'e}}1 {è}{{\`e}}1 {ê}{{\^e}}1 {ë}{{\"e}}1 {à}{{\`a}}1 {ù}{{\`u}}1 {û}{{\^u}}1 {ç}{{\c{c}}}1 {â}{{\^a}}1 {î}{{\^i}}1 {ï}{{\"i}}1
}


\lstdefinestyle{Haskell}{
  frame=none,
  xleftmargin=2pt,
  stepnumber=1,
  numbers=left,
  numbersep=5pt,
  numberstyle=\ttfamily\tiny\color[gray]{0.3},
  belowcaptionskip=\bigskipamount,
  captionpos=b,
  escapeinside={*'}{'*},
  language=haskell,
  tabsize=2,
  emphstyle={\bf},
  %commentstyle=\it,
  stringstyle=\mdseries\ttfamily,
  showspaces=false,
  keywordstyle=\bfseries\ttfamily,
  columns=flexible,
  basicstyle=\small\ttfamily,
  showstringspaces=false,
  morecomment=[l]\%,
}



\lstdefinestyle{CSSDraculaLight}{
    basicstyle=\bfseries\scriptsize\ttfamily\color{htmltext},
    commentstyle=\color{htmlcomment},
    keywordstyle=\bfseries\color{htmlvalue},
    stringstyle=\color{htmlvalue},
    emph={DOCTYPE,html,head,body,div,span,a,script},
    emphstyle={\color{htmltag}\bfseries},
    sensitive=true,
    showstringspaces=false,
    backgroundcolor=\color{white},
    inputencoding=utf8,
    extendedchars=true, % Support extended characters
    frame=none, 
    %frame=tb,
    tabsize=4,
    breaklines=true,
    breakatwhitespace=true,
    numbers=left,
    numbersep=10pt,
    numberstyle=\small\bfseries\ttfamily\color{htmlcomment},
    escapeinside={<@}{@>},
    rulecolor=\color{htmlbackground},
    xleftmargin=20pt,
    % Add a vertical line for opening and closing tags
    %frame=lines,
    framesep=2pt,
    framexleftmargin=4pt,
    % Add a vertical line for closing tags that go through multiple lines
    breaklines=true,
    postbreak=\mbox{\textcolor{gray}{$\hookrightarrow$}\space},
    showlines=true,
    % Add a rule to apply commentstyle to keywords inside comments
    moredelim=[s][\color{htmlcomment}]{/*}{*/},
    literate={é}{{\'e}}1
             {è}{{\`e}}1
             {ê}{{\^e}}1
             {ë}{{\"e}}1
             {à}{{\`a}}1
             {ù}{{\`u}}1
             {û}{{\^u}}1
             {ç}{{\c{c}}}1
             {â}{{\^a}}1
             {î}{{\^i}}1
             {ï}{{\"i}}1,
    morekeywords={color, background, background-color, font-size, font-weight, border, border-radius, padding, margin, display, position, top, right, bottom, left, flex, grid, width, height, min-width, max-width, min-height, max-height, transition, transform, animation, keyframes, content, z-index,id,class,type,name,value,placeholder,checked,src,href,alt},
    morestring=[s][\color{htmltag}]{:}{;},
}












\title{\huge{MATH1400}\\\Huge{Calcul à plusieurs variables}\\\vspace{2em}Travail Pratique 1 }
\author{\huge{Franz Girardin}}
\date{\today}


   

\begin{document}

\maketitle
\newpage% or \cleardoublepage
% \pdfbookmark[<level>]{<title>}{<dest>}
\pdfbookmark[section]{\contentsname}{toc}
\tableofcontents
\pagebreak


\titleformat*{\section}{%
    \normalsize\bfseries%
}

\titleformat{\section}[block]{\normalsize\bfseries}{}{0pt}{}


    \chapter*{Exercices sur les suite numériques}
  \section{Définition de convergence}
  \begin{Exercice}{(Stewart 1.1.2)}{}
      Qu'est-ce qu'une \textit{suite convergente} ? Donnez \textbf{deux exemples}. 
      Qu'est-ce qu'une \textit{suite divergente} ? Donnez \textbf{deux} exemples   
  \end{Exercice}

  Est \textcolor{myb}{\textbf{convergente}} 
  toute \textbf{suite} $\{a_n\}$ dont les termes $a_n$ 
  s'approchent autant 
  que l'on veut d'une valeur $L$ lorsque l'entier $n$ devient arbitrairement grand : 

  \begin{align*}
      \left(\lim\limits_{n\to\infty}a_n = L\right) \implies a_n \; \textcolor{myb}{\textbf{conv}}.    
  \end{align*}

  Plus formellement, soit une valeur positive arbitrairement petite $\varepsilon$, 
  il existe toujours un entier $N(\varepsilon)$ qui représente un rang 
  à partir duquel s'il y a un entier naturel $n > N(\varepsilon)$, \textbf{alors}
  l'image de cet entier naturel, $a_n$ sera aussi proche que l'on veut d'une 
  \textbf{valeur} $L$ représentant \textit{le point de convergence de la suite} :

  \begin{align*}
    \forall \varepsilon > 0 : \exists N(\varepsilon) > 0 : 
    n > N \implies |a_n - L| < \varepsilon
  \end{align*}

    \textbf{Exemple de suites convergentes} : 
    $\{a_n \} = \dfrac{1}{n}, \quad% 
    \{b_n\} = \dfrac{1}{n^2}$   


    \vspace{2em}
    Est \textcolor{myr}{\textbf{divergente}} toute suite dont les termes $a_n$ ne 
    convergent vers acune valeur particulière; ils s'approchent plutôt 
    de $\pm \infty$ :

      \begin{align*}
          \left(\lim\limits_{n\to\infty}a_n = \pm \infty \right) \implies 
          a_n \; \textcolor{myr}{\textbf{div}}.    
      \end{align*}

    Plus formellement, soit un nombre positif arbitrairement grand $M > 0$, 
    on pourra toujours trouver une valeur $N(M) \in \mathbb{N}$ à partir duquel tous les 
    entiers $n > M$ auront une image $a_n$ \textbf{plus grande} 
    que nombre arbitrairement 
    grand. Cela signifie que les termes de la suites ne s'approchent d'aucune 
    valeur. 

    \begin{align*}
    \forall \; M > 0 : \exists N(M) \in \mathbb{N} > 0 : n > N 
    \implies a_n > M (\textcolor{myr}{\textbf{div}}. \; \textbf{+}\infty ) 
    \end{align*}

    
    \section{Identification des termes}
    \begin{Exercice}{(Stewart 1.1.8)}{}
        Donnez les \textbf{cinq premiers termes} de la suite 
        $a_{n}=\frac{\left(-1\right)^{n}n}{n!+1}$ 
    \end{Exercice}


    \begin{align*}
        a_1 = -\dfrac{1}{2}, \; a_2 = \dfrac{2}{3}, \; 
        a_3 = -\dfrac{3}{7}, \; 
        a_4 = \dfrac{4}{25} 
        a_5 = -\dfrac{5}{121} 
    \end{align*}                


    \begin{Exercice}{(Stewart 1.1.11)}{}
        Donnez les \textbf{cinq premiers termes} de la suite 
        $a_n = 2, \; a_{n+1} = \dfrac{a_n}{1 + a_n}$
    \end{Exercice}              


    \begin{align*}
        a_1 = \textcolor{myb}{2}, \; a_2 = \textcolor{myb}{\dfrac{2}{3}}, \; 
        a_3 = \dfrac{2}{3} \times \dfrac{3}{5} = \textcolor{myb}{\dfrac{2}{5}}, \; 
        a_4 = \dfrac{2}{5} \times \dfrac{5}{7}  = \textcolor{myb}{\dfrac{2}{7}}, \;
        a_5 = \dfrac{2}{7} \times \dfrac{7}{9} = \textcolor{myb}{\dfrac{2}{9}}
    \end{align*}    


    \section{Trouver la règle}
    \begin{Exercice}{(Stewart 1.1.15)}{}
       Trouver la formule du \textbf{terme général} $a_n$ de la suite    
       $\left\{-3, 2,-\frac43, \frac89, -\frac{16}{27},\ldots\right\}$ 
    \end{Exercice}              


    \begin{align*}
        a_n = \dfrac{(-1)^n 2^{n-1}}{3^{n - 2}}
    \end{align*}            


    \begin{Exercice}{(Stewart 1.1.18)}{}
    Trouver la formule du \textbf{terme général} $a_n$ de la suite    
    $\left\{1,0,-1,0,1,0,-1,0,\cdots \right\}$
    \end{Exercice}

    \begin{align*}
        a_n = \sin\left(\dfrac{n\pi}{2}\right)
    \end{align*}        

    \section{Estimer un somme}


    \begin{Exercice}{(Stewart 1.1.19 - 1.1.22)}{}
        Calculez, \textbf{à la quatrième décimale}, 
        les dix premiers termes de la suite et 
        utilisez-les pour tracer le graphique de la suite à la main. 
        La suite semble-t-elle avoir une limite ? 
        Si oui, \textbf{calculez cette limite}. 
        Si non, expliquez pourquoi.
    \end{Exercice}

    \vspace{1em}%
    \noindent\textbf{1.1.19}  $a_n = \dfrac{3n}{1 + 6n}$ 


    \begin{align*}
        a_1 = 0.4286, \; a_2 = 0.4615, \; 
        a_3 = 0.4736, \; a_4 = 0.4800, \; 
        a_5 = 0.4838, \;
        \\
        a_6 = 0.4864, \; a_7 = 0.4883, \; 
        a_8 = 0.4998, \; a_9 =0.4909, \; 
        a_{10} = 0.4918
    \end{align*}    


    \begin{center}
    \begin{tikzpicture}
        \begin{axis}[
            title={Graphe de la suite $a_n = \dfrac{3n}{1 + 6n}$ 
            et de la fonction $f(x) =\dfrac{3x}{1 + 6x}$ },
            xlabel={$n$},
            ylabel={$f(x)$},
            grid=major,
            xmin=1, xmax=10,
            ymin=0.42, ymax=0.5,
            xtick={1,2,...,10},
            ytick={0.42, 0.43, 0.44, 0.45, 0.46, 0.47, 0.48, 0.49, 0.50},
            legend pos=south east
        ]
        
        % Plot des valeurs des dix premiers termes donnés
        \addplot coordinates {
            (1, 0.4286)
            (2, 0.4615)
            (3, 0.4736)
            (4, 0.4800)
            (5, 0.4838)
            (6, 0.4864)
            (7, 0.4883)
            (8, 0.4898)
            (9, 0.4909)
            (10, 0.4918)
        };
        \addlegendentry{Valeurs de $a_n$}

        % Tracé de la fonction a_n = 3n / (1 + 6n)
        \addplot[domain=1:10, samples=100, thick, red] 
        {3*x / (1 + 6*x)};
        \addlegendentry{Fonction $f(x) = \frac{3x}{1 + 6x}$}

        \end{axis}
    \end{tikzpicture}

    \end{center}


    \section{Déterminer la convergence}

    \begin{Exercice}{(Stewart 1.1.23 - 1.1.56)}{}
    Déterminez si la suite converge ou diverge. Si elle 
    converge, \textbf{trouvez sa limite}.       
    \end{Exercice}      




    \noindent\textbf{1.1.23}  $a_n = \dfrac{3 + 5n^2}{n + n^2}$  
    \vspace{1em}
    \noindent 
    $\lim\limits_{n\to+\infty }a_n \implies \textcolor{red}{\dfrac{\infty }{\infty }}$. 

    \begin{align*}
        \dfrac{3 + 5n^2}{n + n^2} = 
        \dfrac{n^2(\frac{1}{3n^2} + 5)}{n^2(1+ \frac{1}{n})} = a_n \quad \text{(Simplifié)}
    \end{align*}

    \begin{align*}
        \lim\limits_{n \to+\infty }a_n  = 
        \lim\limits_{n \to+\infty }\dfrac{5  + \frac{1}{n}
        \cancelto{0}{\frac{1}{n}}}{1 + \cancelto{0}{\frac{1}{n}}}  =
        \lim\limits_{n \to+\infty } \frac{5}{1} = \dfrac{5}{1} 
    \end{align*}



    \noindent\textbf{1.1.42}  $a_n = \dfrac{\cos^{2}n}{n}$  
    \vspace{1em}

    Nous savons que la fonction $a_n$ est bornée entre les valeurs
    $\cos^{2}n \in [0, 1]$. Ainsi, pour tout $n \in \mathbb{N}$ 
    nous avons les équivalences suivantes : 

    \begin{align*}G
        0 \leq cos^{2}n  \leq 1 \\ 
        \dfrac{0}{n} \leq \dfrac{cos^{2} n}{n}  \leq \dfrac{1}{n}\\ 
        \lim\limits_{n \to+\infty }\cancelto{0}{\dfrac{0}{n}} \leq  
        \lim\limits_{n \to+\infty }\dfrac{cos^{2} n}{n} \leq 
        \lim\limits_{n \to+\infty }\cancelto{0}{\dfrac{1}{n}}
    \end{align*}

    Ainsi nous savons que la limite est bornée 
    \textbf{inférieurement et supérieurement} 
    par zéro, lorsque $n \longrightarrow \infty$. Ainsi, nous pouvons conclure 
    que la limite est égale à zéro et que la suite $a_n$ converge vers 
    $L = 0$. 

        
    \vspace{1em}%
    \noindent\textbf{1.1.44}  $a_n = \sqrt[n]{2^{1 + 3n}}$

    \begin{align*}
        a_n 
        = \sqrt[n]{2^{1 + 3n}} 
        = \left(2 \cdot 2^{3n} \right)^{\frac{1}{n}}
        = 2^{\frac{1}{n}} \cdot 2^{\frac{3n}{n}} \quad \text{(Développé)} \\ 
        \lim\limits_{n \to+\infty }a_n 
        = 
        \lim\limits_{n \to+\infty }  
        = 
        2^{\cancelto{0}{\frac{1}{n}}} \cdot 2^{3} = 8
    \end{align*}        

    \vspace{1em}
    \noindent\textbf{1.1.46}  $a_n = 2^{-n}\cos n\pi$ 

    Nous savons que la suite $\{a_n\}$ est bornée par les valeurs 
    $-1$ et $1$ :
    $\forall n \in \mathbb{N}, -1 \leq a_n \leq 1$. En utilisant l'identité 
    $\cos\pi n = (-1)^n$, nous avons : 
    \begin{align*}
        a_n = 2^{-1}\cos\pi n  =  b_n = 2^{-1}(-1)^n 
    \end{align*}
    Nous pouvons ainsi évaluer la limite lorsque $n \longrightarrow 0$ :

    \begin{align*}
        \lim\limits_{n \to+\infty }a_n 
        = 
        \lim\limits_{n \to+\infty }b_n  
        =
        \lim\limits_{n \to+\infty }\cancelto{0}{\dfrac{1}{2^n}} \cdot (-1)^n  
        = \lim\limits_{n \to+\infty } 0 \cdot (-1)^n = 0  
    \end{align*}

    
    \noindent\textbf{1.1.52}  $a_n = \arctan(\ln n)$ 

    Nous savons que la fonction $\ln x$ tend vers $+ \infty$ lorsque 
    $x \longrightarrow + \infty$. Et par le théorème d'association d'une 
    fonction à une suite, nous savons que la suite analogue $b_n = 
    \ln n$ tend également vers $+ \infty$. Nous avons alors :


    \begin{align*}
        \lim\limits_{n \to+\infty } a_n = 
        \lim\limits_{n \to+\infty } \arctan\cancelto{\infty}{(\ln n)} =
        \lim\limits_{M \to+\infty } \arctan(M) = \frac{\pi}{2} 
    \end{align*}        

    \vspace{1em}%

    \noindent\textbf{1.1.54}  
    $a_n = \left\{\frac11,\frac13,\frac12,\frac14,\frac13,\frac15,\frac14,\frac16,...\right\}$ 
    
    En observant la suite, on constate que le dénominateur pour 
    les termes $n$ \textbf{impairs} est équivalent à $\frac{n + 1}{2}$ :

    \begin{align*}      
        \dfrac{1}{\left( \frac{\textcolor{myb}{1} + 1}{2}  \right)}
        = \textcolor{myr}{\dfrac{1}{1}}, \quad  
        \dfrac{1}{\left( \frac{\textcolor{myb}{3} + 1}{2}  \right)}
        = \textcolor{myr}{\dfrac{1}{2}}, \quad  
        \dfrac{1}{\left( \frac{\textcolor{myb}{5} + 1}{2}  \right)}
        = \textcolor{myr}{\dfrac{1}{3}}, \quad  
        \dfrac{1}{\left( \frac{\textcolor{myb}{7} + 1}{2}  \right)}
        = \textcolor{myr}{\dfrac{1}{4}}, \dots  
    \end{align*}        

    Et en observant la suite, on constate que le dénominateur pour 
    les termes $n$ \textbf{pairs}   est équivalent à $\frac{n}{2} + 2$ :


    \begin{align*}      
        \dfrac{1}{\left( \frac{\textcolor{myb}{2}}{2} +2  \right)}
        = \textcolor{myr}{\dfrac{1}{3}}, \quad  
        \dfrac{1}{\left( \frac{\textcolor{myb}{4}}{2} +2  \right)}
        = \textcolor{myr}{\dfrac{1}{4}}, \quad  
        \dfrac{1}{\left( \frac{\textcolor{myb}{6}}{2} +2  \right)}
        = \textcolor{myr}{\dfrac{1}{5}}, \quad  
        \dfrac{1}{\left( \frac{\textcolor{myb}{8}}{2} +2  \right)}
        = \textcolor{myr}{\dfrac{1}{6}}, \quad  
    \end{align*}

    On peut donc conclure que la suite obéit à la règle 
    $a_n = \dfrac{1}{\frac{n+1}{2}}$ pour les termes \textbf{impairs} et 
    $a_n = \dfrac{1}{\frac{n + 4}{2}}$ pour les
    termes \textbf{pairs}. En simplifiant les fractions, on obtient :


    \begin{align*}
        a_n = 
        \begin{cases}
            \dfrac{2}{n+1} & \quad \quad n \; \text{impairs} \\ 
            \\
            \dfrac{2}{n+4} & \quad \quad n \; \text{pairs} \\ 
        \end{cases}
    \end{align*}        




    \vspace{1em}


    \noindent \textbf{1.1.56}  $a_n = \dfrac{(-3)^n}{n!}$




    \begin{align*}
        \lim\limits_{n\to \infty} a_n = \lim\limits_{n \to+\infty }\dfrac{(-3)^n}{n!} 
        &= 
        \lim\limits_{n \to+\infty } (-1)^n  \dfrac{3^n}{n!} \\ 
        &=
        \lim\limits_{n\to +\infty} -\cos(n\pi)\dfrac{3^n}{n!} \\ 
        &= 
        -\lim\limits_{n\to +\infty} \cos(n\pi)\cancelto{0}{\dfrac{3^n}{n!}}\\ 
        &\implies a_n \xrightarrow[n\to \infty]{} 0
    \end{align*}


    La partie importante ici est le rapport entre \( (-3)^n \) et \( n! \). Bien que \( 3^n \)
    croisse exponentiellement, \( n! \) croît beaucoup plus rapidement que \( 3^n \), car 
    \( n! \) est un produit d'entiers successifs qui croît super-exponentiellement. Cela 
    signifie que pour des \( n \) suffisamment grands, le dénominateur \( n! \) va dominer 
    le numérateur \( 3^n \), ce qui entraînera la limite de \( a_n \) vers 0.


    \begin{Exercice}{(Stewart 1.1.64)}{}
        Déterminez si la suite définie par récurrence est convergente ou divergente : 
        \begin{align*}
            a_1 = 1, \; a_{n+1} = 4 - a_n \; \forall n \geq 1
        \end{align*}
    \end{Exercice}

    \noindent \textbf{1.1.64a}  Les premiers termes de la suite sont les suivantes :
    \begin{align*}
        a_1 = 1, \quad a_2 = 3, \quad a_3 = 1, \quad a_4 = 3, \quad a_5 = 1, \cdots
    \end{align*}

    On voit que la suite oscille entre les valeurs $1$ et $3$, en fonction du fait que $n$ est 
    \textbf{pair} ou \textbf{impair} :                  

    \begin{align*}
        a_1 = 1, \quad a_2 = 3, \quad a_3 = &1, \quad a_4 = 3, \quad a_5 = 1, \cdots \\ \\
                                                     &\Updownarrow \\\\
        a_n = 
        \begin{cases} 
            1 & \quad \quad n \; \text{impair} \\ \\
            3 & \quad \quad n \; \text{pair}
        \end{cases}
    \end{align*}
    

    Puisque la suite oscille entre deux valeurs, la limite $\lim\limits_{n \to+\infty }a_n$ est \textbf{indéfinie}                    et on peut conclure que la suite diverge.
    \vspace{1em}%

    \noindent \textbf{1.1.64b}

    Si la valeur de $a_1$ était $a_1 = 2$, on aurait les premiers termes suivants : 
    
    \begin{align*}
        a_1 = 2, \quad a_2 = 2, \quad a_3 = 2, \quad a_4 = 2, \quad \cdots \\ \\
    \end{align*}

    Ainsi, on constate qu'après le premier termes, la suite a une valeur constante $a_n = 2 \forall n > 1$. On 
    peut donc conclure que la suite converge vers l'entier naturel 2.


    \section{Convergence de \(a_{n+1} \)}

    \begin{Exercice}{(Stewart 1.1.70a)}{}
        Si $\{a_n \}$ converge, montrez que  
        \begin{align*}
            \lim\limits_{n \to+\infty }a_{n+1} = \lim\limits_{n \to+\infty } a_n  
        \end{align*}
    \end{Exercice}


    \textbf{Preuve directe}. La définition de convergence d'une suite suggère que $a_n$ 
    est \textbf{convergente} si pour toute valeur arbitrairement petite et positive, 
    $\varepsilon > 0$, il existe un entier $N(\varepsilon)$ 
    qui représente un seuil à partir duquel 
    pour toute valeur $n > N(\varepsilon)$, la distance entre $a_n$ et $L$ 
    est suffisamment petite ($|a_n - L| < \varepsilon)$. Après ce seuil $N$ 
    les images $a_n$ de chaque $n > N$ sont suffisamment proche d'une valeur limite 
    $L$. Autrement dit :

    \begin{align*}
        a_n \; \textbf{\textcolor{myb}{conv}.}  \implies 
        \forall \; \varepsilon > 0 : \exists N(\varepsilon) > 0 : n > N(\varepsilon) 
        \implies |a_n - L| < \varepsilon
    \end{align*} 


    \textbf{Supposons} que la suite $\{a_n\}$ converge vers $L$. Cela signifie que pour tout 
    $\varepsilon > 0$, il existe un entier $N(\varepsilon)$ tel que pour tout $n > N(\varepsilon)$, 
    on a $|a_n - L| < \varepsilon$. 

    Maintenant, considérons la suite \(\{a_{n+1}\}\). Lorsque $n > N(\varepsilon)$, 
    il est évident que $n+1 > N(\varepsilon)$ également. Ainsi, pour $n > N(\varepsilon)$, 
    on a aussi :
    \[
    |a_{n+1} - L| < \varepsilon.
    \]
    Cela montre que \(\lim\limits_{n \to +\infty} a_{n+1} = L\), puisque la distance 
    entre \(a_{n+1}\) et \(L\) devient arbitrairement petite pour des \(n\) suffisamment grands.

    Ainsi, nous avons montré que si \(\lim\limits_{n \to +\infty} a_n = L\), alors 
    \(\lim\limits_{n \to +\infty} a_{n+1} = L\).

    \textbf{Conclusion} : Puisque les deux suites \(\{a_n\}\) et \(\{a_{n+1}\}\) convergent 
    vers la même limite \(L\), nous avons :

    \[
    \boxed{\lim\limits_{n \to +\infty} a_{n+1} = \lim\limits_{n \to +\infty} a_n.}
\]

                

    \begin{Exercice}{(Stewart 1.1.70b)}{}
        Une suite $\{a_n\}$ est définie par $a_1 = 1$ et $a_{n+1} = \dfrac{1}{1+a_n}, \; \forall n\geq 1$. 
        En supposant que $\{a_n\}$ converge, trouvez sa limite.
    \end{Exercice}

    Les premiers termes de la suite sont :

    \begin{align*}
        a_1 = 1, \quad a_2 = \dfrac{1}{2}, \quad a_3 = \dfrac{2}{3}, \quad a_4 = \dfrac{3}{5}, \quad 
        a_5 = \dfrac58, \quad a_6 = \dfrac{8}{13}, \quad
        a_7 = \dfrac{13}{21}, \quad a_8 = \dfrac{21}{34}, \quad
        a_9 = \dfrac{34}{55}, \quad a_{10} = \dfrac{55}{89}
    \end{align*}

    Supponsons, comme suggère l'énoncé, que la suite $\{ a_n \}$ converge vers $L$ :
    \begin{align*}
        \lim\limits_{n \to+\infty } a_n = L  
    \end{align*}            

    Or, la relation de récurrence, $a_{n+1} = \dfrac{1}{1 + a_n}$. Et puisque 
    $a_n \rightarrow L$ lorsque $n \to \infty$, la récurrence engendre l'équivalence :

    \begin{align*}
        a_{n+1} = \dfrac{1}{1 + L}
    \end{align*}

    Nous avons montré que si $a_n$ converge vers $L$, alors $a_{n+1}$ converge également vers $L$. 
    Ona  donc : 

    \begin{align*}
        L = \dfrac{1}{1 + L} \\ 
        L(1 + L) = 1 \\ 
        L + L^2 = 1 \\ 
        L^2 + L - 1 = 0
    \end{align*}

    En appliquant la formule quadratique, on obtient :
    \begin{align*}
        L = \dfrac{-1 \pm \sqrt{1^2 -4(1)(-1)}}{2} = \dfrac{-1\pm \sqrt{5}}{2}
    \end{align*}
    Cela engendre les solutions $L_1 \approx 0.618$ et $L_2 \approx -1.628$. Puisque les valeurs 
    des termes de la suite sont positives (voir \(a_1\) à \(a_{10} \)), on peut \textbf{rejeter la solution négative}.                 
    La limite de la suite $\{ a_n \}$ est :
    
    \[% 
    \boxed{L = \dfrac{-1 + \sqrt{5}}{2}}
    \]%

    \section{Théorie sur les suites monotones}

    \begin{Exercice}{(Stewart 1.1.71)}{}
        Supposez que vous savez que est une suite décroissante
        et que tous ses termes sont compris entre les nombres 5 et 8.
        Expliquez pourquoi cette suite possède une limite. Que
        pouvez-vous dire à propos de la valeur de cette limite ?
    \end{Exercice}      

    Par la propriété des suites monotones, toute suite décroissante et bornée est également convergente. 
    Plus formellement, supposons que $\{a_n\}$ est décroissante et que toutes les valeurs de 
    $\{a_n\}$ sont comprises entre $5$ et $8$. Sachant que $\{a_n\}$ est décroissante, nous savons également que 
    $\forall \; n \geq 1, a_{n+1} \leq a_n$. Ainsi, chaque valeur $a_{n+1}$ est
    au moins  égale ou plus petite que son prédécesseur $a_n$. Ainsi, lorsque $n \to infty$, 
    valeurs de la suite s'approchent de la borne inférieure et la limite de $\{a_n \}$ 
    est une certaine valeur $L \geq 5$. 


    \section{Déterminer la convergence de suites monotones}

    \begin{Exercice}{(Stewart 1.1.72-1.1.78)}{}
        Déterminez si la suite est croissante, décroissante ou non monotone. Est-elle bornée ?
    \end{Exercice}

    \noindent \textbf{1.1.73} \; \( a_n = \dfrac{1}{2n + 3} \). La suite \( \{a_n \} \) est strictement décroissante; 
    il s'agit donc d'une suite monotome. On constate que $a_n \xrightarrow[n\to \infty]{} L = 0$ 
    Ainsi, on peut conclure que $\{ a_n\}$ est bornée inférieurement. 
    
    \vspace{2em}
    \noindent \textbf{1.1.75} \; \( a_n = n(-1)^n \). La suite \( \{a_n \} \) oscille entre des valeurs 
    positives et négatives à cause de l'exposant $(-1)^n$. Ainsi, lorsque $n\to \infty$, 
    $a_n$ ne s'approche d'aucune valeur particulière. Ainsi, on peut conclure que 
    $\{a_n \}$ est non bornée, divergente et non monotone. 

    \begin{Exercice}{(Stewart 1.1.82)}{}
        Montrer que la suite définie par 
        \begin{align*}
                a_n = 2, \quad  a_{n+1} = \dfrac{1}{3 - a_n} 
        \end{align*}
        satisfait à $0 < a_n \leq 2$ et qu'elle est décroissante. Désuisez que cette suite 
        converge et trouvez sa limite
    \end{Exercice}

Soit la suite définie par :
\[
a_1 = 2, \quad a_{n+1} = \frac{1}{3 - a_n}.
\]

\textbf{1. Montrons que la suite \((a_n)\) est décroissante.}

\vspace{1em}
\begin{center}
    Preuve par récurrence
\end{center}

\textbf{Description} : Nous voulons montrez $P(n)$, c'est-à-dire que 
$0 < a_{n+1} \leq a_n \leq 2, \;\; \forall n \in \mathbb{N}$. Autrement 
dit, nous voulons montrer que la suite est décroissante. 

\vspace{1em}


\textbf{Initialisation :} Pour \( n = 1 \), calculons \( a_n \) et \( a_{n+1} \) ; vérifions $P(1)$ :
\[
    a_n = a_1 = 2, \quad a_{n+1} = a_2 = \frac{1}{3 - a_1} = \frac{1}{3 - 2} = 1.
\]
Ainsi, on a  
\begin{align*}
    0 < \underbrace{a_{n+1}}_{= 1}  \leq \underbrace{a_n}_{= 2} \leq 2 
\end{align*}




et le \textbf{cas de base} est vérifié; $P(1)$ est vrai. Nous voulons maintenant montrer que si $P(n)$ est vrai, 
cela implique que $P(n+1)$ est aussi vrai:

\begin{align*}
    P(n) \implies P(n+1)
\end{align*}


\textbf{Hérédité :} Supposons que pour un certain \( n \geq 1 \), on a \( 0 < a_{n+1} \leq a_{n} \leq 2\).
Montrons que cela implique que \( a_{n+2} \leq a_{n+1} \), c'est-à-dire :

\begin{align*}
    &a_{n+2} \leq a_{n+1}  
    &\text{À vérifier} 
    \\ 
    &\dfrac{1}{3 - a_{n+1}} \leq a_{n+1} 
    &\text{Définition de } a_{n+2} 
    \\ 
    &0 < \dfrac{1}{3 - a_{n+1}} \leq a_{n+1} \leq 2  
    &\text{Inclusion des bornes}
\end{align*}    

Ainsi, nous constatons que si l'hypothèse d'hérédité  est vraie, le terme suivant 
$a_{n+2}$ sera toujours compris entre les bornes $0$ et $2$ tel que $0 < a_{n+2} 
\leq a_{n+1} \leq 2$. Et un terme quelconque $n+2$ sera donc toujours plus petit que 
son prédécesseur $n+1$. 

\textbf{2. Montrons que la suite converge.}


Puisque nous savons que la suite est monotone (décroissante) et bornée, 
par le \textcolor{myb}{\textbf{théorème des suites bornées}}, 
nous pouvons conclure qu'elle est également convergente. Pour trouver 
vers quelle valeur la suite converge, on peut utiliser les expressions qui 
définissent la suite. 

\begin{align*}
    \lim\limits_{n \to+\infty } a_{n+1} = \; &L = \dfrac{1}{3 - \cancelto{L}{a_n}}  \\
                                             &L = \dfrac{1}{3 - L} \\ 
    &
    \implies L(3 - L) = 1 \\ 
    &
    \implies 3L -L^2 = 1 \\ 
    & 
    \implies L^2 -3L + 1 = 0
\end{align*}            

Les deux solutions possibles pour cette équation sont : 
\begin{align*}
    L = \dfrac{3 \pm \sqrt{9 -4(1)(1)}}{2(1)} = \dfrac{3 \pm \sqrt{5}}{2}
\end{align*}


La solution $\dfrac{3+\sqrt{5}}{2}$ engendre une valeur $\alpha \approx 2.618 > 2$ qui est supérieure à 
notre borne supérieure $a_1 = 2$. La seule valeur possible est donc :

\begin{align*}
    \boxed{0 < L = \dfrac{3-\sqrt{5}}{2} \approx  1.882 \leq 2 }
\end{align*}
 

\textbf{Conclusion :} La suite \( (a_n) \) est décroissante, elle vérifie \( a_n \leq 1 \)
pour tout \( n \geq 2 \), et elle converge vers \( L = \dfrac{3 - \sqrt{5}}{2} \).


\begin{Exercice}{(Stewart 1.1.92a)}{}
    Montrez que si $\lim\limits_{n \to+\infty }a_{2n} = L$ et si  
    $\lim\limits_{n \to+\infty }a_{2n + 1} = L$, alors $\{a_n\}$ converge et
    $\lim\limits_{n \to+\infty }a_n = L$ 
\end{Exercice}

Soit $\varepsilon > 0$. Nous devons montrer qu'il existe un entier $N \in 
\mathbb{Z}^+$ tel que pour tout $n > N$, on ait $|a_n - L| < \varepsilon$.

\textbf{Hypothèses} :
\begin{enumerate}
    \item $\lim\limits_{n \to +\infty} a_{2n} = L$, c'est-à-dire qu'il existe 
    $N_1 \in \mathbb{Z}^+$ tel que pour tout $n > N_1$,
    \[
    |a_{2n} - L| < \varepsilon.
    \]
    \item $\lim\limits_{n \to +\infty} a_{2n+1} = L$, c'est-à-dire qu'il existe 
    $N_2 \in \mathbb{Z}^+$ tel que pour tout $n > N_2$,
    \[
    |a_{2n+1} - L| < \varepsilon.
    \]
\end{enumerate}

\textbf{Cas 1 : $n$ est pair} \\
Si $n$ est pair, alors $n = 2m$ pour un certain entier $m$. On utilise la limite 
des termes pairs :
\[
|a_n - L| = |a_{2m} - L| < \varepsilon, \quad \text{pour } m > N_1.
\]
Cela implique que pour $n = 2m$ et $n > 2N_1$, on a $|a_n - L| < \varepsilon$.

\textbf{Cas 2 : $n$ est impair} \\
Si $n$ est impair, alors $n = 2m + 1$ pour un certain entier $m$. On utilise la 
limite des termes impairs :
\[
|a_n - L| = |a_{2m+1} - L| < \varepsilon, \quad \text{pour } m > N_2.
\]
Cela implique que pour $n = 2m + 1$ et $n > 2N_2 + 1$, on a $|a_n - L| < 
\varepsilon$.

\textbf{Conclusion} \\
Il suffit de prendre $N = \max(2N_1, 2N_2 + 1)$. Ainsi, pour tout $n > N$, que 
$n$ soit pair ou impair, on a $|a_n - L| < \varepsilon$. Par conséquent, 
$\lim\limits_{n \to +\infty} a_n = L$.












\end{document}
