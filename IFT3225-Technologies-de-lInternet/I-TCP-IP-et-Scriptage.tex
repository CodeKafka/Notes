\documentclass[a4paper, 14pt]{report}
\usepackage{mathtools}
\usepackage{romannum} 




% Custom files; preamble and choice of font
%==================
% Custom Files 
%==================
\input{preamble.tex}
\usetikzlibrary{positioning}
\usepackage{xpatch,letltxmacro}
\LetLtxMacro{\cminted}{\minted}
\let\endcminted\endminted
\xpretocmd{\cminted}{\RecustomVerbatimEnvironment{Verbatim}{BVerbatim}{}}{}{}
\renewcommand{\figurename}{Listing}







\usemintedstyle{default}
\begin{document}
%==================
% Title page 
%==================
\frontmatter                    % Declares numbering style of chapters
\begin{titlepage}
    \centering
    \vspace*{1cm}

\scalebox{1.65}{
    \Huge\bfseries\color{gray}
    \sffamily Technologies
    }
    \\
    \vspace{0.2cm}%
    {\LARGE\sffamily\textcolor{gray}{de l'internet}}\par 
    \vspace{4cm}%
    {\LARGE\textcolor{gray}{\sffamily FRANZ GIRARDIN
        }
    }\par%
    {\large\sffamily Université de Montréal}
    \par%
    
    \vspace{1cm}
    
    {\large\sffamily CHARGÉE DE COURS}\par
    {\normalsize\sffamily Philippe Langlais}\par
    {\small\sffamily 
        Département d'Informatique et de Recherche Opérationnelle
    }\par%
    \vspace{1cm}
    
    {\large\textsc
        {\sffamily \textbf
            {Notes de cours}  
        }%
    }\par
    {\normalsize\sffamily 
            TPC/IP, sciptage
    }\par
    \vfill
    {\Large\sffamily Montréal 2025}\par
    {\small\sffamily\textit{\today}}

\end{titlepage}
\pagebreak


%==================
% Table of contents
%==================
% table of contents numered in romain numerals
\thispagestyle{plain}
\tableofcontents



\mainmatter                     % Declares numbering style of chapter
\thispagestyle{plain}           % Simple header after each new chapter

\chapter{Modèle TCP-IP}
    \begin{Définition}[TCP-IP]
        Architecture simplifiée 
        en \textbf{4 couches} décrivant 
        comment les données sont échangées sur un réseau, 
        notamment l'Internet, en référençant les technologies nécessaires. 
        \begin{align*}
            \underbrace{
            \textsf{Application
            }}_{\scriptsize{
                \texttt{HTTP, FTP, SMTP, SSH, Telnet}
            }} 
            + 
            \underbrace{
            \textsf{\phantom{A}
            Transport\phantom{A}
            }}_{\scriptsize{
            \texttt{TCP, UDP}
            }}  
            +\;
            \underbrace{
            \textsf{\phantom{T}
            Internet
            \phantom{p}} }_{
            \scriptsize{\texttt{IP, ICMP}
            }}
            \;\;+ \;
            \underbrace{
            \textsf{\phantom{T}
            Réseau
            \phantom{p}}}_{
            \scriptsize{\texttt{Ethernet, Wi-Fi}
            }}
        \end{align*}

        La couche
        \textit{application} 
            gère les interactions utilisateurs; 
        \textit{transport} 
            permet la communication machine;   
        \textit{internet}   
            sert à l'adressage et routage des paquets;
        et la couche
        \textit{réseau} 
            est responsable du transport physique des données.  
    \end{Définition}

    \section{SSH}
        SSH est un protocole de réseau qui permet d'établir 
        une connexion sur une machine distante afin d'exécuter 
        des opérations en ligne de commande et d'effectuer le transfert 
        de fichiers via des technologies auxiliaires telles SCP et SFTP. 
        \begin{figure}[h]       
\begin{center}
    \scalebox{0.85}{
    \begin{tikzpicture}
        \node (laptop) {\faLaptopCode};
        \node[above of=laptop, yshift=-14] (pin1) {$\blacktriangledown$};
        \node[above of=pin1, yshift=-18] (description1) {\textsf{\small{client ssh}}};

        \node (serveur) at ($(laptop) + (10.5cm, 0)$) {\faLayerGroup};
        \node[above of=serveur, yshift=-14] (pin2) {$\blacktriangledown$};
        \node[above of=pin2, yshift=-18] (description2) {\textsf{\small{serveur ssh}}};

        \draw[->Stealth] (laptop) -- (serveur);

        \node[align=left, yshift=1.25cm] (export) at ($(laptop)!0.5!(serveur)$)
            {$\overbracket{\texttt{Hello, bonjour}}^{\textsf{message/commande envoyé}}\underbracket{\cdots\cdots\cdots}_{\textsf{encryption}}\texttt{\$yb38r}\underbracket{\cdots\cdots\cdots}_{\textsf{décryption}}\overbracket{\texttt{Hello, bonjour}}^{\textsf{message/commande reçu}}$};

\end{tikzpicture}}
\caption{Transmission sécurisée d’un message via SSH}
\end{center}
\end{figure}

        % \[ \arraycolsep=1.4pt\def\arraystretch{1.5}
        % \begin{array}{lll}
        %     \underset{\blacktriangledown}{\texttt{client ssh}} & & 
        %     \underset{\blacktriangledown}{\texttt{serveur ssh}}
        %     \\
        %     \hspace{1.75em}\text{\faLaptopCode} & 
        %     \overset{\overbracket{\texttt{Hello, bonjour}}^{\text{message/commande envoyé}}\underbracket{\cdots\cdots\cdots}_{\text{encryption}}\texttt{\$yb38r}\underbracket{\cdots\cdots\cdots}_{\text{décryption}}\overbracket{\texttt{Hello, bonjour}}^{\text{message/commande reçu}}}{\xrightarrow{\hspace*{7cm}}} & 
        %     \hspace{2.175em}
        %     \text{\faLayerGroup}
        % \end{array}
        % \]





    \section{FTP, SFTP, SMTP, POP, IMAP, UDP}
        \textbf{\textsf{SFTP}} (\textit{Secure File Transfer Protocol})
        date de 1995 et
        est un successeur de FTP (1970). Contrairement à son prédecesseur, 
        SFTP permet le transport sécuritaire des données 
        de fichiers en communicant sur le port \( 22 \) et 
        en utilisant un seul canal chiffré et authentifié par SSH. 
\begin{figure}[H]
\begin{center}
% [>=stealth, node distance=1.4cm, every node/.style={align=center}]

\scalebox{0.625}{
    \begin{tikzpicture}
        [scale=1.0,node distance=1.4cm, >=stealth,
        box/.style={rectangle, draw,rounded corners, minimum width=2.9cm, minimum height=2cm, align=center, fill=myb!5},
        arrow/.style={-{Stealth}, },
        dashedarrow/.style={-{Stealth}, thick, dashed},
        label/.style={font=\small}
    ]
  % Nodes (top row)
  \node (client)     [box] {\faLaptopCode\\\texttt{Client}};
  \node (cloud)      [box, right=of client] {\faCloud\\\texttt{Connexion}};
  \node (interface)  [box, right=of cloud] {\faCode\\\texttt{Interface SFTP}};
  \node (request)    [box, right=of interface] {\faArrowDown\\\texttt{Requête}};

  % Node (right side)
  \node (server)     [box, below=of request] {\faServer\\\texttt{Serveur}};

  % Bottom row
  \node (encrypted)  [box, left=of server] {\faFolder\\\faLock\\\texttt{Fichier chiffré}};
  \node (decrypt)    [box, left=of encrypted] {\faUnlock\\\texttt{Déchiffrement}};
  \node (decrypted)  [box, left=of decrypt] {\faFolderOpen\\\texttt{Fichier clair}};

  % Arrows (top row)
  \draw[->] (client) -- (cloud);
  \draw[->] (cloud) -- (interface);
  \draw[->] (interface) -- (request);
  \draw[->] (request) -- (server);

  % Arrows (bottom path)
  \draw[->] (server) -- (encrypted);
  \draw[->] (encrypted) -- (decrypt);
  \draw[->] (decrypt) -- (decrypted);
\draw[->] (decrypted.west) -- ++ (-0.5, 0)|- (client.west);

\end{tikzpicture}}
\end{center}
\caption{ Étapes d’un transfert de fichier sécurisé via SFTP}
\end{figure}

\paragraph{}%
\textsf{\textbf{SMTP}} (\textit{Simple Mail Transfer Protocol}) est un protocole standard de l'Internet destiné 
à l’\textit{envoi} et au relais de courriels électroniques. 
Il fonctionne sur le port \(25\), en mode non sécurisé, 
sauf lorsqu’il est encapsulé via des alternatives chiffrées telles que SMTPS 
ou TLS.


\begin{figure}[H]
\begin{center}
\scalebox{0.75}{
\begin{tikzpicture}[scale=1.0,node distance=2.2cm and 2.5cm, >=stealth,
        box/.style={rectangle, draw,rounded corners, minimum width=2.9cm, minimum height=2cm, align=center, fill=myb!5},
        arrow/.style={-{Stealth}, },
        dashedarrow/.style={-{Stealth}, thick, dashed},
        label/.style={font=\small}
    ]
  % Ligne du haut : expéditeur -> SMTP -> internet
  \node (sender)   [box] {\faDesktop\\\texttt{Expéditeur}};
  \node (smtp)     [box,right=of sender] {\faServer\\\texttt{Serveur SMTP}};
  \node (cloud)    [box, right=of smtp] {\faCloud\\\texttt{Internet}};

  % Ligne du bas : POP/IMAP -> destinataire
  \node (popimap)  [box, below=of cloud] {\faServer\\\texttt{Serveur POP/IMAP}};
  \node (receiver) [box, left=of popimap] {\faLaptop\\\texttt{Destinataire}};

  % Flèches en haut (envoi)
  \draw[->] (sender) -- (smtp);
  \draw[->] (smtp) -- (cloud);

  % Flèche vers le bas : Internet vers serveur POP/IMAP
  \draw[->] (cloud) -- (popimap);

  % Flèche bas → gauche : POP/IMAP vers Destinataire
  \draw[->] (popimap) -- (receiver);
\end{tikzpicture}}
\end{center}
\caption{Chemin de transmission d’un courriel via SMTP et POP/IMAP}
\end{figure}
    Notons que SMTP est un protocole à \textit{flux unidirectionnel} : 
    il est \textit{utilisé exclusivement pour l’envoi de messages}. 
    Il ne permet ni la réception, ni la consultation du courrier par le destinataire.
        


    \textbf{\textsf{POP}}  (\textit{Post Office Protocol}) 
    est un protocole de la couche 
        application du modèle qui est utilisé 
        par les clients de messagerie pour récupérer
        ou pour supprimer les courriels 
        depuis un serveur mail. Le flux est unidirectionnel. Il permet 
        uniquement de stocker localement le courriel disponible 
        sur un serveur. 
        Le protocole opère de façon 
        non sécurisée sur le port \(110\) ou de façon sécurisée 
        sur le port \( 995 \). 

\begin{figure}[H]
\begin{center}
\scalebox{0.85}{
\begin{tikzpicture}[
    >=Stealth,
    node distance=5cm and 3cm,
    every node/.style={align=center}
  ]
  % Noeuds
  \node (email)   [] {\faEnvelope\\\texttt{e-mail}};
  \node (server)  [right=of email] {\faServer\\\texttt{Serveur mail}};
  \node (client)  [right=of server] {\faLaptop\\\texttt{Ordinateur utilisateur}};

  % SMTP
  \draw[->] (email) -- node[above]{\textsf{envoie via \textsf{SMTP}}} (server);

  % POP3 téléchargement (haut)
  \coordinate (aboveServer) at ($(server.north)+(0,1)$);
  \coordinate (aboveClient) at ($(client.north)+(0,1)$);
  \draw[->] (server.north) -- (aboveServer) -- (aboveClient) -- (client.north);
  \node at ($(aboveServer)!0.5!(aboveClient)$) [above] {\textsf{\textsf{POP3} télécharge les messages}};

  % POP3 suppression (bas)
  \coordinate (belowClient) at ($(client.south)-(0,1)$);
  \coordinate (belowServer) at ($(server.south)-(0,1)$);
  \draw[->] (client.south) -- (belowClient) -- (belowServer) -- (server.south);
  \node at ($(belowClient)!0.5!(belowServer)$) [below] {\textsf{Supprime les messages après récupération}};

\end{tikzpicture}}
\end{center}
\caption{Fonctionnement du protocole POP3 pour la réception des courriels}
\end{figure}
        \textbf{\textsf{IMAP}} (\textit{Internet Message Access Protocol})
        est un protocole de la couche \textit{application} du modèle 
        qui permet lire, organiser les courriels à distance  
        sur le serveur. Le protocole a un 
        \textit{flux bidirectionnel} : les changements                  
        effectué localement sont répercuté sur le serveur. 
        IMAP fonctionne via le port \(143\) par défaut 
        (non sécurisé) ou de façon sécurisée via le port 
        \( 993 \) en TLS.


        \begin{figure}[H]
\begin{center}
\scalebox{0.75}{
\begin{tikzpicture}
[scale=1.0,node distance=2.2cm and 3cm, >=stealth,
        box/.style={rectangle, draw,rounded corners, minimum width=2.8cm, minimum height=2cm, align=center, fill=myb!5},
        arrow/.style={-{Stealth}, },
        dashedarrow/.style={-{Stealth}, thick, dashed},
        label/.style={font=\small}
    ]

  % Clients
  \node (client1) [box] 
    {\faLaptop\\\texttt{Client IMAP }};
    

  
  % Serveur IMAP
  \node (server) [box, left=2.25cm of client1] 
    {\faServer\\\texttt{Serveur IMAP}};
  \node (server2) [box, left=5.5cm of server] 
    {\faServer\\\texttt{Serveur IMAP }};
  \node (client2) [box] 
      at ($(server)!0.5!(server2)+(0,-5cm)$) 
    {\faDesktop\\\texttt{Client IMAP }};

  % Flèches bidirectionnelles
  \draw[<->] ($(client2.north) + (0.25, 0)$) |-  ($(server.west) + (0,-0.75)$);
  \draw[<->] ($(client2.north) + (-0.25, 0)$) |-  ($(server2.east) + (0,-0.75)$);
  \draw[<->] (server.east) --  (client1.west);

  % Encadrés de texte explicatif
  \node[anchor=west, align=left, text width=5cm, right=0.5cm of client2] (rightbox) {
          \small\textsf{Il a un contrôle total sur ses dossiers courriel. 
    Il peut les créer, supprimer, déplacer ou vider, même entre 
différents comptes ou serveurs.}
  };

  \node[anchor=east, align=left, text width=5cm, below=0.5cm of client2] (leftbox) {
    \small \textsf{Le client décide où stocker les messages. 
    Il peut les copier localement ou les laisser sur le serveur.}
  };
\node[align=left] (multiple) at ($(server)!0.5!(server2)$)  {\small \textsf{Un client peut communiquer} \\ \textsf{avec plusieurs serveurs}};

    \textbf{\textsf{TCP}} \textit{Transmission Control Protocol}
    est un protocole de la couche \textit{transport} qui foutnit 
    un service de transimission fiablle, ordonné 
    et avec un contrôle de flux entre hôtes sur un réseau 
    IP. 


  \node[align=left, text width=7cm] (bottombox) 
      at ($(client1.south) + (-2.2cm, -1.5)$){
    \small\textsf{IMAP permet à plusieurs clients d'accéder au même compte 
    en conservant les messages sur le serveur depuis n'importe où.}
  };

\end{tikzpicture}}
\end{center}
\caption{Architecture IMAP avec accès simultané à plusieurs serveurs}
\end{figure}

        \textbf{\textsf{TCP}} \textit{Transmission Control Protocol}
        est un protocole de la couche \textit{transport} qui foutnit 
        un service de transimission fiablle, ordonné 
        et avec un contrôle de flux entre hôtes sur un réseau 
        IP. 

        \begin{figure}
\begin{center}
\scalebox{0.75}{
\begin{tikzpicture}[scale=1.0,node distance=2.2cm and 3.5cm, >=stealth,
        box/.style={rectangle, draw,rounded corners, minimum width=2.8cm, minimum height=2cm, align=center, fill=myb!5},
        arrow/.style={-{Stealth}, },
        dashedarrow/.style={-{Stealth}, thick, dashed},
        label/.style={font=\small}
    ]
  % Nœuds client et serveur
  \node (client) [box] 
    {\faLaptop\\\texttt{Client TCP}};
    
  \node (server) [box, right=6.5cm of client] 
    {\faServer\\\texttt{Serveur TCP}};

  % Flèches handshake
  \draw[->] (client.north) -- ++ (0, 1) -|  (server.north);
  \node at ($(client)!0.5!(server) +(0,2.25)$) {\textsf{1.} \texttt{SYN}};
  \draw[->] ($(server.west)+(0,0)$) -- node[above] {\textsf{2.} \texttt{SYN-ACK}} ($(client.east)+(0,0)$);

  \draw[->] (server.south) -- ++ (0, -1) -|  (client.south);

  \node at ($(client.south)!0.5!(server.south) +(0,-1.25)$) {\textsf{3.} \texttt{ACK}};
  % Échange de données
  \draw[<->, thick, dashed] ($(server.south) +(-0.5,0)$) -- ($(server.south) +(-0.5,-0.5)$)  
      |-($(client.south) +(0.5,-0.5)$) -- ++ (0, 0.5) ;

  % Encadrés explicatifs
  \node[align=left, text width=6.5cm, anchor=north] (expl1) at ($(client.north)+(1.8,+3)$) {
      \small  \textsf{Le client envoie un paquet \texttt{SYN} pour initier une connexion TCP. 
    Il propose un numéro de séquence initial.} };

  \node[align=left, text width=5.2cm, anchor=north] (expl2) at ($(server.north)+(-0.5,3)$) {
          \small \textsf{Le serveur répond avec \texttt{SYN-ACK} pour accepter la connexion et envoyer son propre numéro de séquence.}
  };

  \node[align=left, text width=7.5cm, anchor=north] (expl3) at ($(client.south)!0.5!(server.south)+(0,-1.85)$) {
          \small \textsf{Le client confirme avec un \texttt{ACK} final. La connexion est alors établie et les données peuvent être échangées de manière fiable.}
  };

\node[] (dataEx) at ($(client.south)+(0, 0.25)!0.5!(server.south)+(0, 0.25)$)  
    {\textsf{Échange de données}};

\end{tikzpicture}}
\end{center}
\caption{Établissement d’une connexion TCP selon le \textit{three-way handshake}}
\end{figure}


        \begin{Remarque}[Modèle TCP/IP et protocole TCP]
            Il faut savoir distinguer le modèle TCP/IP qui fait référence à un concept décrivant 
            les quatres couches de technologies impliquées dans le système TCP/IP de la 
            \textit{technologie} TCP 
            qui est un protocole à part entière faisant 
            partie de la couche transport du modèle TCP/IP.   
        \end{Remarque}
        \textbf{\textsf{UDP}}  (\textit{User Datagram Protocol}) est un protocole de la couche 
        \textit{transport} de la suite IP, conçu pour envoyer des datagrammes---des unités de données échangées 
        sur les réseau à commutation de paquets---sans établir de connexion préalable et sans garantie de 
        livraison. 

\begin{figure}[H]
\begin{center}
\scalebox{0.75}{
\begin{tikzpicture}
[scale=1.0,node distance=1.4cm, >=stealth,
        box/.style={rectangle, draw,rounded corners, minimum width=2.9cm, minimum height=2cm, align=center, fill=myb!5},
        arrow/.style={-{Stealth}, },
        dashedarrow/.style={-{Stealth}, thick, dashed},
        label/.style={font=\small}
    ]

  % Nœuds client et serveur
  \node (client) [box] 
    {\faLaptop\\\texttt{Client UDP}};
    
  \node (server) [box, right=6.5cm of client] 
    {\faServer\\\texttt{Serveur UDP}};

  % Flèche unidirectionnelle
  \draw[->] (client.east) -- node[above]{\textsf{Datagramme envoyé}} (server.west);

  % Flèche retour optionnelle (UDP est bidirectionnel, mais sans état)
  \draw[->, dashed] ($(server.west) + (0, -0.25)$) -- node[below]{\textsf{Datagramme de réponse (optionnel)}} ($(client.east) + (0, -0.25)$);

  % Encadré 1 : envoi direct
  \node[align=left, text width=6cm, anchor=north] at ($(client.north)+(1.65,+1.75)$) {
      \small \textsf{Le client envoie un datagramme sans établir de connexion. 
      Aucun numéro de séquence n’est négocié.}
  };

  % Encadré 2 : pas de garantie
  \node[align=left, text width=5.5cm, anchor=north] at ($(server.south)+(-0.5,-0.5)$) {
      \small \textsf{Le serveur reçoit le message, mais il n’y a aucune garantie de réception, d’ordre ou de duplication.}
  };



\end{tikzpicture}}
\end{center}
\caption{Envoi de datagrammes sans connexion avec UDP}
\end{figure}

UDP est un protocole léger, sans connexion, 
      adapté aux applications qui privilégient la rapidité à la fiabilité, 
      comme le streaming ou le DNS.

        \section{Datagramme}
        Un datagramme est une unité de données qui comporte deux compsant structurels, 
        soit une en-tête aussi appelée \textit{header} et  une charge utile 
        aussi appelé \textit{payload}.  

        L'en-tête contient les informations de routate (adresse, longueur, type, etc.), alors que la 
        charge utile transporte les données de l'utilisateur. Dans le contexte de la suite 
        TCP/IP, les paquets IP sont essentiellement des datagrammes IP.
\begin{figure}[H]
\begin{center}
\scalebox{0.75}{
\begin{tikzpicture}[scale=1.0,node distance=0cm, >=stealth,
        box/.style={rectangle, draw,rounded corners, minimum width=4cm, minimum height=1cm, align=center, fill=myb!5},
        arrow/.style={-{Stealth}, },
        dashedarrow/.style={-{Stealth}, thick, dashed},
        label/.style={font=\small}
    ]
  % Ports (première ligne)
  \node[box] (src) {\textsf{Port source (16 bits)}};
  \node[box, right=0cm of src] (dst) {\textsf{Port destination (16 bits)}};

  % Longueur + Checksum (deuxième ligne)
  \node[box, below=0cm of src] (len) {\textsf{Longueur (16 bits)}};
  \node[box, below=0cm of dst] (chk) {\textsf{Checksum (16 bits)}};

  % Données (troisième ligne, pleine largeur)
  \node[align=left,box, below=0.5cm of len.east |- chk.east] (data) {\textsf{Données (variable)}};

  % Légende en bas
  \node[align=left, text width=10cm, below=0.5cm of data] (expl) {
    \small \textsf{Un datagramme UDP est minimaliste : 8 octets d’en-tête (ports, longueur, checksum) suivis de la charge utile. 
    Contrairement à TCP, il ne gère ni la connexion ni l’ordre des paquets.}
  };

\end{tikzpicture}}
\end{center}
\caption{Format de l’en-tête UDP}
\end{figure}

        \section{Procesus sans connexion UDP et TCP}
        Contraitement à TCP qui effectue un \textit{handshake} en trois temps pour négocier 
        l'ouverture d'une session avant d'envoyer des données, UDP se contente 
        d'envoyer des datagrammes de façon aveugle. Par conséquent, l'expéditeur 
        n'attend pas de confirmation (SYN) que le destinataire est prêt à 
        recevoir. 

        Par ailleurs, UDP ne prévoit ni accusé de réception (ACK), ni réordonnancement, ni 
        de retransmission automatique. Ainsi, il n'y a pas de garantie que 
        les données atteignent leur destination ou que leur intégrité soit préservée. 
        UDP est donc considéré comme un protocole \textit{connectionless} et 
        \textit{unreliable}---sans garantie de fidelité.   
        
        \section{Identificateur et localisateur}
        Une URI ou \textit{Universal Ressource Identifier} est un identifiant générique 
        de ressource sur le Web. La syntaxe générale d'un URI est la suivante :
        \begin{align*}
        \fontfamily{lmss}\selectfont
         \underbrace{\texttt{\phantom{t}mailto\phantom{g}}}_{\text{schéma}}%
        \texttt{:}%
        \underbrace{\phantom{t}\texttt{[alice\@gmail.com]\phantom{g}}}_{\text{chemin}} \; % 
        \underbrace{\phantom{t}\texttt{[?...]\phantom{g}}}_{\text{requête}} \; % 
        \underbrace{\phantom{t}\texttt{[\#...]\phantom{g}}}_{\text{fragment}} 
        \\ 
        \end{align*}
        Il existe \textbf{plusieurs type d'\uppercase{uri}} : les \textbf{\uppercase{urn} }  pour les technologies de courrier,
        les \uppercase{urn} pour 
        les numéro d'\uppercase{isbn}, et les \textbf{\uppercase{URL}}  
        pour les pages Web sont des exemples d'\uppercase{uri}. 


        \begin{Concept}[URL]
            Une \uppercase{url} est composée, entre autres, 
            d'un \textit{schéma}, p. ex. \uppercase{HTTP}, 
        d'un \textit{nom de domaine}, d'un \textit{chemin}, de 
        \textit{paramètres}  
        et d'une \textit{ancre}.   
        \begin{align*}
        \fontfamily{lmss}\selectfont
         \underbrace{\texttt{\phantom{t}scheme\phantom{g}}}_{\text{protocole}}%
        \texttt{:}%
        \underbrace{\texttt{[//authority]}}_{\text{authentification ou domaine}}%
        \underbrace{\texttt{\phantom{t}path\phantom{g}}}_{\text{chemin}} \; %
        \underbrace{\phantom{t}\texttt{[?query]}\phantom{g}}_{\text{paramètres}} \;%
        \underbrace{\phantom{t}\texttt{[\#anchor]\phantom{g}}}_{\text{ancre}}
        \\ 
        \end{align*}
        C'est un \textbf{type particulier} d'\uppercase{uri} qui, en plus d'identifier, localise la ressource 
        et indique comment y accéder. 
    \end{Concept}
        \section{Fonctionnement du Web et URI}
        \begin{center}
            \textsf{Web} = \textsf{URI} + \textsf{HTTP} + \textsf{HTML}        
        \end{center}
        Le modèle ci-haut permet de mettre en évidences les trois briques fondamentales du Web. 
        Le protocole \textbf{\uppercase{http}} régit la communication entre différentes sources et destinataires 
        selon le modèle \textit{client-serveur}. Le langage \textbf{\uppercase{html}}   permet la conception 
        de page Web, et les technologies d'\textbf{\uppercase{uri}} permettent d'associer une ressource telle qu'une page 
        à des caractéristiques d'identification, de localisation et destinations.

        

        \begin{figure}[H]
            \[
            \small
            \texttt{http://}\underbrace{\texttt{www.iro.umontreal.ca}}_{\text{nom de domaine}}%
            \texttt{/}\underbrace{\texttt{\~felipe/new-home/frontal.php}}_{\textit{path}}%
            \texttt{?}\underbrace{\texttt{page=resume}}_{\textit{query}}
            \] 
            \normalsize
            \caption{Exemple d'\uppercase{URI}}
        \end{figure}                        


    \section{DNS}
    Le \uppercase{DNS} est un système distribué 
        utilisé pour \textit{traduire les noms de domaine}  
        lisibles par l'humain 
        en adresses IP compréhensibles 
        par les machines. 
     
\begin{figure}[H]
    \begin{center}
        \scalebox{0.85}{
    \begin{tikzpicture}[scale=0.25,
        box/.style={rectangle, draw, myb, rounded corners, minimum width=1.5cm, minimum height=0.5cm, align=center, fill=myb!5},
        arrow/.style={-{Stealth}, },
        dashedarrow/.style={-{Stealth}, dashed},
        label/.style={font=\small}
    ]

    \sffamily
    % Nodes
    \node[box] (client) {Client (example.com)};
    \node[box, below=1.5cm of client] (resolver) {DNS Resolver};
    \node[box, below right=1.5cm and 2cm of resolver] (rootserver) {Serveur Racine};
    \node[box, below=1.5cm of rootserver] (tldserver) {Serveur TLD  (.com, .org)};
    \node[box, below=1.5cm of tldserver] (authserver) {Serveur Authoritative};
    \node[box, right=3.5cm of client] (webserver) {Serveur Web};

    % Arrows (Recursive Queries)
    \draw[arrow] (client.south) -- node[label, left] {1. Requête Initiale} (resolver.north);

    % Arrows (Iterative Queries)
    \draw[dashedarrow] ([yshift=0.25cm]resolver.east) --  ++(0.5,0) -| node[label, below right, yshift=-0.095cm] {2. Requête} ([xshift=0.35cm]rootserver.north);
    \draw[dashedarrow] ([xshift=-0.1cm]rootserver.north) |- node[label, below left] {3. Réponse} ([yshift=-0.25cm]resolver.east);

    \draw[dashedarrow] ([xshift=1cm]resolver.south)  |- node[label, above right] {4. Requête} ([yshift=0.15cm]tldserver.west);
    \draw[dashedarrow] ([yshift=-0.25cm]tldserver.west)   -| node[label, below, xshift=1.20cm] {5. Réponse} ([xshift=0.5cm]resolver.south);

    \draw[dashedarrow] ([xshift=-0.25cm]resolver.south)  |- node[label, above right] {6. Requête} ([yshift=0.15cm]authserver.west);
    \draw[dashedarrow] ([yshift=-0.25cm]authserver.west) -- ++(-0.5,0) -| node[label, below right] {7. Réponse} ([xshift=-0.75cm]resolver.south);

    % Arrows (Réponse to client)
    \draw[arrow] ([xshift=0.4cm]resolver.north) -- node[label, right] {8. Résponse Finale} ([xshift=0.4cm]client.south);
a
    % Web Serveur
    \draw[arrow] (client.east) -- node[label, above] {9. Requête HTTP} (webserver.west);
    \draw[arrow] ([yshift=-0.4cm]webserver.west) --  node[label, below] {10. Réponse HTTP } ([yshift=-0.4cm]client.east);

    % Additional labels
    \node[align=center, below=0.25cm of rootserver, font=\small, text width=3cm] (note1) 
        {[A-M] Serveur K reçoit $\sim$20k requêtes/s};
        
\end{tikzpicture}}
    \end{center}
    \caption{Étapes de requête d'une résolution DNS} 
\end{figure}



\noindent La résolution DNS débute par la requête initiale du client
vers son \textit{resolver} local configuré par le FAI (étape 1).
Le resolver envoie une requête itérative au serveur racine
(étape 2), qui répond en fournissant l’adresse d’un TLD adapté (étape 3).
Le resolver interroge alors le serveur TLD responsable des 
adresses \textit{.com} et \textit{.org}, par exemple (étape 4),
lequel renvoie l’adresse du serveur \textit{authoritative} pour le domaine
(étape 5). Celui-ci retourne l’enregistrement A/AAAA demandé (étape 7). Le
resolver assemble les réponses puis envoie la réponse finale au client
(étape 8). Le client initie ensuite une connexion HTTP vers le serveur Web
obtenu (étape 9) et reçoit la réponse HTTP (étape 10).

\section{DNS Resolver}
    Le \textit{stub resolver} est intégré à l’OS et transmet les
    requêtes DNS aux resolvers récursifs (BIND, Unbound, services
    FAI ou tiers publics).  Le resolver est avant tout un logiciel, côté client
    et côté serveur. Certains routeurs ou appliances réseau
    intègrent un resolver optimisé en firmware.

\begin{figure}[H]
    \begin{center}

\scalebox{0.75}{
    \begin{tikzpicture}[scale=0.25,
        box/.style={rectangle, draw, myb, rounded corners, minimum width=2.8cm, minimum height=1.2cm, align=center, fill=myb!5},
        arrow/.style={-{Stealth}},
        label/.style={font=\small}
    ]

    \sffamily
    % --- Noeuds principaux ---
    \node[box] (os) {Système d'exploitation \\ (\textit{Stub Resolver})};
    \node[box, below=3cm of os] (resolver) {Resolver DNS récursif \\ (BIND, Unbound, FAI, services publics)};
    \node[box, below left=3cm and 0.5cm of resolver] (router) {Routeur / Appliance réseau \\ (Resolver embarqué)};
    \node[box, above right=0.5cm and  -0.75cm of resolver] (public) {Resolvers publics \\ (Google DNS, Cloudflare, etc.)};
    \node[box, below=3cm of resolver] (dnsinfra) {Infrastructure DNS globale \\ (Serveurs racine, TLD, Authoritative)};

    % --- Flèches principales ---
    \draw[arrow] (os.south) -- node[label, right] {Requête DNS} (resolver.north);

    \draw[arrow] (resolver.south) -- node[label, right] {Résolution récursive} (dnsinfra.north);
    \draw[arrow] (dnsinfra.north) -- ++(0,1.5) node[label, right] {Réponse DNS} (resolver.south);

    \draw[arrow] (resolver.north) -- ++(0,1.5) node[label, left] {Réponse DNS} (os.south);

    % --- Flèches supplémentaires ---
    \draw[arrow, dashed] (os.west) -- ++(-3,0) node[label, above, xshift=-2.25cm, yshift=0.5] {Certaines appliances intègrent un resolver} -| (router.north);

    \draw[arrow, dashed, xshift=6.75] (resolver.east) -|  (public.south);
    \node[label, above] at ($(public.south) + (0, -9.5)$) {\textsf{Resolver publics (alternatifs)}};
    % --- Notes supplémentaires ---
    \node[align=left, below=0.5cm of router, font=\small, text width=5cm] (note-router)
        {Firmware optimisé : résolveur intégré \\ pour les environnements locaux.};

\end{tikzpicture}}
    \end{center}
    \caption{Fonctionnement du DNS Resolver (Stub, récursif, public, intégré)}
\end{figure}
\section{Emplacement du Resolver}
    Le stub resolver vit dans chaque poste utilisateur.
    Le resolver récursif peut résider dans le routeur local,
    chez le FAI ou chez un fournisseur DNS public (1.1.1.1).
\section{Serveur Racine}
    Le serveur racine gère la zone « . » et renvoie les adresses
    anycast des serveurs TLD responsable 
    des adresses .com, .org, .fr, etc., sans fournir
    directement l’enregistrement final.

\begin{figure}[H]
    \begin{center}
    \scalebox{0.75}{
    \begin{tikzpicture}[scale=0.3,
        box/.style={rectangle, draw, myb, rounded corners, minimum width=2.8cm, minimum height=1.2cm, align=center, fill=myb!5},
        arrow/.style={-{Stealth}, thick},
        dashedarrow/.style={-{Stealth}, thick, dashed},
        label/.style={font=\small}
    ]

    \sffamily
    % --- Noeuds principaux ---
    \node[box] (resolver) {Resolver DNS récursif};
    \node[box, below=6.25cm of resolver] (root) {Serveur Racine \\ (zone « . »)};
    \node[box, below left=0.5cm and 1cm of root] (tldcom) {Serveur TLD \\ .com};
    \node[box, below=2cm of root] (tldorg) {Serveur TLD \\ .org};
    \node[box, below right=0.5cm and 1cm of root] (tldfr) {Serveur TLD \\ .fr};

    % --- Flèches principales ---
    \draw[arrow] (resolver.south) -- node[right, label, above, sloped] {Requête : example.com} (root.north);
    
    % Réponses vers Resolver
    \draw[arrow] (root.north) --  node[right, sloped, label, right, xshift=-2.5cm, yshift=0.25cm] {Réponse : Adresses anycast TLD} (resolver.south);

    % Flèches root vers TLD
    \draw[dashedarrow] (root.south west) -| node[label, above, xshift=1.5cm] {Info .com} (tldcom.north);
    \draw[dashedarrow] (root.south) -- node[label, right] {Info .org} (tldorg.north);
    \draw[dashedarrow] (root.south east) -| node[label, above, xshift=-1.5cm] {Info .fr} (tldfr.north);

    % --- Notes ---
    \node[align=left, below=0.5cm of tldorg, font=\small, text width=10cm] (note) 
        {\textbf{Rôle :} Le serveur racine gère la zone « . » et renvoie \\ 
        les adresses \textit{anycast} des serveurs TLD. \\ 
        };

\end{tikzpicture}}
    \end{center}
    \caption{Fonctionnement du Serveur Racine DNS}
\end{figure}

\section{Enregistrements A et AAAA}
    Les enregistrements A associent un nom de domaine
    à une adresse IPv4, tandis que AAAA fournit une
    adresse IPv6. Lorsqu’un resolver interroge un serveur DNS pour 
    obtenir les enregistrements
    A/AAAA, il reçoit l’adresse IP cible pour établir la connexion.


\begin{figure}[H]
    \begin{center}
    \scalebox{0.75}{
    \begin{tikzpicture}[scale=0.3,
        box/.style={rectangle, draw, myb, rounded corners, minimum width=2.8cm, minimum height=1.2cm, align=center, fill=myb!5},
        arrow/.style={-{Stealth}, thick},
        dashedarrow/.style={-{Stealth}, thick, dashed},
        label/.style={font=\small}
    ]

    \sffamily
    % --- Noeuds ---
    \node[box] (client) {Client / Resolver DNS};
    \node[box, right=6cm of client] (dns) {Serveur DNS};
    \node[box, below =2cm of dns] (server) {Serveur Web cible};

    % --- Flèches ---
    \draw[arrow] ($(client.east) + (0,0.25)$) -- node[label, above] {Requête : "A / AAAA pour \texttt{example.com}"} ($(dns.west) + (0,0.25)$);
    \draw[arrow] ($(dns.west) + (0,-0.25)$) -- node[label, below] {Réponse : IPv4 (A)  ou IPv6 (AAAA) } ($(client.east) + (0,-0.25)$);

    % Connexion IP finale
    \draw[arrow, dashed] (client.south) -- ++(0,-4.5) -- ++(40,0) |- (server.east);

    % --- Légendes IP ---
    \node[below=0.75cm of dns, align=left,xshift=-14, font=\small\sffamily, text width=12cm] (notes) {
         \textbf{A} : Associe le domaine à une \textbf{IPv4} (ex. \texttt{93.184.216.34})};
    \node[below=1.5cm of dns, align=left, xshift=-100, font=\small\sffamily, text width=12cm] (notes2) {
        \textbf{AAAA} : Associe le domaine à une \textbf{IPv6} (ex. \texttt{2606:2800:220:1:248:1893:25c8:1946})
    };


\end{tikzpicture}}
    \end{center}
    \caption{Résolution DNS avec enregistrements A (IPv4) et AAAA (IPv6)}
\end{figure}


\section{Serveur TLD}
    Un serveur TLD gère un \textit{Top-Level Domain}
    (.com, .org, .fr). Il délègue la résolution des noms aux
    serveurs \textit{authoritative} via des enregistrements NS.

\section{Serveur authoritative}
    Le serveur \textit{authoritative} détient la copie officielle
    des enregistrements DNS d’une zone (maître ou secondaire).
    Il répond de manière définitive aux requêtes sans délégation
    supplémentaire.

\section{Avantages de la répartition}
    Cette hiérarchie garantit la scalabilité en partageant
    la charge, améliore la performance grâce aux caches
    et à la répartition géographique, renforce la
    robustesse via la redondance et permet une
    gestion déléguée et sécurisée.


   \section{HTTP}
    Protocole basé sur le modèle client-serveur où le \( C \) envoie une requête 
    au \( S \) et le \( S \) répond avec les ressources demandées. Plusieurs 
    proxy intermédiaires peuvent intervenir entre l'envoie et la réception. 

    \textbf{\textsf{Sans état}} : HTTP  est un des rares 
    protocoles sans état  . SMTP  , IMAP et FTP maintiennent 
    une session durant laquelle des 
    informations de contexte sont échangées.


    \textbf{\textsf{Types de requêtes}} : GET récupère la ressource; POST   
    crée une ressource en envoyant des données; PUT  remplace ou crée la ressource; 
    DELETE la supprime; HEAD   renvoie uniquement les entêtes.

    \chapter{Scriptage}
    La connaissance de commandes UNIX et une compréhension générale 
    du langage bash évite souvent d'avoir à programmer des scripts 
    élaborés qui peuvent être remplacé par des commandes
    simples mais sophistiquées.  

    \section{Commande linux de navigation et de gestion de fichiers}
    \begin{tikzpicture}[dirtree]
        \node[directory] {\fontfamily{lmss}\selectfont Commandes Linux}
            child { node[directory] {ls}
                child { node {\fontfamily{lmss}\selectfont Liste le contenu d'un répertoire} }
            }
            child[missing] {}
            child { node[directory] {cd rep}
                child { node {\fontfamily{lmss}\selectfont Se déplace dans le répertoire \texttt{rep}} }
            }
            child[missing] {}
            child { node[directory] {rm file}
                child { node {\fontfamily{lmss}\selectfont Supprime le fichier \texttt{rep}} }
            }
            child[missing] {}
            child { node[directory] {mkdir rep}
                child { node {\fontfamily{lmss}\selectfont Crée le répertoire \texttt{rep} dans le répertoire courant} }
            }
            child[missing] {}
            child { node[directory] {cd}
                child { node {\fontfamily{lmss}\selectfont Se déplace dans le répertoire \texttt{home}} }
            }
            child[missing] {}
            child { node[directory] {pwd}
                child { node {\fontfamily{lmss}\selectfont Indique le répertoire courant} }
            }
            ;
    \end{tikzpicture}


    \subsection{Pipeline de redirection}
    \begin{Concept}[Pipeline]
        La pipeline dénotée par la syntaxe \xfbox[black!80!cyan!40]{|} permet 
    de diriger la sortie d'une première commande 
    sur l'entrée standard d'une seconde commande afin que l'\textit{output}
    de la première serve d'\textit{input} à la seconde.
    \end{Concept}

    \begin{Exemple}[Utilisation d'une pipeline pour inverser l'affichage d'un string]

        La commande \xfbox[black!80!cyan!40]{\texttt{rev}} renverse le texte en \textit{input} 
        ligne à ligne.   
            \begin{minted}{bash}
                            echo "bonjour" | rev
            \end{minted}
    \end{Exemple}
    On contaste que, contrairement à \xfbox[black!80!cyan!40]{\texttt{tac}}, 
    la commande \textsf{rev} ne modifie par l'ordre des lignes. Par ailleurs, 
    \texttt{rev} n'inverse pas tous le contenu d'un  fichier, mais 
    plutôt \textbf{chaque ligne individuellement}. 
    \verb|rev|. 

    \begin{Note}[Directionnalité]
        L'inversion de l'ordre de la commande n'est pas possible puisque 
        \xfbox[black!80!cyan!40]{\texttt{rev}} lit uniquement 
        sur des entrées standard (\texttt{input} d'une pipeline) ou un fichier, et ne prend pas 
        directement d'argument de texte en ligne de commande : 
        \begin{center}
            \begin{minted}{bash}
                        rev "bonjour" | echo
            \end{minted}
        \end{center}
        L'exemple ci-haut \textbf{ne respecte pas la syntaxe}. D'ailleurs,  
        \xfbox[black!80!cyan!40]{echo} s'attend à un argument qui le succède.
    \end{Note}




    \begin{Concept}[Redirection en \texttt{csh}]
            
        Le symbole de redirection
        \noindent\xfbox[black!80!cyan!40]{\texttt{>}}  
        redirige la sortie standard vers un fichier. 
        Si le fichier existe déjà, \textbf{il est écrasé}.
        \vspace{0.5em}

        L'opérateur \noindent\xfbox[black!80!cyan!40]{\texttt{>!}}  
        Force la redirection de la sortie standard vers un fichier. 
        Cela écrase le fichier existant \textbf{sans avertissement}.

        \vspace{0.5em}

        L'opérateur \noindent\xfbox[black!80!cyan!40]{\texttt{>>}}  
        ajoute la sortie standard à la fin d'un fichier existant. Si le fichier 
        n'existe pas, il est créé. 
    \end{Concept}

    \begin{Exemple}{Redirection vers un fichier}{}
        \begin{minted}{bash}
                    echo "bonjour" | rev > fichier.txt
        \end{minted}
        La commande \xfbox[black!80!cyan!40]{>} permet d'enregistrer 
        un \textit{input} dans le fichier \texttt{fichier.txt}.   
    \end{Exemple}

    \section{Manipulation de fichiers textes}
    \begin{center}
        \begin{tikzpicture}[dirtree]
            \node[directory] {\fontfamily{lmss}\selectfont Commandes UNIX sur fichier texte}
                child { node[directory] {head}
                    child { node {\fontfamily{lmss}\selectfont Affiche les premières lignes d'un fichier} }
                }
                child[missing] {}
                child { node[directory] {tail}
                    child { node {\fontfamily{lmss}\selectfont Affiche les dernières lignes d'un fichier} }
                }
                child[missing] {}
                child { node[directory] {tr}
                    child { node {\fontfamily{lmss}\selectfont Transforme ou supprime des caractères dans un texte} }
                }
                ;
        \end{tikzpicture}
    \end{center}

    \section{La commande \texttt{tr}}

    \begin{Définition}[Commande \texttt{tr}]
        La commande \xfbox[black!80!cyan!40]{\texttt{tr}} 
        est utilisée pour transformer ou supprimer des 
        caractères dans une entrée standard. 
    \end{Définition}
    \subsection{Utilisation de \texttt{tr} et \texttt{-s}}

    \begin{Exemple}{Découpe en mots une ligne avec ponctuation}{}



    \begin{center}
  \begin{adjustbox}{max width=\linewidth}
    \begin{cminted}{bash}
head -n 1 zola1.txt | tr    '[[:punct:]]' ' ' 
                    | tr    '[[:space:]]' '\n' 
                    | tr -s '[[:space:]]'
    \end{cminted}
  \end{adjustbox}
    \end{center}
        Cette commande découpe la première ligne de \texttt{zola1.txt} en mots en 
        remplaçant les ponctuations par des espaces, divisant chaque mot sur une 
        nouvelle ligne et éliminant les espaces redondants.

        \vspace{0.5em}

        L'utilisation de \texttt{tr}  
        est justifiée par le fait que la commande différencie correctement 
        un mot suivi d'une virgule ou d'un 
        point. Par exemple, on obtient la transformation suivante :
        \begin{center}
            \( \texttt{Bourse,} \xrightarrow{\hspace*{3cm}} \texttt{Bourse}\)
        \end{center}
        \hfill
    \end{Exemple}


    \begin{Note}{}{}
        L'option \xfbox[black!80!cyan!40]{\texttt{-s}} supprime les \textbf{répétitions successives} 
        des caractères spécifiés dans l'ensemble donné. Ainsi :



\begin{center}
\begin{cminted}{bash}
    echo "aaa   bbb    ccc" | tr -s ' '
\end{cminted}
\end{center}

        engendre le \textit{output} \texttt{"aaa bbb ccc"} en supprimant les répétitons 
        d'espaces, soit \xfbox[black!80!cyan!40]{\texttt{' '}}.
        \\\hfill
    \end{Note}
    \begin{Exemple}{Découpage sélectif des mots alphabétiques}{}

\begin{center}
\begin{minipage}{0.5\linewidth}
\begin{cminted}{bash}
head -n 1 zola1.txt | tr -sc '[[:alpha:]]' '\n'
\end{cminted}
\end{minipage}
\end{center}

        Cette commande découpe la première ligne de \texttt{zola1.txt} en mots, 
        remplaçant \textit{tous caractères non alphabétiques}, 
        y compris ponctuation, par des 
        sauts de ligne.

        \vspace{0.5em}
        Cette approche est plus concise et efficace. Elle ne nécessite pas de 
        multiples appels à \texttt{tr} et évite les mots collés à une ponctuation.
    \end{Exemple}

    \subsection{Utilisation de \texttt{tr} et \texttt{-sc}}
    \begin{Exemple}[Combinaison avec l'option complément -c]
        Dans l'exemple 
        ci-bas, les options \noindent\xfbox[black!80!cyan!40]{\texttt{-sc}} 
        combinent deux fonctionnalités.
        L'option \xfbox[black!80!cyan!40]{\texttt{-c}} 
        prend le complément de l’ensemble spécifié qui est 
        \texttt{:alpha:}, soit
        \textbf{tout sauf les lettres alphabétiques}, et
        l'option
        \noindent\xfbox[black!80!cyan!40]{\texttt{-c}}  
        compresse les répétitions
        des caractères du complément. 
        \begin{center}
        \begin{minipage}{0.5\linewidth}
            \begin{cminted}{bash}
echo "abc123@@@def456" | tr -sc '[:alpha:]' '\n'
            \end{cminted}
        \end{minipage}
        \end{center}


        \vspace{0.5em}

        \noindent 
        Ainsi, on obtient la transformation suivante :
        \begin{align*}
            \textcolor{gray}{\texttt{1.}}  \texttt{abc123@@@def456} \quad \xrightarrow{\hspace*{3cm}} \quad &\textcolor{gray}{\texttt{1.}} \texttt{abc}
            \\
                                                 &\textcolor{gray}{\texttt{2.}} \texttt{def}
        \end{align*}

        Ainsi, les caractères non alphabétiques sont remplacés par des
        sauts de ligne compressés.
        \hfill\\
    \end{Exemple}

    \section{La commande \texttt{grep}}
    \begin{Concept}[Commande grep]
        La commande grep est utilisée pour recherches des lignes correspondant 
        à un \textit{motif}  dans un fichier ou dans une entrée standard. 
    \end{Concept}
    \vspace{1em}


    \begin{Exemple}[Recherche insensible à la casse avec \texttt{grep -i}]
        Cette commande recherche toutes les lignes contenant \textit{bonjour}   
        dans le fichier spécifié, sans tenir compte de la casse. 
\begin{center}
\begin{adjustbox}{max width=\linewidth}
        \begin{cminted}{bash}
grep -i "bonjour" fichier.txt
        \end{cminted}
\end{adjustbox}
\end{center}
    \end{Exemple}



    \begin{Exemple}{Recherche du motif \texttt{ven} insensible à la casse}{}
\begin{center}
  \begin{adjustbox}{max width=\linewidth}
    \begin{cminted}{bash}
head -n 60 zola1.txt | tr '[[:space:]]' '\n' | grep -i ven
    \end{cminted}
  \end{adjustbox}
\end{center}

        Cette commande extrait les 60 premières lignes du fichier \texttt{zola1.txt},
        divise chaque mot sur une nouvelle ligne en remplaçant les espaces par des 
        sauts de ligne, puis recherche toutes les occurrences du motif \texttt{ven}
        sans tenir compte de la casse.
    \end{Exemple}

    \vspace{1em}
    \begin{Exemple}{Recherche avec surlignage du motif \texttt{ven}}{}
\begin{center}
  \begin{adjustbox}{max width=\linewidth}
    \begin{cminted}{bash}
head -n 60 zola1.txt | tr '[[:space:]]' '\n' | grep --colour -i ven
    \end{cminted}
  \end{adjustbox}
\end{center}


        Cette commande effectue la même recherche que l'exemple précédent, mais 
        utilise l'option \texttt{--colour} pour surligner en couleur toutes les 
        occurrences du motif \texttt{ven}, rendant les résultats plus visibles.
    \end{Exemple}


    \section{La commande \text{awk}}
    \begin{Définition}[Commande \texttt{awk}]
        La commande awk est un langage de traitement de texte utilisé pour
        analyser et manipuler des fichiers ou des flux de données structurés, ligne
        par ligne, en fonction de motifs et d'actions spécifiés.
    \end{Définition}
    \vspace{2em}
    \begin{Exemple}{Filtrage de mots fréquents selon leur ordre}{}
    \begin{center}
        \begin{adjustbox}{max width=\linewidth}
            \begin{cminted}{bash}
head -n 200 zola1.txt | tr '[[:space:]]' '\n' 
                      | grep -i '^ven' 
                      | sort | uniq -c 
                      | sort -k1,1nr | awk '{print $2}'
            \end{cminted}
        \end{adjustbox}
    \end{center}                    
    \end{Exemple}

    Notons ici que \xfbox[black!80!cyan!40]{\texttt{n}} force le trie 
    numérique plutôt que le tri par défaut en ASCII, afin que, 
    par exemple, la chaîne \guillemotleft 10\guillemotright 
     soit considéré plus grande que 
    la chaîne \guillemotleft 2\guillemotright. Par ailleurs,
    le mot clé \xfbox[black!80!cyan!40]{\texttt{r}} permet 
    d'effectuer le tri en ordre inverse. L'option 
    \xfbox[black!80!cyan!40]{\texttt{-k1,1}} quant à elle 
    définit une \guillemotleft clé\guillemotright de tri 
    qui va du champ 1 au champ 1, c'est-à-dire que seule 
    la première colonne est inspectée pour décider de l'ordre. 
    Finalement, \xfbox[black!80!cyan!40]{\texttt{\$2}} désigne 
    le deuxième champ de chaque ligne. Ainsi, la commande 
    \xfbox[black!80!cyan!40]{\texttt{awk '{print \$2}'}} 
    permet d'afficher la deuxième colonne de chaque ligne. 


    \begin{Exemple}{Filtrage des mots selon leur occurence}
        \begin{center}
            \begin{adjustbox}{max width=\linewidth}
                \begin{cminted}{bash}
cat zola1.txt | tr '[[:punct:]]' ' ' 
              | tr '[[:space:]]' '\n' 
              | grep -i '^ven' 
              | sort | uniq -c 
              | sort -k1,1nr 
              | awk '$1 > 3 {print $0}'
                \end{cminted}
            \end{adjustbox}
        \end{center}
    \end{Exemple}
    
    Le mot clé \xfbox[black!80!cyan!40]{\texttt{awk '\$1 > 3 {print \$0}'}}
    affiche uniquement les mots avec plus de 3 occurences, avec leur 
    fréquence. 


    \section{La commande awk}
    \begin{Définition}[La commande \text{wc}]
        La commande wc ou \textit{word count} est utilisé 
        pour compter les lignes, le smots ou les caractères dans uen entrée 
        standard ou un fichier. 
    \end{Définition}

    \vspace{1em}
    \begin{Exemple}{Compter les mots uniques}
        \begin{center}
            \begin{adjustbox}{max width=\linewidth}
                \begin{cminted}{bash}
cat zola1.txt | tr      '[[:punct:]]' ' ' 
              | tr   -s '[[:space:]]' 
              | tr      '[[:space:]]' '\n' 
              | sort  
              | uniq -c 
              | wc   -l
                \end{cminted}
            \end{adjustbox}
        \end{center}
    \end{Exemple}

    La commande \xfbox[black!80!cyan!40]{\text{\texttt{uniq -c}}} 
    regroupe les lignes identiques adjacentes. À titre 
    d'exemple, cette même commande engendrerait la transformation suivante 
        \begin{align*}
            &\texttt{arbre} \quad \phantom{\xrightarrow{\hspace*{3cm}}} &
            \\
            &\texttt{arbre} \quad \phantom{\xrightarrow{\hspace*{3cm}}}  &\texttt{2 arbre}
            \\
            &\texttt{vache} \;\;\quad \xrightarrow{\hspace*{4cm}} &\texttt{1 vache}
            \\
            &\texttt{chien} \quad \phantom{\xrightarrow{\hspace*{3cm}}} &\texttt{2 chien}
            \\
            &\texttt{chien} \quad \phantom{\xrightarrow{\hspace*{3cm}}} &
        \end{align*}

        La commande \xfbox[black!80!cyan!40]{\texttt{wc -l}} quant à elle 
    compte uniquement le nombre de lignes. Ainsi, dans l'exemple c







                                                                                    
    








\end{document}


