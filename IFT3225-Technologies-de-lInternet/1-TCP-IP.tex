\documentclass[a4paper]{report}





% Custom files; preamble and choice of font
%==================
% Custom Files 
%==================
\input{preamble/main.tex}
\input{preamble/language/font-choice.tex}




\begin{document}
%==================
% Title page 
%==================
\frontmatter                    % Declares numbering style of chapters
\input{preamble/title-page.tex}
\pagebreak


%==================
% Table of contents
%==================
% table of contents numered in romain numerals
\thispagestyle{plain}
\tableofcontents


\mainmatter                     % Declares numbering style of chapter
\thispagestyle{plain}           % Simple header after each new chapter
\chapter{Modèle TPC-IP}
\begin{center}
    \begin{tikzpicture}[
        node distance=1.5cm and 1.5cm,
        every node/.style={font=\sffamily},
        box/.style={rectangle, draw, fill=gray!30, minimum height=1.5cm, minimum width=3.2cm, align=center},
        desc/.style={rectangle, draw, text width=3.5cm, align=left, font=\sffamily\small},
        arrow/.style={-{Stealth}, thick}
    ]
        % Noeuds principaux (les couches)
        \node[box] (box1) at (0, 6) {Couche Application};
        \node[box, below=0cm of box1] (box2) {Couche Transport};
        \node[box, below=0cm of box2] (box3) {Couche Internet};
        \node[box, below=0cm of box3] (box4) {Interface Réseau};

        % Descriptions alternées (droite/gauche)
        \node[desc, above right=1.25cm and 0cm of box1] (desc1) {
            \textbf{Protocoles :} HTTP, FTP, SMTP, etc. \\
            Gère les interactions utilisateur.
        };

        \node[desc, below left=0.25cm and 0.75 of box2] (desc2) {
            \textbf{Protocoles :} TCP, UDP \\
            Assure la communication fiable ou rapide.
        };

        \node[desc, above right=0.25cm and 0.75cm of box3] (desc3) {
            \textbf{Protocoles :} IP, ICMP \\
            Gère l'adressage et le routage des paquets.
        };

        \node[desc, below left=1.25cm and 0cm of box4] (desc4) {
            \textbf{Technologies :} Ethernet, Wi-Fi \\
            Transport physique des données.
        };

        % Flèches reliant couches et descriptions
        \draw[thick, -{Stealth}] (box1.east) -- ++(1.5,0) -- (desc1.south);
        \draw[thick, -{Stealth}] (box2.west) -- ++(-2.25,0) -- (desc2.north);
        \draw[thick, -{Stealth}] (box3.east) -- ++(2.45,0) -- (desc3.south);
        \draw[thick, -{Stealth}] (box4.west) -- ++(-1.25,0) -- (desc4.north);
    \end{tikzpicture}
\end{center}

    \begin{Définition}[Modèle TCP/IP]
        Le modèle TCP/IP est une architecture simplifiée 
        en \textbf{4 couches}  qui décrit 
        comment les données sont échangées sur un réseau, 
        notamment l'Internet. 
    \end{Définition}        

    \section{Couche Application}
    Cette couche inclut des protocoles tels que \texttt{HTTP}, 
    \texttt{FTP}, \texttt{SSH}, \texttt{Telnet}, \texttt{SMTP}, etc. 
    Elle regroupe les fonctionnalités des couches \textit{Application}, 
    \textit{Présentation}, et \textit{Session} du modèle OSI.

    \begin{note}{}{}
        Le Modèle OSI (\textit{Open System Interconnection}) est un modèle 
        en \textbf{sept couches} qui décrits les fonctionnalités nécessaires 
        à la communication des systèmes informatiques en réseau
    \end{note}

    \section{Couche Transport}
    Elle est responsable de la \textbf{communication de bout en bout} entre 
    les applications sur différents hôtes. 
    Les protocoles principaux sont \texttt{TCP}   
    (\textit{Transmission Control Protocol}  ) 
    et \texttt{UDP}  (\textit{User Datagram Protocol}).

    \section{Couche Internet}
    Responsable de l'\textbf{adressage}, du \textbf{routage}  
    et de l'\textbf{envoi}  des paquets entre les réseaux. 
    Les protocoles inclus sont \texttt{IP}   (\textit{Internet Protocol}) 
    et \texttt{ICMP} (\textit{Internet Control Message Protocol}).

    \section{Couche Interface Réseau}
    Correspond à la liaison de données et à la 
    \textbf{couche physique} du modèle OSI. 
    Elle s'occupe de la transmission physique des données 
    à travers le réseau, par exemple via \texttt{Ethernet}.


\begin{tikzpicture}[
    every node/.style={font=\sffamily},
    box/.style={rectangle, draw, fill=gray!30, minimum height=1cm, minimum width=4cm, align=left, font=\sffamily\small},
    largebox/.style={rectangle, fill=gray!30, draw, minimum height=3cm, minimum width=4cm, align=left, font=\sffamily\small},
    arrow/.style={-{Stealth}, thick},
    largebox2/.style={rectangle, draw, fill=gray!30, minimum height=2cm, minimum width=4cm, align=left, font=\sffamily\small},
    arrow/.style={-{Stealth}, thick},
    node distance=0cm
]

% Modèle OSI
\node[box] (osi1) at (0, 6) {Couche Application };
\node[box, below=of osi1] (osi2) {Couche Présentation };
\node[box, below=of osi2] (osi3) {Couche Session };
\node[box, below=of osi3] (osi4) {Couche Transport };
\node[box, below=of osi4] (osi5) {Couche Réseau};
\node[box, below=of osi5] (osi6) {Couche d'Adressage };
\node[box, below=of osi6] (osi7) {Couche Physique };

\node[above=0.25cm of osi1, font=\bfseries] {Modèle OSI};

% Modèle TCP/IP
\node[largebox, right=3cm of osi1, yshift=-1cm] (tcp1) {Couche Application };
\node[box, below=of tcp1] (tcp2) {Couche Transport };
\node[box, below=of tcp2] (tcp3) {Couche Internet };
\node[largebox2, below=of tcp3] (tcp4) {Couche Interface Réseau };

\node[above=0.25cm of tcp1, font=\bfseries] {Modèle TPC/IP};

% Flèches avec segmentation pour angles droits
\draw[thick] (osi1.east) -- ++(1.5,0) -- ++(0, -1);
\draw[arrow] (osi2.east) --  (tcp1.west);
\draw[thick] (osi3.east) -- ++(1.5,0) -- ++(0, 1);
\draw[arrow] ([yshift=0.035cm]osi4.east) --  (tcp2.west);
\draw[arrow] ([yshift=0.035cm]osi5.east) --  (tcp3.west);
\draw[thick] (osi6.east) -- ++(1.5,0) -| ++(0,-0.5);
\draw[arrow] (osi7.east) -- ++(1.5,0) -- ++(0,+0.55) -- (tcp4.west);

\end{tikzpicture}





    \section{DARPA}
    Fondé en 1969, DARPA est l'organisme à l'orgine du premier 
    \textbf{réseau à commutation de paquets} : \texttt{DARPANET}. 
    La première démonstration du \texttt{DARPANET} remonte à 1972, 
    mais c'est seulement en 1983, que DARPA adopte la suite de protocoles 
    \texttt{TPC/IP} qui sera la base de la communication Internet. 




    \section{Port}
    \begin{Concept}[Port de Communication]
    Un \textbf{port de communication} est une sorte de point d'accès logique 
    qu'un ordinateur ou un serveur utilise pour identifier 
    un service ou une application spécifique avec laquelle communiquer. 
    \end{Concept}                       

    \subsection{Isolation des Services}
    Chaque service ou protocole écoute sur un port spécifique, 
    ce qui permet à plusieurs services de fonctionner simultanément sur 
    la même machine \textbf{sans interférer}.

    \subsection{Gestion des Connexions Multiples}
    Un serveur peut gérer plusieurs connexions en utilisant 
    des \textbf{combinaisons uniques} de \texttt{IP}  et port pour chaque client.

    \subsection{Sécurité}
    Les ports peuvent être \textbf{ouverts ou fermés }  par des 
    pare-feux pour limiter l'accès. Connaître les ports 
    associés à des services permet de configurer des règles de sécurité réseau.


    \begin{rndtable}{|l|c|l|}
        \rowcolor{tablecolor} 
        \multicolumn{3}{|c|}{\color{white} 
        \textbf{PORTS PAR DÉFAUT}} 
        \\ 
        \hline
        \textbf{Protocole} 
        & 
        \textbf{Port} 
        & 
        \textbf{Description}  
        \\ 
        \hline
        \texttt{FTP}   & \entoure{21\phantom{1}} & File Transfer Protocol   \\ \hline
        \texttt{SMTP}   & \entoure{25\phantom{1}} & Simple Mail Transfer Protocol   \\ \hline
        \texttt{HTTP}   & \entoure{80\phantom{1}} & Hyper Text Transfert Protocol   \\ \hline
        \texttt{POP}   & \entoure{110} & Post Office Protocol   \\ \hline
        \texttt{IMAP}   & \entoure{143} & Internet Messagew Access Protocol \\ \hline
        \texttt{Telnet}   & \entoure{23\phantom{1}} & Connexion à une autre machine  \\ \hline
        \texttt{Finger}   & \entoure{79\phantom{1}} & Récupération d'info. utilisateur   \\ \hline
    \end{rndtable}

    \chapter{Protocoles Courants}
    \section{SSH}
    \begin{Protocole}[SSH]
        Permet de se connecter et de contrôler un autre ordinateur via le réseau; 
        souvent utilisé pour la \textbf{connexion à distance}.  
    \end{Protocole}

    \begin{center}
        \fontfamily{lmss}
        \( \underbrace{\texttt{\phantom{000}ssh\phantom{000}}}_{\text{connexion}}  \)
        \(\underbrace{\texttt{username}}_{\text{nom d'u./machine}} \)\texttt{@}
        \( \underbrace{\texttt{remote\_host}}_{\text{adr. IP}}  \)
        $\longrightarrow$
        \entoure{ssh username@remote\_host}
        

    \begin{EExample}{Connexion aux Serveurs du DIRO}{}
        \begin{center}
            \texttt{ssh arcade.iro.umontreal.ca}  
        \end{center}
    \end{EExample}

    \begin{note}{}{}
        Après avoir établit une connexion sur \texttt{arcade} (DIRO), la commande 
        \texttt{finger} affiche le contenu des fichier \texttt{.plan} et \texttt{.project} 
        du \texttt{\mytexttilde/home/}  
    \end{note}
    \end{center}            

    \section{FTP}
    \begin{Protocole}[SFPT]
        Permet d'enclencher un mécanisme de connexion à distance pour effectuer 
        le \textbf{transfert sécuritaire de données}      
    \end{Protocole}

    \begin{rndtable}{|l|l|l|}
        \rowcolor{tablecolor} 
        \multicolumn{3}{|c|}{\color{white} 
        \textbf{DIFFÉRENCES ENTRE FPT ET SFTP}} 
        \\ 
        \hline
        \textbf{Caractéristique} 
        & 
        \textbf{FTP} 
        & 
        \textbf{SFTP}  
        \\ 
        \hline
        \texttt{Sécurité}   & Non sécurisé & Sécurisé grâce à SSH   \\ \hline
        \texttt{Port par défaut}   & \entoure{21} & \entoure{22}   \\ \hline
        \texttt{Transfert de données}   & Non chiffré & Chiffré \\ \hline
        \texttt{Protocole sous-jacents}   & Protocole \texttt{FTP}   & Protocoles \texttt{FTP},  \texttt{SSH} 
        \\ \hline
    \end{rndtable}


    \begin{center}
        \fontfamily{lmss}
        \( \underbrace{\texttt{\phantom{000}sftp\phantom{000}}}_{\text{connexion}}  \)
        \(\underbrace{\texttt{username}}_{\text{nom d'u./machine}} \)\texttt{@}
        \( \underbrace{\texttt{remote\_host}}_{\text{adr. IP}}  \)
        $\longrightarrow$
        \entoure{sftp username@remote\_host}
    \end{center}
        

    \begin{EExample}{Transfert Sécuritaire de Fichiers sur les Serveurs du DIRO}{}
        \begin{center}
            \texttt{sftp arcade.iro.umontreal.ca}  

        \vspace{1em}

       \begin{tikzpicture}[dirtree]
            \node[directory] {\fontfamily{lmss}\selectfont Commandes courantes SFTP}
            child { node {ls}}
            child { node {pwd}}
            child { node {lcd}}
            child { node[directory] {get file.txt}}
            child { node[directory] {put file.txt}}
            child { node[directory] {exit}};
        \end{tikzpicture}%
        \end{center}
    \end{EExample}

    \section{SMTP}
    \begin{Protocole}[SMTP]
        Permet d'\textbf{envoyer des e-mails} d’un client de messagerie 
        (comme \texttt{Outlook}  ou \texttt{Thunderbird}) 
        à un serveur de messagerie, ou entre serveurs de messagerie.
    \end{Protocole}

    \texttt{SMTP}   utilise le protocole \texttt{TCP} (\textit{Transmission control Protocol}) 
    et peut être utilisé sur différents ports, dépendamment du besoin.

    \begin{center}
       \begin{tikzpicture}[dirtree]
           \node[directory] {\fontfamily{lmss}\selectfont Ports pour les tâches SMTP}
               child { node {25\phantom{7} : \fontfamily{lmss}\selectfont Port par défaut SMTP non sécurisé}}
            child { node {587 : \fontfamily{lmss}\selectfont Pour pour le traffic SMTP sécurisé par STARTLS}}
            child { node {465 : \fontfamily{lmss}\selectfont Pour pour SMTP avec SSL/TLS}};
        \end{tikzpicture}%
    \end{center}

    \texttt{SMTP} peut invoqué par différents clients tels que  
    \texttt{telnet}, \texttt{OpenSSL}  et \texttt{netcat}, puisque \texttt{SMTP} repose 
    sur \texttt{TCP} et que ces trois clients \textbf{prennent en charge} \texttt{TCP}.      

    \begin{center}
        \fontfamily{lmss}
        \( \underbrace{\texttt{telnet\phantom{p}}}_{\text{\phantom{p}client}}  \)
        \( \underbrace{\texttt{smtp}}_{\text{prot.}}  \)
        .
        \( \underbrace{\texttt{remote\_host}}_{\text{adr. IP}}  \)
        \( \underbrace{\texttt{25}}_{\text{port}}  \)
        $\longrightarrow$
        \entoure{telnet smtp.example.com 25}
    \end{center}


    \begin{EExample}{Utilisation de SMTP via \texttt{netcat}}{}
        \texttt{nc -v mail.iro.umontreal.ca 25}  
        \begin{lstlisting}
        HELO mydomain.com
        MAIL FROM:<your_email@example.com>
        RCPT TO:<recipient@example.com>
        DATA
        Subject: Test Email with Netcat
        Hello! This email was sent using SMTP and netcat.
        .
        QUIT
        \end{lstlisting} 
        \begin{center}
            \hyperlink{Utilisation de SMTP via netcat}{
                \xtargetbox[myb]{
                    \fontfamily{lmss}
                    \selectfont Détails des options \;}
                }
        \end{center}
    \end{EExample}


    \section{STMP via l'outil curl}
    \begin{Définition}[Commande \texttt{curl}]
        L'outil \texttt{curl} est polyvalent et permet de \textbf{gérer différents protocoles}  
        en combinant des séquences de commandes.
    \end{Définition}

    \begin{EExample}{Utilisation de \texttt{curl} pour envoyer un e-mail}{}
        \begin{center}
            \begin{lstlisting}
        curl --url 'smtps://mail.iro.umontreal.ca:465' \
             --ssl-reqd                                \
             --mail-from 'felipe@iro.umontreal.ca'     \
             --mail-rcpt 'felipe@iro.umontreal.ca'     \
             --upload-file mail.txt                    \
             --user 'felipe@iro.umontreal.ca:passwd'   \
             --insecure               
            \end{lstlisting}
            \hyperlink{Utilisation de curl pour envoyer un e-mail}{
                \xtargetbox[myb]{
                    \fontfamily{lmss}
                    \selectfont Détails des options \;}
                }
        \end{center}
    \end{EExample}              
    


    \section{POP}
    \begin{Protocole}[POP]
        Permet de \textbf{récupérer des e-mails} depuis un serveur de messagerie 
        distant jusqu'à un client de messagerie (comme \texttt{Outlook} ou 
        \texttt{Thunderbird}), en téléchargeant les messages localement. Le protocole 
        permet aussi de \textbf{supprimer tous les messages} lus.  
    \end{Protocole}

    \texttt{POP} utilise le protocole \texttt{TCP} (\textit{Transmission Control 
    Protocol}) et fonctionne généralement sur deux ports standards, selon le 
    niveau de sécurité.

    \begin{center}
       \begin{tikzpicture}[dirtree]
           \node[directory] {\fontfamily{lmss}\selectfont Ports pour POP}
               child { node {110 : Port standard sans chiffrement}}
               child { node {995 : Port pour POP sécurisé via SSL/TLS}};
       \end{tikzpicture}%
    \end{center}

    Contrairement à \texttt{IMAP}, \texttt{POP} télécharge les messages et 
    les supprime par défaut du serveur, bien que cette option puisse être 
    configurée dans certains clients.

    \texttt{POP} peut être invoqué par des outils comme \texttt{telnet} et 
    \texttt{openssl}, car il repose également sur \texttt{TCP}. Ces outils 
    permettent de tester la connectivité et d'interagir manuellement avec un 
    serveur \texttt{POP}.     

    \begin{center}
        \fontfamily{lmss}
        \( \underbrace{\texttt{openssl}}_{\text{\phantom{p}client}}  \)
        \( \underbrace{\texttt{pop3s}}_{\text{prot.}}  \)
        .
        \( \underbrace{\texttt{remote\_host}}_{\text{adr. IP}}  \)
        \( \underbrace{\texttt{995}}_{\text{port}}  \)
        $\longrightarrow$
        \entoure{openssl s\_client -connect pop.example.com:995}
    \end{center}

    \begin{EExample}{Utilisation de POP via \texttt{telnet}}{}
        \begin{center}
         \texttt{telnet pop.example.com 110}  
        \begin{lstlisting}
        +OK POP3 server ready
        USER your_email@example.com
        +OK User accepted
        PASS your_password
        +OK Password accepted
        LIST
        +OK 2 messages
        1 1024
        2 2048
        RETR 1
        +OK Message follows
        Subject: Test Email
        Hello! This email was retrieved using POP and telnet.
        .
        QUIT
        +OK Goodbye
        \end{lstlisting} 
        \hyperlink{Utilisation de POP via telnet}{
                \xtargetbox[myb]{
                    \fontfamily{lmss}
                    \selectfont Détails des options \;}
                }           
        \end{center}

    \end{EExample}


    \section{POP via l'outil curl}


    \begin{EExample}{Utilisation de pop3 via \texttt{curl}}{}
        \begin{center}
            \begin{lstlisting}
curl -o mail.pop -v --ssl-reqd -u 'felipe:passwd' \
                    --request UIDL                \ 
                    --url pop3://mail.iro.umontreal.ca
            \end{lstlisting}
        
        \hyperlink{Utilisation de op3 via curl}{
                \xtargetbox[myb]{
                    \fontfamily{lmss}
                    \selectfont Détails des options \;}
                }   
        \end{center}
    \end{EExample}




    \section{IMAP}
    \begin{Protocole}[IMAP]
        Permet de \textbf{gérer des e-mails à distance} depuis un serveur de 
        messagerie tout en maintenant les messages sur le serveur. Contrairement 
        à \texttt{POP}, \texttt{IMAP} synchronise les actions entre le client et 
        le serveur (e.g., marquer un e-mail comme lu).
    \end{Protocole}

    \texttt{IMAP} utilise le protocole \texttt{TCP} et fonctionne sur des ports 
    standards.

    \begin{center}
       \begin{tikzpicture}[dirtree]
           \node[directory] {\fontfamily{lmss}\selectfont Ports pour IMAP}
               child { node {143 : \fontfamily{lmss}\selectfont IMAP sans chiffrement}}
               child { node {993 : \fontfamily{lmss}\selectfont IMAP avec SSL/TLS}};
       \end{tikzpicture}%
    \end{center}

    \texttt{IMAP} peut être invoqué par des outils comme \texttt{curl}, offrant une 
    interface scriptable pour interagir avec les boîtes de réception distantes.

    \begin{EExample}{Connexion simple via \texttt{curl} avec \texttt{IMAP}}{}
        \begin{lstlisting}
    curl --insecure \
         --url 'imaps://mail.iro.umontreal.ca' \
         --user 'felipe:passwd'
        \end{lstlisting}
        \begin{center}
            \hyperlink{Connexion simple via curl avec IMAP}{
                \xtargetbox[myb]{
                    \fontfamily{lmss}
                    \selectfont Détails des options \;}
            }
        \end{center}
    \end{EExample}

    \begin{EExample}{Examiner une boîte de réception avec \texttt{EXAMINE INBOX}}{}
        \begin{lstlisting}
    curl --insecure \
         --url 'imaps://mail.iro.umontreal.ca' \
         --user 'felipe:passwd' \
         --request "EXAMINE INBOX"
        \end{lstlisting}
        \begin{center}
            \hyperlink{Examiner une boîte de réception avec EXAMINE INBOX}{
                \xtargetbox[myb]{
                    \fontfamily{lmss}
                    \selectfont Détails des options \;}
            }
        \end{center}
    \end{EExample}

    \begin{EExample}{Télécharger un message avec \texttt{UID}}{}
        \begin{lstlisting}
    curl -o mail.imap \
         --insecure \
         --url 'imaps://mail.iro.umontreal.ca/INBOX;UID=14050' \
         --user 'felipe:passwd'
        \end{lstlisting}
        \begin{center}
            \hyperlink{Télécharger un message avec UID}{
                \xtargetbox[myb]{
                    \fontfamily{lmss}
                    \selectfont Détails des options \;}
            }
        \end{center}
    \end{EExample}

    \begin{EExample}{Afficher l'en-tête d'un e-mail avec \texttt{head}}{}
        \begin{lstlisting}
    head -n 8 mail.imap
        \end{lstlisting}
        \begin{center}
            \hyperlink{Afficher l'en-tête d'un e-mail avec head}{
                \xtargetbox[myb]{
                    \fontfamily{lmss}
                    \selectfont Détails des options \;}
            }
        \end{center}
    \end{EExample}

    \begin{EExample}{Afficher la fin d'un e-mail avec \texttt{tail}}{}
        \begin{lstlisting}
    tail -n 20 mail.imap
        \end{lstlisting}
        \begin{center}
            \hyperlink{Afficher la fin d'un e-mail avec tail}{
                \xtargetbox[myb]{
                    \fontfamily{lmss}
                    \selectfont Détails des options \;}
            }
        \end{center}
    \end{EExample}

    \chapter{Internet et HTTP}
    \section{Fonctionnement du Web et URI}

    \begin{Concept}[Fonctionnement du Web]
        Le Web est la conséquence de l'interaction entre 
        \textbf{trois ensembles de technologies}  , soient les \texttt{URI} 
        (\textit{Universal Ressource Locator}), le protocole 
        \texttt{HTTP}, et le langage \texttt{HTML}    
        \begin{center}
            \fontfamily{lmss}\selectfont Web = 
            \texttt{URI} + \texttt{HTTP} + \texttt{HTML}      
        \end{center}
    \end{Concept}

    Une \texttt{URI}  fait référence au \textbf{nom général}  de la ressource, alors 
    que l'\texttt{URL} permet de \textbf{localiser spécifiquement} la 
    ressource et la portion de la ressource qu'on désire consulter.

    \begin{Concept}
        Une URL est peut être composée, entre autres, 
        d'un \textbf{schéma} (p. ex \texttt{http}), 
        d'un \textbf{nom de domaine}, d'un \textbf{chemin}, de 
        \textbf{paramètres}  
        et d'une \textbf{ancre}.   
        \fontfamily{lmss}\selectfont
        \[
        \underbrace{\texttt{scheme}}_{\text{protocol}}%
        \texttt{:}%
        \texttt{[}\underbrace{\texttt{//authority}}_{\text{authentification ou domaine}}\texttt{]}%
        \underbrace{\texttt{path}}_{\text{chemin}}%
        \texttt{[}\underbrace{\texttt{?query}}_{\text{paramètres}}\texttt{]}%
        \texttt{[}\underbrace{\texttt{\#anchor}}_{\text{ancre}}\texttt{]}
        \]
    \end{Concept}


    \fontfamily{lmss}\selectfont\small
    \begin{EExample}{URL simple}{}
           \[
    \texttt{http://}\underbrace{\texttt{www.iro.umontreal.ca}}_{\text{nom de domaine}}%
    \texttt{/}\underbrace{\texttt{\~felipe/new-home/frontal.php}}_{\text{path}}%
    \texttt{?}\underbrace{\texttt{page=resume}}_{\text{query}}
    \]
    \end{EExample}


    \begin{Concept}[Encodage URI]
        L'encodage d'un \texttt{URI}  repose sur un 
        \textbf{remplacement des caractères 
        non valides}  
        ou réservés par un code hexadécimal précédé du caractère       
    \end{Concept}
 

    \begin{rndtable}{|c|c|c|}
    \hline
    \textbf{Caractère} & \textbf{Signification}         & \textbf{Encodage} \\ \hline
    \%                 & Pourcentage (réservé)          & \texttt{\%25}     \\ \hline
    \texttt{ } (espace) & Espace                        & \texttt{\%20}     \\ \hline
    \#                 & Fragment (ancre)              & \texttt{\%23}     \\ \hline
    \&                 & Délimiteur de paramètres       & \texttt{\%26}     \\ \hline
    '                  & Apostrophe                    & \texttt{\%27}     \\ \hline
    /                  & Séparateur de chemin          & \texttt{\%2F}     \\ \hline
    ?                  & Début de la query string      & \texttt{\%3F}     \\ \hline
    =                  & Affectation dans une query    & \texttt{\%3D}     \\ \hline
    +                  & Addition ou espace            & \texttt{\%2B}     \\ \hline
    \end{rndtable}



    \section{DNS}
    \begin{Concept}[Domain Name System]
        Le \texttt{DNS} est un système distribué 
        utilisé pour \textbf{traduire les noms de domaine}   
        lisibles par l'humain 
        en adresses \texttt{IP}   compréhensibles 
        par les machines. 
    \end{Concept}


\begin{center}
\begin{tikzpicture}[
    box/.style={rectangle, draw, myb, rounded corners, minimum width=3.5cm, minimum height=1cm, align=center, fill=myb!5},
    arrow/.style={-{Stealth}, thick},
    dashedarrow/.style={-{Stealth}, thick, dashed},
    label/.style={font=\small}
]

% Nodes
\node[box] (client) {Client (example.com)};
\node[box, below=1.5cm of client] (resolver) {DNS Resolver};
\node[box, below right=1.5cm and 2cm of resolver] (rootserver) {Serveur Racine};
\node[box, below=1.5cm of rootserver] (tldserver) {Serveur TLD  (.com, .org)};
\node[box, below=1.5cm of tldserver] (authserver) {Serveur Authoritative};
\node[box, right=3.5cm of client] (webserver) {Serveur Web};

% Arrows (Recursive Queries)
\draw[arrow] (client.south) -- node[label, left] {1. Requête Initiale} (resolver.north);

% Arrows (Iterative Queries)
\draw[dashedarrow] ([yshift=0.25cm]resolver.east) --  ++(0.5,0) -| node[label, below right, yshift=-0.095cm] {2. Query} ([xshift=0.35cm]rootserver.north);
\draw[dashedarrow] ([xshift=-0.1cm]rootserver.north) |- node[label, below left] {3. Réponse} ([yshift=-0.25cm]resolver.east);

\draw[dashedarrow] ([xshift=1cm]resolver.south)  |- node[label, above right] {4. Query} ([yshift=0.15cm]tldserver.west);
\draw[dashedarrow] ([yshift=-0.25cm]tldserver.west)   -| node[label, below, xshift=1.20cm] {5. Réponse} ([xshift=0.5cm]resolver.south);

\draw[dashedarrow] ([xshift=-0.25cm]resolver.south)  |- node[label, above right] {6. Query} ([yshift=0.15cm]authserver.west);
\draw[dashedarrow] ([yshift=-0.25cm]authserver.west) -- ++(-0.5,0) -| node[label, below right] {7. Réponse} ([xshift=-0.75cm]resolver.south);

% Arrows (Réponse to client)
\draw[arrow] ([xshift=0.1cm]resolver.north) -- node[label, right] {8. Résponse Finale} ([xshift=0.1cm]client.south);

% Web Serveur
\draw[arrow] (client.east) -- node[label, above] {9. Requête HTTP} (webserver.west);
\draw[arrow] ([yshift=-0.1cm]webserver.west) --  node[label, below] {10. Réponse HTTP } ([yshift=-0.1cm]client.east);

% Additional labels
\node[align=center, below=0.25cm of rootserver, font=\small, text width=3cm] (note1) 
    {[A-M] Serveur K reçoit \\ $\sim$20k requêtes/s};
    
\end{tikzpicture}
\end{center}


    \section{HTTP}


    \begin{Concept}[HyperText Transfer Protocol (HTTP)]
        Le \textbf{HTTP}  est un protocole 
        de communication utilisé pour \textbf{transférer des données sur le Web}. Il est 
        basé sur un modèle client-serveur où le client (p. ex., un navigateur 
        web) envoie une requête au serveur, et le serveur répond avec les 
        ressources demandées. 
    \end{Concept}

    \begin{center}
        \begin{tikzpicture}[dirtree]
            \node[directory] {\fontfamily{lmss}\selectfont Types de Requêtes HTTP}
            child { node {GET}}
            child { node {POST}}
            child { node {PUT}}
            child { node {DELETE}};
        \end{tikzpicture}%
    \end{center}

    \normalfont

    \begin{note}{}{}
        Chaque requête \texttt{HTTP}  est \textbf{indépendante des autres}  , 
                  ce qui simplifie les interactions mais nécessite des mécanismes 
                  supplémentaires (comme les cookies) pour maintenir l'état.
    \end{note}



    \section{Utilisation de \texttt{curl} pour requêtes \texttt{HTTP}  }

    \begin{EExample}{Requête simple avec \texttt{curl}}{}
        \begin{lstlisting}
    curl https://www.google.ca/
        \end{lstlisting}
        \begin{center}
            \hyperlink{Requête simple avec curl}{
                \xtargetbox[myb]{\fontfamily{lmss}\selectfont Détails des options \;}
            }
        \end{center}
    \end{EExample}

    \begin{EExample}{Requête détaillée (\texttt{-v}) avec \texttt{curl}}{}
        \begin{lstlisting}
    curl -v https://www.google.ca/
        \end{lstlisting}
        \begin{center}
            \hyperlink{Requête détaillée avec curl}{
                \xtargetbox[myb]{\fontfamily{lmss}\selectfont Détails des options \;}
            }
        \end{center}
    \end{EExample}

    \begin{EExample}{Télécharger une page web avec \texttt{-o}}{}
        \begin{lstlisting}
    curl -o out -v https://diro.umontreal.ca/accueil/
        \end{lstlisting}
        \begin{center}
            \hyperlink{Télécharger une page web avec curl}{
                \xtargetbox[myb]{\fontfamily{lmss}\selectfont Détails des options \;}
            }
        \end{center}
    \end{EExample}

    \begin{EExample}{Télécharger des données JSON}{} 
        \begin{lstlisting}
    curl -o x.json -v https://jsonplaceholder.typicode.com/users
        \end{lstlisting}
        \begin{center}
            \hyperlink{Télécharger des données JSON}{
                \xtargetbox[myb]{\fontfamily{lmss}\selectfont Détails des options \;}
            }
        \end{center}
    \end{EExample}

    \begin{EExample}{Envoyer des données via un fichier avec \texttt{-d}}{}
        \begin{lstlisting}
    curl -v -d @x.json https://jsonplaceholder.typicode.com/posts
        \end{lstlisting}
        \begin{center}
            \hyperlink{Envoyer des données via un fichier avec curl}{
                \xtargetbox[myb]{\fontfamily{lmss}\selectfont Détails des options \;}
            }
        \end{center}
    \end{EExample}

    \begin{EExample}{Envoi d'une requête \texttt{POST} avec des données JSON}{} 
        \begin{lstlisting}
    curl -X POST -d  x.jons                         \
         https://jsonplaceholder.typicode.com/posts \
         -H 'Content-Type: application/json'
        \end{lstlisting}
        \fontfamily{lmss}\selectfont Fichier \texttt{x.json} :  
        \begin{lstlisting}
            {
                "title": "foo",
                "body": "bar",
                "userId": 1
            }
        \end{lstlisting}
        \begin{center}
            \hyperlink{Requête POST avec curl}{
                \xtargetbox[myb]{\fontfamily{lmss}\selectfont Détails des options \;}
            }
        \end{center}
    \end{EExample}

    \begin{EExample}{Modification de données avec une requête \texttt{PUT}}{}
        \begin{lstlisting}
    curl -X PUT -v -d                                       \
    '{"id": 101, "title": "foo","body": "bar","userId": 1}' \
         https://jsonplaceholder.typicode.com/posts/1       \
         -H 'Content-Type: application/json'
        \end{lstlisting}
        \begin{center}
            \hyperlink{Requête PUT avec curl}{
                \xtargetbox[myb]{\fontfamily{lmss}\selectfont Détails des options \;}
            }
        \end{center}
    \end{EExample}


\backmatter
\chapter{Appendices}
\section{Protocoles Courants (Détails des Commandes UNIX)}%
\fontfamily{lmss}\selectfont

\subsection{Utilisation de SMTP via \texttt{netcat}}
    \xfbox[black!80!cyan!40]{\hypertarget{Utilisation de SMTP via netcat}{\texttt{HELLO}}}  Identifie le client auprès du serveur \texttt{SMTP} 

    \xfbox[black!80!cyan!40]{\texttt{MAIL FROM}}  Indique l'adresse e-mail de l'expéditeur

    \xfbox[black!80!cyan!40]{\texttt{RCP TO}} Indique l'adresse email du destinataire

    \xfbox[black!80!cyan!40]{\texttt{DATA}} Permet de saisir le contenu du message

    \xfbox[black!80!cyan!40]{\texttt{ .\;\,}} Termine le message 


    \xfbox[black!80!cyan!40]{\texttt{QUIT}} Termine la session. 

    \subsection{Utilisation de curl pour envoyer un e-mail}
\xfbox[black!80!cyan!40]{\hypertarget{Utilisation de curl pour envoyer un e-mail}{\texttt{--url}}} spécifie l'URL du serveur \texttt{SMTP}    

    \xfbox[black!80!cyan!40]{\texttt{--ssl-reqd}} force l'utilisation de \texttt{SSL/TLS};  

    \xfbox[black!80!cyan!40]{\texttt{--mail-from}} indique l'adresse e-mail de l'expéditeur 

    \xfbox[black!80!cyan!40]{\texttt{--mail-rcpt}} 
    spécifie l'adresse e-mail du destinataire;  

    \xfbox[black!80!cyan!40]{\texttt{--upload-file}} 
    fournit le contenu du message  

    \xfbox[black!80!cyan!40]{\texttt{--user}} inclut les informations
    d'identification 
    pour l'authentification sur le serveur \texttt{SMTP};  

    \xfbox[black!80!cyan!40]{\texttt{--insecure}} 
    désactive la vérification du certificat SSL/TLS.

\subsection{Utiliation de POP via telnet}
\xfbox[black!80!cyan!40]{\hypertarget{Utilisation de POP via telnet}{\texttt{USER}}} spécifie l'adresse e-mail de l'utilisateur;  

    \xfbox[black!80!cyan!40]{\texttt{PASS}} fournit le mot de passe pour 
    l'authentification;  

    \xfbox[black!80!cyan!40]{\texttt{LIST}} affiche la liste des messages 
    disponibles avec leur taille;  

    \xfbox[black!80!cyan!40]{\texttt{RETR}} télécharge un message spécifique 
    depuis le serveur;  

    \xfbox[black!80!cyan!40]{\texttt{DELE}} suprimme un email en fournissant 
    \textbf{le numéro}                                                      

    \xfbox[black!80!cyan!40]{\texttt{TOP}} affiche les lignes d'un message

    \xfbox[black!80!cyan!40]{\texttt{QUIT}} termine la session.


\subsection{Utilisation de pop3 via curl}
\xfbox[black!80!cyan!40]{\hypertarget{Utilisation de pop3 via curl}{\texttt{-o mail.pop}}} 
    spécifie que la sortie (le contenu des messages) doit être écrite 
    dans un fichier nommé \texttt{mail.pop};  

\xfbox[black!80!cyan!40]{\texttt{-v}} 
    active le mode \textit{verbose}, qui affiche des informations détaillées 
    sur la connexion et le transfert des données, utile pour le débogage;  

\xfbox[black!80!cyan!40]{\texttt{--ssl-reqd}} 
    force l'utilisation de \texttt{SSL/TLS} pour sécuriser la connexion au 
    serveur POP;  

\xfbox[black!80!cyan!40]{\texttt{-u 'felipe:passwd'}} 
    fournit les informations d'identification  
    pour s'authentifier auprès du serveur \texttt{POP};  

\xfbox[black!80!cyan!40]{\texttt{--request UIDL}} 
    envoie la commande \texttt{UIDL} au serveur, 
    qui renvoie une liste des messages présents dans la boîte de réception 
    avec leurs identifiants uniques, sans les télécharger;  

\xfbox[black!80!cyan!40]{\texttt{--url pop3://mail.iro.umontreal.ca}} 
    spécifie l'URL du serveur \texttt{POP}, permettant de 
    se connecter au serveur à l'adresse \texttt{mail.iro.umontreal.ca}.  

\subsection{Connexion simple via \texttt{curl} avec \texttt{IMAP}}
\xfbox[black!80!cyan!40]{\hypertarget{Connexion simple via curl avec IMAP}{\texttt{--url}}} 
    spécifie l'URL du serveur \texttt{IMAP} pour établir une connexion;

\xfbox[black!80!cyan!40]{\texttt{--insecure}} 
    désactive la vérification des certificats SSL/TLS (à éviter en production);

\xfbox[black!80!cyan!40]{\texttt{--user}} 
    fournit les informations d'identification pour s'authentifier 
    auprès du serveur.

\subsection{Examiner une boîte de réception avec \texttt{EXAMINE INBOX}}
\xfbox[black!80!cyan!40]{\hypertarget{Examiner une boîte de réception avec EXAMINE INBOX}{\texttt{--request}}} 
    permet d'envoyer une commande IMAP spécifique, comme 
    \texttt{EXAMINE INBOX}, pour afficher des informations détaillées 
    sur la boîte de réception.

\subsection{Télécharger un message avec \texttt{UID}}
\xfbox[black!80!cyan!40]{\hypertarget{Télécharger un message avec UID}{\texttt{-o}}} 
    enregistre le contenu du message dans un fichier local (e.g., 
    \texttt{mail.imap});

\xfbox[black!80!cyan!40]{\texttt{UID}} 
    identifie un message spécifique dans la boîte de réception pour 
    téléchargement.

\subsection{Afficher l'en-tête d'un e-mail avec \texttt{head}}
\xfbox[black!80!cyan!40]{\hypertarget{Afficher l'en-tête d'un e-mail avec head}{\texttt{head -n}}} 
    affiche les premières lignes du fichier contenant le message pour 
    examiner l'en-tête.

\subsection{Afficher la fin d'un e-mail avec \texttt{tail}}
\xfbox[black!80!cyan!40]{\hypertarget{Afficher la fin d'un e-mail avec tail}{\texttt{tail -n}}} 
    affiche les dernières lignes du fichier pour examiner la fin 
    du message (e.g., signature ou pièces jointes encodées).


\section{Détails des commandes \texttt{curl}}
\subsection{Requête simple avec \texttt{curl}}
\xfbox[black!80!cyan!40]{\hypertarget{Requête simple avec curl}{\texttt{curl}}}
envoie une requête GET par défaut à l'URL spécifiée et affiche la réponse brute 
dans le terminal.

\subsection{Requête détaillée avec \texttt{curl}}
\xfbox[black!80!cyan!40]{\hypertarget{Requête détaillée avec curl}{\texttt{-v}}}
active le mode verbose pour afficher les détails de la requête et de la réponse 
(e.g., en-têtes HTTP).

\subsection{Télécharger une page web avec \texttt{curl}}
\xfbox[black!80!cyan!40]{\hypertarget{Télécharger une page web avec curl}{\texttt{-o}}}
enregistre la réponse dans un fichier nommé spécifiquement (\texttt{out} dans cet 
exemple).

\subsection{Télécharger des données JSON avec \texttt{curl}}
\xfbox[black!80!cyan!40]{\hypertarget{Télécharger des données JSON}{\texttt{-o}}}
enregistre les données JSON retournées par l'URL dans un fichier (\texttt{x.json}).

\subsection{Envoyer des données via un fichier avec \texttt{curl}}
\xfbox[black!80!cyan!40]{\hypertarget{Envoyer des données via un fichier avec curl}{\texttt{-d}}}
permet d'envoyer des données dans le corps de la requête HTTP. Le symbole 
\texttt{@} indique que les données doivent être lues depuis un fichier 
(\texttt{x.json}).

\subsection{Requête POST avec des données JSON}
\xfbox[black!80!cyan!40]{\hypertarget{Requête POST avec curl}{\texttt{-X POST}}}
spécifie que la requête doit utiliser la méthode \texttt{POST}. 

\noindent\xfbox[black!80!cyan!40]{\texttt{-H 'Content-Type: application/json'}}
indique que les données envoyées sont en format JSON.

\subsection{Modification avec une requête PUT}
\xfbox[black!80!cyan!40]{\hypertarget{Requête PUT avec curl}{\texttt{-X PUT}}}
utilise la méthode \texttt{PUT} pour modifier ou remplacer une ressource existante. 
Le contenu JSON est transmis dans le corps de la requête, avec l'en-tête
\texttt{-H 'Content-Type: application/json'} pour indiquer le type de données.



















\end{document}


