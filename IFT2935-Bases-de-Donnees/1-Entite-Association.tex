\documentclass[a4paper]{report}





% Custom files; preamble and choice of font
%==================
% Custom Files 
%==================
\input{preamble/main.tex}
\input{preamble/language/font-choice.tex}




\begin{document}
%==================
% Title page 
%==================
\frontmatter                    % Declares numbering style of chapters
\input{preamble/title-page.tex}
\pagebreak


%==================
% Table of contents
%==================
% table of contents numered in romain numerals
\thispagestyle{plain}
\tableofcontents


\mainmatter % Declares numbering style of chapter
\thispagestyle{plain}           % Simple header after each new chapter
\chapter{Introduction}

    \section{Concepts généraux}

    \begin{Définition}[Donnée]
        Une donnée  est une \textbf{représentation d'un fait} à l'aide 
        d'un code binaire stocké en mémoire. 
    \end{Définition}



    \begin{center}
        \begin{tikzpicture}[
            node distance=1.5cm and 1.5cm,
            every node/.style={font=\sffamily},
            box/.style={rectangle, draw, fill=gray!30, minimum height=0.75cm, minimum width=1cm, align=center},
            desc/.style={rectangle, draw, text width=3.5cm, align=left, font=\sffamily\small},
            arrow/.style={-{Stealth}, thick}
        ]
            % Noeuds principaux (les couches)
            \node[box] (box1) at (0, 6) {Phénomène};
            \node[right=4.5cm of box1] (data1) {1011100 \faWindows \; \faApple \; \faLinux 
};
            \draw[arrow, dashed] (box1.east) -- (data1.west);
        \end{tikzpicture}

        Les \textbf{types de données} représentent la nature du codage utilisé 
        pour \textbf{représenter les données} ainsi que l'ensemble 
        des opérations exécutables sur celles-ci.
    \end{center}

    \begin{Concept}
        Il faut savoir faire la \textbf{distinction} entre une donnée et une information :
        \begin{center}
                \fontfamily{lmss}\selectfont 
                \undertext{\texttt{Marie a 5 ans}}{\text{Information}} \quad\quad : 
                \undertext{\texttt{5}}{\text{Donnée (âge)}}
        \end{center}                
    \end{Concept}                   

    \section{Opérations Simples}
    \begin{center}
        \begin{tikzpicture}[dirtree]
            \node[directory] {\fontfamily{lmss}\selectfont Opération simples}
            child { node {Sélection}}
            child { node[directory] {Modification}
                child { node {Insertion}}
                child { node {Mise à Jour}}
            }
            child [missing] {} child [missing] {} child [missing] {} child [missing] {}
            child { node {Suppression}}
            ;
        \end{tikzpicture}%
    \end{center}                

    Soit le tableau suivant, nous pouvons utiliser des commandes \texttt{SQL} pour manipuler 
    les données. 
    \begin{rndtable}{|c|c|l|c|c|}
            MAT & NOM       & FONCTION   & COURS   & AN\_ENT \\ \hline
            \texttt{62945}  & Gilles    & Prof\_adj  & \texttt{MRT1111}   & 1997 \\ \hline
            \texttt{34560}  & Myriam    & Prof\_agr  & \texttt{MRT2221}   & \texttt{1993}   \\ \hline
            \texttt{21539}  & Claudine  & Prof\_adj  & \texttt{MRT3331}   & \texttt{1999}   \\ \hline
            \texttt{80200}  & Bernard   & Prof\_tit  & \texttt{MRT1112}   & \texttt{1982}   \\ \hline
            \texttt{75902}  & Yida      & Prof\_agr  & \texttt{MRT1664}   & \texttt{1990}   \\ \hline
    \end{rndtable}

 

    \begin{EExample}{Sélection d'éléments via SQL}{}
        \begin{center}
            \begin{minted}[baselinestretch=1.2, fontsize=\small]{sql}
                        SELECT NOM, FONCTION
                        FROM PROFESSEUR
                        WHERE AN_ENT > 1992;
            \end{minted} 
            \( \downarrow \)
        \end{center}


        \begin{rndtable}{|c|c|}
                        NOM      & FONCTION   \\ \hline
                        Gilles   & Prof\_adj  \\ \hline
                        Myriam   & Prof\_agr  \\ \hline
                        Claudine & Prof\_adj  \\ \hline
        \end{rndtable}

        \fontfamily{lmss}\selectfont
        \begin{tabular}{l l}
            % Ligne 1
            \xfbox[black!80!cyan!40]{\texttt{SELECT}} & spécifie la colonne à sélectioner  \\
            % Ligne 2
            \xfbox[black!80!cyan!40]{\texttt{FROM}} & spécifie la BD, soit le tableau \texttt{PROFESSEUR} 
            \\
            % Ligne 3
            \xfbox[black!80!cyan!40]{\texttt{WHERE}} & spécifie permet d'introduire une condition  \\
        \end{tabular}
          
    \end{EExample}              


    \begin{EExample}{Insertion d'un élément via SQL}{}
        \begin{center}
            \begin{minted}[baselinestretch=1.2, fontsize=\small]{sql}
        INSERT INTO PROFESSEUR (MAT, NOM, FONCTION, COURS, AN_ENT)
        VALUES (662301, 'Jian', 'Prof_adj', 'MRT2323', 1996);
            \end{minted}
            \( \downarrow \)
        \end{center}

        \begin{rndtable}{|c|c|c|c|c|}
            MAT     & NOM    & FONCTION  & COURS    & AN\_ENT \\ \hline
            \texttt{62945}  & Gilles    & Prof\_adj  & \texttt{MRT1111}   & 1997 \\ \hline
            \texttt{34560}  & Myriam    & Prof\_agr  & \texttt{MRT2221}   & \texttt{1993}   \\ \hline
            \texttt{21539}  & Claudine  & Prof\_adj  & \texttt{MRT3331}   & \texttt{1999}   \\ \hline
            \texttt{80200}  & Bernard   & Prof\_tit  & \texttt{MRT1112}   & \texttt{1982}   \\ \hline
            \texttt{75902}  & Yida      & Prof\_agr  & \texttt{MRT1664}   & \texttt{1990}   \\ \hline
            662301  & Jian   & Prof\_adj & MRT2323  & 1996    \\ \hline
        \end{rndtable}

        \fontfamily{lmss}\selectfont
        \begin{tabular}{l l}
            % Ligne 1
            \xfbox[black!80!cyan!40]{\texttt{INSERT INTO}} & insère des données dans la table  \\
            % Ligne 2
            \xfbox[black!80!cyan!40]{\texttt{VALUES}} & fournit les valeurs à insérer pour
            chaque colonne \\
        \end{tabular}
    \end{EExample}

    \begin{EExample}{Mise à jour d'un élément via SQL}{}
        \begin{center}
            \begin{minted}[baselinestretch=1.2, fontsize=\small]{sql}
                        UPDATE PROFESSEUR
                        SET COURS = 'MRT2325'
                        WHERE MAT = 662301;
            \end{minted}
            \( \downarrow \)
        \end{center}

        \begin{rndtable}{|c|c|c|c|c|}
            MAT     & NOM    & FONCTION  & COURS    & AN\_ENT \\ \hline
            \texttt{62945}  & Gilles    & Prof\_adj  & \texttt{MRT1111}   & 1997 \\ \hline
            \texttt{34560}  & Myriam    & Prof\_agr  & \texttt{MRT2221}   & \texttt{1993}   \\ \hline
            \texttt{21539}  & Claudine  & Prof\_adj  & \texttt{MRT3331}   & \texttt{1999}   \\ \hline
            \texttt{80200}  & Bernard   & Prof\_tit  & \texttt{MRT1112}   & \texttt{1982}   \\ \hline
            \texttt{75902}  & Yida      & Prof\_agr  & \texttt{MRT1664}   & \texttt{1990}   \\ \hline
            662301  & Jian   & Prof\_adj & MRT2325  & 1996    \\ \hline
        \end{rndtable}

        \fontfamily{lmss}\selectfont
        \begin{tabular}{l l}
            % Ligne 1
            \xfbox[black!80!cyan!40]{\texttt{UPDATE}} & modifie les données dans une table \\
            % Ligne 2
            \xfbox[black!80!cyan!40]{\texttt{SET}} & spécifie les nouvelles valeurs \\
            % Ligne 3
            \xfbox[black!80!cyan!40]{\texttt{WHERE}} & définit une condition pour la mise à
            jour \\
        \end{tabular}
    \end{EExample}

    \begin{EExample}{Destruction d'un élément via SQL}{}
        \begin{center}
            \begin{minted}[baselinestretch=1.2, fontsize=\small]{sql}
                    DELETE FROM PROFESSEUR
                    WHERE MAT = 662301;
            \end{minted}
            \( \downarrow \)
        \end{center}

        \begin{rndtable}{|c|c|c|c|c|}
            MAT     & NOM    & FONCTION  & COURS    & AN\_ENT \\ \hline
            \texttt{62945}  & Gilles    & Prof\_adj  & \texttt{MRT1111}   & 1997 \\ \hline
            \texttt{34560}  & Myriam    & Prof\_agr  & \texttt{MRT2221}   & \texttt{1993}   \\ \hline
            \texttt{21539}  & Claudine  & Prof\_adj  & \texttt{MRT3331}   & \texttt{1999}   \\ \hline
            \texttt{80200}  & Bernard   & Prof\_tit  & \texttt{MRT1112}   & \texttt{1982}   \\ \hline
            \texttt{75902}  & Yida      & Prof\_agr  & \texttt{MRT1664}   & \texttt{1990}   \\ \hline
            % Aucune ligne correspondante après la suppression
        \end{rndtable}

        \fontfamily{lmss}\selectfont
        \begin{tabular}{l l}
            % Ligne 1
            \xfbox[black!80!cyan!40]{\texttt{DELETE FROM}} & supprime les données d'une table \\
            % Ligne 2
            \xfbox[black!80!cyan!40]{\texttt{WHERE}} & limite la suppression aux lignes qui
            respectent une condition \\
        \end{tabular}
    \end{EExample}

    \section{Base de données}
    \begin{Concept}[Bases de données]
        Les larges collections de données nécessitent des \textbf{systèmes}---les bases de données---et 
        des \textbf{logiciels} pour être gérées. 
    \end{Concept}                   

    \begin{center}
        \begin{tikzpicture}[dirtree]
            \node[directory] {\fontfamily{lmss}\selectfont Propriétés des bases de données}
            child { node[directory] {Structuré}}
            child { node[directory] {Cohérent}
                child { node {L'ensemble a une signification}}
            }
            child [missing] {} child [missing] {} child [missing] {}
            child { node[directory] {Paratagé}
                child { node {Utilisé par \textbf{plusieurs utilisateurs}}}
            }
            child [missing] {} child [missing] {} 
            child { node[directory] {A une raison d'être}}
            ;
        \end{tikzpicture}%
    \end{center}

    \begin{note}{}{}
        On retrouve \textbf{trois type d'acteurs} qui interagissent avec les BD, soit les 
        concepteurs, les administrateurs et les utilisateurs finaux.
    \end{note}


    \begin{center}
        \begin{tikzpicture}[dirtree]
            \node[directory] {\fontfamily{lmss}\selectfont Avantages des bases de données}
            child { node[directory] {Réduit la redondance}}
            child [missing] {}
            child { node[directory] {Évite l'incohérence}}
            child [missing] {} 
            child { node[directory] {Permet le partage des données}
            }
            child [missing] {} child [missing] {} 
            child { node[directory] {Garantie la sécurité}
                    child{ node {Pour les utilisiteurs}}
                    child{ node {Pour les Administration}
                           child[missing] {}
                           child{ node {consultation, destruction, insertion}}
                    }
            }
            child [missing] {} child [missing] {} 
            child { node[directory] {Assure l'intégrité}
            }
            child [missing] {} child [missing] {} 
            child { node[directory] {Établir des priorités}
            }
            child [missing] {} child [missing] {} 
            child { node[directory] {Assure l'indépendance des données}
            }
            ;
        \end{tikzpicture}%
    \end{center}

    \section{SGBD}
    \begin{Concept}[Système de gestion de base de données]
        Les SGBD sont des \textbf{logiciels} spécialisés qui permettent 
        la définition, la construction, la manipulation et la maintenance des BD. 
        Ils reposent sur les \textbf{langage de définition de données} (LDD) 
        et les \textbf{langage de manipulation de données} (LMD) : 
        \begin{center}
            \texttt{SGBD} = \texttt{LDD} + \texttt{LMD}      
        \end{center}
    \end{Concept}

    \begin{tikzpicture}[dirtree]
        \fontfamily{lmss}\selectfont
        \node[directory] {Composantes d'un SGBD}
            child{ node[directory] {LDD}
                    child{ node {Définit les \textbf{types}   de données de la BD}}
                    child{ node {Définit les \textbf{structures de données}  de données de la BD}}
            }
            child[missing] {} child[missing] {} 
                        child{ node[directory] {LMD}
            child{ node {Lire}}
            child{ node {Ajouter}}
            child{ node {Supprimer}}
            child{ node {Modifier}}
            }
         ;
    \end{tikzpicture}



    \fontfamily{lmss}\selectfont

    \begin{center}
    \begin{tikzpicture}[node distance=2cm]

        % Styles des formes
        \tikzstyle{entity} = [rectangle, draw=black, fill=gray!30, 
                              minimum width=1.75cm, minimum height=1.5cm, text centered]
        \tikzstyle{arrow} = [thick,->,>=stealth]
        \tikzstyle{description} = [text width=6cm, align=center, font=\small]

        % Noeuds
        \node (users) [circle, draw=black, fill=gray!30, 
                       minimum size=2cm, text centered] 
                       {Utilisateurs};
                       
        \node (sgbd) [entity, right of=users, xshift=3.5cm] 
                       {SGBD};
                       
        \node (database) [cylinder, draw=black, fill=gray!30, shape border rotate=90,
                          minimum height=1.5cm, minimum width=1.5cm, right of=sgbd, 
                          xshift=4cm, text centered] 
                          {BD};

        % Flèches
        \draw [arrow] (users) -- (sgbd) node[midway, above, font=\small] 
                       {Interface utilisateur};
        \draw [arrow] (sgbd) -- (database) node[midway, above, font=\small] 
                       {Interface d'accès physique};
        \draw [arrow] (database) -- (sgbd) node[yshift=1.2cm, midway, above, font=\small] 
                       {Tâches : Stockage / accès aux données};
        \draw [arrow] (sgbd) -- (users) node[yshift=-1.2cm, midway, below, font=\small] 
                       {Tâches : Déf. contenu BD, Interrogation BD, MAJ BD};

    \end{tikzpicture}

    \end{center}

    \begin{Concept}[Architecture à 3 niveaux d'un SGBD]
        L'architecture d'un SGBD est organisée en trois niveaux distincts. 
        Le \textbf{niveau externe} représente la vue des utilisateurs et des programmes 
        qui interagissent avec le système, adaptée à leurs besoins spécifiques. 
        Le \textbf{niveau conceptuel} offre une vue commune et intermédiaire qui est indépendante 
        des applications et de l'implantation physique, 
        permettant de se concentrer sur la structure logique des données. 
        Enfin, le \textbf{niveau interne} se concentre sur la gestion des 
        données telles qu'elles sont stockées physiquement, en prenant en 
        compte les détails techniques de leur organisation. 
        Cette architecture à trois niveaux garantit une abstraction efficace, 
        une indépendance des données et une gestion optimisée.
    \end{Concept}



    \begin{EExample}{Illustration des niveaux conceptuel, interne et externe}{}
        \fontfamily{lmss}\selectfont
        \textbf{Conceptuel :} Définition logique des données.

        \begin{rndtable}{|l|l|}
            \hline
            \texttt{Employee}   &                                  \\ \hline
            \texttt{Num\_emp}   & \texttt{CHARACTER (6)}           \\ \hline
            \texttt{Num Dept}   & \texttt{CHARACTER (46)}          \\ \hline
            \texttt{Salaire}    & \texttt{NUMERIC}                 \\ \hline
        \end{rndtable}

        \vspace{0.5cm}
        \textbf{Interne :} Définition physique et stockage des données.

        \begin{rndtable}{|l|l|l|}
            \hline
            \texttt{STORED\_EMP} & \texttt{LENGTH=20}              &                     \\ \hline
            \texttt{PREFIX}      & \texttt{TYPE=BYTE(6)}           & \texttt{OFFSET=0}   \\ \hline
            \texttt{EMP\#}       & \texttt{TYPE=BYTE(6)}           & \texttt{OFFSET=6,}  \\
                                 &                                 & \texttt{INDEX=EMPX} \\ \hline
            \texttt{DEPT\#}      & \texttt{TYPE=BYTE(4)}           & \texttt{OFFSET=12}  \\ \hline
            \texttt{PAY}         & \texttt{TYPE=FULLWORD}          & \texttt{OFFSET=16}  \\ \hline
        \end{rndtable}

        \vspace{0.5cm}
        \textbf{Externe 1 :} Représentation spécifique pour une application.

        \begin{minted}[baselinestretch=1.2, fontsize=\small]{c}
            typedef struct {
                char mat[6];
                float sal;
            } employe;
        \end{minted}

        \vspace{0.5cm}
        \textbf{Externe 2 :} Une autre vue externe adaptée à une application.

        \begin{minted}[baselinestretch=1.2, fontsize=\small]{c}
            typedef struct {
                char mat[6];
                char dept[46];
            } employe2;
        \end{minted}
    \end{EExample}


    \begin{Définition}[Architecture client-serveur d'un SGBD]
    L'architecture client-serveur dans un système de gestion de base de données (SGBD) repose 
    sur la séparation des rôles entre un client et un serveur. Le programme d'application 
    agit comme client, fournissant une interface utilisateur graphique (GUI) et prenant 
    en charge le traitement lié au domaine d'application. Le SGBD, quant à lui, joue le rôle 
    de serveur, en fournissant l'accès aux données, souvent appelé "data server". 
    \end{Définition}






    



%==================
% Appendix Chapters 
%==================
% \backmatter
% \chapter{Appendices}
% \section{Protocoles Courants (Détails des Commandes UNIX)}%
% \fontfamily{lmss}\selectfont


\end{document}
